\section*{Remerciements}

En attendant, je l'espère, une version plus complète de ces remerciements, je tiens d'abord à remercier les chercheuses et chercheurs qui ont accepté d'évaluer mon travail de thèse, particulièrement les rapportrices. 
Je remercie également mes directrices de thèse pour leur suivi pendant ces trois années. Je souhaite remercier spécifiquement l'une d'entre elles pour avoir régulièrement ouvert mes horizons, m'avoir permis d'explorer de nombreux sujets connexes à ma problématique de thèse, et ainsi d'ajouter des branches à l'arbre (pourtant déjà bien fourni) de mes sujets d'intérêt sur le plan scientfique.
Et je remercie particulièrement l'autre pour avoir su élaguer certaines de ces branches afin d'éviter une dispersion d'énergie et permettre d'aboutir à un ensemble réellement fructueux et qualitatif. Bien qu'aucune de ces deux caractéristiques ne soit exclusive, je pense qu'elle n'auront aucun mal à se reconnaître sur ces deux aspects.

Je remercie également celles et ceux avec qui j'ai pu partager l'expérience de la thèse au quotidien, à METIS (avec une mention spéciale à mes cobureaux, notamment sur la période de rédaction, et aux utilisateurs d'ORCHIDEE), et au LMD (avec une mention spéciale à celles et ceux qui m'ont aidé à m'intégrer au groupe à mon arrivée en thèse, et aux adeptes du LAM qui m'ont aidé à essuyer de nombreux plâtres), et Mariame Maiga pour son implication dans son stage de M1. 
Je remercie plus généralement les personnes avec qui j'ai pu collaborer ou échanger sur mon travail, l'hydrologie, ou la physique du climat: les membres des deux laboratoires (chercheurs, ingénieurs, équipes administratives et de direction), les membres de mon Comité de Suivi Individuel, de l'équipe DEPHY, des projets MOSAI, LIAISE, des groupes ClimAction et du Centre Climat et Société de l'IPSL, et toutes les personnes qui s'engagent pour donner du sens au travail de recherche.
Je remercie également les équipes administratives de l'Ecole Doctorale GRNE, Sorbonne Université (notamment celles et ceux qui veillent à l'entretien et la propreté des locaux dans lesquels j'ai travaillé pendant 3 ans), et l'IPSL.

Sur un plan plus personnel, je souhaite remercier ma famille, et particulièrement mes parents, pour le cadre bienveillant et stimulant dans lequel j'ai pu grandir et cultiver ma curiosité scientifique, et pour le soutien et l'intérêt qu'ils ont manifesté pour mon travail tout au long de cette thèse. 
Enfin, je souhaite remercier les colocataires et amis avec qui j'ai partagé mon quotidien en dehors du travail depuis de nombreuses années, en particulier l'équipage (étendu) du Frankiz, qui ont contribué à la meilleure motivation possible pour rendre une thèse dans les temps.


\begin{quote}
    \textit{The story so far: in the beginning the Universe was created. This has made a lot of people very angry and been widely regarded as a bad move.}
\end{quote}
Douglas Adams - Opening line of HHGG II
