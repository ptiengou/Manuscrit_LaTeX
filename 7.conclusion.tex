\section{Synthesis}
Résumé des problématiques dans la littérature actuelle

Feedback atmospherique dans GCM ?

\hfill\\
Main outputs of my work

First use case of \native routing (obviously first use case with the LAM)
%validated a new version against previous one

First use of \native with 2km DEM
% showed the need for a new choice of parameters
% routing parameters only get you so far, afterwards irrigation is the main driver (and atmospheric forcing a little bit)

First use case of LAM for land surface-atmosphere interactions
% it works
% only ERA5 lateral forcing and quite good performance overall
% identification of some biases 

Irrigation in ICOLMDZOR with atmospheric feedbacks
% temperature impacts are very local but mositening impacts are more diffuse
% recycling exists but is very partial
% stabilization dominates in intensely irrigated area so recycling happens elsewhere

\section{Perspectives}
Limitations

Irrigation is not performing well everywhere

Impacts on river discharge are still error compensation since lack of seasonality

Local impacts might also be due to error compensation (evapnu vs tran)

ABL impacts are not visible without boosting irrigation

\hfill\\
Model improvements

More water for irrigation
% more appropriate routing scheme tuning with GW ressources evaluation
% routing interp_HTU -> more relevant scale than just DEM
% Inclusion of adduction in the irrigation scheme -> non-local ressources, especially if routing grid is very small (less problematic with routing HTU)
% Articulte these elements with river dam representation, limit error 

% irrig with beta=0.9 globally probably depletes reservoirs too fast in other semi-arid climates

LAM with CMIP

LMDZ new physics (CMIP7)

Appropriate nudging options inside the LAM to study surface processes

\hfill\\
Further work that could be done from here

Better understanding of the LAM and new possibilities
% more extensive anlaysis of the influence of forcing on the LAM 
% simplified metrics/test to setup a new configuration
% LAM with CMIP format output -> lots of possibilities for studies under CC
% explore changes in LAM resolution (resolved vs parametrized ?)

ABL
% hopefully no need to cheat on the irrigation if irrig is fixed
% similar study on areas with larger irrigation patches to look for dynamical effects (mousson Guimberteau)
%explore heterogeneity of wind speed (roughness, lack of thermals ?)

Exploit LIAISE data better
% LIAISE 1D case (proper synoptic forcing and initialisation) with a focus on LA interactions 
%LAM centered on LIAISE 

More down-to-Earth water ressources analysis
% give more meaning to 