\documentclass{report}
\usepackage[a4paper, total={6in, 9in}]{geometry}
\usepackage{graphicx} % Required for inserting images
\graphicspath{ {./images/} }
\usepackage{caption}
\usepackage{subcaption}
\usepackage{pgfplotstable}
\usepackage{xcolor} % For coloring
\usepackage{amsmath} 
\usepackage{rotating} % Required for sidewaysfigure

\usepackage[utf8]{inputenc}
\usepackage[T1]{fontenc}
\usepackage{lmodern} % Use modern Latin font
\usepackage{geometry} % Custom margin settings
\usepackage{fancyhdr} % For custom headers/footers (running heads)
\usepackage{titlesec} % For custom chapter and section titles
\usepackage{lipsum} % Package for generating placeholder text
\usepackage{glossaries} 

\usepackage{minitoc}
\setlength{\mtcindent}{1.5em}
\setlength{\baselineskip}{10pt}
\renewcommand{\mtcfont}{\normalsize}
\renewcommand{\mtcSfont}{\normalsize}
\renewcommand{\mtcSSfont}{\normalsize}
% \setlength{\mtcskip}{0.3em}
% \setlength{\mtcsep}{0.2em}
\mtcsetfeature{minitoc}{before}{\vspace{-0.5em}}
\mtcsetfeature{minitoc}{after}{\vspace{-0.3em}}

\usepackage{setspace}
% \doublespacing
\onehalfspacing

%Bibliography
% \usepackage{natbib}
% \bibliographystyle{plainnat}
\usepackage[backend=biber, style=apa, backref=true, sorting=nyt, sortcites=false]{biblatex}
\addbibresource{biblio_all.bib}

\newcommand{\citep}{\parencite}
\newcommand{\citet}{\textcite}

\pgfplotsset{compat=1.18}

\usepackage{xspace}
\newcommand{\native}{\textit{interp\_topo}\xspace}
\newcommand{\std}{\textit{subgrid\_halfdeg}\xspace}
\newcommand{\noirr}{\textit{no\_irr}\xspace}
\newcommand{\irr}{\textit{irr}\xspace}
\newcommand{\irrboost}{\textit{irr\_boost}\xspace}
\newcommand{\betairrig}{$\beta_{irrig}$\xspace}
\newcommand{\presnoirr}{\textit{present\_no\_irr}\xspace}
\newcommand{\futnoirr}{\textit{future\_no\_irr}\xspace}
\newcommand{\futirr}{\textit{future\_irr}\xspace}
\newcommand{\smalld}{\textit{small\_domain}\xspace}
\newcommand{\interd}{\textit{intermediate\_domain}\xspace}
\newcommand{\larged}{\textit{large\_domain}\xspace}
\newcommand{\forcingERA}{\textit{forcing\_ERA}\xspace}
\newcommand{\forcingICO}{\textit{forcing\_ICOLMDZOR}\xspace}
\newcommand{\forcingoneh}{\textit{forcing\_1h}\xspace}
\newcommand{\forcingsixh}{\textit{forcing\_6h}\xspace}
\newcommand{\atmchapters}{Chapters \ref{chap:forcing}, \ref{chap:monthly}, and \ref{chap:liaise}\xspace}
\newcommand{\mesoexact}{\textit{mesoNH\_exact}\xspace}
\newcommand{\mesomean}{\textit{mesoNH\_mean}\xspace}

% \newcommand{\tcst1}{\textit{TCST\_1}\xspace}
% \newcommand{\tcst2}{\textit{TCST\_2}\xspace}
% \newcommand{\tcst3}{\textit{TCST\_3}\xspace}
% \newcommand{\tcst4}{\textit{TCST\_4}\xspace}
%units
\newcommand{\perday}{d$^{-1}$\xspace}
\newcommand{\persqm}{m$^{-2}$\xspace}
\newcommand{\persec}{s$^{-1}$\xspace}
\newcommand{\perkg}{kg$^{-1}$\xspace}


\usepackage{hyperref}
\usepackage{booktabs} %doesn't work ?

% This file contains all package imports and custom configuration settings
% for geometry, chapter titles (titlesec), and running heads (fancyhdr).

% --- PACKAGES ---

% Set geometry for standard thesis margins
\geometry{
    a4paper,
    top=3cm,
    bottom=2cm,
    left=3cm, % Slightly wider for binding
    right=2.5cm,
    headheight=13.6pt % Required for fancyhdr
}

% --- 1. Custom Chapter Title Styling (titlesec) ---
% Redefine the chapter title format for a bolder, display style
\titleformat{\chapter}[display]
  % Format for the entire title area
  {\normalfont\bfseries\centering\color{black}}
  % The chapter label (e.g., "CHAPTER 1")
  {\Huge\chaptertitlename\ \thechapter}
  % Vertical space between the label and the title text
  {20pt}
  % Format for the chapter name itself
  {\sffamily\Huge} % Using a sans-serif font for the title for contrast

% Explicitly control Chapter spacing: 0pt left, 50pt before, 40pt after
\titlespacing*{\chapter}{0pt}{50pt}{40pt}

% Ensure sections are also clear
\titleformat{\section}
  {\normalfont\Large\bfseries\sffamily}
  {\thesection}{1em}{}

% Explicitly control Section spacing to resolve minitoc spacing issues
% 0pt left, 12pt before, 8pt after (tight spacing)
\titlespacing*{\section}{0pt}{12pt}{8pt}

% --- 2. Custom Running Heads (fancyhdr) ---
\pagestyle{fancy}
\fancyhead{} % Clear all previous header settings
\fancyfoot{} % Clear all previous footer settings

% Page Numbers are placed on the outside edge
\fancyhead[LE, RO]{\thepage}

% Content for Odd Pages (Section title on the left)
% \nouppercase{} prevents the default behavior of capitalizing titles in headers
\fancyhead[LO]{\nouppercase{\rightmark}} 

% Content for Even Pages (Chapter title on the right)
\fancyhead[RE]{\nouppercase{\leftmark}}

% --- Custom Marks for fancyhdr ---
% Redefine what goes into \leftmark (Chapter) and \rightmark (Section)
% \MakeUppercase ensures the chapter title is capitalized in the mark register
\renewcommand{\chaptermark}[1]{\markboth{\MakeUppercase{#1}}{}}
\renewcommand{\sectionmark}[1]{\markright{\thesection\quad #1}{}}

% Add a line under the header for a professional look
\renewcommand{\headrulewidth}{0.4pt}
\renewcommand{\footrulewidth}{0pt} % No footer line

% Remove the running head from the Chapter Start page (where the title is printed)
\fancypagestyle{plain}{
    \fancyhead{}
    \renewcommand{\headrulewidth}{0pt}
}
 

\begin{document}

\begin{titlepage}
    \centering
    \vspace*{2cm}

    {\Huge\bfseries Modelling the regional impacts of irrigation on the climate, water cycle, and atmospheric boundary layer over the Iberian Peninsula\par}
    \vspace{1cm}

    {\Large Pierre Tiengou\par}
    \vspace{0.5cm}

    {\large Sorbonne Université\par}
    \vspace{0.5cm}

    {\large 01/12/2025\par}

    \vfill
    
    {\large Composition du jury \par
        Mme Fabienne LOHOU \\
        Mme Isabelle BRAUD \\
        M. Jean-Louis DUFRESNE \\
        M. Romain ROEHRIG \\
        M. Martin BEST \\
        M. Aaron BOONE \\
        Mme Agnès DUCHARNE \\
        Mme Frédérique CHERUY\\
    }
%todo:roles et grades

    % \includegraphics[width=0.3\textwidth]{logo.png}
\end{titlepage}

\section*{Abstract}

Irrigation is a widespread agricultural practice expected to keep expanding in the future. It is recognized as a major anthropogenic driver of land-atmosphere interactions, significantly affecting regional climate, continental and atmospheric water cycle components, surface energy balance, and the atmospheric boundary layer (ABL).
Under climate change, in semi-arid regions like most of the Mediterranean basin, it is associated with major challenges regarding water availability, increased evaporative demand, and disrupted precipitation patterns, justifying efforts to understand and represent the diversity of its impacts.

This thesis investigates the regional effects of irrigation on the water cycle, climate, and the atmospheric boundary layer over the Iberian Peninsula using the ICOLMDZOR limited area model (LAM) at 25-kilometre resolution. 
This new regional climate model stems from the global climate model developed at Institut Pierre-Simon Laplace (IPSL-CM), using the ORCHIDEE land surface model with a new routing scheme, and the ICOLMDZ atmospheric model.
The LAM is evaluated over the region under recent climate (2010-2022), and simulations with and without irrigation are compared to isolate its impacts.
Simulations of future climate (2050-2062) under the SSP5-8.5 scenario are also analysed to assess how these impacts interact with those of climate change.

This work first demonstrates the dominant influence of irrigation on the ability of the land surface model to simulate river discharge in offline simulations. It identifies an appropriate set of parameters for the river routing and irrigation schemes over the Iberian Peninsula, to adapt to a 1-arcminute resolution topography and better reflect regional irrigation practices. 
Using this improved representation of the land surface in coupled LAM simulations reveals that the atmospheric impacts of irrigation mainly consist in a cooling and moistening over irrigated areas and a stabilization of the ABL. Partial recycling of atmospheric moisture is identified at the scale of the Peninsula, with increases in precipitation in mountainous regions surrounding the intensely irrigated Ebro Valley.
The atmospheric processes at play are analysed in more details by comparing the ICOLMDZOR LAM to point-based observations (surface measurements and radio-soundings) from the Land Surface Interactions with the Atmosphere over the Iberian Semi-Arid Environment (LIAISE) campaign, held in the Ebro valley in July 2021.
In a sensitivity experiment with increased water availability for irrigation, surface fluxes are found to be greatly improved compared to observations. In comparison to Meso-NH simulations at 2-kilometre resolution, surface variables in the ICOLMDZOR grid cell for the irrigated observation site match the average of Meso-NH grid cells it contains. This suggests that the modelling approach for irrigation and surface fluxes used in ICOLMDZOR is sufficient to represent grid-cell average impacts at the surface. However, although the cooling and moistening effects of irrigation are found to extend vertically into the ABL, ICOLMDZOR does not achieve as good performance in ABL representation as in surface variables. This is likely attributed partly to a lack of sub-grid heterogeneities, in surface fluxes, but also in wind speed and direction, and to advection terms that do not perfectly reflect observed weather conditions.

Overall, this PhD work presents a first use-case of the new ICOLMDZOR LAM for the study of land-surface interactions at the regional and climatic scales, with visible impacts of irrigation on river discharge, precipitation, surface variables and ABL development.
Several sources of biases of the ICOLMDZOR LAM over the region were also identified and partly corrected, and possible improvements are presented for upcoming regional climate modelling studies.

% \hfill
% \textbf{Keywords:} irrigation, regional climate modelling, land-atmosphere interactions, hydrological modelling

\clearpage

\section*{Résumé}

L'irrigation est une pratique agricole répandue, dont l'expansion devrait se poursuivre à l'avenir. Elle est reconnue comme une des activités anthropiques impactant fortement les interactions entre la surface terrestre et l'atmosphère, et peut significativement influencer le climat régional, le cycle de l'eau continental et atmosphérique, le bilan énergétique à la surface et la couche limite atmosphérique (CLA).
Dans le contexte du changement climatique, notamment dans les régions semi-arides du bassin méditerranéen, elle est associée à des défis majeurs de gestion des ressources en eau, d'augmentation de la demande évaporative et de perturbations des régimes de précipitations, qui justifient les efforts pour mieux comprendre et représenter la diversité de ses impacts.

Cette thèse étudie les effets de l'irrigation sur le cycle de l'eau, le climat régional et la couche limite atmosphérique au-dessus de la péninsule Ibérique, à l'aide du modèle à aire limitée (LAM) ICOLMDZOR. Ce nouveau modèle climatique régional est dérivé du modèle climatique global de l'Institut Pierre-Simon Laplace (IPSL-CM). Il utilise le modèle de surface continentale ORCHIDEE, doté d'un nouveau schéma de routage des rivières, le modèle atmosphérique ICOLMDZ, et des conditions aux limites issues des réanalyses ERA5.
Le modèle est évalué sur la région, et des simulations avec et sans irrigation sont comparées pour isoler ses impacts.

Ce travail démontre d'abord l'influence dominante de l'irrigation sur la capacité du modèle ORCHIDEE à simuler les débits. Un ensemble de paramètres pour le routage et l'irrigation est identifié, afin d'adapter le nouveau routage à une topographie à haute résolution (1 arcmin) et de mieux refléter les pratiques d'irrigation de la péninsule Ibérique.
Ensuite, dans les simulations couplées, les impacts atmosphériques de l'irrigation se manifestent principalement par un refroidissement et une humidification de l'air au dessus des zones irriguées, ainsi qu'une stabilisation de la CLA. Un recyclage partiel de l'humidité atmosphérique est identifié à l'échelle de la péninsule, avec une augmentation des précipitations dans les régions montagneuses qui entourent la vallée de l'Ebre, fortement irriguée.
Les processus atmosphériques en jeu sont analysés plus en détail en comparant le LAM ICOLMDZOR à des données d'observation issues de la campagne LIAISE (Land Surface Interactions with the Atmosphere over the Iberian Semi-Arid Environment) dans la vallée de l'Ebre, ainsi qu'à des simulations Meso-NH à une résolution de 2 kilomètres sur la zone d'étude de LIAISE.
Dans une expérience de sensibilité avec une disponibilité accrue en eau pour l'irrigation, les flux de surface sont grandement améliorés par rapport aux observations. Les variables de surface dans la maille ICOLMDZOR du site irrigué correspondent également à la moyenne des mailles Meso-NH qu'elle contient, suggérant que l'approche en mosaïque de l'irrigation est suffisante pour représenter les impacts à la surface. Cependant, bien que le refroidissement et l'humidification générés par l'irrigation se propage verticalement dans la CLA, ICOLMDZOR ne parvient pas à atteindre la même performance dans la représentation de la CLA que pour les variables de surface. Cela est attribué en partie à l'influence des hétérogénéités sous-maille, non seulement dans les flux de surface, mais aussi dans la vitesse et la direction du vent, ainsi qu'à des termes d'advection imparfaits dans ICOLMDZOR.

Dans l'ensemble, ce travail présente une première application du nouveau LAM ICOLMDZOR pour l'étude des interactions surface-atmosphère à l'échelle du climat régional, mettant en évidence des impacts visibles de l'irrigation sur les débits des rivières, les précipitations, les variables de surface et le développement de la CLA.
Plusieurs sources de biais du LAM ICOLMDZOR sur la région sont également identifiées et en partie corrigées, et des pistes d'amélioration sont proposées pour les futures études de modélisation climatique régionale.
%todo:inclure corrections version anglais

% \hfill
% \textbf{Mots clés :} irrigation, modélisation climatique régionale, interactions surface-atmosphère, modélisation hydrologique
\clearpage

\section*{Résumé long}
\section*{Remerciements}

\clearpage
\dominitoc
\renewcommand*\contentsname{Contents}
\tableofcontents
\newpage

\chapter{Introduction}
\label{chap:introduction}
\minitoc
\pagebreak
\chapter{Introduction}
\label{chap:introduction}
\minitoc
\pagebreak

\section{Irrigation in the world and in the Iberian Peninsula}
Irrigation is a widespread practice in the world which consists in providing additional water to cultivated soil to favor crop growth and development.
Irrigated fields are estimated to cover about 20\% of global cropland (3.5 million km$^2$), which account for 40\% of the food produced in the world.

This practice is widespread globally in various forms (gravity irrigation, sprinklers, drip irrigation) and enables agriculture to maintain yields that would not be achievable in many regions otherwise. The historical reconstruction by \citet{siebert_global_2015} estimates that the total irrigated area increased fivefold during the 20th century due to population growth and industrialization, rising from 63 Mha in 1900 to 306 Mha in 2005. Certain regions stand out, such as South Asia, the Western United States, Eastern China, and Western Europe (Figure \ref{irrig_evolution_map}).

\begin{figure}[ht]
    \centering
    \includegraphics[width=\textwidth]{images/irrig_evolution_Siebert.png}
    \caption{Percentage of area equipped for irrigation in 1900, 1960, and 2005 according to the HID (Historical Irrigation Dataset). Extracted from \citet{siebert_global_2015}.}
    \label{irrig_evolution_map}
\end{figure}

%various methods : flooding, sprinkler, drip
%various sources : river, dams, pumping, advection from mountainous areas
%order of magnitudes : world fraction, volumes / same for Spain+Portugal
%how it's estimated ?

\section{Climate modelling}
Climatology aims at describing statistical distribution of multiple variables of interest, among which temperature, pressure, humidity, precipitation, and wind speed.
For centuries, it remained a science based on observations, 
%todo : koppen geiger

which then led to the formulation of conceptual models and to mathematical representations of the energy balance and radiative tranfer processes.
%todo cite Edwards
The Navier-Stokes equations describe the fluid mechanics that control the motions of the atmosphere. However these equations do not have any known analytical solution, forbidding their direct use to predict motions of air, water and other components of the atmosphere. 
Modern climate modelling originated in the 1950s with the development of computer simulations which enable numerically estimating solutions of these equations.

%todo: cite Manabe, others ?
General circulation models (GCMs) use a simplified version of the Navier-Stokes equations, referred to as \textit{primitive equations}, to represent the complex motions of the atmosphere. 
The globe is discretized into grid cells, which can range from a few tens to a few hundred kilometers. Using an appropriate temporal discretization, this enables approximating solutions of the primitive equations to represent atmospheric dynamics. This part of the model is often referred to as \textit{the dynamical core}. However, several major processes of the climate system are not described by fluid mechanics and require additional \textit{parameterizations}, which compute the mean effect of various processes in each vertical column, independently of neighbouring grid cells. Most importantly, they represent all radiative emission and transfer processes which largely dictate the energy budget of the atmosphere.
Thermodynamic processes involved in the phase changes of water are also essential to represent energy transfers between phases, cloud formation and precipitation. Parameterizations are also used to account for the processes that occur at a smaller scale than the resolution of the GCM and cannot be described by the dynamical core, such as turbulent diffusion or shallow and deep convection processes. Finally, parametererization are used to represent interactions of the atmosphere with the various surfaces it can be interfaced with: sea surface, ice caps or sea ice, bare soil, vegetation, urban areas... Interactions between the land surface and the atmosphere and the feedback loops between the two systems are a major focus of this thesis, and the following section %tocheck
provides a detailed description of the state of knowledge in this field. 


\begin{figure}[ht]
    \centering
    \includegraphics[width=\textwidth]{images/GCM_structure.png}
    \caption{GCM structure %todo:ref
    }
    \label{fig:GCM}
\end{figure}


Historically, the first climate models primarily focused on the atmosphere, with
a very simplified representation of the surface. Over the past three decades, climate models have evolved to distincly represent continental surfaces by coupling atmospheric models with ocean models and land surface models (LSMs), and are now often referred to as Earth system models (ESMs). Nowadays, ESMs are not only used to simulate temperature, precipitation, and wind, but can have a much wider range of applications in geoscience, paleoclimatology, oceanography, glaciology, subsurface hydrology, biology, biogeochemistry, etc... 
Regarding land surface in particular, the complexity of LSMs has gradually increased, for example, to better account for the influence of vegetation on the atmosphere, or to represent additional processes of interest such as rivers and groundwater, and biogeochemical cycles of carbon or nitrogen. They can also be used as standalone models using atmospheric forcings instead of a dynamical coupling with a GCM.

%limits to increase in resolution : computing power, physical hypotheses
%how do we evaluate them : point based and satelite obs

%specificities of regional modelling ? Lateral forcing
%mention hierarchy of models, CRM/mesoscale/NWP, RANS/LES, (DNS)

% notion of climate variability, sensitivity to initial (+/- boundary conditions). Idea that running a climate model over a given year may not represent correctly what happenned on that year, but that over a decade (or more) it should be able to reproduce the mean state of climate.

\subsection{Climate change}
As early as 1824, Joseph Fourier, followed by Claude Pouillet, theoretized than some atmospheric components of the atmosphere can influence the temperature of the air more than others, which was demonstrated in 1838 by Eunice Newton Foote for water vapour and carbon dioxyde (CO2). This was linked to infrared absorption and emissions by the experimental work of John Tyndall, and, in 1896, Svante Arrhenius conducted the first estimate of the global temperature increase caused by a hypothetical doubling of CO2 in the atmosphere.

Since the industrial revolution in the 19th century, anthropic activities have increased %todo : estimate
the amounts of greenhouse gases in the atmosphere, mainly as a consequence of fossil fuel combustion and by destruction of natural sinks to give way to agricultural land.
By the end of the 1950s, it was established that concentration of CO2 in the atmosphere was increasing, %todo: cite and show Keeling curve
and that although water vapour was overwhelmingly dominant in the atmospheric composition, other greenhouse gases present in the upper atmosphere, such as CO2, could have a significant impact on global energy balance, and therefore temperature. Therefore, concerns arose about the possible rise of global temperatures and the impacts it could have on natural ecosystems and human activities.
Concurrent research in paleoclimatology showed that previous evolutions of greenhouse gases in the atmosphere had never been this fast, and were all associated with large-scale changes. 

As this knowledge progressed coincidentally with modelling capablilities in numerical climate modelling, models started to be used not only to reproduce past and present climate, but also to simulate future climate scenarios. They quickly confirmed the risk of rapid increases in global temperature, and hinted at the global and regional consequences on ice cap and glacier melting, sea level rise, and disruption of the water cycle among others. In 1988, the International Panel on Climate Change was established by the World Meteorological Organisation (WMO) and the United Nations Environment Program (UNEP) to unify efforts in climate change science, and on the socio-economic impacts of global warming. This group still carries on today in its mission to "provide governments at all levels with scientific information that they can use to develop climate policies".

At the end of the 20th and the beginning of the 21st centuries, there were also many developments in the field of remote sensing which enabled the monitoring of climate variables from satellites. 


%SSP emission pathways

\section{Land-atmosphere interactions}

Over land, interactions between the surface and the lower layers of the atmosphere have significant impacts on meteorological (air temperature and humidity, precipitation, wind) and hydrological (runoff, stream flow, soil moisture) variables. The two systems influence each other's water and energy budgets, and multiple feedback loops can be identified depending on the spatial and temporal scales considered. The complexity of this coupling and the variety of impacts it can have on ecosystems and human activities make it a subject of interest for climate science and a necessary component of climate or numerical weather prediction (NWP) models. This section aims to provide a state-of-the-art description of these interactions and their modelling, as well as the impacts irrigation can have on them.

\subsection{Water and energy budgets at the surface}
To understand the components of the land-atmosphere coupling, it is necessary to recall the main fluxes of water and energy at the surface. 
Figure \ref{fig:budgets} depicts the various components of these two budgets.

\begin{figure}[ht]
    \centering
    \includegraphics[width=\textwidth]{images/budgets.png}
    \caption{Surface water and energy budgets. \\Based on similar figures from \citet{seneviratne_investigating_2010}%add these tanguy
    }
    \label{fig:budgets}
\end{figure}

The surface water budget accounts for:
\begin{itemize}
    \item Precipitation (transfer from the atmosphere to the surface), $P$.
    \item Evapotranspiration (transfer of liquid or solid water from the surface to the atmosphere in gaseous form), $ET$. It is the sum of direct evaporation and of transpiration, a process in  which vegetation releases water is has collected in the soil.
    \item Drainage of water in the soil to lower layers, denoted  as $D$. 
    \item Surface runoff, $R_{surf}$, water that does not infiltrate in the soil and flows out of the area considered.
\end{itemize}

The equation governing the evolution of water quantity in the upper soil layer (denoted here as $W$) is:
\begin{equation}
    \frac{dW}{dt} = P - ET - R_{surf} - D
\end{equation}

The surface energy budget includes:
\begin{itemize}
    \item Shortwave radiation (SW), with an incoming term ($SW_{dn}$) corresponding to incident solar radiation and an outgoing term ($SW_{up}$) corresponding to the portion reflected by the surface .

    The difference between these two terms and the albedo of the considered surface are defined as:

    $SW_{net} = SW_{dn} - SW_{up}$

    $\alpha_{SW} = SW_{up}/SW_{dn}$.
    \item Longwave radiation (LW), with an incoming term ($LW_{dn}$) corresponding to the infrared radiation reflected or emitted by clouds and atmospheric gases reaching the surface and an outgoing term ($LW_{up}$) corresponding mostly to the infrared radiation emitted by the surface based on its temperature, as well as a partial reflection of the incoming LW radiation.

    The difference between these two terms is defined as $LW_{net} = LW_{dn} - LW_{up}$.
    
    The net radiation is also defined as the sum of the two radiation terms: $R_{n} = SW_{net} + LW_{net}$.
    \item Sensible heat flux $H$, which is a thermal transfer between the air and the surface.
    \item Latent heat flux, which corresponds to the energy used to evaporate water at the surface. This energy flux is directly related to evapotranspiration (water flux) through the enthalpy of vaporization ($\lambda$), and is therefore denoted as $LE = \lambda ET$.
    \item Heat flux to the soil, which is a thermal conduction transfer between the considered surface layer and the lower soil layers, denoted as $G$.
\end{itemize}

The equation governing the evolution of energy in the surface soil layer (denoted as $E$ in Figure \ref{fig:budgets}) is:
\begin{equation}
    \frac{dE}{dt} = R_{n} - G - \lambda ET - H
\end{equation}

\subsection{Role of soil moisture in land-atmosphere interactions}

The term \textit{coupling} between the surface and the atmosphere encompasses multiple influences and feedbacks between the two systems. Soil moisture plays a central role in this coupling through its direct and indirect interactions with evapotranspiration, precipitation, and the surface energy budget.

\subsubsection*{Atmospheric boundary layer development and air temperature}

In meteorology, the atmospheric boundary layer (ABL) is defined as the lower part of the troposphere directly influenced by the presence of the surface. This layer is where shallow convection and turbulent diffusion phenomena occur, contributing to energy diffusion and mixing of the air.
The lowest part of the boundary layer, on the order of a few tens of meters, is called the surface layer. The influence of the Coriolis force is negligible compared to that of the surface, and the wind speed generally follows a logarithmic profile. The empirical similarity theory developed by Monin and Obukhov describes the mean flow, temperature, and humidity in this layer \citep{monin1954osnovnye}.
The height of the boundary layer varies during the diurnal cycle depending on air stability, which is related to the presence of vertical temperature and humidity gradients, and wind. It can measure a few tens of meters at night and up to a few kilometers during the day in arid regions \citep{garratt_review_1994}.

The partitionning of energy between the two turbulent fluxes at the surface ($\lambda ET$ and $H$) plays an essential role in the development of the boundary layer. As a tought experiment, for a given net radiation $R_n$ and a nearly constant soil heat flux (e.g., over 24 hours if there is equilibrium between daytime and nighttime), the remaining energy is distributed between the two turbulent fluxes. The evaporative fraction (defined as $EF = \lambda ET / R_n$) %attention défini autrement dans l'article EF=LE/(LE+H)
and the Bowen ratio (defined as $B = H / \lambda ET$) quantify this partitionning. If the latent heat flux is very high compared to the sensible heat flux ($EF$ high, $B$ low), the air temperature in the surface layer remains low because the energy is primarily used for evapotranspiration, and the soil transfers little heat to the air. Conversely, if the Bowen ratio is high, a larger portion of the incident energy is transmitted directly to the air, leading to a higher air temperature near the surface and more pronounced development of the boundary layer \citep{betts_fife_1995}.
In reality, the processes are a bit more complex since the surface temperature (and therefore the $LW_{up}$ flux) may also react to changes in latent heat flux, but the reciprocal behaviour of the latent and sensible heat fluxes has been identified in multiple observations and modelling experiments \citep{betts_fife_1995, seneviratne_investigating_2010}. %todo:source

Moreover, soil moisture also affects the thermal properties of the soil, as wet soil has greater thermal inertia than dry soil. In contexts where the latent heat flux is limited, this can significantly impact air temperature by affecting nighttime cooling \citep{ait-mesbah_role_2015}. The absence of solar radiation at night usually leads to radiative cooling of the soil and air in the surface layer. However, for wetter soil, this cooling will be less pronounced due to higher thermal inertia. During daytime, the impact of this inertia is often negligible, and other processes dominate the evolution of air temperature. Still, on a daily average, an increase in soil moisture can lead to an increase in air temperature since nighttime cooling is less significant \citep{cheruy_role_2017}.

\subsubsection*{Coupling between soil moisture and evapotranspiration}

Soil moisture is a key factor in the description of evapotranspiration regimes initially established by \citet{Budyko_1956, Budyko_1974}. Three main regimes are identified for the evolution of the evaporative fraction $EF$. Figure \ref{fig:evap_regimes} represents these regimes.

\begin{itemize}
    \item If soil moisture is below a threshold $\theta_{WILT}$, called the wilting point, plants cannot extract water to transpire, bare soil no longer evaporates, and evapotranspiration (and thus the evaporative fraction) is zero. This first regime is called the \textbf{dry regime}.
    \item If soil moisture is above a critical threshold $\theta_{CRIT}$, soil moisture has no impact on $EF$, which is maximal. Evaporation is limited by the available incident energy, it is the \textbf{wet regime}.
    \item Between $\theta_{WILT}$ and $\theta_{CRIT}$, there is a \textbf{transition regime} where evapotranspiration is primarily conditioned by soil moisture. As in the dry regime, evapotranspiration is limited by soil moisture.
\end{itemize}

\begin{figure}[ht]
    \centering
    \includegraphics[width=0.7\textwidth]{images/evap_regimes.png}
    \caption{Representation of different evapotranspiration regimes. Extracted from \citet{seneviratne_investigating_2010}.}
    \label{fig:evap_regimes}
\end{figure}

Through its influence on evapotranspiration, soil moisture directly impacts surface water and energy budgets, particularly the distribution of energy between turbulent fluxes. As explained earlier, this has consequences for air temperature and humidity in the surface layer.

However, there are also several the feedbacks of evapotranspiration on soil moisture. First, an increase in evapotranspiration directly leads to a decrease in soil moisture and an increase in air humidity. This contributes to reducing the vertical humidity gradient between the air and the surface, which tends to limit evaporation \citep{allen_crop_2000}, forming a negative feedback loop. It is also established that an increase in air temperature leads to higher evaporative demand \citep{jarvis_stomatal_1986}. If enough water is available in the soil, a temperature increase will increase evapotranspiration, thus decreasing soil moisture. This can form a positive feedback loop where dry soil leads to high air temperatures, resulting in even drier soil and possibly initiating extreme drought events \citep{quesada_asymmetric_2012}. If no more water is available or if there are no more gradients between the soil surface and the air, the feedback of air temperature on soil moisture becomes neutral, and the feedback loop is interrupted.

\subsubsection*{Coupling between soil moisture and precipitation}

The most complex processes of surface-atmosphere coupling concern the link between soil moisture and precipitation. Several opposing effects are at play, with a large importance of spatial heterogeneities of soil moisture at the surface, which may derive from the diversity of vegetation, soil types, orographic features, and anthropogenic processes.

Increases in SM and ET have been associated with direct increases in precipitation in both modelling and observational studies \citep{koster_observational_2003, guo_glace_2006, wei_dissecting_2012, findell_probability_2011}, constituting a positive feedback loop \citep[moisture recycling, as presented in ][]{eltahir_precipitation_1996}.

However, high soil moisture can lead to a stabilisation of the boundary layer. %todo:citation ? details ? mention that it's driven by H ?
This stabilisation can inhibit vertical development and convective processes involved in cloud formationa and precipitation \citep{findell_atmospheric_2003-1, ek_influence_2004}. 
This constitutes a negative feedback loop where convective rainfall is more likely to occur over drier soil patches, which was noticed in observations \citep{taylor_afternoon_2012, klein_dry_2020}. These findings suggest that indirect processes related to boundary layer structure and heterogeneities can locally exceed the direct process of atmospheric moisture recycling, although demonstrating causality in such observational studies remains challenging \citep{salvucci_investigating_2002, guillod_land-surface_2014}. Building on the purely spatial analysis of \citet{taylor_afternoon_2012}, a spatiotemporal analysis of correlations between soil moisture and precipitation highlighted the importance of temporal variability \citep{guillod_reconciling_2015}. It showed that while precipitation is more frequently triggered over drier areas, it occurs on days that are wetter relative to the season and region concerned.
Finally, spatial heterogeneities in soil moisture have also been identified as factors influencing precipitation through mesoscale circulations that can either favour or inhibit convection triggering \citep{findell_atmospheric_2003, taylor_frequency_2011, rochetin_morphology_2017}.

\subsection{Land-atmosphere interactions in climate models}

In modern ESMs of NWP models %todo:check que NWP défini avant
he modelling of land-atmosphere interactions involves both the atmospheric model and the land surface model, and the modelling choices for their coupling.
From the perspective of an atmospheric model, latent and sensible heat fluxes at the surface constitute necessary boundary conditions for solving turbulent diffusion equations throughout the considered atmospheric column. These conditions impact essential meteorological variables such as air temperature, wind, and humidity. From the perspective of the LSM, precipitation computed in the atmospheric model constitutes a water input for the soil column, while surface layer characteristics (humidity, temperature, wind) condition evapotranspiration demand.

Various experiments have been designed to quantify the importance of surface coupling processes for atmospheric models. In particular, the GLACE experiments \citep{koster_glace_2006} compared atmospheric simulations with different prescribed soil moisture conditions to isolate the influence of soil moisture on precipitation. This led to the identification of hotspots: regions where this coupling is particularly pronounced. These are mainly semi-arid regions (Sahel, Great Plains in the United States) where the transitional evaporation regime described in Figure \ref{fig:evap_regimes} is more frequent than dry and wet regimes \citep{koster_regions_2004}.
This was confirmed by other modelling studies that also identified various mechanisms through which land surface conditions can impact the atmosphere in these coupling hotspots, and metrics to quantify them \citep{dirmeyer_terrestrial_2011, zou_precipitation_2023}.
The GLACE-CMIP5 experiments \citep{seneviratne_impact_2013} extended these conclusions by highlighting the importance of this coupling in the global warming response observed in these hotspots \citep{berg_interannual_2015}.

A particular challenge in modelling this coupling within an ESM arises from the size of the grid cells used (generally around 100 km for long climate simulations). At this scale, numerous subgrid heterogeneities exist at the surface, requiring the aggregation and averaging of highly diverse land-atmosphere interactions depending on vegetation cover, elevation, or anthropogenic factors (cities, irrigation).

For example, regarding the triggering of convective rainfall in heterogeneous areas, \citet{moon_soil_2019} showed that CMIP5 models correctly reproduced the positive temporal feedbacks (rain on wetter days) described in \citet{guillod_reconciling_2015} but not the negative spatial feedbacks identified in \citet{taylor_afternoon_2012}. The influence of resolution and parametrization choices (especially for deep convection) on the importance of land-atmosphere coupling in models has also been highlighted by \citet{tuinenburg_high-resolution_2020} and \citet{lee_weaker_2024}.

Several ongoing projects aim to better document the impact of these heterogeneities and explore ways to better account for them in models. This is the case of the Land Surface Interactions with the Atmosphere over the Iberian Semi-Arid Environment (LIAISE, \cite{boone_land_2019}) and Models and Observations for Surface-Atmosphere Interactions (MOSAI, \cite{lohou_model_2022}) projects, which rely on measurement campaigns specifically dedicated to studying land-atmosphere interactions and comparing multiple models with these observations.

It is now well known that accurately representing surface-atmosphere couplings is essential for accurately representing climate, particularly temperature and precipitation extremes \citep{jaeger_impact_2011, van_den_hurk_acceleration_2011}. Furthermore, the study of CMIP5 simulations has revealed a warm bias in most models, particularly in the Mediterranean basin \citep{christensen_temperature_2012, mueller_systematic_2014}. By comparing CMIP5 model biases with satellite observations, \citet{al-yaari_satellite-based_2019} also showed that these temperature biases are correlated with soil moisture biases. More generally, surface interaction processes have been identified as a crucial factor in the appearance of these biases, especially the partitioning of energy between latent and sensible heat fluxes and the underestimation of evapotranspiration \citep{cheruy_combined_2013, cheruy_role_2014}. 

\subsection{Impacts of irrigation on land-atmosphere interactions}

Human activities can influence the various components of land-atmosphere coupling. This is particularly the case with irrigation, which directly modifies soil moisture through artificial water inputs and can create strong spatial heterogeneities in SM. 

Observational studies have established that irrigation leads to a moister and cooler atmosphere near the surface \citep{bonfils_empirical_2007, mcdermid_irrigation_2023}, which is explained by an increase in the latent heat flux and a decrease in the sensible heat flux in irrigated areas \citep{rappin_landatmosphere_2022, boone_land_2025}.
Based on observation records in California, \citet{bonfils_empirical_2007} estimated a decrease of 1.8 to 3°C in the average summer temperature since the development of irrigation practices. This cooling effect can limit temperature values during extreme meteorological events \citep{thiery_present-day_2017, thiery_warming_2020}. However, the cooling due to increased latent heat flux can be partially compensated on a daily scale by a weakening of nighttime cooling due to thermal inertia \citep{chen_irrigation_2018}.
In the American Midwest, \cite{nocco_observation_2019} showed that such effects could impact the regional climate and even mask the rise in temperature induced by global warming. Different regional responses of precipitation have also been observed, such as a decrease in irrigated areas \citep{alter_rainfall_2015} or an increase in downwind regions \citep{deangelis_evidence_2010}.
The other impacts on the atmosphere (structure of the boundary layer, mesoscale circulations, precipitation) are more complex to characterize because they do not always occur solely over the irrigated area.

To get more insights than what point-based obersvation provide and analyse the atmospheric processes involved, mesoscale modelling studies have often been used to complement observational campaigns. 
The Great Plains irrigation experiment \cite[GRAINEX,][]{rappin_great_2021} measurements and simulations with the Weather Research and Forecasting (WRF) model revealed a lower ABL over irrigated areas and a reduction of existing mesoscale slope-induced circulations in the presence of irrigation \citep{rappin_landatmosphere_2022, phillips_influence_2022}. 
In the Ebro Valley (northern Spain), Meso-NH simulations at 2-km and 400-m resolution were conducted over the area of the Land surface Interactions with the Atmosphere over the Iberian Semiarid Environment campaign \citep[LIAISE][]{boone_land_2019}. They reproduced the observed differences between irrigated areas and neighboring areas in turbulent fluxes and air temperature and were greatly improved by representing irrigation \citep{lunel_irrigation_2024}. They also highlighted a weakening of the regional sea-breeze regime due to irrigation, which reduces the pressure gradient force \citep{lunel_marinada_2024}.  

Multiple regional modelling studies were also conducted without being associated to specific observation campaigns, particularly with the Weather Research and Forcasting (WRF) atmospheric model and the Noah LSM, showing that the representation of irrigation improves model performance \citep{qian_modeling_2013, yang_impact_2017, liu_simulating_2021}. They identified a lowering of the boundary layer and lifting condensation level over the American Great Plains \citep{qian_modeling_2013} but very heterogeneous effects on precipitation in China \citep{liu_simulating_2021}. \citet{yang_impact_2017} showed an increase in precipitation linked to irrigation in the Colorado River basin due to precipitation recycling over the Sierra Nevada mountains. This positive feedback loop and strengthening of the water cycle at the basin scale had also been described in simulations using the global Community Atmosphere Model (CAM) coupled with the Community Land Model (CLM) to study the local and remote impacts of irrigation in California's Central Valley \citep{lo_irrigation_2013}. Conversely, a negative feedback loop on precipitation in Saudi Arabia was described in \citep{lo_intense_2021}, with moisture convergence leading to increased precipitation in a remote area west of the irrigated region.

These results, however, were obtained with very diverse representations of irrigation since not all models target the same objectives. For example, the experiments with WRF-Noah or MesoNH-ISBA \citep{lunel_irrigation_2024} used idealized representations that maintain SM at a certain level in irrigated areas to limit stress for the vegetation. This type of modelling is suitable for short simulations over a limited domain but for Earth System Models (ESMs), water-conservative approaches are necessary to run long-term global simulations while keeping a consistent water cycle representation.

The impacts of irrigation on regional climate have been studied for several decades, but interest in its impacts on global climate is more recent \citep{boucher_direct_2004}. By comparing coupled simulations with and without irrigation, \citet{sacks_effects_2009} found no influence on the global mean temperature but small regional cooling (less than 1 K) at mid-latitudes. In more recent studies, LSMs that represent irrigation were shown to represent the shift in energy partitioning between the turbulent fluxes, achieving large increases in latent heat flux \citep{pokhrel_incorporating_2012, arboleda-obando_validation_2024, al-yaari_role_2022}. Irrigation modelling was found to induce a cooling on yearly average values, with seasonal and regional impacts that remain highly varied \citep{puma_effects_2010, cook_irrigation_2015}. In general, effects are mostly visible in irrigation hotspots such as India, Eastern China and the United States of America. 
%tod:arboleda 2025+ ?
%distinction modèle couplé/LSM offline pas très nette

Using water-conservative representations of irrigation also allows the study of the continental water cycle as a whole, accounting for impacts of irrigation withdrawals. More complex models can now differentiate between irrigation methods, withdrawals from surface or groundwater ressources and represent storage and adduction, offering an integrated vision of the anthropized water cycle. %todo:more refs..............
For instance, \citep{leng_significant_2017} showed that the responses in groundwater levels and river flows to irrigation were very contrasted and highly dependent on the irrigation methods and sources used.  \citet{yao_implementation_2022} confirmed these findings, and also showed that representing distinct irrigation methods in CLM improved the agreement with surface fluxes measurements.
%todo : pedro ? costantini +/- decharme

Finally, global simulations allow the study of very remote connection and long term patterns. \citep{de_vrese_asian_2016} highlighted a link between irrigation in Asia and increased precipitation in East Africa, while \citep{wei_where_2013} identified moisture-importing and -exporting regions at the global scale. Running 30-years simulations with and without irrigation, \citep{guimberteau_global_2012} studied the temporal disruptions of seasonal phenomena, and identified a delay in the onset of the Indian summer monsoon due to regional irrigation effects.

However, it is important to note that most ESMs participating in CMIP6 did not include irrigation modelling. In heavily irrigated regions, models representing irrigation show different historical trends in several climate variables (soil moisture, latent heat flux) that better match satellite observations \citep{al-yaari_role_2022}. 
The IRRMIP intercomparison project is designed to study in detail the various impacts of irrigation representation in ESMs to better understand the uncertainty of responses to this modelling \citep{yao_irrigation-expansion-induced_2023}.
%todo: which models, what are the first results 

%Guoshuai Liu 2023 : secondary impacts since vegetation is more developped (positive feedback loop enhancing impacts of irrig on LE)

\section{Scientific questions and thesis outline}

\textbf{Which regional impacts of irrigation on land-atmosphere interaction processes and on the water cycle can be represented by an ESM?}

\textbf{How are irrigation-induced surface heterogeneities accounted for in ESMs ?}

% 
%ESMs are not expected to achieve the same level of precision as convection resolving models in their accounting for irrigation, but the extent to which their representation of irrigation can impact the simulated climate in irrigated regions is still not well constrained. In this context, this study aims to understand which regional impacts of irrigation on surface-atmosphere couplings and the water cycle can be represented by a climate model. 
%It leverages a new limited area model configuration developed for the Institut Pierre-Simon Laplace climate model (IPSL-CM) to perform regional simulations with the land surface and atmospheric components of the global model, ORCHIDEE \citep{krinner_dynamic_2005, cheruy_improved_2020} and ICOLMDZ \citep{dubos_dynamico-10_2015, hourdin_lmdz6a_2020}.This configuration allows for insight on the parameterizations of the global model while running with a higher resolution (25 km) and lower computational costs than in global applications.


%Nous avons vu dans la section précédente que le couplage surface - atmosphère est un élément central du système climatique et qu'il est une composante clé des modèles de climat. La poursuite de son étude est justifiée par les nombreuses incertitudes auxquelles il est associé et l'impact que peut avoir sa modélisation dans les ESM. En particulier, dans un contexte de forte irrigation, les processus associés à ce couplage et les hétérogénéités spatiales peuvent être exacerbés, ce qui en fait un cas d'étude pertinent.
% Deux principaux types d'études de modélisation sur ces processus en lien avec l'irrigation se distinguent : modélisation globale et modélisation régionale à haute résolution. Les premières ne permettent en général pas de rentrer dans le détail des processus impactés par l'irrigation ou des différentes réponses constatées en fonction des régions. Les secondes le peuvent, souvent avec beaucoup de précision, mais elles s'appuient souvent sur des représentations de l'irrigation idéalisées qui ne permettent pas une étude complète du cycle de l'eau, et des paramétrisations adaptées à la haute résolution uniquement. Les processus représentés dans ces simulations ne pourraient donc pas l'être de la même manière dans un modèle global.

% Dans ce contexte, mon travail de thèse vise donc àbeaucoup apporter des réponses à la question scientifique suivante : \textbf{Quels impacts régionaux de l'irrigation sur les processus du couplage surface - atmosphère et le cycle de l'eau un modèle de climat global est-il capable de représenter ?}

%Ce travail est conduit en utilisant deux composantes du modèle de climat de l'IPSL : le modèle de circulation générale atmosphérique LMDZ et le modèle de surface continentale ORCHIDEE. Il repose sur des simulations couplées avec une nouvelle configuration régionale qui utilise la dynamique et les paramétrisations du modèle global, y compris pour la modélisation de l'irrigation.Des simulations avec et sans représentation de l'irrigation sont comparées à des observations (de terrain et sous forme de produits grillés), puis entre elles, dans le but d'isoler specifiquement les impacts de l'irrigation.

%lol



\chapter{Methods and tools: regional simulations with the ICOLMDZOR LAM}
\label{chap:methods}
\minitoc
\pagebreak
This work uses the atmosphere and land surface components of the IPSL-CM, which has been a regular participant in CMIP exercises, including CMIP6 \citep{boucher_presentation_2020}. 

\section{ICOLMDZ General Circulation Model}
\subsection{General structure and parameterizations}
The atmospheric component of the model is the association of the dynamical core DYNAMICO \citep{dubos_dynamico-10_2015}, which uses an icosahedral grid, and the LMDZ6A physics used for CMIP6: version NPv6.2 with 79 vertical levels \citep{hourdin_lmdz6a_2020}. The physics of the model are run every 15 mn and include the following parameterizations:
\begin{itemize}
    \item a surface layer description based on \citet{louis_parametric_1979} and \citet{king_sensitivity_2001}; 
    \item an Eddy-Diffusivity Mass Flux (EDMF) scheme of boundary layer vertical transfer composed of a turbulent diffusion scheme based on \citet{yamada_simulations_1983} with recent improvements described in \citet{vignon_modeling_2018}, and a thermal plume model for shallow convection \citep{hourdin_unified_2019}; 
    \item a stochastic triggering scheme for deep convection \citep{rochetin_deep_2014, rochetin_deep_2014-1}; 
    \item a large scale condensation scheme based on a statistical distribution of subgrid total water content, from which cloud fraction and water contents are derived \citep{madeleine_improved_2020}; 
    \item radiative transfer model RRTM \citep{mlawer_radiative_1997}.
\end{itemize}

%todo:detail forcings
All simulations were run with prescribed sea surface temperature and sea ice content from the AMIP dataset.


\subsection{Limited Area Model configuration}
Simulations are run in a Limited Area Model (LAM) configuration, first used and well described in \citet{raillard_leveraging_2024}.
Lateral boundary conditions for the LAM are read at each time step of the dynamics, and are taken from ERA5 reanalysis hourly values at 0.25° resolution \citep{hersbach_era5_2020}.
The domain comprises 3 zones: a raw forcing zone which contains values directly given by the forcing, a transition zone  where the model is nudged toward the forcing with decreasing strength, and a free zone at the center of the domain where there is no direct influence of the lateral forcing.
In this study, the LAM domain is a hexagon centered on (40.4°N, -3.7°E) and of radius 1500 km (Fig. \ref{fig:domain_full_hex}). The radius is composed of 60 grid cells, so the diameter of a cell is 25 km. 
Outputs of the model are natively on a hexagonal grid but are interpolated to a more traditional longitude-latitude grid of similar resolution, in order to simplify post-treatment and comparisons to evaluation products.

\section{ORCHIDEE Land Surface Model}
\subsection{General structure, running modes, and input data}
ORCHIDEE (Organizing Carbon and Hydrology In Dynamic EcosystEms) is a land surface model. Various branches exist within the model, as the base structure presented in \citet{krinner_dynamic_2005} has been adapted and extended to meet different research objectives.
This work uses branch 2.2, which serves as the land component of the IPSL climate model and has notably been used for CMIP6 simulations \citep{boucher_presentation_2020}. Coupling with an atmosphere and ocean model imposes significant computational constraints, leading to the simplification or omission of certain processes for global climate simulations.
The description of this model version provided here is partly based on Section 2 of Aurélien Campoy’s PhD thesis \citep{campoy_influence_2013}.

ORCHIDEE v2.2 includes several modules:
\begin{itemize}
    \item SECHIBA (\textit{Schématisation des Échanges Hydriques à l’Interface entre la Biosphère et l’Atmosphère}, \cite{ducoudre_sechiba_1993}). Initially integrated into the surface scheme of the LMDZ GCM, this module represents surface energy and water balances, including interactions with the atmosphere. It models water infiltration into the soil and horizontal transfers, allowing for river representation.
    \item STOMATE \citep{krinner_dynamic_2005}. This module simulates biochemical surface processes such as photosynthesis and phenology evolution, enabling the representation of seasonal variations in transpiration.
    \item LPJ \citep{sitch_evaluation_2003}. This module dynamically evolves vegetation to represent land-use changes. In this work, to reduce computational costs, this module is inactive, and vegetation evolution is instead prescribed using annual land-cover maps from \citet{belward1999igbp}.
\end{itemize}

ORCHIDEE can interface with the atmosphere in either forced mode (also called offline) or coupled mode. 
In the first case, meteorological forcing (usually obtained from a reanalysis) provides values for downward radiation (shortwave and longwave), precipitation (rain and snow), air temperature and specific humidity at 2m, wind speed at 10m (eastward and northward components), and surface pressure. 
In the second case, ORCHIDEE is coupled with an atmospheric model that calculates these variables in real time while receiving certain variables computed by ORCHIDEE (surface roughness, albedo and turbulent fluxes). 
Within the IPSL climate model, ORCHIDEE can be coupled with the atmospheric physics model LMDZ in a standard configuration called LMDZOR, using a semi-implicit coupling scheme \citep{polcher_proposal_1998}. %option : detail that this schemes enables computing thermal and turbulent diffusion in soil and atmospheric column (respectively).
ORCHIDEE discretizes the simulation domain into grid cells, with a resolution that adapts to the atmospheric forcing or the grid of the coupled atmospheric model.

In addition to meteorological variables, the model requires other input data. Each grid cell must be assigned a soil texture from three main classes (sandy loam, silt loam, clay loam), which determines various hydrological and thermal soil parameters. 
For CMIP6 simulations, the map from \citet{zobler87802world} was used to assign the dominant  texture in each grid cell, with a resolution of 1°. For the coupled simulations described in this work, the dominant USDA texture was obtain from the map \citet{reynolds_estimating_2000}, which has a resolution of 5 arcmin.

To characterize vegetation, ORCHIDEE defines 15 Plant Functional Types (PFTs), each associated with a set of characteristic parameters such as average height, leaf area index, and albedo. For CMIP6 simulations, the fraction of each PFT in a grid cell is derived from a map based on the LUHv2 database \citep{lurton_implementation_2020}. Each grid cell is divided into three columns, grouping multiple PFTs: one for bare soil, one for tall vegetation (trees), and one for low vegetation (crops and grasses). A separate water budget is computed for each soil tile but only one energy budget per grid cell.

\subsection{Water and energy budgets}
\subsubsection{from article LAM}
The ICOLMDZ GCM is coupled to the Land Surface Model (LSM) ORCHIDEE v2.2 \citep{cheruy_improved_2020}. The spatial grid of the LSM is the same as for the GCM and the timestep is dictated by the atmospheric physics (15 mn).
% This LSM represents a 2-m soil column discretized over 11 vertical levels of increasing depth. 
It computes the water and energy budgets at the surface and simulates water infiltration and thermal diffusion in the soil column. 
% Each grid cell is assigned the dominant USDA soil texture according to the 5 arcmin resolution map from \citet{reynolds_estimating_2000}.
Vegetation is described using 15 Plant Functional Types (PFTs) on a 0.1° resolution input map based on the Land Use Harmonization 2 (LUHv2) dataset \citep{hurtt_harmonization_2020, lurton_implementation_2020}. PFTs are distributed in each grid cell over 3 soil tiles for bare soil, forests, and low vegetation (which notably includes C3 and C4 crops).
A separate water budget is computed for each soil tile but only one energy budget per grid cell, using a composite approach with aggregated parameters (roughness length, albedo) to compute surface temperature and turbulent fluxes. 
%option:other parameters aggregated ? Beta for Evap (resistance to evaporation ? other name ?)
%option:implicit coupling for turbulent and thermal diffusion into the ground -> jugé plutôt pas nécessaire

\subsubsection{from CSI + translation}
This section describes how the different terms of water and energy budgets identified in the introduction are modeled in ORCHIDEE.  

To determine the soil water content in a grid cell, ORCHIDEE first distinguishes between the fraction of precipitation intercepted by vegetation and the fraction that actually reaches the ground. In the case of liquid precipitation (snow is treated separately), it further differentiates between the portion that infiltrates into the soil and surface runoff.  

Soil hydrology is based on the one-dimensional Richards equation, with a 2-m soil column discretized over 11 vertical levels of increasing depth \citep{de_rosnay_impact_2002, dorgeval_sensitivity_2008}. A free drainage condition is applied at the bottom of the column. ORCHIDEE models the infiltration front velocity, which progressively saturates the layers. The values of hydraulic conductivity and diffusivity are calculated based on soil texture according to \citet{mualem_new_1976, van_genuchten_closed-form_1980}.

The missing term to complete the surface water balance is evapotranspiration $E$, which also appears in the energy balance via the latent heat flux $\lambda E$. The total evaporation value depends on potential evaporation, which represents the evaporation that would occur over a water surface and thus serves as an upper limit. It is calculated using the formulation of \citet{Budyko_1956}:  

\begin{equation}
    E_{pot} = \frac{\rho_{air} (q_{sat}(T_S) - q_{air})}{r_a}
\end{equation}

where $\rho_{air}$ is the air density, $q_{air}$ is the specific humidity of the air, $q_{sat}(T_S)$ is the specific humidity of saturated air at the surface temperature $T_S$, and $r_a$ is the aerodynamic resistance to evaporation.  
Evapotranspiration $E$ is obtained from the ratio $\beta = \frac{E}{E_{pot}}$, which depends on four terms associated with different components of evapotranspiration:  

\begin{itemize}
    \item Snow sublimation, which only occurs in certain regions and seasons.  
    \item Evaporation of water intercepted by vegetation, which depends on vegetation structure.
    \item Vegetation transpiration, which involves the stomatal resistance of vegetation.  
    \item Bare soil evaporation, which is the minimum between $E_{pot}^*$, the potential evaporation reduced according to \citet{milly_potential_1992}, and $Q_{up}$, the maximum volume that can be extracted from the soil column.  
\end{itemize}

The factor $\beta$ is calculated as a conductance that depends on the evaporation resistances of each of these terms and the fractions of the grid cell occupied by vegetation and bare soil.  

% \hfill

% Bare soil evaporation is expressed as a relationship between supply and demand:
% \begin{equation}
%     E_g = min(E^*_{pot}, Q_{up})
% \end{equation}
% where $Q_{up}$ is the maximum volume that can be extracted from the soil column, and $E^*_{pot}$ is the reduced potential evaporation from \citet{Milly_1992}. Note that in most cases, $E^*_{pot} < E_{pot}$.
%NB Provide more details on the calculation of Qup? The available water volume is estimated by integrating Richards' equation over the soil column to maintain a water volume in each layer above......

% \hfill

% An alternative version exists in ORCHIDEE, which allows limiting the demand by modeling a bare soil resistance to evaporation $r_{soil}$ based on \citet{Sellers_1992}:
% \begin{equation}
%     E_g = min(\frac{E^*_{pot}}{1+\frac{r_{soil}}{r_a}}, Q_{up})
% \end{equation}
% \begin{equation}
%     r_{soil} = exp(8.206 - 4.255 \frac{W_L}{W_L^s})
% \end{equation}
% where $W_L$ is the moisture content in the top four layers of the soil column, and $W_L^s$ is the saturation moisture content in these same layers.

In the energy balance, ORCHIDEE does not modify the incoming radiation terms but determines the outgoing terms by calculating the albedo (which is split into two values, one for infrared and one for visible light) and the surface temperature, which controls the outgoing longwave radiation. 

For turbulent fluxes, ORCHIDEE calculates roughness lengths, which affect surface drag coefficients ($C_{drag}$). Two roughness lengths are distinguished:  
\begin{itemize}
    \item $z_{0m}$, which represents the height above the ground where wind speed is zero.
    \item $z_{0h}$, below which temperature is assumed to be equal to the soil temperature. 
\end{itemize} 

The default parameterization in ORCHIDEE is based on \citet{su_evaluation_2001} and dynamically evaluates both roughness lengths using empirical formulations. However, some of the obtained values are not always consistent (particularly for $z_{0h}$), and an older parameterization is sometimes used. This simpler approach relies on two prescribed parameters for each PFT:  
\begin{itemize}
    \item The ratio between $z_{0m}$ and the canopy height, with a typical value of $\frac{1}{15}$.
    \item The ratio between $z_{0m}$ and $z_{0h}$, with a typical value of $\frac{1}{10}$.  
\end{itemize}

\subsection{Routing scheme}
\subsubsection{from article}
River discharge is simulated using a routing scheme based on the one described in \citet{ngo-duc_validation_2007}.
This scheme solves water transfers directly on the MERIT Digital Elevation Model (DEM) \citep{yamazaki_merit_2019} which provides topographic data at a 2-km resolution. Each DEM grid cell contains three linear reservoirs that represent surface runoff, groundwater and rivers. Surface runoff and drainage computed for each ORCHIDEE soil column are interpolated to the routing grid to feed the surface runoff and groundwater reservoirs (respectively). All three reservoirs then flow into the river reservoir of the downstream grid cell, with a residence time that depends on the slope and on a time constant specific to each reservoir. 
In this scheme, the river reservoir is the only representation of horizontal water transfers in ORCHIDEE.

\subsubsection{from CSI + translation}
The routing scheme is a sub-module of SECHIBA, which circulates water between various cascading reservoirs to represent horizontal transfers and transport it to the oceans \citep{ducharne_development_2003, ngo-duc_validation_2007}. 
Routing is necessary to maintain water conservation on a global scale in coupled simulations with the ocean, and it also allows for simulating river discharge and representing groundwater volumes. 
This scheme generally runs at a larger time step than the rest of the model (once per day by default); however, the irrigation modeling (detailed in the next section) requires routing to run at each time step of the ORCHIDEE model.

The scheme subdivides the simulation domain into Hydrological Transfer Units (HTU) and uses a Digital Elevation Model (DEM, constructed from topographic data) to define flow directions and characterize the HTUs (dimensions, slope, upstream/downstream relationships).
Within each HTU, three linear reservoirs are distinguished: the fast reservoir (surface runoff), the slow reservoir (groundwater), and the stream reservoir (rivers). These reservoirs differ based on their respective time constants, which determine the residence time as a function of slope (denoted as TCST\_FAST, TCST\_SLOW, and TCST\_STREAM in Figure \ref{fig:routing_principles}).
The slow reservoir is fed by drainage, the fast reservoir by runoff, and all three reservoirs flow into the river reservoir in the downstream grid cell. The river reservoir is therefore the only one connected to neighboring cells, as no other horizontal transfer is modeled in ORCHIDEE. This assumption may have limitations at high resolutions as it ignores direct groundwater transfers between grid cells.

The outgoing volume $Q$ from a reservoir at a routing time step is expressed in terms of the reservoir volume in the grid cell, $V$, the time constant $TCST$ ($day \cdot km^{-1}$), a topographic index $topoindex$ ($km$) provided by the DEM, and the routing time step $dt_{routing}$ ($s$):

\begin{equation}
    Q = \frac{V}{topoindex \times TCST} \times \frac{dt_{routing}}{86400}
\end{equation}

\begin{figure}[ht]
    \centering
    \includegraphics[width=1\linewidth]{images/routing_structure.png}
    \caption{Operating principles of the \textit{interp\_topo} routing}
    \label{fig:routing_principles}
\end{figure}

Based on the same flow physics, different versions of routing have been developed, corresponding to different relationships between the DEM grid, the ORCHIDEE grid, and the definition of the HTUs used:

\begin{itemize}
\item \textbf{\textit{Subgrid\_halfdeg} routing}, which was the default option in branch 2.2 until recently. It is designed to work with a specific DEM at a resolution of 0.5°. Additionally, it requires that an ORCHIDEE grid cell contains an integer number of DEM grid cells, which limits its use to resolutions higher than 0.5°.

\item \textbf{\textit{Interp\_topo} routing}, recently implemented into ORCHIDEE to become the default version in IPSL-CM7. It is designed to adapt to any DEM and to be completely independent of ORCHIDEE’s resolution and even the shape of the grid cells. This is particularly necessary for compatibility with the new hexagonal-grid atmospheric dynamics model (DYNAMICO) of IPSL-CM7.

In this routing, HTUs are exactly the DEM grid cells, requiring interpolation of runoff and drainage from the ORCHIDEE grid to the DEM grid, where they serve as inputs for the fast and slow reservoirs, respectively. Horizontal transfers between river reservoirs of the HTUs are thus performed directly on the DEM grid. Once these transfers are completed, water volumes in the reservoirs are re-interpolated from the DEM back to the ORCHIDEE grid, allowing them to be used for irrigation within ORCHIDEE grid cells.

\item \textbf{\textit{Subgrid\_HTU} routing}, developed before the introduction of the \textit{interp\_topo} routing to operate at a higher resolution than \textit{subgrid\_halfdeg} routing. In this version, HTUs are constructed from a high-resolution DEM by aggregating grid cells. However, this routing still maintains the constraint of having an integer number of HTUs within each ORCHIDEE grid cell, making it incompatible with DYNAMICO. I have not used it in simulations, but this manuscript occasionally refers to it.
\end{itemize}

\subsection{Irrigation scheme}
\subsubsection{from article}
Irrigation is modeled using the scheme extensively described in \citet{arboleda-obando_validation_2024}, based on a water-conservative supply-and-demand approach.
It computes a moisture deficit by comparing SM in the upper layers of the ORCHIDEE soil column (corresponding to the root zone) to a target SM described as a fraction of the SM at field capacity.
In the default version of the global model, this target is set to 90 \% of SM at field capacity, but it reflects a wide variety of irrigation practices, including flooding in rice paddies. In the Iberian Peninsula, the irrigation methods are less water-intensive, and after a calibration of the routing and irrigation schemes (not shown), the target value was adjusted to 60 \% of SM at field capacity. 
To avoid computing irrigation requirements on grid cells without plants, the SM deficit is set to zero if the Leaf Area Index (LAI) is below a given threshold $LAI_{min}$.
At each time step, the SM deficit is combined to the irrigated fraction of each ORCHIDEE grid cell (based on the HID database of \citet{siebert_quantifying_2010} at 5 arc-min resolution) to define the irrigation requirement. 
This demand is then compared to the available water in the various reservoirs of the routing scheme. Water is withdrawn preferentially from either surface water (runoff and rivers) or groundwater depending on the nature of irrigation equipments, according to a map of areas equipped for irrigation \citep{siebert_groundwater_2010}.
The amount of water withdrawn from the reservoirs is then added at the top of the ORCHIDEE soil column at the next time step and infiltrates.
%option:show here map of irrig_frac and AEI ? -> I think they are fine with results on irrigation for better visualization of the links

\subsubsection{from CSI}
A scheme for representing irrigation in ORCHIDEE has recently been developed, presented, and validated in \citet{arboleda-obando_validation_2024}. An irrigation representation already existed in ORCHIDEE \citep{de_rosnay_integrated_2003, guimberteau_global_2012}, but this work relies solely on this new modeling approach detailed here and summarized in Figure \ref{fig:schema_pedro}.

\begin{figure}[t]
    \centering
    \includegraphics[width=1\textwidth]{images/schema_pedro.png}
    \caption{Principles of the new irrigation scheme. Extracted from \citet{arboleda-obando_validation_2024}.}
    \label{fig:schema_pedro}
\end{figure}

\hfill

First, a root zone is defined using parameter $Root_{lim} \in [0;1]$, which determines the fraction of the root system that must be included in the zone. This results in the inclusion of a certain number of vertical layers depending on the cumulative root density. It is important to note that irrigation only applies to the column of low vegetation PFTs (grasses and crops), and the root zone is defined exclusively within this column.

A target soil moisture value is also defined to sustain plant growth. It is expressed as a fraction of the soil moisture at field capacity water content (parameter $\beta \in [0;1]$).

Given this target value and the root zone, a soil moisture deficit $D$ (mm) is calculated as the sum of deficits across all layers of the root zone:
\begin{equation}
    D = \sum_{i \in Rootzone} min(0,\beta \times W_i^{fc} - W_i)
\end{equation}
where $W_i$ and $W_i^{fc}$ are respectively the soil moisture and field capacity soil moisture of layer $i$ (in mm), .

In the absence of plants (notably in winter), it is not relevant to compute a soil moisture deficit and an irrigation demand. A parameter $LAI_{lim}$ is therefore defined as a threshold LAI value below which the soil moisture deficit is zero.

For a SECHIBA module time step $dt$ (by default 15 minutes in ICOLMDZOR simulations), the water demand to be added to the column to represent irrigation is thus $D/dt$ (mm/h). However, if the hourly irrigation rate is significantly higher than the infiltration rate into the soil, the model may produce excessive runoff that is not representative of reality. To prevent this, even if the moisture deficit is very high, the hourly irrigation rate is capped by a maximum value $I_{max}$.

Finally, the water demand for the column is weighted by the fraction of the grid cell that is irrigated: $f_{irr}$. This fraction is obtained from a map of irrigated fractions, which can be updated over the course of the simulation.
This irrigated fraction must not exceed the fraction of the soil column containing grasses and crops, as irrigation is only applied to this column.

The irrigation demand $I_{req}$ is therefore:
\begin{equation}
    I_{req} = f_{irr} min(D/dt, I_{max})
\end{equation}

Once this demand is determined, the irrigation scheme searches for available water in the three reservoirs of the routing scheme mapped on the ORCHIDEE grid.
The volume of water available for irrigation is expressed as:
\begin{equation}
    A_w = f_{sw} (a_1 S_1 + a_2 S_2)+ f_{gw}a_3 S_3
\end{equation}
where $S_i$ represents the water volume (in mm) in each of the reservoirs (rivers, surface water, groundwater), and for each reservoir, a parameter $a_i \in [0;1]$ limits the total volume that can be withdrawn. This ensures that a certain amount of water remains in each reservoir, representing physical constraints and environmental regulations on irrigation withdrawals.
The fractions $f_{sw}$ and $f_{gw}$, ranging from 0 to 1, represent the accessibility of different reservoirs (surface water and groundwater, respectively) within the grid cell. The global map from \citet{siebert_groundwater_2010} of irrigation-equipped areas is used to define these fractions. A key feature of this map is that a given area cannot be equipped for both surface water and groundwater use simultaneously, which translates to $f_{sw} + f_{gw} =1$.

Once the demand $I_{req}$ and supply $A_w$ are calculated, the applied irrigation is determined as:
\begin{equation}
    I = min(A_w/dt, I_{req})
\end{equation}

Water is primarily withdrawn from the river reservoir, and only if this is insufficient to meet the demand are the other reservoirs tapped, in the limit of $f_{sw}$ and $f_{gw}$.
This withdrawn water is added in the code simultaneously with water infiltration into the soil, simulating a gravity-fed or drip irrigation method.

\hfill

To summarize the functioning of this irrigation scheme, the following inputs are used:
\begin{itemize}
    \item Map of irrigated fractions, to obtain $f_{irr}$.
    \item Map of irrigation-equipped areas, to obtain $f_{sw}$ and $f_{gw}$.
\end{itemize}

Additionally, to use this scheme, the following parameters must be set:
\begin{itemize}
    \item $Root_{lim}$, which determines the portion of the root system considered in the soil moisture deficit calculation.
    \item $\beta$, which defines the target soil moisture.
    \item $LAI_{lim}$, the minimum LAI threshold for irrigation activation.
    \item $I_{max}$, which limits the hourly irrigation rate.
    \item $a_i$, which restricts the volume that can be withdrawn from each reservoir.
    % \item $a_{i,add}$, which limits the maximum portion that can be supplied from the river reservoir of a neighboring grid cell.
\end{itemize}

The sensitivity analyses described in \citet{arboleda-obando_validation_2024} have shown that the parameter $\beta$ has the greatest impact on the volume of water withdrawn for irrigation. These analyses have allowed the determination of default values for all parameters to best represent the impact of irrigation on global climate. However, these values can be adjusted for simulations focused on a specific region.


\chapter{Evaluation and parameter tuning of a new routing scheme over the Iberian Peninsula}
\label{chap:routing}
\minitoc
\pagebreak
%\chapter{Evaluation and calibration of a new routing scheme over the Iberian Peninsula}
\label{chap:routing}
\minitoc
\pagebreak

\section{Chapter introduction}

This chapter presents the preliminary work to evaluate and calibrate the new version of the routing scheme (\native) before running coupled simulations with ICOLMDZOR. As a reminder, the pre-existing version of the routing scheme \std cannot be used with ICOLMDZOR because it imposes constraints on the ORCHIDEE and routing grids that are incompatible with the icosahedral grid.
The \native routing is based on the same modelling principles (described in Chapter \ref{chap:methods}) as \std but relies on a new code to interpolate between the ORCHIDEE grid and the routing grid. 

This preliminary work had two main objectives:
\begin{itemize}
    \item Ensure that the new code could replicate the behaviour of the \std version if given the same parameters and DEM as input. 
    Indeed, this routing has long been the default version in ORCHIDEE and in the IPSL-CM, including for CMIP6 coupled model runs and has been subject to calibration and evaluation studies. %todo:ref ?
    \item Evaluate and calibrate the new code using a high resolution DEM over the Iberian Peninsula to identify an appropriate set of parameters for coupled simulations.
\end{itemize}

%tood:reprendre pour mieux poser une question scientifique

\section{Methods for the routing scheme evaluation and calibration}

%todo: explain 2 DEMs better

The relevant routing outputs for evaluation and calibration are river discharge and water volumes in each of the reservoirs (groundwater, surface runoff, rivers). 
It must be noted that these volumes cannot be easily compared to observations, since large-scale measurements of groundwater or river volumes are complex and may not physically correspond to the abstraction level of the reservoirs modelled in ORCHIDEE.
There importance in this work is mainly justified by the fact that the irrigatin scheme withdraws water from these reservoirs to satisfy the irrigation demand while conserving water quantities.

All simulations for this chapter are run in offline mode, meaning that ORCHIDEE is not coupled to any atmospheric model but takes meteorological data as input. The choice of this meteorological forcing may impact the results of the offline simulation.

In all this chapter, the simulated river discharge is evaluated against monthly observation data from discharge stations of the Global Runoff Data Center \cite[GRDC, https://grdc.bafg.de,][]{fekete_global_2003}.
Stations were positioned on the 0.5° DEM grid using only the GPS position, and on the MERIT DEM grid with tools presented in \cite{polcher_hydrological_2023}, which use the GPS position of the stations as well as the upstream catchment area to find the most appropriate grid cell for comparison with the observations. 
Four stations were used between 2003 and 2012, which are all on large rivers (Ebro, Douro, Tagus, Guadiana) and show large discharge values, meaning they represent the integrated discharge over a large share of the basins. 

\begin{figure}[htbp]
    \centering
    \includegraphics[width=0.7\linewidth]{images/eval_halfdeg/river_discharge/halfdeg_4stations_map.png}
    \caption{Discharge stations selected for the routing scheme evaluation and calibration.}
    \label{fig:halfedg_stations_map}
\end{figure}

%todo : expliquer période 2000-2012 + spinup -> forçages historiques et obs beaucoup plus dispo (et on n'était pas encore sûrs de la période pour les simus couplées...)

\section{Consistency of \native with \std routing}
\subsubsection{Simulation setup}

This first analysis used the \native routing in the same setup as the \std routing. 
The input DEM was the same (0.5° resolution), and the time constants for the three reservoirs were the same as the default values for the global model (Table \ref{table:tcst_refs}). Four simulations were run, two without irrigation (labeled as \textit{no\_irr}) to assess the routing schemes without its influence, and two with irrigation (labeled as \textit{irr}). It must be noted that these simulations were not aimed at evaluating the realism of simulated irrigation (which will be addressed in Section \ref{section:calib}) but only to assess if the new routing code behaves similarly to the previous version.

\begin{table}[h]
\centering
\begin{tabular}{|c|c|c|}
\hline
\textbf{TCST\_SLOW} & \textbf{TCST\_FAST} & \textbf{TCST\_STREAM} \\ \hline
25            & 3             & 0.24            \\ \hline
\end{tabular}
\caption{Routing time constant for the consistency analysis ($day \cdot km^{-1}$).}
\label{table:tcst_consistency}
\end{table}

Simulations were run from 2000 to 2012, with the WFDEI forcing. %todo:version/ref
The first three years were considered as a spinup and removed from the analysis, to allow the vegetation and hydrological variables to reach an equilibrium. The regional domain covers the Iberian Peninsula and part of Morocco, since at the time, this region was also considered as a study area for coupled simulations. Since this idea was not pursued, the analysis only focuses on the Iberian Peninsula.

\subsubsection{Results}

%figure : 12 maps of mean reservoir volumes and hydrographs, for both sims and diff
%todo:subcaptions
\begin{figure}[htbp]
    \centering
    \begin{tabular}{ccc}
        \begin{subfigure}[b]{0.33\textwidth}
            \caption{}
            \includegraphics[width=\linewidth]{images/eval_halfdeg/maps/slowr_subgrid.png}
        \end{subfigure} &
        \begin{subfigure}[b]{0.33\textwidth}
            \caption{}
            \includegraphics[width=\linewidth]{images/eval_halfdeg/maps/slowr_interp.png}
        \end{subfigure} &
        \begin{subfigure}[b]{0.33\textwidth}
            \caption{}
            \includegraphics[width=\linewidth]{images/eval_halfdeg/maps/slowr_diff.png}
        \end{subfigure} \\
        
        \begin{subfigure}[b]{0.33\textwidth}
            \caption{}
            \includegraphics[width=\linewidth]{images/eval_halfdeg/maps/fastr_subgrid.png}
        \end{subfigure} &
        \begin{subfigure}[b]{0.33\textwidth}
            \caption{}
            \includegraphics[width=\linewidth]{images/eval_halfdeg/maps/fastr_interp.png}
        \end{subfigure} &
        \begin{subfigure}[b]{0.33\textwidth}
            \caption{}
            \includegraphics[width=\linewidth]{images/eval_halfdeg/maps/fastr_diff.png}
        \end{subfigure} \\
        
        \begin{subfigure}[b]{0.33\textwidth}
            \caption{}
            \includegraphics[width=\linewidth]{images/eval_halfdeg/maps/streamr_subgrid.png}
        \end{subfigure} &
        \begin{subfigure}[b]{0.33\textwidth}
            \caption{}
            \includegraphics[width=\linewidth]{images/eval_halfdeg/maps/streamr_interp.png}
        \end{subfigure} &
        \begin{subfigure}[b]{0.33\textwidth}
            \caption{}
            \includegraphics[width=\linewidth]{images/eval_halfdeg/maps/streamr_diff.png}
        \end{subfigure} \\
        
        \begin{subfigure}[b]{0.33\textwidth}
            \caption{}
            \includegraphics[width=\linewidth]{images/eval_halfdeg/maps/hydrographs_subgrid.png}
        \end{subfigure} &
        \begin{subfigure}[b]{0.33\textwidth}
            \caption{}
            \includegraphics[width=\linewidth]{images/eval_halfdeg/maps/hydrographs_interp.png}
        \end{subfigure} &
        \begin{subfigure}[b]{0.33\textwidth}
            \caption{}
            \includegraphics[width=\linewidth]{images/eval_halfdeg/maps/hydrographs_diff.png}
        \end{subfigure} \\
    \end{tabular}
    \caption{Annual mean (2003-2012) of reservoir volumes and river discharge for \std (\noirr), \native (\noirr) and difference between them.}
    \label{fig:routing_reservoirs_halfdeg}
\end{figure}

The comparison of the mean volumes for the fast and slow reservoirs (Fig. \ref{fig:routing_reservoirs_halfdeg}a-f) shows differences only on coastal grid cells, which have null values in \std but can have significant values in \native, particularly in the northern moutain ranges, which are regions of high precipitation, and therefore high drainage and surface runoff. %null, almost, ???
This is explained by a new modelling choice in the \native code, to represent these two reservoirs on coastal grid cells whereas drainage and surface runoff were considered to flow directly in the ocean in \std. %todo:check if this is true for standard... if not, we're missing an explanation...

Regarding the stream reservoir (Fig. \ref{fig:routing_reservoirs_halfdeg}g-i), an opposite choice was made on coastal grid cells. In \std, there was a stream reservoir for these grid cells, whereas it was removed in \native. This choice was made during the development of the scheme and might be changed in future versions to ensure consistency with the other two reservoirs. 
This change in coastal grid cells does not have an impact on the river discharge (Fig. \ref{fig:routing_reservoirs_halfdeg}j-l) because the way to compute it (\textit{hydrographs} variable in the model) was also changed from grid cell output to grid cell input. Coastal grid cell therefore have similar values of river discharge, but not of volume in the river reservoir. 
%todo:commenter l'augmentation de hydrographs sur les côtes. 

The most noticeable discrepancy on the stream reservoir and river discharge is that some non-coastal points have different values in both versions. It can be seen on Fig. \ref{fig:routing_reservoirs_halfdeg}j-l particularly that the water does not flow exactly in the same path, meaning that the DEM is not interpreted exactly in the same way by both versions. This has been attributed to small preprocessings of the DEM implemented in \std to avoid creating dead-ends, grid cells where the water would not keep flowing and accumulate in the stream reservoir. This was implemented when developing the routing scheme for global applications, possibly with a focus on other ORCHIDEE grid resolutions (1° or 2° instead of the 0.5° used here) and do not seem necessary for this domain and resolution, which is why they were not implemented in \native. These modifications may impact the routing graph, even though there do not seem to be any dead-ends in the routing graph computed by \native. In the large rivers where there are differences, \native seems to use shorter paths that \std, flowing diagonally to the next grid cell instead of taking two 90° turns. In most places, on annual average, this seems to lead to smaller amounts of water in the river reservoir and smaller discharge in \native compared to \std, as clearly visible on the Duero river in Fig. \ref{fig:routing_reservoirs_halfdeg}l.

%figure : 3 time series (3 reservoirs)on average over the IP for both routings
%todo:subcaptions
\begin{figure}[htbp]
    \centering
    \begin{subfigure}[b]{0.32\textwidth}
        \caption{}
        \includegraphics[width=\linewidth]{images/eval_halfdeg/time_series/slowr_time_series.png}
    \end{subfigure}
    \begin{subfigure}[b]{0.32\textwidth}
        \caption{}
        \includegraphics[width=\linewidth]{images/eval_halfdeg/time_series/fastr_time_series.png}
    \end{subfigure}
    \begin{subfigure}[b]{0.32\textwidth}
        \caption{}
        \includegraphics[width=\linewidth]{images/eval_halfdeg/time_series/streamr_time_series.png}
    \end{subfigure} \\

    \begin{subfigure}[b]{0.32\textwidth}
        \caption{}
        \includegraphics[width=\linewidth]{images/eval_halfdeg/time_series/slowr_seasonal_cycle.png}
    \end{subfigure}
    \begin{subfigure}[b]{0.32\textwidth}
        \caption{}
        \includegraphics[width=\linewidth]{images/eval_halfdeg/time_series/fastr_seasonal_cycle.png}
    \end{subfigure}
    \begin{subfigure}[b]{0.32\textwidth}
        \caption{}
        \includegraphics[width=\linewidth]{images/eval_halfdeg/time_series/streamr_seasonal_cycle.png}
    \end{subfigure}
    \caption{Time series and seasonal cycles of reservoir volumes on average over the Iberian Peninsula domain.}
    \label{fig:reservoir_time_series}
\end{figure}

On average over the Peninsula, these differences are visible in the groundwater and surface reservoirs where \native has larger volumes due to water in coastal grid cells (Fig. \ref{fig:reservoir_time_series}a, b). On the river reservoir however, \std has larger volumes, mostly due to the absence of this reservoir on coastal points with river outlets, which are the grid cells with the largest volumes in \std.
If all coastal grid cells were to be removed from the average (not shown)%todo: faire la figure ? ou annexe ?
, there would be no difference on the groundwater and surface reservoirs, and a much difference on the river reservoir, although \std still has more water stored in this reservoir than \native, particularly in winter and spring, when there is more rain in the region. This confirms the previous observation that water follows shorter paths to the sea in \native and therefore, does not remain in the river reservoir as long as in \std.
In the \irr simulations (purple and green lines in Fig. \ref{fig:reservoir_time_series}), all reservoirs are partly depleted by the irrigation withdrawals. For the groundwater and surface reservoirs, the differences between \std and \native mostly remain, since they are located in coastal grid cells, which are seldom intensely irrigated.
Regarding the river reservoir (Fig. \ref{fig:reservoir_time_series}f), \std simulates much lower volumes in the \irr than in \noirr simulation (-75\% in winter and complete depletion in summer). This decrease is also seen with \native, and the differences between the two routing versions almost disappear. This is consistent with the source of these differences, since the \irr simulations simulate much lower volumes in rivers, the discrepancies in the routing graph and at the river outlets have much less impact on the average volume over the domain.

%figure : 3 maps of simulated irrigation (subgrid, interp, diff), TS and SC of irrig for both simulations
%todo:subcaptions
\begin{figure}[htbp]
    \centering
    \begin{subfigure}[b]{0.32\textwidth}
        \caption{}
        \includegraphics[width=\linewidth]{images/eval_halfdeg/maps/irrigation_subgrid.png}
    \end{subfigure}
    \begin{subfigure}[b]{0.32\textwidth}
        \caption{}
        \includegraphics[width=\linewidth]{images/eval_halfdeg/maps/irrigation_interp.png}
    \end{subfigure}
    \begin{subfigure}[b]{0.32\textwidth}
        \caption{}
        \includegraphics[width=\linewidth]{images/eval_halfdeg/maps/irrigation_diff.png}
    \end{subfigure} \\

    \begin{subfigure}[b]{0.48\textwidth}
        \caption{}
        \includegraphics[width=\linewidth]{images/eval_halfdeg/time_series/irrigation_time_series.png}
    \end{subfigure}    
    \begin{subfigure}[b]{0.48\textwidth}
        \caption{}
        \includegraphics[width=\linewidth]{images/eval_halfdeg/time_series/irrigation_seasonal_cycle.png}
    \end{subfigure}

    \caption{Simulated irrigation in \std and \native. Annual mean for both simulations (a,b) and difference (c). Average time series (d) and seasonal cycle (e) over the Iberian Peninsula.}
    \label{fig:irrigation_halfdeg_eval}
\end{figure}

Both simulations show a similar simulated irrigation (Fig. \ref{fig:irrigation_halfdeg_eval}a, b), but the differences on reservoir volumes affect the available water, and therefore irrigation. This is particlarly visible in the Ebro valley where some grid cells are highly irrigated in one simulation and not in the other. This is due to the aforementionned differences in the routing graph, which affect the path of the Ebro river (Fig. \ref{fig:irrigation_halfdeg_eval}c), the main source for irrigation withdrawals in the region.
On average over the domain (Fig. \ref{fig:irrigation_halfdeg_eval}d, e) \std exhibits slightly higher volumes of irrigation than \native, which is consistent with the higher volumes in the river reservoir. As previously explained, the differences seen in the reservoirs on coastal grid cells seem to have little impacts on the irrigated volumes, since these regions are seldom intensely irrigated. %todo:ref to irrigmap figure ? In methods ?
%todo: comment more on depletion, irrigation limited by available water, and second peak in autumn which is enabled by new rain

%figure : river discharge seasonal cycle for 4 stations (large rivers but not Guadalquivir...), 2 routing versions, irr and no_irr for each
\begin{figure}[htbp]
    \centering
    \includegraphics[width=\linewidth]{images/eval_halfdeg/river_discharge/halfdeg_4stations_SC.png}
    \caption{River discharge mean seasonal cycle (2003-2012) for 4 large stations of the Peninsula.}
    \label{fig:halfdeg_stations_SC}
\end{figure}

The seasonal cycle of river discharge reflects the general agreement of the two routing versions, without irrigation, the red and yellow lines follow each other closely, and with irrigation, the purple and green are also very similar. Some differences persist, due to the differences in the routing graph already described, mostly for the Ebro river since the station is downstream of several discrepancies visible on Fig. \ref{fig:routing_reservoirs_halfdeg}l.

\subsection{Conclusions}

This comparison of the \native routing version with \std showed that this new version is generally reproducing the modelling principles correctly, but several differences were identified between the two versions. All three reservoirs are handled differently in \native on coastal grid cells, which affects the average volumes stored over the Iberian Peninsula, but does not have a significant impact on irrigation or river dicharge. However, differences in the path followed by water as it flows from one grid cell to another in the river reservoirs are also clearly identified in large rivers of the Peninsula. This leads to local changes in the simulated river dicharge, and slightly less water stored in the river reservoir by \native since it simulated shorter paths in several occurrences. These discrepancies also have an impact on irrigation since it is strongly reliant on the volumes of water available in the river reservoir, but this impact remains limited on average over the domain.

\section{Calibration over the Iberian Peninsula}
\label{section:calib}

Once the general consistency of the \native routing had been assessed, it became possible to evaluate the specific setup that would be used for the coupled simulation, using the MERIT DEM at 2km resolution and accounting for irrigation. To use irrigation, the routing time step must be the same as the time step of ORCHIDEE, which is 15mn (900s) in coupled simulations. 

This calibration was initially focused on the three time constants of the routing reservoirs: TCST\_SLOW, TCST\_FAST, TCST\_STREAM. Although these parameters could theoretically be independent of the spatial and temporal time steps and of the DEM used as input, they can strongly influence the results. In particular, the ORCHIDEE routing scheme had never been used with a high-resolution topography like the MERIT DEM at 2-km resolution, since the \std routing ran only with the 0.5° resolution DEM used in the previous section, and often with a 24-hours routing time step. Multiple tests were therefore carried to identify an appopriate set of time constants for the reservoirs.
Moreover, the sensitivity of river discharge to the activation of irrigation was identified as very important, with large differences in the responses depending on the intensity of irrigation demand. As a consequence, the parameter $\beta$, which defines the soil moisture target in the irrigation scheme, was also included in this calibration and modified compared to the default version of the model.

% For these offline calibration simulations, given the better availability of flow observations and meteorological forcings before 2014, I did not work exactly on the period 2010-2022 but rather on the previous decades.. 

% It is also the version that was used for the calibration of the irrigation scheme developped in \citet{arboleda-obando_validation_2024} and since the scheme withdraws

\subsection{Simulation setup}

At the begining of the calibration, three reference sets of TCST parameters (presented in Table \ref{table:tcst_refs}) were considered:
\begin{itemize}
\item The values established during an initial calibration of the \native routing performed over the Danube basin, which focused only on river discharge observations \citep{kilic_evaluation_2023}.
\item The default values used by the \std routing on a global scale, which were used in the previous section with the 0.5° DEM. These values were identified empirically by a calibration study over the Senegal river basin and validated globally in \citet{ngo-duc_validation_2007} %todo:add ngoduc 2005
using river discharge and groundwater measurements from Gravity Recovery and Climate Experiment (GRACE) satellite data.
\item The values used for regional studies with the \textit{subgrid\_HTU} routing \citep{rinchiuso_improving_2022, huang_multi-objective_2024}.
\end{itemize}

\begin{table}[h]
\centering
\begin{tabular}{|l|c|c|c|}
\hline
\textbf{} & \textbf{TCST\_SLOW} & \textbf{TCST\_FAST} & \textbf{TCST\_STREAM} \\ \hline
Initial \native calibration & 1.2 & 0.9 & 0.03 \\ \hline
\std default values & 25 & 3 & 0.24 \\ \hline
\textit{Subgrid\_HTU} calibration & 600 & 80 & 6.3 \\ \hline
\end{tabular}
\caption{Reference parameter sets considered to calibrate the \native routing ($day \cdot km^{-1}$).}
\label{table:tcst_refs}
\end{table}

In these three sets of values, there is approximately an order of magnitude between the three time constants, and very significant differences (one or two orders of magnitude) between each set. 
This can be explained by the different DEMs used, the way HTUs are defined, and the calibration choices that led to these sets of values.%todo:compléter ? 

Three initial experiments were defined, based on these ratios:\\$TCST\_SLOW \approx 10 \times TCST\_FAST \approx 100 \times TCST\_STREAM$.
Table \ref{table:tcst_exp} presents these three sets of values based on the orders of magnitude of each of the reference value sets (TCST1, TCST2, TCST3), as well as the one selected after the calibration (TCST4), whose development steps are detailed hereafter.

\begin{table}[h]
\centering
\begin{tabular}{|l|c|c|c|}
\hline
\textbf{} & \textbf{TCST\_SLOW} & \textbf{TCST\_FAST} & \textbf{TCST\_STREAM} \\ \hline
TCST1 & 3 & 0.3 & 0.03 \\ \hline
TCST2 & 30 & 3 & 0.3 \\ \hline
TCST3 & 300 & 30 & 3 \\ \hline
TCST4 & 700 & 100 & 0.1 \\ \hline
\end{tabular}
\caption{Value sets of the simulations for the calibration of the \native routing ($day \cdot km^{-1}$).}
\label{table:tcst_exp}
\end{table}

Four simulations were therefore carried out with these four sets of TCST values, over the period 2000-2012, without activating irrigation.
The first three years are considered as a spin-up to allow the vegetation, soil moisture, and routing reservoir volumes to reach equilibrium, and the results presented therefore cover 10 years of simulation (2003-2012). %todo:remove if this part is already mentionned in the chapter methods

\subsection{Reservoir volumes and time constants}

%figure : 3 time series (3 reservoirs) on average over the IP for 4 tcsts and std as reference
%todo:subcaptions
\begin{figure}[htbp]
    \centering
    \begin{subfigure}[b]{0.32\textwidth}
        \caption{}
        \includegraphics[width=\linewidth]{images/eval_halfdeg/time_series/slowr_time_series_tcsts.png}
    \end{subfigure}
    \begin{subfigure}[b]{0.32\textwidth}
        \caption{}
        \includegraphics[width=\linewidth]{images/eval_halfdeg/time_series/fastr_time_series_tcsts.png}
    \end{subfigure}
    \begin{subfigure}[b]{0.32\textwidth}
        \caption{}
        \includegraphics[width=\linewidth]{images/eval_halfdeg/time_series/streamr_time_series_tcsts.png}
    \end{subfigure} \\

    \begin{subfigure}[b]{0.32\textwidth}
        \caption{}
        \includegraphics[width=\linewidth]{images/eval_halfdeg/time_series/slowr_seasonal_cycle_tcsts.png}
    \end{subfigure}
    \begin{subfigure}[b]{0.32\textwidth}
        \caption{}
        \includegraphics[width=\linewidth]{images/eval_halfdeg/time_series/fastr_seasonal_cycle_tcsts.png}
    \end{subfigure}
    \begin{subfigure}[b]{0.32\textwidth}
        \caption{}
        \includegraphics[width=\linewidth]{images/eval_halfdeg/time_series/streamr_seasonal_cycle_tcsts.png}
    \end{subfigure}
    \caption{Time series and seasonal cycles of reservoir volumes on average over the Iberian Peninsula domain.}
    \label{fig:reservoir_time_series_tcsts}
\end{figure}

%figure : river discharge seasonal cycle for 4 stations (large rivers but not Guadalquivir...), 2 routing versions, irr and no_irr for each
\begin{figure}[htbp]
    \centering
    \includegraphics[width=\linewidth]{images/eval_halfdeg/river_discharge/merit_tcst_4stations_SC.png}
    \caption{River discharge mean seasonal cycle (2003-2012) for 4 large stations of the Peninsula.}
    \label{fig:merit_tcsts_stations_SC}
\end{figure}

\subsection{Impact of irrigation on river discharge}

blabla
blabla

blabla


\begin{figure}[htbp]
    \centering
    \includegraphics[width=\linewidth]{images/eval_halfdeg/river_discharge/merit_irr_4stations_SC.png}
    \caption{River discharge mean seasonal cycle (2003-2012) for 4 large stations of the Peninsula.}
    \label{fig:merit_irr_stations_SC}
\end{figure}

\subsubsection{Sensitivity to atmospheric forcing}


\begin{figure}[htbp]
    \centering
    \includegraphics[width=\linewidth]{images/eval_halfdeg/river_discharge/merit_forcing_4stations_SC.png}
    \caption{River discharge mean seasonal cycle (2003-2012) for 4 large stations of the Peninsula.}
    \label{fig:merit_forcing_stations_SC}
\end{figure}

\section{Chapter conclusions}

This chapter presents the evaluation and calibration of the new version of the routing scheme using ORCHIDEE offline simulations  over the Iberian Peninsula, which was necessary to prepare for coupled simulations.

This calibration and evaluation work first showed that the new routing code behaves similarly to the pre-existing version if given the same input map and parameters.%todo : some differences identified ?

The new routing was then used with a high-resolution DEM over the Iberian Peninsula, and several options were tested for the reservoir time constants. 
Comparison of reservoir volumes with the previous version of the routing (\std), which served as a first reference, allowed the identification of values for TCST\_SLOW and TCST\_FAST to simulate similar water volumes on average over the Peninsula. This was essential because of the dependance of the irrigation scheme on the available water in the routing reservoirs.
Using discharge observations, the value of TCST\_STREAM was then adjusted to correctly represent the seasonal cycle of the major rivers of the Iberian Peninsula.

Sensitivity experiments highlighted the very large response of river discharge to the activation of irrigation (except in winter). 
This is not shown here, %tocheck and/or todo
but it is worth noting that irrigation stood out significantly compared to other sensitivity analyses conducted (on meteorological forcing, on the use of bare soil evaporation resistance). 
The magnitude of this sensitivity also contributed to the decision not to further calibrate the TCST values, whose impact on fidelity to observations remains limited compared to that of irrigation.
As a consequence, the $\beta$ parameter of the irrigation scheme was also included in the calibration, and 0.6 was identified as a more suitable value than the default 0.9. This choice allows avoiding excessively low discharge in summer for the four rivers considered, and is considered more consistent with a representation of irrigation practices in the region.

Therefore, the set of parameters from the simulations \textit{tcst4\_no\_irr} and \textit{tcst4\_irr\_reduced} were used to study the impact of irrigation in the coupled simulations presented in CHapter \ref{chap:monthly}.
%over

\chapter{Influence of lateral boundary conditions on the consistency of the ICOLMDZOR LAM}
\label{chap:forcing}
\minitoc
\pagebreak
%\section{Chapter introduction}

The main objective of this PhD work was to study the impact of irrigation in coupled ICOLMDZOR LAM simulations.
In the process of identifying an adequate simulation setup for the LAM, considering how recent the model is, particular attention was paid to the consistency of the model. 
Structural biases were identified in the transition zone on the edges of the LAM, with possible associated impacts in the central free zone. 
A sensitivity analysis on the size of the domain enabled pursuing the work with a limited influence of these biases, leading to the results presented in Chapters \ref{chap:monthly} and \ref{chap:liaise}, but several secondary questions were identified as to their source and other ways to reduce them. 
These were further investigated separately, during Mariame Maiga's internship for her first year of Masters at Sorbonne Université \citep{maiga2025}, which I co-supervised with Frédérique Cheruy from April to July 2025. 

These analyses provided valuable understanding of the sensitivities of the LAM to its lateral forcing and addressed the following questions: 
\begin{itemize}
    \item Which variables have an inconsistent behaviour in the transition zone ? How can these inconsistencies be explained ?
    \item How can the size of the domain limit the extent of these inconsistencies to ensure the LAM is not influenced in the free zone ?
    \item How dependent on the choice of lateral forcing are the inconsistencies in the transition zone and the general performance of the model ?
    \item How much does the sampling frequency of the forcing file influence the consistency and performance of the LAM ?
\end{itemize}

\hfill

In this chapter, the inconsistencies identified in the transition zone are analysed in several coupled simulations without irrigation, presented in Section \ref{sec:sim_setups}. Their key characteristics are summarised in Table \ref{table:coupled_simulations_chap4}.

%table:all coupled simulation details
%todo:add SST/SIC ?
\begin{table}[htbp]
    \centering
    \resizebox{\textwidth}{!}{
        \begin{tabular}{|l|l|l|l|l|l|l|l|}
            \hline
            \textbf{Section} & \textbf{Simulation name} & \textbf{Period} & \textbf{Diameter} & \textbf{NBP} & \textbf{Forcing} & \textbf{Irrigation} \\
            \hline
            \ref{sec:domain_size} & \smalld & 2010-2014 & 1000 km & 40 & ERA5, 1h & No \\
            \hline
            \ref{sec:domain_size}  & \interd & 2010-2014 & 1500 km & 60 & ERA5, 1h & No \\
            \hline
            \ref{sec:domain_size}  & \larged & 2010-2014 & 2000 km & 80 & ERA5, 1h & No \\
            \hline
            \ref{sec:forcing_source} & \forcingERA & 2010-2022 & 1000 km & 40 & ERA5, 1h & No \\
            \hline
            \ref{sec:forcing_source} & \forcingICO & 2010-2022 & 1000 km & 40 & ICOLMDZOR, 1h & No \\
            \hline
            \ref{sec:forcing_frequency} & \forcingoneh & 2013 (* 10) & 1500 km & 60 & ERA5, 1h & No \\
            \hline
            \ref{sec:forcing_frequency} & \forcingsixh & 2013 (* 10) & 1500 km & 60 & ERA5, 6h & No \\
            \hline
        \end{tabular}
    }
    \caption{Characteristics of simulations used in this chapter.}
    \label{table:coupled_simulations_chap4}
\end{table}

Throughout this chapter, the model is compared to the ERA5 reanalysis. Although it is not considered the most realistic reference product for all variables, the study of the inconsistencies mostly focused on the spatial structure of the variables, rather than on the absolute values.
For this purpose, ERA5 has two advantages: it is available over all the simulation domain, at a resolution similar to the simulations (0.25° is close to 25 km in mid-latitudes), and provides all the variables of the model.
Most figures presented are maps of biases relative to ERA5 for six variables of interest, and their values in the reanalysis over the period 2010-2014 are shown in Fig. \ref{fig:ERA_var_maps} to provide a perspective on the expected structure and order of magnitudes.

%figure : maps of 6 vars in ERA to see variable structure
\begin{figure}[htbp]
    \centering
    \begin{tabular}{ccc}
        %precip
        \begin{subfigure}[b]{0.33\textwidth}
            \caption{Precipitation (mm \perday)}
            \includegraphics[width=\textwidth]{images/chap4/domain_size/var_map_precip_ERA5.png}
        \end{subfigure} &
        \begin{subfigure}[b]{0.33\textwidth}
            \caption{Downwelling shortwave \\radiation (W \persqm)}
            \includegraphics[width=\textwidth]{images/chap4/domain_size/var_map_SWdnSFC_ERA5.png}
        \end{subfigure} &
        \begin{subfigure}[b]{0.33\textwidth}
            \caption{Total cloud cover \\(cloud fraction in \%)}
            \includegraphics[width=\textwidth]{images/chap4/domain_size/var_map_cldt_ERA5.png}
        \end{subfigure} \\

        \begin{subfigure}[b]{0.33\textwidth}
            \caption{Evapotranspiration (mm \perday)}
            \includegraphics[width=\textwidth]{images/chap4/domain_size/var_map_evap_ERA5.png}
        \end{subfigure} &
        \begin{subfigure}[b]{0.33\textwidth}
            \caption{Downwelling longwave \\radiation (W \persqm)}
            \includegraphics[width=\textwidth]{images/chap4/domain_size/var_map_LWdnSFC_ERA5.png}
        \end{subfigure} &
        \begin{subfigure}[b]{0.33\textwidth}
            \caption{Low cloud cover \\(cloud fraction in \%)}
            \includegraphics[width=\textwidth]{images/chap4/domain_size/var_map_cldl_ERA5.png}
        \end{subfigure}
    \end{tabular}
    \caption{Annual mean of the six variables of interest for the analysis of the LAM structural biases in ERA5 over the period 2010-2014.}
    \label{fig:ERA_var_maps}
\end{figure}
%todo:voir si meilleure mise en page des légendes est possible?

\section{Structural inconsistencies and influence of domain size}
\label{sec:domain_size}
The inconsistencies were identified with the initial simulation setup \smalld, by noticing that it presents little to no precipitation on the edges of the domain. The LAM displays a strong underestimation compared to ERA5 in the transition zone (Fig. \ref{fig:domain_size_P_ET_ERA_diff_maps}a), although it remains hard to determine if these biases have an influence on the rest of the domain. In particular,, the Iberian Peninsula exhibits underestimated precipitation along its northern and western coasts which might either be considered as the continuity of the bias on the edges, or as the consequence of other issues linked to rainfall at the transition between ocean and land.
This lack of precipitation is correlated to a lower ET on continents, which receive less water. However, over the Atlantic ocean and the Mediterranean sea, ET was surprisingly high, overestimating the values of ERA5 (Fig. \ref{fig:domain_size_P_ET_ERA_diff_maps}b). 

Simulations with the intermediate and large domains show that the strongest biases in precipitation are located in the transition zone, especially in the north-west of the domain which is not influenced by the presence of continents (Fig. \ref{fig:domain_size_P_ET_ERA_diff_maps}b-c).
In all simulations, the differences compared to ERA5 in the very edge of the domain correspond to a relative decrease between -80\% and -100\%, meaning the model computes almost no precipitation in these grid cells (similar maps to Fig. \ref{fig:domain_size_P_ET_ERA_diff_maps} but with relative differences can be found in supplementary Fig. \ref{fig:domain_size_ERA_reldiff_maps}).
In the free zone, precipitation remains underestimated over the sea but much less than in \smalld, and the bias decreases towards the centre. Along the northern coast of the Peninsula, the underestimation is still present but seems structurally independent from the behaviour of the model on the edges of the domain.
The overestimation in ET is also confined to the northwestern edge of the domain and the large overestimation in the Mediterranean is reversed to an underestimation, of smaller magnitude, for both \interd and \larged (Fig. \ref{fig:domain_size_P_ET_ERA_diff_maps}e-f).
As mentioned before, although ERA5 cannot be considered as the best reference product for all the variables considered, it is the spatial structure of the biases that is striking here, since they are independent of the structure of the variable (Fig. \ref{fig:ERA_var_maps}), and follow the edges of the domain even when the radius changes.

%figure : maps of diff vs ERA for 3 domain sizes : precip, evap
\begin{figure}[htbp]
    \centering
    \begin{tabular}{ccc}
        %precip
        \begin{subfigure}[b]{0.33\textwidth}
            \caption{Precipitation bias to ERA5\\(mm \perday, \smalld)}
            \includegraphics[width=\textwidth]{images/chap4/domain_size/diff_map_precip_era_LAM_1000km_NBP40.png}
        \end{subfigure} &
        \begin{subfigure}[b]{0.33\textwidth}
            \caption{Precipitation bias to ERA5\\(mm \perday, \interd)}
            \includegraphics[width=\textwidth]{images/chap4/domain_size/diff_map_precip_era_LAM_1500km_NBP60.png}
        \end{subfigure} &
        \begin{subfigure}[b]{0.33\textwidth}
            \caption{Precipitation bias to ERA5 \\(mm \perday, \larged)}
            \includegraphics[width=\textwidth]{images/chap4/domain_size/diff_map_precip_era_LAM_2000km_NBP80.png}
        \end{subfigure} \\
        
        %evap
        \begin{subfigure}[b]{0.33\textwidth}
            \caption{ET bias to ERA5\\(mm \perday, \smalld)}
            \includegraphics[width=\textwidth]{images/chap4/domain_size/diff_map_evap_era_LAM_1000km_NBP40.png}
        \end{subfigure} &
        \begin{subfigure}[b]{0.33\textwidth}
            \caption{ET bias to ERA5\\(mm \perday, \interd)}
            \includegraphics[width=\textwidth]{images/chap4/domain_size/diff_map_evap_era_LAM_1500km_NBP60.png}
        \end{subfigure} &
        \begin{subfigure}[b]{0.33\textwidth}
            \caption{ET bias to ERA5\\(mm \perday, \larged)}
            \includegraphics[width=\textwidth]{images/chap4/domain_size/diff_map_evap_era_LAM_2000km_NBP80.png}
        \end{subfigure} 
    \end{tabular}
    \caption{Precipitation (a-c) and evapotranspiration (d-f) biases compared to ERA5 over 2010-2014 for three simulations with small, intermediate and large domain sizes.}
    \label{fig:domain_size_P_ET_ERA_diff_maps}
\end{figure}

These first findings led to the hypothesis that the physics of the LAM is not behaving in a normal way in the transition zone, and more specifically the large-scale water condensation scheme. 
Comparisons with ERA5 show an underestimation of total cloud cover in the transition zone (Fig. \ref{fig:domain_size_clouds_ERA_diff_maps}a-c) and particularly of low cloud cover (Fig. \ref{fig:domain_size_clouds_ERA_diff_maps}d-f), which constitutes its most important component (Fig. \ref{fig:ERA_var_maps}c, f). %option (commentaire FC: profil moyen de nuage ou sur 1 ou 2 points de l'atlantique)
The absence of clouds allows more solar radiation to reach the surface, leading to higher values of the downwelling shortwave radiation flux (Fig. \ref{fig:domain_size_clouds_ERA_diff_maps}g-i). This can explain the higher ET values since it provides more available energy for evaporation of seawater.
Having less condensed water also leads to a smaller downwelling longwave radiation flux in the transition zone (Fig. \ref{fig:domain_size_clouds_ERA_diff_maps}j-l). This relationship is further illustrated in the central part of the domain, where the downwelling longwave radiation flux and low cloud cover are both overestimated, following similar spatial patterns.

%figure : maps of diff vs ERA for 3 domain sizes : SW, LW, cloud cover
\begin{figure}[htbp]
    \centering
    \begin{tabular}{ccc}
        %total
        \begin{subfigure}[b]{0.33\textwidth}
            \caption{Total cloud cover bias\\(\%, \smalld)}
            \includegraphics[width=\textwidth]{images/chap4/domain_size/diff_map_cldt_era_LAM_1000km_NBP40.png}
        \end{subfigure} &
        \begin{subfigure}[b]{0.33\textwidth}
            \caption{Total cloud cover bias\\(\%, \interd)}
            \includegraphics[width=\textwidth]{images/chap4/domain_size/diff_map_cldt_era_LAM_1500km_NBP60.png}
        \end{subfigure} &
        \begin{subfigure}[b]{0.33\textwidth}
            \caption{Total cloud cover bias\\(\%, \larged)}
            \includegraphics[width=\textwidth]{images/chap4/domain_size/diff_map_cldt_era_LAM_2000km_NBP80.png}
        \end{subfigure} \\
        
        %low
        \begin{subfigure}[b]{0.33\textwidth}
            \caption{Low cloud cover bias\\(\%, \smalld)}
            \includegraphics[width=\textwidth]{images/chap4/domain_size/diff_map_cldl_era_LAM_1000km_NBP40.png}
        \end{subfigure} &
        \begin{subfigure}[b]{0.33\textwidth}
            \caption{Low cloud cover bias\\(\%, \interd)}
            \includegraphics[width=\textwidth]{images/chap4/domain_size/diff_map_cldl_era_LAM_1500km_NBP60.png}
        \end{subfigure} &
        \begin{subfigure}[b]{0.33\textwidth}
            \caption{Low cloud cover bias\\(\%, \larged)}
            \includegraphics[width=\textwidth]{images/chap4/domain_size/diff_map_cldl_era_LAM_2000km_NBP80.png}
        \end{subfigure} \\

        %SWdn
        \begin{subfigure}[b]{0.33\textwidth}
            \caption{Downwelling SW flux bias\\(W \persqm, \smalld)}
            \includegraphics[width=\textwidth]{images/chap4/domain_size/diff_map_SWdnSFC_era_LAM_1000km_NBP40.png}
        \end{subfigure} &
        \begin{subfigure}[b]{0.33\textwidth}
            \caption{Downwelling SW flux bias\\(W \persqm, \interd)}
            \includegraphics[width=\textwidth]{images/chap4/domain_size/diff_map_SWdnSFC_era_LAM_1500km_NBP60.png}
        \end{subfigure} &
        \begin{subfigure}[b]{0.33\textwidth}
            \caption{Downwelling SW flux bias\\(W \persqm, \larged)}
            \includegraphics[width=\textwidth]{images/chap4/domain_size/diff_map_SWdnSFC_era_LAM_2000km_NBP80.png}
        \end{subfigure} \\
        
        %LWdn
        \begin{subfigure}[b]{0.33\textwidth}
            \caption{Downwelling LW flux bias\\(W \persqm, \smalld)}
            \includegraphics[width=\textwidth]{images/chap4/domain_size/diff_map_LWdnSFC_era_LAM_1000km_NBP40.png}
        \end{subfigure} &
        \begin{subfigure}[b]{0.33\textwidth}
            \caption{Downwelling LW flux bias\\(W \persqm, \interd)}
            \includegraphics[width=\textwidth]{images/chap4/domain_size/diff_map_LWdnSFC_era_LAM_1500km_NBP60.png}
        \end{subfigure} &
        \begin{subfigure}[b]{0.33\textwidth}
            \caption{Downwelling LW flux bias\\(W \persqm, \larged)}
            \includegraphics[width=\textwidth]{images/chap4/domain_size/diff_map_LWdnSFC_era_LAM_2000km_NBP80.png}
        \end{subfigure}
    \end{tabular}
    \caption{Cloud cover and downwelling radiative fluxes biases to ERA5 over 2010-2014 for three simulations with small, intermediate and large domain sizes.}
    \label{fig:domain_size_clouds_ERA_diff_maps}
\end{figure}

These results confirmed the hypothesis that the LAM is hindered from condensing water in the transition zone, where it is nudged towards the forcing data from the ERA5 reanalysis. This was attributed to the fact that the ERA5 data is obtained with a model that has its own physics scheme, possibly functioning very differently from the LMDZ physics, and that the continuous nudging does not give the physics of the LAM the freedom it needs to condense water. This explanation is very consistent with the very low values of cloud cover and precipitation in this zone.

In the free zone of the LAM, the model once again behaves normally, but can be influenced by the transition zone via the advection terms for atmospheric moisture computed by the dynamics. This means that the influence of the forcing and of the discrepancies between the physics of ERA5 and ICOLMDZ is not strictly confined to the transition zone.
The comparison of the three domain sizes show that in \smalld, the underestimation of precipitation is impacting some parts of the Peninsula, and that biases in ET are not limited to the edges but extended to large portions of the continental study area. With the two larger domains, although some biases persist over the Peninsula, their structure suggests that they are not the direct consequence of the discrepancies in the transition zone. Clear progress in the consistency of the LAM was therefore achieved by changing from the small to the intermediate domain. As there was no obvious improvements between the intermediate and large domain, the less computationally-intensive \interd simulation setup with $R_{domain} = 1500$ km and $NBP = 60$ was used for the rest of the study (results presented in Section \ref{sec:article1} and Chapter \ref{chap:liaise}).

Meanwhile, to explore the hypothesis that the biases in the transition zone are due to discrepancies between the physics of ICOLMDZ and ERA5, LAM simulations were run using a global ICOLMDZOR simulation as the source for the lateral boundary conditions. The results of this analysis are presented in the following Section \ref{sec:forcing_source}.

\clearpage

\section{Impact of the source of the lateral forcing}
\label{sec:forcing_source}

Here, two simulations using the small simulation domain over the period 2010-2022 are compared to assess the influence of the source used for the lateral forcing: one with forcing data from the ERA5 reanalysis (\forcingERA) and one using outputs of a global ICOLMDZOR simulation as forcing data (\forcingICO). 
The \forcingICO simulation was run by Frédérique Cheruy, and Mariame Maiga contributed to the analysis during her internship.

%figure : maps of diff vs ERA for 2 forcing sources
\begin{figure}[!h]
% \begin{figure}[b]
    \centering
    \begin{tabular}{cc}
        %precip
        \begin{subfigure}[b]{0.33\textwidth}
            \caption{Precipitation bias\\(mm \perday, \forcingERA)}
            \includegraphics[width=\textwidth]{images/chap4/forcing_source/diff_map_precip_era_era.png}
        \end{subfigure} &
        \begin{subfigure}[b]{0.33\textwidth}
            \caption{Precipitation bias\\(mm \perday, \forcingICO)}
            \includegraphics[width=\textwidth]{images/chap4/forcing_source/diff_map_precip_ico_era.png}
        \end{subfigure} \\
        %evap
        \begin{subfigure}[b]{0.33\textwidth}
            \caption{ET bias\\(mm \perday, \forcingERA)}
            \includegraphics[width=\textwidth]{images/chap4/forcing_source/diff_map_evap_era_era.png}
        \end{subfigure} &
        \begin{subfigure}[b]{0.33\textwidth}
            \caption{ET bias\\(mm \perday, \forcingICO)}
            \includegraphics[width=\textwidth]{images/chap4/forcing_source/diff_map_evap_ico_era.png}
        \end{subfigure} \\
        %SWdn
        \begin{subfigure}[b]{0.33\textwidth}
            \caption{Downwelling SW flux bias\\(W \persqm, \forcingERA)}
            \includegraphics[width=\textwidth]{images/chap4/forcing_source/diff_map_SWdnSFC_era_era.png}
        \end{subfigure} &
        \begin{subfigure}[b]{0.33\textwidth}
            \caption{Downwelling SW flux bias\\(W \persqm, \forcingICO)}
            \includegraphics[width=\textwidth]{images/chap4/forcing_source/diff_map_SWdnSFC_ico_era.png}
        \end{subfigure}\\
        %LWdn
        \begin{subfigure}[b]{0.33\textwidth}
            \caption{Downwelling LW flux bias\\(W \persqm, \forcingERA)}
            \includegraphics[width=\textwidth]{images/chap4/forcing_source/diff_map_LWdnSFC_era_era.png}
        \end{subfigure} &
        \begin{subfigure}[b]{0.33\textwidth}
            \caption{Downwelling LW flux bias\\(W \persqm, \forcingICO)}
            \includegraphics[width=\textwidth]{images/chap4/forcing_source/diff_map_LWdnSFC_ico_era.png}
        \end{subfigure}
    \end{tabular}
    \caption{Annual biases (2010-2022) of precipitation, evapotranspiration, and downwelling radiative fluxes for the simulations \forcingERA and \forcingICO, compared to ERA5.}
    \label{fig:forcing_source_ERA_diff_maps_endvars}
\end{figure}

In the transition zone, the biases in precipitation and downwelling shortwave flux compared to ERA5 are much smaller in \forcingICO than in \forcingERA, and more clearly limited to the edges of the domain (Fig. \ref{fig:forcing_source_ERA_diff_maps_endvars}a, b, e, f). For these variables, as well as for the downwelling longwave flux (Fig. \ref{fig:forcing_source_ERA_diff_maps_endvars}g, h), the spatial pattern of the bias in \forcingICO is similar to the patterns previously seen in simulations with the intermediate or large domain (Fig. \ref{fig:domain_size_P_ET_ERA_diff_maps} \ref{fig:domain_size_clouds_ERA_diff_maps}), confirming the idea that the model is much more consistent when it is forced by ICOLMDZOR. There is no clear improvement of ET over the Atlantic ocean, but over the Mediterranean, the underestimation is reversed to a small overestimation, as previously seen in simulations with larger domains (Fig. \ref{fig:domain_size_P_ET_ERA_diff_maps}e, f).
All these findings confirm that the behaviour of the LAM is dependent on the source of the lateral boundary conditions. Forcing it with outputs from a global ICOLMDZOR simulation rather than ERA5 allows a more consistent behaviour of the condensation scheme in the transition zone, and avoids the spread of the biases from the edges of the domain to the central free zone.

\hfill

However, the global ICOLMDZOR simulation is not a reanalysis and is not expected to reproduced observed synoptic conditions like ERA5. Since the period analysed here remains short compared to climatic time scales (13 years), inter-annual variability can lead to large differences between reality and the simulation. Therefore, the comparison of the \forcingICO to observed conditions reveals new biases. Over the Iberian Peninsula, precipitation is clearly overestimated in northern mountainous areas (Fig. \ref{fig:forcing_source_ERA_diff_maps_endvars}b), while the larger low cloud cover affects downwelling radiative fluxes along the Atlantic coast, leading to an underestimation of the shortwave flux and an overestimation of the longwave flux (Fig. \ref{fig:forcing_source_ERA_diff_maps_endvars}f, h) which were not present in \forcingERA.
To analyse this further, precipitation and ET in the two simulation were respectively compared to the GPCC and GLEAM products over the Iberian Peninsula, from 2010 to 2019 (Fig. \ref{fig:forcing_source_SC}).

%figure : SC of precip and evap with GLEAM and GPCC
\begin{figure}[htbp]
    \centering
    %precip
    \begin{subfigure}[b]{0.49\textwidth}
        \caption{Seasonal cycle of precipitation (2010-2019)}
        \includegraphics[width=\textwidth]{images/chap4/forcing_source/IP_seasonal_cycle_precip.png}
    \end{subfigure}
    \begin{subfigure}[b]{0.49\textwidth}
        \caption{Seasonal cycle of ET (2010-2019)}
        \includegraphics[width=\textwidth]{images/chap4/forcing_source/IP_seasonal_cycle_evap.png}
    \end{subfigure}
    \caption{Mean seasonal cycle of precipitation and evapotranspiration, on average over the Iberian Peninsula, for the LAM forced by ERA (red) and forced by ICOLMDZOR (blue), and the GPCC and GLEAM products (2010-2019).}
    \label{fig:forcing_source_SC}
\end{figure}

When forced with ICOLMDZOR rather than ERA5, the LAM simulates similar amounts of precipitation in spring and larger precipitation from July to October (Fig. \ref{fig:forcing_source_SC}a) matching GPCC. This improvement is possible because the precipitation underestimation in the transition zone is largely reduced in \forcingICO compared to \forcingERA, and does not affect the central part of the domain. 
However, in winter, the LAM now strongly overestimates precipitation, mainly due to excessive snow and rainfall in mountainous areas, as visible in Fig. \ref{fig:forcing_source_ERA_diff_maps_endvars}b. This is a known bias of climate models and was already described for the LMDZ physics in particular \citep{arjdal_modeling_2024, adhikari_evaluation_2024}. 
Over this season, model performance may not be improved by changing the forcing source, but it has the advantage of being self-consistent over the domain, and of producing biases that have already been studied and analysed in the IPSL modelling community.

The increase in summer precipitation in \forcingICO lead to an increase in ET over the Peninsula (Fig. \ref{fig:forcing_source_SC}b), although a gap remains between the GLEAM product and the model. This dry biasmight derive from various causes, among which an underestimation of the downwelling shortwave radiation in some regions, the parametrization of evapotranspiration components in the LSM, and the absence of irrigation in the simulations. In other seasons, the difference between the two simulation remains limited. In particular, the increases in precipitation in winter do not lead to significant changes in ET since in the mountainous areas where they occur, precipitation is already large and ET is not limited by the available soil moisture.

\hfill

In summary, the comparison of the two LAM simulations, \forcingERA and \forcingICO, corroborated the hypothesis that the inconsistent behaviour of the LAM in the transition zone was due to discrepancies between the physics used to produce the ERA5 reanalysis and the LMDZ physics of the LAM. Indeed, biases on the edges of the domain are largely reduced when ICOLMDZOR outputs are used as forcing data, and their influence on the Iberian Peninsula almost disappears. However, although the LAM gains consistency, its performance compared to observation-based reference products may not be systematically improved, because the ICOLMDZOR model has its own biases, that can be present in the forcing and perpetuated by the LAM, as seen with the overestimation of precipitation over mountainous areas in winter.

\clearpage

\section{Impact of the forcing file sampling frequency}
\label{sec:forcing_frequency}
%figure : maps of diff vs ERA for 2 forcing sampling freqs
\begin{figure}[!h]
    \centering
    \begin{tabular}{cc}
        %precip
        \begin{subfigure}[b]{0.33\textwidth}
            \caption{Precipitation bias\\(mm \perday, \forcingoneh)}
            \includegraphics[width=\textwidth]{images/chap4/forcing_sampling_freq/diff_map_precip_lmdz1h_era.png}
        \end{subfigure} &
        \begin{subfigure}[b]{0.33\textwidth}
            \caption{Precipitation bias\\(mm \perday, \forcingsixh)}
            \includegraphics[width=\textwidth]{images/chap4/forcing_sampling_freq/diff_map_precip_lmdz6h_era.png}
        \end{subfigure} \\
        %evap
        \begin{subfigure}[b]{0.33\textwidth}
            \caption{ET bias\\(mm \perday, \forcingoneh)}
            \includegraphics[width=\textwidth]{images/chap4/forcing_sampling_freq/diff_map_evap_lmdz1h_era.png}
        \end{subfigure} &
        \begin{subfigure}[b]{0.33\textwidth}
            \caption{ET bias\\(mm \perday, \forcingsixh)}
            \includegraphics[width=\textwidth]{images/chap4/forcing_sampling_freq/diff_map_evap_lmdz6h_era.png}
        \end{subfigure} \\
        %SWdn
        \begin{subfigure}[b]{0.33\textwidth}
            \caption{Downwelling SW flux bias\\(W \persqm, \forcingoneh)}
            \includegraphics[width=\textwidth]{images/chap4/forcing_sampling_freq/diff_map_SWdnSFC_lmdz1h_era.png}
        \end{subfigure} &
        \begin{subfigure}[b]{0.33\textwidth}
            \caption{Downwelling SW flux bias\\(W \persqm, \forcingsixh)}
            \includegraphics[width=\textwidth]{images/chap4/forcing_sampling_freq/diff_map_SWdnSFC_lmdz6h_era.png}
        \end{subfigure}\\
        %LWdn
        \begin{subfigure}[b]{0.33\textwidth}
            \caption{Downwelling LW flux bias\\(W \persqm, \forcingoneh)}
            \includegraphics[width=\textwidth]{images/chap4/forcing_sampling_freq/diff_map_LWdnSFC_lmdz1h_era.png}
        \end{subfigure} &
        \begin{subfigure}[b]{0.33\textwidth}
            \caption{Downwelling LW flux bias\\(W \persqm, \forcingsixh)}
            \includegraphics[width=\textwidth]{images/chap4/forcing_sampling_freq/diff_map_LWdnSFC_lmdz6h_era.png}
        \end{subfigure} \\
    \end{tabular}
    \caption{Biases of precipitation, evapotranspiration, and downwelling radiative fluxes for the two simulations, with hourly and 6-hourly forcing data, compared to ERA (2013).}
    \label{fig:forcing_sampling_freq_ERA_diff_maps}
\end{figure}

For practical reasons, running long LAM simulations or simulations under future climate using simulation outputs in CMIP format as a source for the forcing data would require to use a forcing file with a 6-hours sampling frequency rather than one hour. 
Although the other simulations used in this manuscript were eventually all run with hourly forcing data, an experiment was conducted to assess the effects of the sampling frequency of the forcing file on the LAM. 
The results presented here compare two simulations with the intermediate domain size and lateral boundary conditions obtained from ERA5, with hourly data (\forcingoneh) and with 6-hourly data (\forcingsixh). Technical issues were encountered when trying to run with 6-hourly forcing data and only the year 2013 could be run without problems. Therefore, this single year was simulated ten times in a row in each simulation setup, constituting two small simulation ensembles.

\hfill

First, it is clear that the biases in the transition zone identified in Section \ref{sec:domain_size} for downwelling radiative fluxes are strengthened in \forcingsixh (Fig. \ref{fig:forcing_sampling_freq_ERA_diff_maps}f, h). 
Compared to \forcingoneh, shortwave flux is increased and longwave flux is decrease over the whole domain, and are associated to a reduced cloud cover (see supplementary Fig. \ref{fig:forcing_sampling_freq_ERA_diff_maps_appendix}).
In \forcingsixh, precipitation remains near-zero in the transition zone, and the bias seems confined to the edges (Fig. \ref{fig:forcing_sampling_freq_ERA_diff_maps}b). However, precipitation is largely underestimated in the central part of the domain, including continental areas in the north-west of the Peninsula, the Ebro valley, and near the Gibraltar straight. Considering the structure of the bias, this does not seem to be a direct consequence of the inconsistencies in the transition zone, but might still be an indirect consequence of the discrepancies between the physics used by ERA5 and LMDZ.
The underestimation of ET is generally strengthened in \forcingsixh than in \forcingoneh, which is not very well correlated with the changes found for radiative fluxes over the sea (Fig. \ref{fig:forcing_sampling_freq_ERA_diff_maps}d). This might be due to more complex feedback loops involving ET, atmospheric humidity, cloud cover, and therefore surface radiation, but it was not investigated further due to the scope of this study.
Over land, the underestimations of ET compare to ERA5 reflect those of precipitations but only to some extent, mostly in the Ebro valley and the western part of the domain.

On average over the Iberian Peninsula, the seasonal cycle of precipitation and ET have similar characteristics in both simulations, but \forcingsixh presents less precipitation all year, leading to a stronger underestimation in summer, autumn and winter compared to GPCC (Fig. \ref{fig:forcing_sampling_freq_SC}a). This lower amount of precipitation is logically associated to a lower ET in the seasons where it is limited by available soil moisture, which also increases the underestimation compared to GLEAM data (Fig. \ref{fig:forcing_sampling_freq_SC}b). 

%figure : SC of precip and evap with GLEAM and GPCC
\begin{figure}[htbp]
    \centering
    %precip
    \begin{subfigure}[b]{0.49\textwidth}
        \caption{Seasonal cycle of precipitation (2013)}
        \includegraphics[width=\textwidth]{images/chap4/forcing_sampling_freq/IP_seasonal_cycle_precip.png}
    \end{subfigure}
    \begin{subfigure}[b]{0.49\textwidth}
        \caption{Seasonal cycle of ET (2013)}
        \includegraphics[width=\textwidth]{images/chap4/forcing_sampling_freq/IP_seasonal_cycle_evap.png}
    \end{subfigure}
    \caption{Mean seasonal cycle of precipitation and evapotranspiration, on average over the Iberian Peninsula, for the LAM forced with hourly (blue) and 6-hourly (red) forcing data, and the GPCC and GLEAM products (2013).}
    \label{fig:forcing_sampling_freq_SC}
\end{figure}

Although this study only focused on a single year, it revealed that the LAM is sensitive to the sampling frequency of the ERA5 forcing file. 
To understand it, one must remember that a lower forcing file sampling frequency does not mean the model is less constrained: variables are still nudged at every time step of the dynamics (30 seconds), with the same intensity. However, with a lower sampling frequency, the diurnal cycle is not as well described in the forcing data, as illustrated in Fig. \ref{fig:diurnal_cycle_sampling}. 
For the variables that follow a marked diurnal cycle conditioned by radiative fluxes or the mixing of the lower atmosphere, the inconsistencies between the LMDZ physics and the ERA5 forcing data can therefore be increased and lead to more intense biases. 

\begin{figure}[ht]
    \centering
    \includegraphics[width=0.5\textwidth]{images/chap4/forcing_sampling_freq/sampling_freq_diurnal_cycle.png}    
    \caption{Illustration of the differences in diurnal cycle sampling using hourly (blue) or 6-hourly (red) forcing data. Adapted from Mariame Maiga's internship defence.
    \label{fig:diurnal_cycle_sampling}}
\end{figure}

\section{Chapter conclusions}

To identify an appropriate simulation setup for the study of the impacts of irrigation, the sensitivities of the LAM to the size of the domain and to the choice of lateral forcing (both its source and sampling frequency) were analysed. They revealed an inconsistent behaviour of the LAM in the transition zone when forced by ERA5 data, in particular unexpectedly low precipitation and cloud cover. These biases were attributed to discrepancies between the physics used to produce the ERA5 reanalysis and the LMDZ physics, which can keep the large-scale condensation scheme from forming clouds and precipitation properly.
This inconsistent behaviour in the transition zone was found to also have impacts in the central part of the domain, including over the Iberian Peninsula. Using larger domains does not make these inconsistencies disappear but limits their direct influence over the Peninsula since the transition zone is further away. More generally, when using lateral boundary conditions from ERA5, a dry bias (on precipitation and continental ET) still affects the whole simulation domain in all the configurations used, which might be a more indirect consequence of these discrepancies.

A simulation forced with outputs from a global ICOLMDZOR simulation was compared to the initial setup with forcing data from ERA5 and showed large improvements in the consistency of the model in the transition zone and a reduction of the precipitation and ET biases over the whole domain.
These findings confirmed that the model is sensitive to the source of the forcing data, and identified a possibly better forcing source for future studies with the LAM. However, with this setup, some inherent biases of the ICOLMDZOR model appear in the central zone (\textit{e.g.} excessive winter precipitation in mountains), which partly limits the improvements in performance over the Iberian Peninsula. 

Sensitivity experiments to look into the impact of the forcing file sampling frequency were also conducted. They showed that using 6-hourly ERA5 data for the lateral forcing instead of hourly data increases the biases of the radiative fluxes in the transition zone and the central free zone. Precipitation and ET are also affected although it remains difficult to interpret the structure of the biases and possible links to the inconsistencies of the transition zone.
Over the Iberian Peninsula, the performance is partly degraded for precipitation and ET, as the LAM displays an increased dry bias when using 6-hourly data. 

However, the limitations identified in this simplified evaluation do not discard LAM simulations with 6-hourly forcing data as a viable option for regional climate modelling.
As a reminder, the setup used for this analysis of the sampling frequency was limited to one simulation year (repeated multiple times), and different conclusions might be obtained in other experiments over a longer period, or with a different forcing source that does not generate such strong inconsistencies in the transition zone.
In his internship report, \citet{conesa2022} had analysed the sensitivity of the LAM (in an idealised setup with simplified physics) and found that hourly data performed very similarly to higher sampling frequency (down to 5 mn), that errors became significant for 3-hourly data but remained largely acceptable for 6-hourly data, but that performance was further degraded for 12-hourly data.
Since \citet{denis_sensitivity_2003}, 6-hourly lateral boundary conditions have generally been considered as a good compromise that can properly simulate regional climate \citep[enabling similar performance to 3-hourly data,][]{dimitrijevic_validation_2005}, while limiting the need for large data storage and high-frequency forcing data generation.
In practice, although this analysis was conducted with the idea that 6-hourly forcing data could be used for future climate simulations, this option was eventually discarded for technical reasons (frequent unexplained model crashes) and all simulations used in the rest of this manuscript were run using hourly forcing data.

\hfill

To summarise, despite several simplifications due to technical and time constraints in the context of Mariame Maiga's internship, several dependences of the ICOLMDZOR LAM on the choice of the domain and forcing data were identified. This work is of high interest for the IPSL modelling community since the LAM is a recent tool, still in development, that is expected to be used in more and more regional modelling and parametrization development studies.
The results would benefit from further consolidation with longer simulations to enable statistical analyses on climatic time scales, and from a study of mixed effects of domain size, forcing source and sampling frequency to identify the most appropriate setups for future regional climate simulations with the LAM.
The influence of the resolution of the forcing could also be investigated as is has long been identified as an important sensitivity of regional climate models \citep{leduc_regional_2009}. In this regard, the zooming capability of the global ICOLMDZOR model might prove very useful to generate forcings that match the resolution of the LAM over the transition zone. Additional research directions to increase the understanding and reliability of the LAM are detailed in Section \ref{sec:lam_perspectives}.

\chapter{Impacts of irrigation on the regional climate of the Iberian Peninsula}
\label{chap:monthly}
\minitoc
\pagebreak
% \section{Chapter introduction}

This chapter presents results of coupled regional climate simulations run with the ICOLMDZOR LAM to study the impacts of irrigation on land-atmosphere interactions and the water cycle over several years of simulation, with a focus on annual and seasonal means. 
The objective was to address the following questions:
\begin{itemize}
    \item Is simulated irrigation realistic in coupled simulations ? What are its limits and perspectives for improvements?
    \item What impacts does irrigation have on the water cycle and land-atmosphere interactions? Does irrigation improve the ability of the model to represent these interactions?
    \item At the scale of the Iberian Peninsula, is the impact of irrigation limited to intensely irrigated areas or can remote effects be identified as a consequence of atmospheric feedbacks?
    \item Under a high-end climate change scenario, how do the regional impacts of irrigation interact with those of climate change?
\end{itemize}


%todo: evaluate need to mention this again 
All the results presented in this chapter are derived from simulation setups described in Chapter \ref{chap:methods}. Their key characteristics are summarised in Table \ref{table:coupled_simulations_chap5}. The routing and irrigation schemes were used with the parameter set identified after the offline calibration presented in Chapter \ref{chap:routing}, and the MERIT DEM at 1-arcminute resolution.

%table:all coupled simulation details
%todo:add SST/SIC ?
\begin{table}[htbp]
    \centering
    \resizebox{\textwidth}{!}{
        \begin{tabular}{|l|l|l|l|l|l|l|l|}
            \hline
            \textbf{Section} & \textbf{Simulation name} & \textbf{Period} & \textbf{Diameter} & \textbf{NBP} & \textbf{Forcing} & \textbf{Irrigation} \\
            \hline
            \ref{sec:article1} & \irr & 2010-2022 & 1500 km & 60 & ERA5, 1h & Yes \\
            \hline
            \ref{sec:article1} & \noirr & 2010-2022 & 1500 km & 60 & ERA5, 1h & No \\
            \hline
            \ref{sec:climate_change} & \presnoirr & 2010-2022 & 1000 km & 40 & ICOLMDZOR, 1h & No \\
            \hline
            \ref{sec:climate_change} & \futnoirr & 2050-2062 & 1000 km & 40 & ICOLMDZOR, 1h & No \\
            \hline
            \ref{sec:climate_change} & \futirr & 2050-2062 & 1000 km & 40 & ICOLMDZOR, 1h & Yes \\
            \hline
        \end{tabular}
    }
    \caption{Characteristics of coupled simulations used in this chapter.}
    \label{table:coupled_simulations_chap5}
\end{table}

Section \ref{sec:article1} presents the impacts of irrigation on land-atmosphere interactions and the water cycle, under the present climate of the Iberian Peninsula. These results were presented in an article currently under review in Earth System Dynamics (\url{https://egusphere.copernicus.org/preprints/2025/egusphere-2025-2491/})
Many elements were reused and adapted, but the article was not included as a whole in this chapter to avoid redundancies with Chapters \ref{chap:introduction} and \ref{chap:methods} and enable a better integration with additional results under climate change and Chapters \ref{chap:forcing} and \ref{chap:liaise}. 
Section \ref{sec:climate_change} presents preliminary results obtained from coupled simulations of future climate under a strong climate change scenario (SSP5-8.5), highlighting the impacts of climate change over the Iberian Peninsula, and how they interact with irrigation.

\clearpage

\section{Impacts of irrigation under present climate}
\label{sec:article1}
\input{5.article_results}
% \clearpage

\section{The Iberian Peninsula and impacts of irrigation under climate change}
\label{sec:climate_change}
This section is based on the results of Mariame Maiga's internship for her 1st year of Masters at Sorbonne Université, which I co-supervised with Frédérique Cheruy from April to July 2025 \citep{maiga2025}. The impact of climate change was studied during the internship but not the effects of irrigation in the future, which was analysed later. All simulations used in this section were run by Frédérique Cheruy. 
Multiple technical issues were encountered to obtain reliable simulations of future climate with and without irrigation, therefore these results remain preliminary.

\hfill

To simulate future climate conditions, the LAM is used with forcing data from a global ICOLMDZOR simulation under climate change scenario SSP5-8.5, which presents the strongest increase in global mean temperature. 
Sea-surface temperature (SST) and sea ice concentration (SIC) for this global simulation were prescribed from a dataset constructed for CMIP6 by applying bias and variance correction \citep{beaumet_assessing_2019} to IPSL-CM6 fully coupled simulations with SSP5-8.5 radiative forcing. 
However, these global simulation did not account for dynamical evolutions of aerosols. 
Directly using CMIP6 outputs as lateral boundary conditions was considered but many model crashes were encountered, which are likely due to the very coarse resolution of CMIP6 outputs (compared to that of the LAM), and possibly to the 6-hourly sampling frequency, as mentioned in Chapter \ref{chap:forcing}.
The global ICOLMDZOR simulation provides hourly outputs and is zoomed to match the resolution of the LAM over its domain.

The mid-century period 2050-2062 was studied and compared to the present period (2010-2022). Longer simulations are under analysis to extend the study period to 30 years for both present and future but they were not yet available at the time of writing this manuscript.
The smaller LAM domain ($R_{domain} = 1000$ km, $NBP=40$) was used for computational efficiency, considering the findings of Chapter \ref{chap:forcing} that the inconsistencies in the transition zone are limited when lateral boundary conditions are obtained from ICOLMDZOR.
Simulations without irrigation in the present (\presnoirr) and future climate (\futnoirr) are compared to characterise the impacts of climate change over the Peninsula. Then, \futnoirr is compared to a simulation over the same period, with irrigation (\futirr) to analyse how irrigation interacts with these impacts.
In \futirr, irrigated fractions were kept similar to present conditions, using the last available map in the HID map at 5 arc-min resolution \citep{siebert_quantifying_2010}, correponding to the year 2012. This assumption is not expected to largely affect the results as the Iberian Peninsula is already a very anthropised region, more likely to undergo changes in water management policy and irrigation methods than in the spatial distribution of irrigated crops.

\subsection{Impact of climate change over the Iberian Peninsula}

The annual impact of climate change under the SSP5-8.5 scenario on land-atmosphere coupling variables is presented in Fig. \ref{fig:diffmaps_present_future}, as the difference between the \futnoirr (2050-2062) and \presnoirr (2010-2022) simulations. 
%temperature
A general warming of the air is observed, as expected with a strong climate change scenario, with 2-metre temperature increases of 2°C in most of the domain and up to 3°C in the northern half of the Peninsula (Fig. \ref{fig:diffmaps_present_future}a). The coasts are slightly less affected by warming, whereas the mountain ranges are the most impacted regions.
%precip

Precipitation decreases over most of the Peninsula, especially in the northern coast and the Pyrenees while the only region where it increases is the Southeast (Fig. \ref{fig:diffmaps_present_future}c). This region usually receives little precipitation so, although the absolute values are small, the relative increase is higher than 10\% over a large area and exceeds 25\% in several grid cells (see supplementary Fig. \ref{fig:reldiffmaps_present_future}). 
%evap and fluxsens
The changes in precipitation translate into similar changes in ET, of lower magnitude (Fig. \ref{fig:diffmaps_present_future}d). The only exceptions are in  Pyrenees and Cantabrian mountain ranges in the North, which are very humid areas where ET is limited by incoming radiation rather than available soil moisture.
These general decreases in precipitation and ET are consistent with CMIP6 projections for the northern Mediterranean basin under climate change, which are attributed to the combined influence of local responses to atmospheric warming and alterations in large-scale circulation \cite{tuel_understanding_2021, arjdal_future_2023}.

%figure : diff maps (no_irr, present - future)
\begin{figure}[ht!]
    \centering
    \begin{tabular}{cc}
        %t2m
        \begin{subfigure}[b]{0.5\textwidth}
            \caption{2-metre temperature difference}
            \includegraphics[width=\textwidth]{images/chap4/future/diffmap_t2m_presfut.png}
        \end{subfigure} &
        %fluxsens
        \begin{subfigure}[b]{0.5\textwidth}
            \caption{Sensible heat flux difference}
            \includegraphics[width=\textwidth]{images/chap4/future/diffmap_fluxsens_presfut.png}
        \end{subfigure} \\
        
        %precip
        \begin{subfigure}[b]{0.5\textwidth}
            \caption{Precipitation difference}
            \includegraphics[width=\textwidth]{images/chap4/future/diffmap_precip_presfut.png}
        \end{subfigure} &
        %evap
        \begin{subfigure}[b]{0.5\textwidth}
            \caption{ET difference}
            \includegraphics[width=\textwidth]{images/chap4/future/diffmap_evap_presfut.png}
        \end{subfigure} \\
        %q2m
        \begin{subfigure}[b]{0.5\textwidth}
            \caption{2-metre specific humidity difference}
            \includegraphics[width=\textwidth]{images/chap4/future/diffmap_q2m_presfut.png}
        \end{subfigure} &
        %rh2m
        \begin{subfigure}[b]{0.5\textwidth}
            \caption{2-metre relative humidity difference}
            \includegraphics[width=\textwidth]{images/chap4/future/diffmap_rh2m_presfut.png}
        \end{subfigure} \\
        %SWdnSFC
        \begin{subfigure}[b]{0.5\textwidth}
            \caption{Downwelling shortwave radiation difference}
            \includegraphics[width=\textwidth]{images/chap4/future/diffmap_SWdnSFC_presfut.png}
        \end{subfigure} &
        %LWdnSFC
        \begin{subfigure}[b]{0.5\textwidth}
            \caption{Downwelling longwave radiation difference}
            \includegraphics[width=\textwidth]{images/chap4/future/diffmap_LWdnSFC_presfut.png}
        \end{subfigure}
    \end{tabular}
    \caption{Impacts of climate change over the Iberian Peninsula. Annual mean difference between \futnoirr (2050-2062) and  \presnoirr (2010-2022).}
    \label{fig:diffmaps_present_future}
\end{figure}

The sensible heat flux is increased over almost all the continental domain  (Fig. \ref{fig:diffmaps_present_future}b). This evolution is not only a consequence of changes in surface energy partitioning since increases are also visible in areas where ET is locally increasing. It is likely dominated by an increase in the incoming energy at the surface, since it closely linked to increases in downwelling shortwave radiation (Fig. \ref{fig:diffmaps_present_future}g).
%q2m and RH2m
Near-surface specific humidity is increased over the whole domain, with the largest increases reaching 1 g \perkg on coastal areas (Fig. \ref{fig:diffmaps_present_future}e), which corresponds to a 10\% increase. Considering the increase in 2-metre air temperature, this can be understood using the Clausius-Clapeyron relation, showing that warmer air can contain more water \citep[in theory up to +7\% for a 1°C increase, Chapter 8 in][]{IPCC_2023}.
However, under climate change, the increase in specific humidity is not as impactful as the increase in temperature since 2-metre relative humidity is lower over all the continental domain (Fig. \ref{fig:diffmaps_present_future}f), especially in the north of the Peninsula. 
%link to rad fluxes and cloud cover
This extends to the rest of the atmospheric column and leads to smaller cloud cover (see supplementary Fig. \ref{fig:clouds_present_future}), which is reflected by the increase in the downwelling shortwave radiation flux (Fig. \ref{fig:diffmaps_present_future}g), particularly in the north-west of the Peninsula. 
The general increase in longwave downwelling radiation (Fig. \ref{fig:diffmaps_present_future}h) is an expected consequence of a warmer atmosphere, but this increase is not as large in the north and north-west of the domain, where changes in relative humidity (and cloud cover) are the largest . 

% \clearpage

\subsection{Impacts of irrigation under climate change}
%todo: FC Est ce que tu peux comparer  l'intensité du recyclage induit par l irrigation, local et remote, en climat modifié à celui analysé pour le climat récent? 

%figure : map and SC of irrigation in the future
\begin{figure}[htbp]
    \centering
    \begin{tabular}{cc}
        %precip
        \begin{subfigure}[b]{0.48\textwidth}
            \caption{Irrigation annual mean}
            \includegraphics[width=\textwidth]{images/chap4/future/map_irrigation_fut.png}
        \end{subfigure} &
        \begin{subfigure}[b]{0.46\textwidth}
            \caption{Irrigation mean seasonal cycle}
            \includegraphics[width=\textwidth]{images/chap4/future/SC_irrigation_fut.png}
        \end{subfigure}
    \end{tabular}
    \caption{Annual mean and seasonal cycle of irrigation over the Iberian Peninsula in the \futirr simulation (2050-2062).}
    \label{fig:future_irrig}
\end{figure}

Irrigation in the \futirr simulation is largest in the valleys of the Ebro, Douro, Tagus, Guadiana and Guadalquivir rivers (Fig. \ref{fig:future_irrig}a). Due to modelling choices, it is near-zero in winter, and follows a rather symmetrical seasonal cycle from spring to autumn, with a peak in July (Fig. \ref{fig:future_irrig}b).
It is important to note that this simulation cannot be directly compared to the simulation of irrigation in present conditions presented in Section \ref{sec:article1}, for two reasons. First, the simulation setup is different since a smaller domain and lateral boundary conditions from ICOLMDZOR are used, instead of the intermediate domain and ERA5 data. Secondly, a technical issue with the routing scheme occurred when running this simulation with irrigation under future climate. A different setting was inappropriately used, which alters the routing scheme parameters determined in Chapter \ref{chap:routing}, dividing them by 1000. This leads to irrelevant values for river discharge (not shown here) and very large volumes of water available in the routing reservoirs. In practice, that means that irrigation was mostly not constrained by available water in this simulation, and that irrigation demand was systematically met, maintaining soil moisture at 60\% of field capacity in irrigated areas.
To some extent, this setup can be considered as a sensitivity experiment representing a situation where irrigation infrastructure can provide virtually infinite amounts of water. Also, this likely circumvents the limitations of the \textit{normal} irrigation scheme setup identified in Section \ref{sec:article1} for southern regions, where irrigation was highly underestimated. Therefore, this simulation was still used for a preliminary study of atmospheric feedbacks of irrigation.

%figure : diff maps (future, irr - no_irr)
\begin{figure}[hb!]
    \centering
    \begin{tabular}{cc}
        %t2m
        \begin{subfigure}[b]{0.5\textwidth}
            \caption{2-metre temperature difference}
            \includegraphics[width=\textwidth]{images/chap4/future/diffmap_t2m_futirr.png}
        \end{subfigure} &
        %fluxsens
        \begin{subfigure}[b]{0.5\textwidth}
            \caption{Sensible heat flux difference}
            \includegraphics[width=\textwidth]{images/chap4/future/diffmap_fluxsens_futirr.png}
        \end{subfigure} \\
        %precip
        \begin{subfigure}[b]{0.5\textwidth}
            \caption{Precipitation difference}
            \includegraphics[width=\textwidth]{images/chap4/future/diffmap_precip_futirr.png}
        \end{subfigure} &
        %evap
        \begin{subfigure}[b]{0.5\textwidth}
            \caption{ET difference}
            \includegraphics[width=\textwidth]{images/chap4/future/diffmap_evap_futirr.png}
        \end{subfigure} \\
        %q2m
        \begin{subfigure}[b]{0.5\textwidth}
            \caption{2-metre specific humidity difference}
            \includegraphics[width=\textwidth]{images/chap4/future/diffmap_q2m_futirr.png}
        \end{subfigure} &
        %rh2m
        \begin{subfigure}[b]{0.5\textwidth}
            \caption{2-metre relative humidity difference}
            \includegraphics[width=\textwidth]{images/chap4/future/diffmap_rh2m_futirr.png}
        \end{subfigure} \\
        %pblh
        \begin{subfigure}[b]{0.5\textwidth}
            \caption{ABL height difference}
            \includegraphics[width=\textwidth]{images/chap4/future/diffmap_s_pblh_futirr.png}
        \end{subfigure} &
        %lcl
        \begin{subfigure}[b]{0.5\textwidth}
            \caption{Lifting condensation level difference}
            \includegraphics[width=\textwidth]{images/chap4/future/diffmap_s_lcl_futirr.png}
        \end{subfigure}
    \end{tabular}
    \caption{Impacts of irrigation under climate change. \\Annual mean difference (\futirr - \futnoirr, 2050-2062).}
    \label{fig:diffmaps_future_irr}
\end{figure}

%turbulent fluxes
Compared to the \futnoirr simulation, the most direct impact of irrigation is an increase in ET of similar amplitude to the irrigation amounts (Fig. \ref{fig:diffmaps_future_irr}d), which largely compensates for the decrease in ET induced by climate change. This increase of the latent heat flux is directly linked to a decrease of the sensible heat flux (Fig. \ref{fig:diffmaps_future_irr}b) reaching -15 W \persqm in the most irrigated valleys. 
%t2m
This new partitioning of surface energy leads to a cooler air over most of the domain (Fig. \ref{fig:diffmaps_future_irr}a). The largest decreases occur  over the intensely irrigated areas where 2-metre temperature is lowered by 0.5°C on annual average and 1°C in summer (see supplementary Fig. \ref{fig:diffmaps_JJA_future_irr} for JJA impacts). 
Contrary to ET, this consequence of irrigation is insufficient to compensate the impact of climate change, which induces 2-metre temperature increases of 2°C annually and larger than 3°C in summer (see supplementary Fig. \ref{fig:SC_future_irr} for a mean seasonal cycle over the whole domain).
Regarding humidity, \futirr is moister than \futnoirr (Fig. \ref{fig:diffmaps_future_irr}e) but some intensely irrigated areas such as the Ebro valley show smaller increases in specific humidity than others, showing that the impacts on humidity are not a local as those on temperature. Combined with the changes in temperature, the increase of specific humidity also leads to an increase in relative humidity  (Fig. \ref{fig:diffmaps_future_irr}f), which opposes the impact of climate change, without reverting it.
%ABL
The structure of the boundary layer is also affected by irrigation, as seen in Section \ref{sec:article1}. Stabilization dominates and follows the same pattern as irrigation, as a consequence of cooling and of the reduced sensible heat flux. The ABL height is lowered by more than a 100 m in the most intensely irrigated grid cells. 
Due to the moistening and cooling of the atmosphere, the lifting condensation level is also lowered. Similarly to the changes in 2-metre humidity, LCL is more affected in the Tagus, Guadiana and Guadalquivir basins than in the Ebro and Douro basins.
As explained in Section \ref{sec:article1}, the lowering of both the ABL height and LCL point to opposite effects on cloud formation and precipitation, since one reflects a more stable atmosphere less likely to form clouds through convection, while the other is associated with a moister and cooler atmosphere where condensation is more likely. As a consequence, the impact of irrigation on precipitation is rather limited and the only structures that emerge are precipitation increases in the highest mountain ranges (the Baetic, Central, Iberian Systems and the Pyrenees), and a decrease in the northern Ebro valley (more clearly visible by looking at relative changes in Fig. \ref{fig:reldiffmaps_future_irr}c).

%IP average if useful 
%precip
% present, no_irr : 2.18771 (mm d⁻¹)
% future, no_irr : 1.86688 (mm d⁻¹)
% future, irr : 1.89212 (mm d⁻¹)
%ET
% present, no_irr : 1.27330 (mm d⁻¹)
% future, no_irr : 1.17734 (mm d⁻¹)
% future, irr : 1.44247 (mm d⁻¹)
%t2m
% present, no_irr : 14.66318 (°C)
% future, no_irr : 16.98357 (°C)
% future, irr : 16.80824 (°C)
%sensible
% present, no_irr : 50.54140 (W m⁻²)
% future, no_irr : 55.53161 (W m⁻²)
% future, irr : 49.91047 (W m⁻²)
%q2m
% present, no_irr : 0.00730 (kg kg⁻¹)
% future, no_irr : 0.00786 (kg kg⁻¹)
% future, irr : 0.00813 (kg kg⁻¹)
%pblh
% present, no_irr : 942.46790 (m)
% future, no_irr : 946.90222 (m)
% future, irr : 909.96997 (m)
%lcl
% present, no_irr : 884.01685 (m)
% future, no_irr : 1039.27222 (m)
% future, irr : 959.86011 (m)
%rh2m
% present, no_irr : 68.83486 (%)
% future, no_irr : 65.23708 (%)
% future, irr : 66.66062 (%)

% \clearpage

\subsection{Aridification and impact of irrigation}

Aridity indices are often used as a way to characterise continental regions and how their climate will evolve in the future, under climate change. Aridity aims to describe long-term moisture deficits that limit the development of vegetation. Several definitions and indices have been proposed and used in the last decades, but the most common is the United Nations Environmental Programme aridity index \citep{unep1992}, defined as the ratio of the annual precipitation to potential evapotranspiration : $AI = \frac{P}{PET}$.
This index is most relevant when averaged over multiple years to reduce sensitivity to specific droughts or extreme precipitation events.

\begin{table}[h]
    \centering
    \begin{tabular}{|l|l|}
        \hline
        \textbf{Aridity class} & \textbf{Index values} \\
        \hline
        Hyper-arid & \( AI < 0.05 \) \\
        \hline
        Arid & \( 0.05 < AI < 0.2 \) \\
        \hline
        Semi-arid & \( 0.2 < AI < 0.5 \) \\
        \hline
        Dry sub-humid & \( 0.5 < AI < 0.65 \) \\
        \hline
        Humid & \( 0.65 < AI \) \\
        \hline
    \end{tabular}
    \caption{Aridity classes based on UNEP aridity index values.}
    \label{table:aridity_classes}
\end{table}

At the end of Mariame Maiga's internship, a short study of the evolution of aridity under climate change was conducted, using this index to classify grid cells into five aridity classes as presented in Table \ref{table:aridity_classes}. 
Potential evapotranspiration was computed using the Penman-Monteith equation, matching the formulation used to compute aridity indices with observation data over the Iberian Peninsula in \citet{begueria_aridity_2025}.
%taken from ORCHIDEE's \textit{evapot\_corr} variable, which is computed by applying a correction from \citet{milly_potential_1992} to the formulation given for $E_{pot}$ in Chapter \ref{chap:methods}.

\hfill

%figure : maps of aridity index classes for present and future
\begin{figure}[htbp]
    \centering
    %pres, no_irr
    \begin{subfigure}[b]{0.45\textwidth}
        \caption{Aridity index in \presnoirr}
        \includegraphics[width=\textwidth]{images/chap4/future/map_AI_pres_noirr.png}
    \end{subfigure}
    \begin{subfigure}[b]{0.54\textwidth}
        \caption{Aridity index classes in \presnoirr}
        \includegraphics[width=\textwidth]{images/chap4/future/aridity_index_pres_noirr.png}
    \end{subfigure} \\

    \vspace{0.2cm}

    %future, noirr
    \begin{subfigure}[b]{0.45\textwidth}
        \caption{Aridity index in \futnoirr}
        \includegraphics[width=\textwidth]{images/chap4/future/map_AI_fut_noirr.png}
    \end{subfigure}
    \begin{subfigure}[b]{0.54\textwidth}
        \caption{Aridity index classes in \futnoirr}
        \includegraphics[width=\textwidth]{images/chap4/future/aridity_index_fut_noirr.png}
    \end{subfigure} \\

    \vspace{0.2cm}

    %future, irr
    \begin{subfigure}[b]{0.45\textwidth}
        \caption{Aridity index in \futirr}
        \includegraphics[width=\textwidth]{images/chap4/future/map_AI_fut_irr.png}
    \end{subfigure}
    \begin{subfigure}[b]{0.54\textwidth}
        \caption{Aridity index classes in \futirr}
        \includegraphics[width=\textwidth]{images/chap4/future/aridity_index_fut_irr.png}
    \end{subfigure}\\
    
    \vspace{1cm}

    \begin{subfigure}[b]{0.31\textwidth}
        \caption{Share of aridity index classes in \presnoirr}
        \includegraphics[width=\textwidth]{images/chap4/future/aridity_index_distribution_pres_noirr.png}
    \end{subfigure}
    \begin{subfigure}[b]{0.31\textwidth}
        \caption{Share of aridity index classes in \futnoirr}
        \includegraphics[width=\textwidth]{images/chap4/future/aridity_index_distribution_fut_noirr.png}
    \end{subfigure}
    \begin{subfigure}[b]{0.31\textwidth}
        \caption{Share of aridity index classes in \futirr}
        \includegraphics[width=\textwidth]{images/chap4/future/aridity_index_distribution_fut_irr.png}
    \end{subfigure}
    \caption{Spatial distribution of aridity index values and classes in the \presnoirr (2010-2022), \futnoirr (2050-2062), and \futirr simulations (2050-2062).}
    \label{fig:aridity_index_v2}
\end{figure}


The spatial distribution in the \presnoirr simulation (Fig. \ref{fig:aridity_index_v2}a, b) shows a dominant semi-arid climate over the Iberian Peninsula, with a humid strip in the northern coast and the Pyrenees, and arid climate in the Ebro and Guadalquivir valleys, and other areas in the South.
Index values are much more arid than the results obtained from observations for the period 1991-2020 in \citet{begueria_aridity_2025}, which did not show any arid area and a larger extent of the dry sub-humid areas. Over 1961-2017, \citep{paniagua_aridity_2019} found a similar proportion of semi-arid areas but no arid ones and a larger extent of humid and subhumid, and confirmed this distribution using another index \citep{DeMartonne1925}.
The differences might be explained by the more recent period considered since climatic changes are already observed in the Iberian Peninsula \citep[][documents an aridification trend since 1961 that accelerates in the 80s for about half of the Peninsula]{paniagua_aridity_2019}.
Furthermore, the way to compute the 2-metre variables used to obtain PET and the more generic biases of the model can have a strong impact on the value of the aridity index in ICOLMDZOR and on the classification of each grid cell. However, this simple index remains an interesting tool to study the future evolution of aridity in the various regions of the Peninsula by comparing simulations to one another.

Under the SSP5-8.5 climate change scenario, in \futnoirr (Fig. \ref{fig:aridity_index_v2}c, d, h), the share of humid grid cells falls from 19.4\% to 11.0\%, the shares of semi-arid and dry sub-humid cells are reduced by respectively 1 and 2 percentage points, while the share or arid grid cells is more than doubled, from 8.0\% to 19.9\%. This corresponds to a shift in aridity index, that affects each aridity class. Most humid areas change to dry sub-humid or even directly to semi-arid, leaving only the heart of the Pyrenees and of the north-western mountain ranges in this aridity class. The arid areas, on the contrary, extend to encompass a large share of the Ebro, Guadiana and Guadalquivir valleys. This shift in aridity is consistent with the changes described in Fig. \ref{fig:diffmaps_present_future}, a decrease of precipitation and an increase in temperature (which partly drives PET), and is also described in \citet{fonseca_agricultural_2022} for water dependent regions like the Ebro and Guadalquivir basins.

The influence of irrigation on this evolution under climate change remains limited (Fig. \ref{fig:aridity_index_v2}). The share of arid grid cells is slightly lower in \futirr (16.2\%) than \futnoirr (19.9\%), which is mostly compensated by a larger share of semi-arid grid cells (65.3\% instead of 62.9\%) whereas the share of humid cells increase by one percentage point. Irrigation has an effect on the aridity index through the moistening and cooling of the lower atmosphere, which can decrease PET, and through the changes in precipitation. As described in Fig. \ref{fig:diffmaps_future_irr}, the cooling induced by irrigation is rather small compared to the warming induced by climate change, and the increases in precipitation are mostly located in the elevated areas. This explains why irrigation is not able to limit the expansion of the arid areas in the valleys under climate change.
The results presented here are clearly dependent on the the choice of the aridity index and on the biases of the model for near-surface variables and precipitation, which would require further investigation. In practical terms, they still suggest that although irrigation is an adaptation strategy which can help sustaining plant growth in the valleys, it has very limited long-term benefits on the local and regional climate over cultivated areas.

% present_noirr
%Arid: 74 grid cells (8.034744842562432)
%Semiarid: 589 grid cells (63.952225841476654)
%Dry subhumid: 79 grid cells (8.577633007600435)
%Humid: 179 grid cells (19.435396308360477)

% future_noirr
%Arid: 183 grid cells (19.86970684039088)
%Semiarid: 579 grid cells (62.866449511400646)
%Dry subhumid: 58 grid cells (6.297502714440825)
%Humid: 101 grid cells (10.966340933767643)

% future_irr
%Arid: 150 grid cells (16.286644951140065)
%Semiarid: 601 grid cells (65.25515743756786)
%Dry subhumid: 59 grid cells (6.406080347448426)
%Humid: 111 grid cells (12.052117263843648)

% \clearpage

\section{Chapter conclusions}

This chapter presented results from coupled simulations with the ICOLMDZOR LAM over the Iberian Peninsula to study the impacts of irrigation on the regional climate and the water cycle, through land-atmosphere interactions. 

\hfill

The impacts of irrigation on land surface-atmosphere coupling variables and the water cycle over the Iberian Peninsula were first studied in the present climate. Two simulations were run, with and without irrigation, using hourly ERA5 forcing data, which was the only option technically available at the time, and the intermediate domain identified in Chapter \ref{chap:forcing} as a good compromise to limit biases over the Iberian Peninsula.

This study first showed that the ORCHIDEE irrigation scheme simulates realistic values from April to September in areas where surface water withdrawals are most important, such as the Ebro Valley. However, it cannot represent winter irrigation, and is enable to satisfy irrigation demand in southern regions, where actual irrigation is more dependent on groundwater pumping and river dams, due to low available volumes in rivers and groundwater routing reservoirs. Ongoing developments to add river dams into the ORCHIDEE routing scheme \citep{baratgin_modeling_2024} could very likely improve this aspect by representing interseasonal water storage, making more water available in summer.
Explicit dam representation could also limit the winter and spring overestimations of river discharge in anthropised areas, since water would be stored in the dam reservoirs during this season instead of flowing in the rivers. 
Overall, the irrigation parametrization reduces river discharge and enables better agreement with observations, but since it is only active when the LAI is above a defined threshold, these impacts are mostly visible in summer and autumn. Future work with a looser activation threshold for irrigation could help to represent winter crop irrigation, although it is not expected to have as significant an impact on discharge as an explicit dam representation since simulated irrigation demand would still remain low in winter. Nevertheless, precipitation biases are very likely to remain a major driver of discharge biases, largely independent of irrigation or dam representation.

The simulation of precipitation and ET over the Iberian Peninsula is satisfactory in winter and spring, but this study highlighted a large underestimation in summer and contrasted spatial patterns with positive precipitation biases in elevated regions and negative biases in plains. ET underestimation is partly improved by simulated irrigation, but remains present on average and over most of the domain. 
It was only a hypothesis at the time, but it is now clear that running simulations with ICOLMZOR output as forcing data would improve the underestimation of summer precipitation and ET, but also degrade the performance in winter river discharge due to excessive precipitation in mountainous areas.
% These linked biases might be improved with a different simulation setup, particularly in the lateral forcing. Preliminary analyses (not shown) revealed an abnormal behaviour of the model in the transition zone between the ERA5 forcing zone and the central free zone, which was attributed to discrepancies between the physics used in the model and in the reanalysis. This resulted in precipitation underestimations throughout the entire simulation domain, which were largely improved by using a larger domain for the simulations presented here. A good lead for future works would be to use lateral forcing from global simulations of the ICOLMDZOR model or nested LAM simulations rather than a reanalysis, but these options are not yet technically available.
To improve these biases would likely require more work in ICOLMDZOR, regarding the modelling of radiative processes, shallow and deep convection (the tuning of which often focuses on tropical regions), surface processes (roughness, albedo, components of ET), and the modelling of snow. This highlighted the fact that the results of this study are necessarily limited by the modelling choices, uncertainties, and biases of the ICOLMDZOR LAM, and therefore remain largely model-specific.

The atmospheric impacts of irrigation were analysed in detail in summer, since it is the season with the largest irrigation values and the most significant response for all variables of interest, although it is the driest season, with very little precipitation. 
In JJA, the strong response of turbulent fluxes to irrigation leads to cooling and moistening of the lower atmosphere and significantly affects its structure (LCL and ABL height), with stronger effects on intensely irrigated regions, which is consistent with the findings of \citet{rappin_landatmosphere_2022}. In contrast, significant increases in precipitation are mostly detected in lightly irrigated mountainous areas surrounding the highly irrigated Ebro Valley. This points to a dominant effect of ABL stabilization, described by \citet{findell_atmospheric_2003-1, ek_influence_2004}, in intensely irrigated areas, and remote effects of atmospheric moistening as in \citet{deangelis_evidence_2010, lo_irrigation_2013, yang_impact_2017}. 
An improved representation of winter and spring irrigation could either allow to generalise the following results or to identify different responses to irrigation under moister atmospheric conditions.
Furthermore, over the Iberian Peninsula, increases in ET are proportional to applied irrigation and actually exceed it for almost every simulation month. This is made possible by small but systematic increases in average precipitation over the domain, forming evidence of continental moisture recycling over the Iberian Peninsula. The precipitation increases are of lower magnitude than those of ET and occur much more in lightly irrigated regions than in intensely irrigated regions, confirming that the recycling is partial and mostly non-local.

These findings called for an analysis of surface-atmosphere coupling processes in the presence of irrigation at the diurnal scale to better describe the impacts on the ABL structure in both irrigated areas and neighbouring regions.
Therefore, the LAM was compared to field observations from the LIAISE campaign, held in the Ebro Valley in July 2021 \citep{boone_land_2025}, and to mesoscale simulations which had already been run over the campaign area and analysed in \citet{lunel_irrigation_2024,lunel_marinada_2024}. These results are presented in Chapter \ref{chap:liaise}.
%High resolution modelling experiments using irrigation parametrizations have shown large improvements of performance relative to LIAISE observations for turbulent fluxes, air temperature and humidity \citep{lunel_irrigation_2024, udina_irrigation_2024}; stressed the importance of the convection parametrization for the response of precipitation to irrigation \citep{udina_irrigation_2024}; and identified interactions of irrigation-induced heterogeneities with regional breeze circulations \citep{lunel_marinada_2024}. Conducting similar analyses with the simulation setup used in this study should provide insights into the ability of an ESM to reproduce the complex structure of these heterogeneities \citep{mangan_surface-boundary_2023} and their impacts on the ABL and atmospheric water cycle. 


\hfill

Following this study of regional climate under present conditions, another experiment was set up to study future climate conditions over the Iberian Peninsula, in the context of Mariame Maiga's internship.
In this setup, hourly lateral boundary conditions for the LAM are obtained from a global ICOLMDZOR simulation under SSP5-8.5 climate change scenario. 
One simulation was run for the present period (2010-2022), without irrigation, and two for the future period (2050-2062), with and without irrigation. This allowed for an analysis of the impacts of climate change on the region, and of the behaviour and impact of irrigation in the future. 
Although some new simulations are needed (and ongoing) to strengthen the preliminary results presented in Section \ref{sec:climate_change}, several conclusions were drawn from this study. 

Under SSP5-8.5, mid-century simulated climate is 2.3°C warmer than present climate on average over the Iberian Peninsula. Although specific humidity increases due to the water-holding capacity of the air, relative humidity decreases over all the domain, particularly in the northern mountain ranges. Cloud cover and precipitation decrease over most of the domain, with the exception of a coastal region in the Southeast, leading to a decrease in latent heat flux and an increase in sensible heat flux. Using the UNEP aridity index, a clear shift in the aridity was identified over the Peninsula, with humid regions in the North evolving towards a semi-arid climate, and the more arid areas in the valleys extending to cover large share of the Ebro, Guadiana and Guadalquivir basins.

The simulation of future climate with irrigation showed that irrigation can counteract the effect of climate change on evapotranspiration, whereas its cooling effect is much weaker than the warming induced by climate change. Irrigation is not associated with large changes in precipitation, and the only significant increases occur in mountainous humid regions, where land-atmosphere interactions are not very impacted by additional moisture. This is consistent with the recycling of atmospheric moisture from irrigation analysed in present climate.
%option:FC peux tu faire la le lien avec le recyclage ?
Therefore, the impact of irrigation on the aridification of the Peninsula induced by climate change remains limited to a few grid cells.
The impacts of irrigation on river discharge were not analysed however, due to technical issues in the river routing scheme.

This work on climate change and the impact of irrigation in the future, conducted in the context of a Masters student's internship, remains preliminary and calls for further investigation. Longer simulations are currently being run over 30-year periods to strengthen the statistical significance of the results. They will also include a simulation of present climate with irrigation, run with the same setup as the other three simulations, to enable a relevant comparison of irrigation demand and irrigated volumes in the two periods.

\clearpage


\chapter{Impacts of irrigation and surface heterogeneities on the atmospheric boundary layer}
\label{chap:liaise}
\minitoc
\pagebreak
\section{Chapter introduction}

Considering its impacts on the energy budget at the surface, the effects of irrigation follow a marked diurnal cycle, with distinct effects in nighttime and daytime and varying amplitude depending on insolation. 
%option:could reuse ref from Intro
Average monthly or seasonal studies (like the work presented in Chapter \ref{chap:monthly}) cannot allow a detailed analysis of these impacts, and simulation outputs at a finer temporal resolution must be used.

Although it's based on the same simulation setup as Chapter \ref{chap:monthly}, this chapter focuses on the month on July 2021, and on part of the Ebro Valley. This study area and period were selected to include the measurement sites and special observation period (SOP) of the LIAISE field campaign. This enables direct comparison to local observations of temperature, humidity, wind and turbulent fluxes on irrigated and rainfed sites, at the surface and in the boundary layer. Furthermore, several mesoscale modelling experiments were conducted over this area and period, in particular using the mesoscale MesoNH model at 2-km resolution. These simulations served as an intermediate between punctual observations and the regional climate model, bringing new perspective on the following questions:

\begin{itemize}
    \item How does the ICOLMDZOR-LAM perform over the LIAISE study area, relative to observations and mesoscale simulations?
    \item To what extent does simulated irrigation improve the performance of the LAM? What limits its ability to reproduce observations on irrigated and rainfed sites at the diurnal scale?
    \item Are the effects of simulated irrigation limited to surface variables or do they propagate to the vertical structure of the ABL? 
    \item Are the effects of simulated irrigation only visible over the irrigated areas or do they also affect neighbouring rainfed areas?
    \item Over such a heterogeneous terrain, how can representativity of the LAM be defined and to what extent is it achieved?
\end{itemize}

This chapter first presents the LIAISE field campaign with details on the measurements used, mesoscale simulations that were conducted over the study area, and the main results of this project.
Then, the ICOLMDZOR LAM simulations used for comparison over LIAISE sites are described, before presenting results on land-atmosphere coupling variables and the vertical structure of the atmosphere.
%todo: improve/rephrase after final structure

\section{The LIAISE field campaign}
This description of the Land Surface Interactions with the Atmosphere over the Iberian Semi-Arid Environment (LIAISE) project is partly based on the various articles which presented the campaign : \citet{boone_land_2019,boone_updates_2021,boone_land_2025}, as well as on the dedicated section in Tanguy Lunel's PhD thesis \citep{lunel_interactions_2024}.

%objectives
LIAISE is an international research campaign aimed at improving the understanding of land-atmosphere-hydrology interactions in a semiarid region characterized by strong surface heterogeneity owing to contrasts between the natural landscape and irrigation-dependent agriculture, and the limitations of models to represent all aspects of the terrestrial water cycle.

%period
Although the campaign included a longer monitoring of surface variables, this work focuses on the special observation period (SOP) which occurred from July  15–29 2021. This SOP has been selected since it is the period where contrasts between irrigated and natural surfaces are generally at or near their maximum. At this time, synoptic scale winds are typically light and from the west, and a  thermal (heat) low pressure area generally forms over the region with daily maximum temperatures in the mid-to-upper  30s °C.
Within this SOP, Intensive Observation Periods (IOPs) were selected, aiming for relatively clear days characterized by a well-mixed  atmospheric boundary layer with weak synoptic scale forcing, so that surface-induced local to regional scale circulations were most likely to be present and detectable. 

%area
The study domain for LIAISE is the Ebro basin in northeastern Spain, which is bound to the north by the Pyrenees and to the south by the Iberian System.
It presents a semiarid hot, dry  Mediterranean climate, with a very sharp delineation between a vast, nearly continuous intensively-irrigated region and the generally drier rainfed zone to the east of the study domain.
%sites
Two supersites were defined at La Cendrosa, within the irrigated zone, and Els Plans, within the rainfed zone. Each was equipped with a 50-meter mast to measure temperature, humidity, wind speed, radiative and turbulent fluxes (using eddy-covariance) at various heights.
This work used 2-meter measurements for most variables, and 10-m wind speed.
On IOP days, hourly radiosondes were launched, roughly from  sunrise through early evening, probing the ABL up to about 3 km.

\begin{figure}[hbtp]
    \centering
    \includegraphics[width=\textwidth]{images/chap5/liaise_sites_picture.png}
    \caption{Instrumented sites of La Cendrosa (a) and Els Plans (b) in July 2021 \citep[taken from ][]{lunel_interactions_2024}.}
    \label{fig:liaise_sites_photos}
\end{figure}


%choice of IOP days, synoptic conditions
%  This thermal situation has  a strong impact on surface winds. Formation of some shallow cumulus is possible, but moist convection, if present, is  generally confined to the surrounding mountain ranges. Owing to the proximity to the Mediterranean Sea to the east, sea  breeze (SB) formation is quite frequent, but its inland progression is slowed by the presence of the Catalan pre-coastal and  coastal ranges that separate the Ebro basin from the sea. The  intensity of the westerlies and strength of the heat low are also  contributing factors to the SB front propagation and intensity. Generally speaking, the SB front usually arrived sometime  between 16:00 and 19:00 local time over the study zone with  the dry zone sites impacted earliest. The SB passage was seen  in the observations as a low level wind shift (to winds with a  significant easterly component) over the entire region, while in  the east, horizontally-scanning lidar observations also revealed  a significant increase in low-level moisture with its arrival. The  sea breeze generally coincided with a collapsing ABL in the late  afternoon. Days with a predicted early arrival of the SB were  not designated as IOPs during the daily briefings. Finally, one  rain event did occur on July 26, which was associated with the  passage of a synoptic scale trough: local precipitation totals of  around 30 mm were recorded over the irrigated zone; however,  the convective cells propagated to the northeast and the dry  zone received relatively little rainfall, thereby reinforcing the  wet-dry zone contrast during the subsequent days of the SOP.

%measurements / instruments
%irrigation practices and timing. Urgell canal adduction
%overview of synoptic conditions, rainfall
% simulation experiment with conceptual and mesoscale models
%main outcomes so far : Mangan, Lunel (2 sites), Lunel Marinada, intermodel comparison

%option:voir si irait bien quelque part: 
%Most of the water used for agriculture, approximately 75%, is stored in reservoirs while the rest is maintained by the snow pack in the mountains.

\section{Simulation experiments}

Several simulations are compared to analyse the diurnal cycle of land-atmosphere coupling variables in the Ebro Valley.
First, the two simulations studied in Section \ref{sec:article1} (\noirr and \irr) were analysed and compared to LIAISE observations using hourly outputs over the month of July 2021.

Then, sensitivity experiments were conducted to increase irrigation over grid cell corresponding to the irrigated site of La Cendrosa, leading to the simulation henceforth referred to as \irrboost. This simulation was only run over the month of July, starting from the same state as \noirr and \irr on July 1st 2021, and involved significant changes in both on the computation of irrigation demand, and on the water availability. 
Regarding demand, the irrigated fraction was slightly increased to make sure that all of the soiltile dedicated to low vegetation and crops was considered as irrigated land. This soiltile represents 83.5\% of the grid cell, the rest being occupied by trees (12.4\%) and bare soil (4.1\%). This fraction was also reduced to 0\% on the Els Plans site to make sure there would be no irrigation demand computed in this grid cell.
More importantly, the \betairrig parameter, which had been set to 0.6 after the offline calibration in Chapter \ref{chap:routing} to avoid excessive depletion of the groundwater and river reservoirs, was increased to 1. This means that the irrigation scheme is aiming to maintain soil moisture in the root zone (first 64cm of soil) to field capacity, making sure that plants would not be limited in their growth by available soil moisture. 
Regarding available water, since this was a short simulation run (one month), very large amounts of water were added in the three routing reservoirs at the simulation initial state, making virtually inifinite amounts of water available for irrigation in the LIAISE region. Obviously, this method is not realistic, nor applicable to longer climate runs, and was only used to test the limits of the irrigation scheme and explore the sensitivities of land-atmosphere interactions in a context where irrigation would be driven by the water demand rather than the supply. 
When comparing to reality, this can also be seen as a way to compensate for the absence of water adduction in the ICOLMDZOR LAM simulations, a practice that is determinant in the actual supply of irrigated water in the area. %todo:see if this explained in previous sec, refer to it
%todo:see also if irrigation simulation method used by tanguy has been detailed (because that's what it does too)

The three ICOLMDZOR LAM simulations (\noirr, \irr, \irrboost) are compared to observations and to the MesoNH simulation from July 14th to July 30th. This covers all the LIAISE SOP and allows for the stabilization of irrigation volumes in the \irrboost simulation, since no long-term spinup was conducted with this setup.

\hfill

%todo:choice of grid cell -> figure
For each site, an ICOLMDZOR grid cell was selected for the comparison to observations. For Els Plans, the grid cell containing the exact site location seemd appropriate since it was a very lightly irrigated cell. however, the exact position of the La Cendrosa site was not in an intensely irrigated cell, which is why the neighbouring grid cell was selected.

%fig:sites and corresponding grid cells
\begin{figure}[hbtp]
    \centering
    \begin{tabular}{cc}
        \begin{subfigure}[t]{0.44\textwidth}
            \caption{Visible image of the LIAISE study area as seen by \textit{Sentinel-2} on July 22nd.}
            \includegraphics[width=\textwidth]{images/chap5/liaise_overview_lunel.png}
        \end{subfigure} &

        \begin{subfigure}[t]{0.5\textwidth}
            \caption{Monthly average irrigation simulated by ORCHIDEE (July 2021, \irr simulation).}
            \includegraphics[width=\textwidth]{images/chap5/liaise_sites_irrig_ORC.png}
        \end{subfigure} 
    \end{tabular} 

    \begin{subfigure}[t]{0.75\textwidth}
            \caption{Average latent heat flux simulated by MesoNH (14-30 July 2021, with irrigation). Red hexagons show the ICOLMDZOR grid cells for each site.}
            \includegraphics[width=\textwidth]{images/chap5/liaise_sites_mean_mesoNH.png}
        \end{subfigure} 
    
    \caption{Correspondance between actual location of La Cendrosa and Els Plans sites (red dots), selected ICOLMDZOR grid cells (centered on green dots), and MesoNH grid. (a) was taken from \citet{lunel_irrigation_2024}.}
    \label{fig:liaise_sites_grid_cells}
\end{figure}

%todo:mesoNH (if not presented in previous section ?)
Comparison to the MesoNH simulations was performed using two values for each site. 
The first one, referred to as \mesoexact, takes the variables from the exact grid cell corresponding to the site location based on the GPS coordinates of the site, as done in \citet{lunel_irrigation_2024}. 
The second, referred to as \mesomean, is an aggregation of all the MesoNH grid cells contained into the selected ICOLMDZOR grid cell for each site. The mean value is shown, as well as an envelope encompassing the 25th and 75th percentiles, when it is appropriate.
On Fig. \ref{fig:liaise_sites_grid_cells}c, the two hexagonal ICOLMDZOR grid cells are drawn upon the average latent heat flux simulated by MesoNH. It can already be noticed that within the grid cell for La Cendrosa, there are some MesoNH grid cells with very low values of ET, and that in the grid cell for Els Plans, there are some MesoNH grid cells which fall into the irrigated zone and exhibit high ET. This will have an influence on the \mesomean values and on the spread of MesoNH surface variables within one ICOLMDZOR grid cell. 
%option:mesoMean is not meant to be more representative of the obs than mesoExact, but to bring information on heterogeneity within one ICOLMDZOR grid cell

\section{Near-surface variables over the SOP}
\label{sec:sop}

Here, surface variables simulated by ICOLMDZOR and MesoNH are compared to the observations on both sites from July 14th to July 30th, the span of the MesoNH simulation, which corresponds to the LIAISE SOP.

\hfill
%%LA CENDROSA%%

At La Cendrosa (Fig. \ref{fig:cendrosa_surfacevars}), observed turbulent fluxes(in black) exhibit a clear diurnal cycle driven by solar irradiation, with a maximum value between 12 and 13UTC.
In the \noirr simulation (in red), ICOLMDZOR simulates a negligible latent heat flux (Fig. \ref{fig:cendrosa_surfacevars}b) and largely overestimates the sensible heat flux (+300\% at 12UTC on Fig. \ref{fig:cendrosa_surfacevars}d). 
These major biases are partly improved in the \irr simulation (in blue), showing that additional soil moisture brought by irrigation can really improve turbulent fluxes. However, latent heat flux simulated in \irr follows a good trajectory from 5UTC to 9UTC but then drops and remains largely underestimated until the evening. This is due to a failure of the irrigation parametrization which cannot sustain the irrigation demand throughout the day since the routing reservoirs (particularly the river reservoirs) cannot provide sufficient irrigation withdrawals. 
In the \irrboost sensitivity experiment (in green), virtually infinite amounts of water are available in the reservoirs and ET is no longer constrained by soil moisture. Both turbulent fluxes follow a diurnal cycle that is consistent with observations, although a small underestimation of latent heat flux and an overestimation of sensible heat flux remain compared to the observations. 
However, the turbulent fluxes in \irrboost are very close to those of \mesomean (in yellow, with the envelope representing the 25th and 75th percentiles of the distribution), suggesting that ICOLMDZOR correctly represents the average fluxes over the grid cell, which does not only include intensely irrigated areas.
It can be noted here that \mesoexact (in purple), corresponding to the exact MesoNH grid cell of the observation site, overestimates latent heat flux on average over the period. This is mostly due to the beginning of the SOP, where the alfalfa crops at La Cendrosa were still freshly cut and had not recovered their original size, LAI and transpiration levels.

The diurnal cycle of 2-meter temperature is also driven by solar irradiation but its shape is slightly shifted with a peak in the afternoon around 4UTC (Fig. \ref{fig:cendrosa_surfacevars}f). Figure \ref{fig:cendrosa_surfacevars}e shows that temperatures were consistently rising throughout the SOP, from July 14th to 22nd, and that this evolution is well captured by ICOLMDZOR and MesoNH.
On average over the SOP, all simulations exhibit a warm bias, of at least 1.5° at night, where \mesoexact is closer to ICOLMDZOR than to observations, and of varying amplitude during the day. The \irr simulation has the strongest warm bias, which is consistent with the simulated turbulent fluxes, and this bias is slightly reduced in \irr. In \irrboost, it is largely improved, with a peak temperature 2°C lower than \irr, and simulated values that follow \mesomean throughout most of the day. In the evening, \irrboost even falls closer to observed values than to \mesomean, which suggests that if turbulent fluxes are correctly simulated, the structural nighttime warm bias is smaller in ICOLMDZOR than in MesoNH.
The diurnal cycle for 2-meter specific humidity is less structured than turbulent fluxes or temperatures.
On Fig. \ref{fig:cendrosa_surfacevars}h, MesoNH appears to present a dry bias a night, with an underestimation of specific humidity in \mesoexact as strong as in \mesomean. With ICOLMDZOR, a similar bias is present in \noirr, but it is absent in \irr and \irrboost, likely thanks to local ET enabled by irrigation.
In daytime, \mesoexact matches observed values from 8UTC to 18UTC, and \mesomean follows a similar structure through the diurnal cycle. 
ICOLMDZOR, on the contrary, presents an incorrect diurnal cycle of specific humidity, with an extensive drying in daytime. The strength of this drying is largest in \noirr but \irrboost still falls below the values of \mesomean, at the limit of the 25th percentile. 
The differences between ICOLMDZOR and observed values increase largely from July 20th, where humidity drops in the simulations but not in reality or in MesoNH simulation (Fig. \ref{fig:cendrosa_surfacevars}g).
This shows that in certain daytime conditions, increasing ET by accounting for irrigation is not sufficient to simulate a realistic 2-meter specific humidity. In particular, incorrect advection terms and surface wind can neutralise the effect of this local improvement of land-atmosphere interactions.

This hypothesis is corroborated by the 10-meter wind regime at La Cendrosa (Fig. \ref{fig:bothsites_wind}a-d), which is quite well structured in the beginning of the SOP with low wind speeds, but shows important variations in direction and an increased wind speed from July 20th. ICOLMDZOR seems to be capturing the variations quite well until July 19th but clearly misses part of the regime changes afterwards. 
On average over the SOP, \mesoexact follows observations closely, and the diurnal cycle simulated by ICOLMDZOR has the same structure as \mesomean, particularly for wind direction.
Irrigation only has an impact on the 10-meter wind speed: in \noirr it is overestimated, and that bias is partly reduced in \irr and \irrboost. This was expected since in the presence of irrigation, the vegetation is more developed and LAI is larger, leading to a larger surface drag and therefore lower surface wind speed. 
Wind direction is not impacted by irrigation (red, blue and green lines overlap in Fig. \ref{fig:bothsites_wind}d), which was also expected since only a few grid cells are intensely irrigated in the region, which limits the possibilities of influencing the dynamics of the LAM.

%fig : Cendrosa turbulent fluxes + t2m, q2m
\begin{figure}[hbtp]
    \centering
    \begin{tabular}{cc}
        \begin{subfigure}[t]{0.5\textwidth}
            \caption{}
            \includegraphics[width=\textwidth]{images/chap5/SOP_TS_DC/time_series_cendrosa_flat.png}
        \end{subfigure} &
        \begin{subfigure}[t]{0.5\textwidth}
            \caption{}
            \includegraphics[width=\textwidth]{images/chap5/SOP_TS_DC/diurnal_cycle_cendrosa_flat.png}
        \end{subfigure} \\
        
        % \vspace{1em} % Add vertical space between the rows
        \begin{subfigure}[t]{0.5\textwidth}
            \caption{}
            \includegraphics[width=\textwidth]{images/chap5/SOP_TS_DC/time_series_cendrosa_sens.png}
        \end{subfigure} &
        \begin{subfigure}[t]{0.5\textwidth}
            \caption{}
            \includegraphics[width=\textwidth]{images/chap5/SOP_TS_DC/diurnal_cycle_cendrosa_sens.png}
        \end{subfigure} \\

        \begin{subfigure}[t]{0.5\textwidth}
            \caption{}
            \includegraphics[width=\textwidth]{images/chap5/SOP_TS_DC/time_series_cendrosa_t2m.png}
        \end{subfigure} &
        \begin{subfigure}[t]{0.5\textwidth}
            \caption{}
            \includegraphics[width=\textwidth]{images/chap5/SOP_TS_DC/diurnal_cycle_cendrosa_t2m.png}
        \end{subfigure} \\
        
        % \vspace{1em} % Add vertical space between the rows
        \begin{subfigure}[t]{0.5\textwidth}
            \caption{}
            \includegraphics[width=\textwidth]{images/chap5/SOP_TS_DC/time_series_cendrosa_q2m.png}
        \end{subfigure} &
        \begin{subfigure}[t]{0.5\textwidth}
            \caption{}
            \includegraphics[width=\textwidth]{images/chap5/SOP_TS_DC/diurnal_cycle_cendrosa_q2m.png}
        \end{subfigure} \\
    \end{tabular}
    \caption{Time series and mean diurnal cycle of surface turbulent fluxes at La Cendrosa (irrigated site), July 14-30 2021. The envelope for \mesomean represents the 25th and 75th percentiles of the distribution.}
    \label{fig:cendrosa_surfacevars}
\end{figure}


%Fig : Wind on both sites
\begin{figure}[hbtp]
    \centering
    \begin{tabular}{cc}
        \begin{subfigure}[t]{0.5\textwidth}
            \caption{}
            \includegraphics[width=\textwidth]{images/chap5/SOP_TS_DC/time_series_cendrosa_wind_speed_10m.png}
        \end{subfigure} &
        \begin{subfigure}[t]{0.5\textwidth}
            \caption{}
            \includegraphics[width=\textwidth]{images/chap5/SOP_TS_DC/diurnal_cycle_cendrosa_wind_speed_10m.png}
        \end{subfigure} \\
        
        \begin{subfigure}[t]{0.5\textwidth}
            \caption{}
            \includegraphics[width=\textwidth]{images/chap5/SOP_TS_DC/time_series_cendrosa_wind_direction_10m.png}
        \end{subfigure} &
        \begin{subfigure}[t]{0.5\textwidth}
            \caption{}
            \includegraphics[width=\textwidth]{images/chap5/SOP_TS_DC/diurnal_cycle_cendrosa_wind_direction_10m.png}
        \end{subfigure} \\

        \begin{subfigure}[t]{0.5\textwidth}
            \caption{}
            \includegraphics[width=\textwidth]{images/chap5/SOP_TS_DC/time_series_elsplans_wind_speed_10m.png}
        \end{subfigure} &
        \begin{subfigure}[t]{0.5\textwidth}
            \caption{}
            \includegraphics[width=\textwidth]{images/chap5/SOP_TS_DC/diurnal_cycle_elsplans_wind_speed_10m.png}
        \end{subfigure} \\
        
        \begin{subfigure}[t]{0.5\textwidth}
            \caption{}
            \includegraphics[width=\textwidth]{images/chap5/SOP_TS_DC/time_series_elsplans_wind_direction_10m.png}
        \end{subfigure} &
        \begin{subfigure}[t]{0.5\textwidth}
            \caption{}
            \includegraphics[width=\textwidth]{images/chap5/SOP_TS_DC/diurnal_cycle_elsplans_wind_direction_10m.png}
        \end{subfigure} \\
    \end{tabular}
    \caption{Time series and mean diurnal cycle of wind speed and direction on both sites, July 14-30 2021. The envelope for \mesomean represents the 25th and 75th percentiles of the distribution.}
    \label{fig:bothsites_wind}
\end{figure}

\hfill

%%ELS PLANS%%
Looking into near-surface conditions at the Els Plans site (Fig. \ref{fig:elsplans_surfacevars}) serves as a useful comparison to assess the general performance of both models in a context that is largely independent of soil moisture.
The energy partitioning between turbulent fluxes is very different, since the average observed latent heat flux is below 20 W \persqm while the sensible heat flux is three times larger than at La Cendrosa (Fig. \ref{fig:elsplans_surfacevars}b, d). 
The three ICOLMDZ simulations are very similar and the lines for \noirr and \irr completely overlap, while latent heat flux is slightly increased (5-10 W \persqm) in \irrboost since ORCHIDEE still computes a small irrigation for this grid cell. 
%option:explain the details ? Demand is 0 but then interpolated to DEM, satisfied on some grid cells near the border of the hexagon and reinterpolated to ORC heaxagonal grid.
ICOLMDZOR underestimates the already weak latent heat flux and overestimates the sensible heat flux with a peak value higher than 400 W \persqm whereas the observed one is around 300 W \persqm. 
Regarding the MesoNH simulation, the \mesoexact diurnal cycle of latent heat flux looks a bit different from the observed one, with a higher and sharper peak around 12UTC instead of the smooth shape seen in observations. The diurnal cycle of sensible heat flux matches the observed shape but presents an overestimation from 11UTC until the evening.
The latent heat flux of \mesomean is clearly driven by the few irrigated grid cells that fall into the hexagon (Fig. \ref{fig:liaise_sites_grid_cells}c), leading the mean value to exceed the 75th percentile. These outliers have much less influence on the sensible heat flux which shows a smaller spread and is even closer to observations than \mesoexact.
On the contrary to La Cendrosa, it was not really expected to see the \irr or \irrboost simulation matching \mesomean because the input map of irrigated fraction was voluntarily modified to limit irrigation on this grid cell. This was done with the initial objectives of creating a constrated situation compared to La Cendrosa and achieving a better match with on-site observations, but clearly made the comparison of grid-cell average fluxes less relevant.
%option:add that no big reservoir here so no irrigation at all if not cheating ? (and too much if cheating ?)

The average diurnal cycle of 2-meter temperature (Fig. \ref{fig:elsplans_surfacevars}f) confirms the presence of a warm bias in MesoNH at night (more than 2°C), and shows a slight underestimation of the peak value in the afternoon. 
%option: further links to fsens, LWup, Tsurf ?
%NB : sur les deux sites, LWdn est sous estimé de ~15W/m² en journée par les deux modèles (sauf \noirr sur La Cendrosa car air plus chaud), probablement ce qui permet d'être bon sur fsens alors que flat est surestimé (manque d'énergie incidente donc pas de compensation parfaite entre les deux)
It also highlights the very good performance of ICOLMDZOR on this non-irrigated site, although the peak is slightly too early compared to the observed one, as also seen at La Cendrosa.
The observed evolution of 2-meter specific humidity over the day is very different from La Cendrosa, since it is highest at night and steadliy decreases throughout the day with a minimal peak at 16UTC on average over the SOP (Fig. \ref{fig:elsplans_surfacevars}h).
As seen in La Cendrosa, MesoNH exhibits a dry bias at night but simulates correct humidity in the day. 
The diurnal evolution simulated by ICOLMDZOR is quite similar to the one at La Cendrosa, except that at Els Plans it is more in agreement with observations and MesoNH. On average over the SOP, ICOLMDZOR strongly underestimates 2-meter specific humidity, but as visible in Fig. \ref{fig:elsplans_surfacevars}g, this is mostly due to the second part of the SOP where excessive drying occurs in all three ICOLMDZOR simulations. 
This supports the idea that the limitations in capturing near-surface humidity variations at La Cendrosa are mostly non-local and largely induced by large-scale advection. 
Finally, it can be noticed that irrigation has a visible impact on specific humidity, although not very large in comparison to the dry bias, also pointing to non-local effects on this variable since irrigation is very small on the Els Plans grid cell in ICOLMDOZOR.%todo:link to irrig TS

Observed wind speed at Els Plans is higher than at La Cendrosa (Fig. \ref{fig:bothsites_wind}f), which relflects the influence on roughness on a site with lower vegtation (Fig. \ref{fig:liaise_sites_photos}). On average, variations in direction are of smaller amplitude than at La Cendrosa (Fig. \ref{fig:bothsites_wind}h) but some quick regime changes can still be identified on most days (Fig. \ref{fig:bothsites_wind}g).
In ICOLMDZOR, the simulated wind speed and direction are very similar at Els Plans and La Cendrosa, which is not surprising since the grid cells are next to each other, but there are no more dfferences between the three simulations. This confirms the hypothesis that the wind speed differences at La Cendrosa were mainly the consequence of increased roughness in \irrboost and \irr. On average, ICOLMDZOR underestimates wind speed but reproduces the diurnal changes in wind direction. As seen at La Cendrosa, it captures the variations in wind direction much better in the beginning of the SOP than after the regime change on July 20th. On this second part of the SOP, a lot of regime changes are not represented by ICOLMDZOR and \mesomean, while only some of them are simulated in \mesoexact.

%Fig : Els Plans turbulent fluxes + t2m, q2m
\begin{figure}[hbtp]
    \centering
    \begin{tabular}{cc}
        \begin{subfigure}[t]{0.5\textwidth}
            \caption{}
            \includegraphics[width=\textwidth]{images/chap5/SOP_TS_DC/time_series_elsplans_flat.png}
        \end{subfigure} &
        \begin{subfigure}[t]{0.5\textwidth}
            \caption{}
            \includegraphics[width=\textwidth]{images/chap5/SOP_TS_DC/diurnal_cycle_elsplans_flat.png}
        \end{subfigure} \\
        
        % \vspace{1em} % Add vertical space between the rows
        \begin{subfigure}[t]{0.5\textwidth}
            \caption{}
            \includegraphics[width=\textwidth]{images/chap5/SOP_TS_DC/time_series_elsplans_sens.png}
        \end{subfigure} &
        \begin{subfigure}[t]{0.5\textwidth}
            \caption{}
            \includegraphics[width=\textwidth]{images/chap5/SOP_TS_DC/diurnal_cycle_elsplans_sens.png}
        \end{subfigure} \\

        \begin{subfigure}[t]{0.5\textwidth}
            \caption{}
            \includegraphics[width=\textwidth]{images/chap5/SOP_TS_DC/time_series_elsplans_t2m.png}
        \end{subfigure} &
        \begin{subfigure}[t]{0.5\textwidth}
            \caption{}
            \includegraphics[width=\textwidth]{images/chap5/SOP_TS_DC/diurnal_cycle_elsplans_t2m.png}
        \end{subfigure} \\
        
        % \vspace{1em} % Add vertical space between the rows
        \begin{subfigure}[t]{0.5\textwidth}
            \caption{}
            \includegraphics[width=\textwidth]{images/chap5/SOP_TS_DC/time_series_elsplans_q2m.png}
        \end{subfigure} &
        \begin{subfigure}[t]{0.5\textwidth}
            \caption{}
            \includegraphics[width=\textwidth]{images/chap5/SOP_TS_DC/diurnal_cycle_elsplans_q2m.png}
        \end{subfigure} \\
    \end{tabular}
    \caption{Time series and mean diurnal cycle of surface turbulent fluxes at Els Plans (rainfed site), July 14-30 2021. The envelope for \mesomean represents the 25th and 75th percentiles of the distribution.}
    \label{fig:elsplans_surfacevars}
\end{figure}

\hfill

%%CCL%%
The key takeaways from this comparison of near-surface variables over the SOP is that the irrigation parametrization can significantly improve the results at La Cendrosa, especially if irrigation demand is fully satisfied, as done in the \irrboost sensitivity experiment.
In particular, even if the observations cannot be matched perfectly, near-surface variables simulated by ICOLMDZOR are very close to the \mesomean values. Considering that the MesoNH simulation is a relevant reference (ensured by the good performance of \mesoexact), it means that a good representation of grid-cell average values is achieved in the \irrboost simulation. 
In the second part of the SOP, a dry bias in ICOLMDZOR was identified, which could be improved be not entirely corrected by irrigation. Since this limitation is also visible at the rainfed site of Els Plans, and occurs simultaneously with a change in near-surface wind regime, it was hypothesized to be mainly the consequence of non-local effects, such as insufficient moisture advection over the LIAISE study area. The next section provides a study of the vertical structure of the ABL over two days, July 15th and July 20th, to investigate the differences between the first and second part of the SOP, and characterise the impact of irrigation and surface heterogeneities in these two different contexts.
\clearpage %todo:keep ?

\section{Vertical structure of the atmosphere}
\label{sec:iop}

Out of the seven IOP days over which radiosoundings were conducted on both sites, two were selected (July 15th and 20th) to investigate the atmospheric behaviour of the ICOLMDZOR LAM and the impact of irrigation on the vertical structure of the ABL.

\subsection{Surface conditions on selected IOP days}
%todo:reorganize, describe obs for all vars and then comment on model performance ?
%mesoexact is very very good on most vars, making it an interesting reference

July 15th was selected as a representative IOP day for the beginning of the SOP, showing similarities with observed profiles on the 16th and 17th. As the SOP progressed, 2-meter temperature increased and the wind regime became much more changeable, with more influence from the sea breeze circulation.
The three following IOP days (20th, 21st, 22nd) therefore presented different conditions, and this is why July 20th was also selected to analyse the impacts of irrigation on the ABL under different conditions.
The time series at La Cendrosa over both days show that July 20th was a warmer day than July 15th (Fig. \ref{fig:iop_days_TS_energy}a,b) and that the lower atmospher was moister (Fig. \ref{fig:iop_days_TS_energy}c,d).
%LMDZ biases in t2m, q2m
At the surface, ICOLMDZOR presents a warm bias on July 15th but almost not temperature bias in the \irrboost simulation on July 20th. As previously noticed in general over the SOP, 2-m specific humidity is overestimated by ICOLMDZOR at night, but decreases during the day. In \irrboost, it remains slightly above observed and MesoNH values during the day on July 15th, but on July 20th it is underestimated.
%link to flat
This may partly be explained by the simulated latent heat flux.%todo:meh
In the first part of the SOP, the alfalfa crops at La Cendrosa were still freshly cut and had not recovered their original size, LAI and transpiration levels. %todo:rephrase to refer to previous explanation
Since the models do not account for this information, the latent heat flux is overestimated by MesoNH (both \mesoexact and \mesomean are above the observations) and ICOLMDZOR (in the \irrboost simulation), whereas the sensible heat flux is underestimated (Fig. \ref{fig:iop_days_TS_energy}e, g). 
On July 20th, observed latent heat flux is higher and sensible heat flux lower, even presenting negative values in the afternoon.
It is difficult to disentangle the increase due to the vegetation growth from the increased evaporative demand due to the warming, but in \mesoexact, the increase is rather small.%blah
For both turbulent fluxes, as noticed in Section \ref{sec:sop}, the \irrboost simulation matches really well the \mesomean aggregated value throughout the day. It is also the case for 2-meter temperature but not really for specific humidity and 10m winds, which might result from limitations of the dynamics rather than the LMDZ physics.

% However, MesoNH does not overestimate 2-meter specific humidity, and  


%Fig : energy fluxes Cendrosa
\begin{figure}[hbtp]
    \centering
    \begin{tabular}{cc}
        %rad fluxes
        \begin{subfigure}[t]{0.5\textwidth}
            \caption{}
            \includegraphics[width=\textwidth]{images/chap5/IOP_TS/TS_2021-07-15_cendrosa_SWdnSFC.png}
        \end{subfigure} &
        \begin{subfigure}[t]{0.5\textwidth}
            \caption{}
            \includegraphics[width=\textwidth]{images/chap5/IOP_TS/TS_2021-07-20_cendrosa_SWdnSFC.png}
        \end{subfigure} \\
        \begin{subfigure}[t]{0.5\textwidth}
            \caption{}
            \includegraphics[width=\textwidth]{images/chap5/IOP_TS/TS_2021-07-15_cendrosa_LWdnSFC.png}
        \end{subfigure} &
        \begin{subfigure}[t]{0.5\textwidth}
            \caption{}
            \includegraphics[width=\textwidth]{images/chap5/IOP_TS/TS_2021-07-20_cendrosa_LWdnSFC.png}
        \end{subfigure} \\

        %turb fluxes
        \begin{subfigure}[t]{0.5\textwidth}
            \caption{}
            \includegraphics[width=\textwidth]{images/chap5/IOP_TS/TS_2021-07-15_cendrosa_flat.png}
        \end{subfigure} &
        \begin{subfigure}[t]{0.5\textwidth}
            \caption{}
            \includegraphics[width=\textwidth]{images/chap5/IOP_TS/TS_2021-07-20_cendrosa_flat.png}
        \end{subfigure} \\
        \begin{subfigure}[t]{0.5\textwidth}
            \caption{}
            \includegraphics[width=\textwidth]{images/chap5/IOP_TS/TS_2021-07-15_cendrosa_sens.png}
        \end{subfigure} &
        \begin{subfigure}[t]{0.5\textwidth}
            \caption{}
            \includegraphics[width=\textwidth]{images/chap5/IOP_TS/TS_2021-07-20_cendrosa_sens.png}
        \end{subfigure} \\
    \end{tabular}
    \caption{}
    \label{fig:iop_days_TS_energy}
\end{figure}

%Fig : surface variables Cendrosa
\begin{figure}[hbtp]
    \centering
    \begin{tabular}{cc}
        %t2m, q2m
        \begin{subfigure}[t]{0.5\textwidth}
            \caption{}
            \includegraphics[width=\textwidth]{images/chap5/IOP_TS/TS_2021-07-15_cendrosa_t2m.png}
        \end{subfigure} &
        \begin{subfigure}[t]{0.5\textwidth}
            \caption{}
            \includegraphics[width=\textwidth]{images/chap5/IOP_TS/TS_2021-07-20_cendrosa_t2m.png}
        \end{subfigure} \\
        \begin{subfigure}[t]{0.5\textwidth}
            \caption{}
            \includegraphics[width=\textwidth]{images/chap5/IOP_TS/TS_2021-07-15_cendrosa_q2m.png}
        \end{subfigure} &
        \begin{subfigure}[t]{0.5\textwidth}
            \caption{}
            \includegraphics[width=\textwidth]{images/chap5/IOP_TS/TS_2021-07-20_cendrosa_q2m.png}
        \end{subfigure} \\

        %turb fluxes
        \begin{subfigure}[t]{0.5\textwidth}
            \caption{}
            \includegraphics[width=\textwidth]{images/chap5/IOP_TS/TS_2021-07-15_cendrosa_wind_speed_10m.png}
        \end{subfigure} &
        \begin{subfigure}[t]{0.5\textwidth}
            \caption{}
            \includegraphics[width=\textwidth]{images/chap5/IOP_TS/TS_2021-07-20_cendrosa_wind_speed_10m.png}
        \end{subfigure} \\
        \begin{subfigure}[t]{0.5\textwidth}
            \caption{}
            \includegraphics[width=\textwidth]{images/chap5/IOP_TS/TS_2021-07-15_cendrosa_wind_direction_10m.png}
        \end{subfigure} &
        \begin{subfigure}[t]{0.5\textwidth}
            \caption{}
            \includegraphics[width=\textwidth]{images/chap5/IOP_TS/TS_2021-07-20_cendrosa_wind_direction_10m.png}
        \end{subfigure} \\
    \end{tabular}
    \caption{}
    \label{fig:iop_days_TS_surfvars}
\end{figure}

%todo:figure of winds at 12UTC (10m and 850hPa) ideally mesoNH and LMDZ
\clearpage

\subsection{Vertical profiles at 12UTC}
%July 15th
%noirr : ABL too high in ICOLMDZ (also visible at Els Plans), fsens too high ?
%ICOLMDZ wind speed lower than obs but mostly agrees with mesoNH, wind direction is Ok
%irrboost : ABL height clearly lowered, getting much closer to mesoMean, although not fully, still missing few 100m
%irrigation only slightly reduced warm bias, moistens whole ABL (closer to mesoExact than mesoMean or obs)

%no impact at ElsPlans, not a surprise

%Fig : profiles 1507 12UTC
\begin{figure}[hbtp]
    \centering
    \makebox[\textwidth][c]{%
    \begin{tabular}{@{}cccc@{}}
        %cendrosa
        \begin{subfigure}[t]{0.382\textwidth}
            \caption{}
            \includegraphics[width=\textwidth]{images/chap5/profiles/profile_cendrosa_theta_1507_.png}
        \end{subfigure} &
        \begin{subfigure}[t]{0.289\textwidth}
            \caption{}
            \includegraphics[width=\textwidth]{images/chap5/profiles/profile_cendrosa_ovap_1507_.png}
        \end{subfigure} &
        \begin{subfigure}[t]{0.283\textwidth}
            \caption{}
            \includegraphics[width=\textwidth]{images/chap5/profiles/profile_cendrosa_wind_speed_1507_.png}
        \end{subfigure} &
        \begin{subfigure}[t]{0.283\textwidth}
            \caption{}
            \includegraphics[width=\textwidth]{images/chap5/profiles/profile_cendrosa_wind_direction_1507_.png}
        \end{subfigure} \\
        %elsplans
        \begin{subfigure}[t]{0.382\textwidth}
            \caption{}
            \includegraphics[width=\textwidth]{images/chap5/profiles/profile_elsplans_theta_1507_.png}
        \end{subfigure} &
        \begin{subfigure}[t]{0.289\textwidth}
            \caption{}
            \includegraphics[width=\textwidth]{images/chap5/profiles/profile_elsplans_ovap_1507_.png}
        \end{subfigure} &
        \begin{subfigure}[t]{0.283\textwidth}
            \caption{}
            \includegraphics[width=\textwidth]{images/chap5/profiles/profile_elsplans_wind_speed_1507_.png}
        \end{subfigure} &
        \begin{subfigure}[t]{0.283\textwidth}
            \caption{}
            \includegraphics[width=\textwidth]{images/chap5/profiles/profile_elsplans_wind_direction_1507_.png}
        \end{subfigure} \\
    \end{tabular}
    }
    \caption{Vertical profiles at 12UTC on July 15th, at La Cendrosa (a-d) and Els Plans (e-h).}
    \label{fig:profiles_cendrosa_1507}
\end{figure}


%July 20th
%noirr : ABL too low (even more striking at els plans), although fsens strongly overestimated, consequence of dryness ?
%irrboost : reduces warm bias (half) and dry bias (matches obs but mesoMean even moister
%irrboost : lower ABL, making it worse...
%winds: link with ABL structure, missing lower jet at Els Plans

%Fig : profiles 2007 12UTC
\begin{figure}[hbtp]
    \centering
    \makebox[\textwidth][c]{%
    \begin{tabular}{@{}cccc@{}}
        %cendrosa
        \begin{subfigure}[t]{0.382\textwidth}
            \caption{}
            \includegraphics[width=\textwidth]{images/chap5/profiles/profile_cendrosa_theta_2007_.png}
        \end{subfigure} &
        \begin{subfigure}[t]{0.289\textwidth}
            \caption{}
            \includegraphics[width=\textwidth]{images/chap5/profiles/profile_cendrosa_ovap_2007_.png}
        \end{subfigure} &
        \begin{subfigure}[t]{0.283\textwidth}
            \caption{}
            \includegraphics[width=\textwidth]{images/chap5/profiles/profile_cendrosa_wind_speed_2007_.png}
        \end{subfigure} &
        \begin{subfigure}[t]{0.283\textwidth}
            \caption{}
            \includegraphics[width=\textwidth]{images/chap5/profiles/profile_cendrosa_wind_direction_2007_.png}
        \end{subfigure} \\
        %elsplans
        \begin{subfigure}[t]{0.382\textwidth}
            \caption{}
            \includegraphics[width=\textwidth]{images/chap5/profiles/profile_elsplans_theta_2007_.png}
        \end{subfigure} &
        \begin{subfigure}[t]{0.289\textwidth}
            \caption{}
            \includegraphics[width=\textwidth]{images/chap5/profiles/profile_elsplans_ovap_2007_.png}
        \end{subfigure} &
        %winds
        \begin{subfigure}[t]{0.283\textwidth}
            \caption{}
            \includegraphics[width=\textwidth]{images/chap5/profiles/profile_elsplans_wind_speed_2007_.png}
        \end{subfigure} &
        \begin{subfigure}[t]{0.283\textwidth}
            \caption{}
            \includegraphics[width=\textwidth]{images/chap5/profiles/profile_elsplans_wind_direction_2007_.png}
        \end{subfigure} \\
    \end{tabular}
    }
    \caption{Vertical profiles at 12UTC on July 20th, at La Cendrosa (a-d) and Els Plans (e-h).}
    \label{fig:profiles_cendrosa_2007}
\end{figure}

\clearpage

\section{Importance of surface fluxes heterogeneities}
\clearpage

\section{Chapter conclusions}
%irrigation without boost does not have very significant impact
%with boostirr, manage to get much closer to observations 

%no impact on Els Plans site, not a surprise


%discussion
% flat in irrboost is mostly driven by bare soil evap : realistic ? error compensation? need further developments in plant behaviour ?
%irrboost very simplified test, setup not good for dynamical effects, very strong irrig for some grid cells and no irrig at Els Plans
%setup does not allow a study of heterogeneity at Els Plans but it was not the main aim here, it rather serves as a reference for noirr case


\clearpage

\section{Chapter appendix}

\begin{figure}[hbtp]
    \centering
    \begin{tabular}{cc}
        \begin{subfigure}[t]{0.5\textwidth}
            \caption{}
            \includegraphics[width=\textwidth]{images/chap5/SOP_TS_DC/time_series_cendrosa_SWdnSFC.png}
        \end{subfigure} &
        \begin{subfigure}[t]{0.5\textwidth}
            \caption{}
            \includegraphics[width=\textwidth]{images/chap5/SOP_TS_DC/diurnal_cycle_cendrosa_SWdnSFC.png}
        \end{subfigure} \\
        
        \begin{subfigure}[t]{0.5\textwidth}
            \caption{}
            \includegraphics[width=\textwidth]{images/chap5/SOP_TS_DC/time_series_cendrosa_LWdnSFC.png}
        \end{subfigure} &
        \begin{subfigure}[t]{0.5\textwidth}
            \caption{}
            \includegraphics[width=\textwidth]{images/chap5/SOP_TS_DC/diurnal_cycle_cendrosa_LWdnSFC.png}
        \end{subfigure} \\

        \begin{subfigure}[t]{0.5\textwidth}
            \caption{}
            \includegraphics[width=\textwidth]{images/chap5/SOP_TS_DC/time_series_elsplans_SWdnSFC.png}
        \end{subfigure} &
        \begin{subfigure}[t]{0.5\textwidth}
            \caption{}
            \includegraphics[width=\textwidth]{images/chap5/SOP_TS_DC/diurnal_cycle_elsplans_SWdnSFC.png}
        \end{subfigure} \\
        
        \begin{subfigure}[t]{0.5\textwidth}
            \caption{}
            \includegraphics[width=\textwidth]{images/chap5/SOP_TS_DC/time_series_elsplans_LWdnSFC.png}
        \end{subfigure} &
        \begin{subfigure}[t]{0.5\textwidth}
            \caption{}
            \includegraphics[width=\textwidth]{images/chap5/SOP_TS_DC/diurnal_cycle_elsplans_LWdnSFC.png}
        \end{subfigure} \\
    \end{tabular}
    \caption{Time series and mean diurnal cycle of radiative fluxes on both sites, July 14-30 2021.}
    \label{fig:bothsites_rad}
\end{figure}


%Fig : energy fluxes ElsPlans
\begin{figure}[hbtp]
    \centering
    \begin{tabular}{cc}
        %rad fluxes
        \begin{subfigure}[t]{0.5\textwidth}
            \caption{}
            \includegraphics[width=\textwidth]{images/chap5/IOP_TS/TS_2021-07-15_elsplans_SWdnSFC.png}
        \end{subfigure} &
        \begin{subfigure}[t]{0.5\textwidth}
            \caption{}
            \includegraphics[width=\textwidth]{images/chap5/IOP_TS/TS_2021-07-20_elsplans_SWdnSFC.png}
        \end{subfigure} \\
        \begin{subfigure}[t]{0.5\textwidth}
            \caption{}
            \includegraphics[width=\textwidth]{images/chap5/IOP_TS/TS_2021-07-15_elsplans_LWdnSFC.png}
        \end{subfigure} &
        \begin{subfigure}[t]{0.5\textwidth}
            \caption{}
            \includegraphics[width=\textwidth]{images/chap5/IOP_TS/TS_2021-07-20_elsplans_LWdnSFC.png}
        \end{subfigure} \\

        %turb fluxes
        \begin{subfigure}[t]{0.5\textwidth}
            \caption{}
            \includegraphics[width=\textwidth]{images/chap5/IOP_TS/TS_2021-07-15_elsplans_flat.png}
        \end{subfigure} &
        \begin{subfigure}[t]{0.5\textwidth}
            \caption{}
            \includegraphics[width=\textwidth]{images/chap5/IOP_TS/TS_2021-07-20_elsplans_flat.png}
        \end{subfigure} \\
        \begin{subfigure}[t]{0.5\textwidth}
            \caption{}
            \includegraphics[width=\textwidth]{images/chap5/IOP_TS/TS_2021-07-15_elsplans_sens.png}
        \end{subfigure} &
        \begin{subfigure}[t]{0.5\textwidth}
            \caption{}
            \includegraphics[width=\textwidth]{images/chap5/IOP_TS/TS_2021-07-20_elsplans_sens.png}
        \end{subfigure} \\
    \end{tabular}
    \caption{}
    \label{fig:iop_days_TS_energy_elsplans}
\end{figure}

%Fig : surface variables ElsPlans
\begin{figure}[hbtp]
    \centering
    \begin{tabular}{cc}
        %t2m, q2m
        \begin{subfigure}[t]{0.5\textwidth}
            \caption{}
            \includegraphics[width=\textwidth]{images/chap5/IOP_TS/TS_2021-07-15_elsplans_t2m.png}
        \end{subfigure} &
        \begin{subfigure}[t]{0.5\textwidth}
            \caption{}
            \includegraphics[width=\textwidth]{images/chap5/IOP_TS/TS_2021-07-20_elsplans_t2m.png}
        \end{subfigure} \\
        \begin{subfigure}[t]{0.5\textwidth}
            \caption{}
            \includegraphics[width=\textwidth]{images/chap5/IOP_TS/TS_2021-07-15_elsplans_q2m.png}
        \end{subfigure} &
        \begin{subfigure}[t]{0.5\textwidth}
            \caption{}
            \includegraphics[width=\textwidth]{images/chap5/IOP_TS/TS_2021-07-20_elsplans_q2m.png}
        \end{subfigure} \\

        %turb fluxes
        \begin{subfigure}[t]{0.5\textwidth}
            \caption{}
            \includegraphics[width=\textwidth]{images/chap5/IOP_TS/TS_2021-07-15_elsplans_wind_speed_10m.png}
        \end{subfigure} &
        \begin{subfigure}[t]{0.5\textwidth}
            \caption{}
            \includegraphics[width=\textwidth]{images/chap5/IOP_TS/TS_2021-07-20_elsplans_wind_speed_10m.png}
        \end{subfigure} \\
        \begin{subfigure}[t]{0.5\textwidth}
            \caption{}
            \includegraphics[width=\textwidth]{images/chap5/IOP_TS/TS_2021-07-15_elsplans_wind_direction_10m.png}
        \end{subfigure} &
        \begin{subfigure}[t]{0.5\textwidth}
            \caption{}
            \includegraphics[width=\textwidth]{images/chap5/IOP_TS/TS_2021-07-20_elsplans_wind_direction_10m.png}
        \end{subfigure} \\
    \end{tabular}
    \caption{}
    \label{fig:iop_days_TS_surfvars_elsplans}
\end{figure}
%todo:relocate appendix

\chapter{Conclusions and perspectives}
\label{chap:conclusion}
\minitoc
\pagebreak
\section{Synthesis and discussion}

%Résumé des problématiques dans la littérature actuelle

The reliance of agriculture on irrigation has been rapidly increasing in the 20th century and is still on the rise. It is projected that irrigation demand will keep increasing in the future, especially as it is often regarded as an adaptation strategy to compensate for the effects of climate change, namely higher temperatures and disrupted precipitation patterns, and maintain high crop yields.
However, it is already responsible for the large majority of freshwater withdrawals worldwide, and many concerns arise regarding the depletion of groundwater ressources and regional tensions over water management challenges. This is particularly the case for semi-arid regions like the Iberian Peninsula (and the rest of the Mediterranean basin), where the effects of climate change are already visible, and precipitation decreases are expected in the 21st century.
To account for its impacts on the continental water cycle and provide relevant insights for current and future management policies, NWP models and ESMs have been increasingly incorporating irrigation parametrizations, as well as mesoscale research models.
This brought the demonstration that its effects were not limited to withdrawals and also included atmospheric feedbacks on evapotranspiration and precipitation, through near-surface cooling and moistening, and changes in the structure of the ABL and regional winds.
However, this is still a recent evolution for ESMs as most of them did not represent irrigation in their CMIP6 simulations, including the IPSL-CM6.
The representation of these atmospheric feedbacks in climate models is therefore still an active research field, paving the way for a complete assessment of its impacts on the water cycle and the atmosphere.

\hfill

This thesis presented a study of the impacts of irrigation with the ICOLMDZOR LAM over the Iberian Peninsula. 
It constitutes the first use-case of this regional climate model focused on land-atmosphere interactions, as well as the first use-case of a new river routing scheme designed for the global IPSL-CM7.

\hfill
% Main outputs of my work and discussion

%% offline routing evaluation and calibration %%
In Chapter \ref{chap:routing}, offline simulations with the ORCHIDEE LSM provided a general assessment of the new routing scheme, \native, compared to the preexisting version, \std, with the same parameter set and 0.5° resolution DEM as input. The \std routing served as a reference since it had been used and validated in multiple studies with ORCHIDEE, in particular to develop and evaluate the irrigation scheme \citep{arboleda-obando_validation_2024}.
Appart from differences in implementation on coastal grid cells, and a few flow direction changes in the processing of the DEM, \native was shown to reproduce the behaviour of \std for reservoir volumes, river discharge and irrigation volumes.
It was then used with a 1-arcminute resolution DEM, enabling better correspondance with discharge station and higher resolutions in ORCHIDEE. 
A tuning of the routing parameters was conducted to match observed discharge in the main rivers of the Iberian Peninsula and enable the irrigation scheme to withdraw sufficient amounts from the reservoirs.
An adequate representation of the seasonal river discharge cycle was achieved, establishing a satisfactory parameter set. However, sensitivity analyses showed that the relevance of a finer parameter calibration would be mitigated by the dependency on the irrigation parametrization.
Therefore, the target parameter for irrigation was also adjusted to a lower value than in the global model, to reduce irrigation withdrawals and better represent regional irrigation practices.
% LIMITATIONS: could have used more discharge stations (and more objective metrics....), a more formal framework for parameter calibration, GW observations (not only previous routing as reference)

%% LAM configuration for regional modelling over IP %%%
After identifying an appropriate set of parameters for the new river routing scheme in ORCHIDEE, coupled simulations with the ICOLMDZOR LAM were run over multiple years, using ERA5 data as lateral boundary conditions for the LAM.
From a technical point of view, this demonstrated that this new modelling tool could be used to study land-atmosphere interactions and the water cycle on climatic time scales, whereas previous use cases simulated only a few months and did not include any routing scheme.
However, the availability of ERA5 data on the IPSL or CNRS supercomputers in the proper format limited the simulations to 13 years, from 2010 to 2022. 
Before using these coupled simulations to study the impacts of irrigation, some tests were conducted on the size of the LAM domain to identify the most adapted setup for this study. 
This work is presented in Chapter \ref{chap:forcing} along with some investigations of the influence of the lateral boundary conditions on the results of the LAM over the region, conducted with Mariame Maiga during her master's internship. 
By comparing the structures of several land-atmosphere coupling variables with the ERA5 reanalysis, physical inconsistencies were identified in the transition zone of the LAM, that propagated to the central \textit{free zone}.
The main hypothesis was that in the transition zone where the nudging towards ERA5 forcing values is strong, the LAM did not behave normally, and was not able to condensate water to form clouds and precipitation. This was attributed to discrepancies between the physics of the model used to produce the ERA5 reanalysis and the LMDZ physics of the LAM.  
Simulations with larger domains showed that the inconsistencies were indeed mainly localized in the transition zone and confirmed their concentric structure, with a progressive decrease of biases towards the center. It also made clear that using a larger domain largely decreases their direct influence on the Iberian Peninsula and, of the three domain size explored, the intermediate one was considered to satisfyingly reduce their impact on this continental region.
Simulation experiments using outputs from ICOLMDZOR global simulations as lateral forcing corroborated the hypothesis that discrepancies with ERA5 were the main origin of the inconsistencies in the transition zone. Although they did not completely disappear, the biases in precipitation and downwelling radiative flux on the edges of the domain were greatly improved. However, using this source of forcing also amplified existing biases of the ICOLMDZOR model over the Iberian Peninsula, such as excessive winter precipitation over mountainous regions.
Finally, a simplified exploration of the sensitivity of the LAM to the sampling frequency of the LAM showed that using 6-hourly instead of hourly forcing data could strengthen most of the biases of the model, particularly leading to underestimated precipitation over the Peninsula.
Considering the short span of the interhsip (4 months) and some technical issues encountered while running LAM simulations in all the different setups, these results still require further consolidation. Nevertheless, they contributed to a greater understanding of the limitations of the ICOLMDZOR LAM and identified good practices for the IPSL community that will help designing new regional climate modelling experiments in the future.

\hfill

%% Coupled sims %%
% only ERA5 lateral forcing and quite good performance overall
Once a proper setup was identified for this study, the impacts of irrigation on climate and on the water cycle were analysed by comparing a simulation with irrigation to the another one without it, over the period 2010-2022.
These results are presented in Chapter \ref{chap:monthly} and were also included in an article submitted to Earth System Dynamics, currently under review.
This was the first evaluation of the new version of the routing and irrigation scheme coupled to an amotpsheric model. It was shown to simulate adequate irrigated volumes in the Ebro Valley, where it is mostly dependent on surface water withdrawals, and a consistent seasonal cycle, although it did not represent winter irrigation due to modelling assumptions.
However, in the Guadiana and Guadalquivir basins, where groundwater provides most of the irrigation withdrawals, the model was not able to satisfy the irrigation demand due to the depletion of the groundwater reservoir in the presence of irrigation. River discharge was evaluated at 18 stations in the five major river basins, and was generally improved by irrigation withdrawals which compensated for overestimations over most of the year.
Several biases in river discharge remain and were linked to overestimated precipitation in mountainous regions and to the absence of river dams in the model.
Atmospheric impacts were most visible in summer, when irrigation is the largest, and in the Ebro valley. In JJA, they consist in a change in energy partitionning between turbulent fluxes over intensely irrigated areas (by up to 50 W \persqm), associated with a near-surface cooling (up to -0.35°C) and a lowering of the ABL (up to -100 m). Moistening of the air is found to extend to less intensely irrigated zones in the center of the Peninsula (Tagus and Douro basins), and to induce a lowering of the LCL over these areas and the Ebro valley (up to -250 m). The only feedback on precipitation however, occurs in the mountains ranges surrounding the Ebro valley, where it is associated by increases in CAPE and moisture convergence. 
This suggests a dominant effect of ABL stabilization in the irrigated zones, which was confirmed by analysing precipitation on average over the Peninsula. Evidence of a partial recycling of atmospheric moisture is provided, but it is shown to be mostly nonlocal, as it occurs over lightly irrigated areas, which included mountainous regions.
% LIMITATIONS : too short, not the best lateral forcing option (?), ET and P still not great, irrigation in half of the domain is limited by reservoir volumes

%% Coupled sims in the future %%
%todo:compléter/changer selon si j'inclus les simus de Frédérique au dernier moment, notamment le fait que l'irrig était partout et pas juste dans l'Ebre...
In the context of Mariame Maiga's interhsip, a similar experiment was designed to study the impacts of irrigation over the region under climate change, leveraging the experience from the simulations of Chapter \ref{chap:forcing} to use global ICOLDMZOR simulations of future climate as a source for the lateral boundary conditions of the LAM. The period 2050-2062 was studied under the SSP5-8.5 scenario, with two simulations, with and without irrigation, and a simulation without irrigation under present climate (2010-2022), also conducted with forcing data from a global ICOLMDZOR simulation.
This allowed to describe the effects of climate change over the region, independently of irrigation, such as a 2-3°C warning, decreases in evapotranspiration and relative humidity over most of the domain, and large decreases in precipitation, most marked over the northen mountain ranges. 
The impacts of irrigation in the future compensate for the changes in evapotranspiration, but less evenly for those in relative humidity, and very partially for the increase in temperature (-0.5°C at most). As found in the present climate, ABL stabilization seems to be the dominant process in the irrigated valleys, while increases in atmospheric moisture only translate to more precipitation in mountainous regions.
The aridification induced by climate change was also quantified using the P/PET aridity index.
It showed a shift of several humid or sub-humid regions to a semi-arid climate, and an expansion of arid areas from the valley floor to most of the Ebro, Guadiana and Guadalquivir basins. 
Although these valleys are expected to receive large irrigation volumes in the future, this barely impacts the aridification process, since PET is only slightly reduced while precipitation does not increase in the valleys.
Although this study remained preliminary in many regards, it constituted a first implementation of future regional climate simulations with the LAM, opening new prospects with this tool. The results are not available at the time of writing this manuscript, but it will be followed by a proper comparisons of longer simulations, with  similar setups for irrigation and river routing over both periods. %todo:check si j'ai mentionné les pb de routage...

\hfill

%% LIAISE
In Chapter \ref{chap:liaise}, to complement these analyses of the impact of irrigation on the water cycle and climate on annual and seasonal scales, hourly outputs over the month of July 2021 were analysed and compared to observation data from the LIAISE field campaign and Meso-NH simulations (at 2-kilometer resolution) over the Ebro valley. 
This campaign was specifically design to quantify the influence of irrigation, and of the surface heterogeneities it creates, on land-atmosphere coupling variables and the structure of the ABL.
Corresponding ICOLMDZOR grid cells were identified for two observation sites located in an irrigated field (La Cendrosa) and in a rainfed area (Els Plans).
The first comparison in La Cendrosa showed that irrigation demand could not be sustained throughout the day by ORCHIDEE, due to a lack of available water in the reservoirs. This was to be expected as the actual supply of water over the LIAISE area mainly relies on the canal d'Urgell, and such advection infrastructure is not represented in the model.
A sensibility experiment was therefore conducted, with virtually infinite amounts of water, to assess the impacts of irrigation if it was not limited by constraints on withdrawals.
This simulation showed great improvements of surface turbulent fluxes and 2-meter temperature and humidity with regards to observations over the two weeks of the LIAISE SOP.
As a small underestimation of latent heat flux and overestimation of sensible heat flux remained, the Meso-NH grid cells contained into the ICOLMDZOR grid cell was also used as a reference and showed that ICOLMDZOR was adequately representing average surface fluxes over the grid cell, which is not fully irrigated.
The effects of the ABL were then investigated on two IOP days (15 July and 20 July), selected for their contrasting situations, both representative of a portion of the SOP.
The cooling and moistening effects of irrigation at La Cendrosa were found to propagate to the ABL on both days simimarly, with a 1K decrease in potential temperature and a 1-1.5 g \perkg increase in specific humidity. A moistening of smaller amplitude (0.5 g \perkg at most) was also identified at Els Plans, constituting the only non-local impact of irrigation.
At La Cendrosa, the ABL was also lowered on both days in the presence of intense irrigation by more than 300 m.
This impacts, although consistent on both days, do not always correspond to relevant improvements. On 15 July, the ABL was excessively high and the lowering lead to a much better agreement with the mean ABL simulated by Meso-NH, but the values of potential temperature and specific humidity were not really improved. On 20 July, the values of both variables in the ABL were improved, but the structure, which was already lower than the observed one, did not benefit from the lowering induced by irrigation.
These differences stressed the influence of external factors which cannot be locally corrected by irrigation such as large-scale advection terms, and the direction of the background wind. This was to be expected in comparisons at such fine time scales since the realism of the LAM only relies on the lateral forcing by ERA5.
The limitations in representing the impacts of irrigation were also linked to sub-grid heterogeneities, mainly of surface fluxes on 15 July, where the average surface flux was found to be insufficient to represent the ABL over the area; and of winds on 20 July, with a sharp front simulated by Meso-NH at the border between rainfed and irrigated areas that contributed to the development of the ABL and was not accounted for in ICOLDMZOR.
% LIMITATIONS : no analysis of SM, can't analyse heterog at Els Plans, irrig sensitivity experiment very simplified so not suitable for regional analysis, only grid cells (because irrigation in rainfed area and a very irrealistic patch)
% discussion Beta=1 yields "perfect" LE, ok because flood irrigation ! 

\section{Perspectives}
Limitations

Irrigation is not performing well everywhere

Impacts on river discharge are still error compensation since lack of seasonality

Local impacts might also be due to error compensation (evapnu vs tran)

ABL impacts are not visible without boosting irrigation

\hfill\\
Model improvements

More water for irrigation
% more appropriate routing scheme tuning with GW ressources evaluation
% routing interp_HTU -> more relevant scale than just DEM
% Inclusion of adduction in the irrigation scheme -> non-local ressources, especially if routing grid is very small (less problematic with routing HTU)
% Articulte these elements with river dam representation, limit error 

% irrig with beta=0.9 globally probably depletes reservoirs too fast in other semi-arid climates

LAM with CMIP

LMDZ new physics (CMIP7)

Appropriate nudging options inside the LAM to study surface processes

\hfill\\
Further work that could be done from here

Better understanding of the LAM and new possibilities
% more extensive anlaysis of the influence of forcing on the LAM 
% simplified metrics/test to setup a new configuration
% LAM with CMIP format output -> lots of possibilities for studies under CC
% explore changes in LAM resolution (resolved vs parametrized ?)

ABL
% hopefully no need to cheat on the irrigation if irrig is fixed
% similar study on areas with larger irrigation patches to look for dynamical effects (mousson Guimberteau)
%explore heterogeneity of wind speed (roughness, lack of thermals ?)
%exploit simulations with high frequency outputs over the rest of the IP/Ebro valley, could be done, and not require irr_boost like on LIAISE zone

Exploit LIAISE data better
% LIAISE 1D case (proper synoptic forcing and initialisation) with a focus on LA interactions 
%LAM centered on LIAISE 

More down-to-Earth water ressources analysis
% give more meaning to GW modelling

%the end

% Appendix
\chapter*{Appendix} % Unnumbered chapter
\addcontentsline{toc}{chapter}{Appendix} % Manually add to toc
\label{chap:appendix}
\renewcommand{\thefigure}{A\arabic{figure}} % Figure numbering: A1, A2
\section*{Appendix to chapter \ref{chap:routing}}

%figure : 3 time series (3 reservoirs) on average over the IP for both routings
\begin{figure}[htbp]
    \centering
    \begin{subfigure}[b]{0.48\textwidth}
        \caption{Groundwater reservoir average}
        \includegraphics[width=\textwidth]{images/chap3/time_series/slowr_time_series.png}
    \end{subfigure}
        \begin{subfigure}[b]{0.48\textwidth}
        \caption{Groundwater reservoir average seasonal cycle}
        \includegraphics[width=\textwidth]{images/chap3/time_series/slowr_seasonal_cycle.png}
    \end{subfigure} \\
    
    \begin{subfigure}[b]{0.48\textwidth}
        \caption{Overland reservoir average}
        \includegraphics[width=\textwidth]{images/chap3/time_series/fastr_time_series.png}
    \end{subfigure}
    \begin{subfigure}[b]{0.48\textwidth}
        \caption{Overland reservoir average seasonal cycle}
        \includegraphics[width=\textwidth]{images/chap3/time_series/fastr_seasonal_cycle.png}
    \end{subfigure} \\
    
    \begin{subfigure}[b]{0.48\textwidth}
        \caption{River reservoir average}
        \includegraphics[width=\textwidth]{images/chap3/time_series/streamr_time_series.png}
    \end{subfigure} 
    \begin{subfigure}[b]{0.48\textwidth}
        \caption{River reservoir average seasonal cycle}
        \includegraphics[width=\textwidth]{images/chap3/time_series/streamr_seasonal_cycle.png}
    \end{subfigure} \\

    \caption{Time series and seasonal cycles of reservoir volumes on average over the Iberian Peninsula domain.}
    \label{fig:reservoir_time_series}
\end{figure}

%figure : maps of diff vs ERA for 2 forcing sampling freqs
\begin{figure}[!h]
    \centering
    \begin{tabular}{cc}
        %total cc
        \begin{subfigure}[b]{0.33\textwidth}
            \caption{Total cloud cover bias\\(\%, \forcingoneh)}
            \includegraphics[width=\textwidth]{images/chap4/forcing_sampling_freq/diff_map_cldt_lmdz1h_era.png}
        \end{subfigure} &
        \begin{subfigure}[b]{0.33\textwidth}
            \caption{Total cloud cover bias\\(\%, \forcingsixh)}
            \includegraphics[width=\textwidth]{images/chap4/forcing_sampling_freq/diff_map_cldt_lmdz6h_era.png}
        \end{subfigure}\\
        %low cc
        \begin{subfigure}[b]{0.33\textwidth}
            \caption{Low cloud cover bias\\(\%, \forcingoneh)}
            \includegraphics[width=\textwidth]{images/chap4/forcing_sampling_freq/diff_map_cldl_lmdz1h_era.png}
        \end{subfigure} &
        \begin{subfigure}[b]{0.33\textwidth}
            \caption{Low cloud cover bias\\(\%, \forcingsixh)}
            \includegraphics[width=\textwidth]{images/chap4/forcing_sampling_freq/diff_map_cldl_lmdz6h_era.png}
        \end{subfigure}
    \end{tabular}
    \caption{Biases of cloud cover for the simulations with hourly and 6-hourly forcing data, compared to ERA (2013).}
    \label{fig:forcing_sampling_freq_ERA_diff_maps_appendix}
\end{figure}

\clearpage

\section*{Appendix to chapter \ref{chap:forcing}}
%Forcing influence

%figure : maps of relative diff vs ERA for 3 domain sizes
%todo:captions
\begin{figure}[htbp]
    \centering
    \begin{tabular}{ccc}
        %precip
        \begin{subfigure}[b]{0.33\textwidth}
            \caption{}
            \includegraphics[width=\textwidth]{images/chap4/domain_size/rel_diff_map_precip_era_LAM_1000km_NBP40.png}
        \end{subfigure} &
        \begin{subfigure}[b]{0.33\textwidth}
            \caption{}
            \includegraphics[width=\textwidth]{images/chap4/domain_size/rel_diff_map_precip_era_LAM_1500km_NBP60.png}
        \end{subfigure} &
        \begin{subfigure}[b]{0.33\textwidth}
            \caption{}
            \includegraphics[width=\textwidth]{images/chap4/domain_size/rel_diff_map_precip_era_LAM_2000km_NBP80.png}
        \end{subfigure} \\
        
        %evap
        \begin{subfigure}[b]{0.33\textwidth}
            \caption{}
            \includegraphics[width=\textwidth]{images/chap4/domain_size/rel_diff_map_evap_era_LAM_1000km_NBP40.png}
        \end{subfigure} &
        \begin{subfigure}[b]{0.33\textwidth}
            \caption{}
            \includegraphics[width=\textwidth]{images/chap4/domain_size/rel_diff_map_evap_era_LAM_1500km_NBP60.png}
        \end{subfigure} &
        \begin{subfigure}[b]{0.33\textwidth}
            \caption{}
            \includegraphics[width=\textwidth]{images/chap4/domain_size/rel_diff_map_evap_era_LAM_2000km_NBP80.png}
        \end{subfigure}
    \end{tabular}
    \caption{}
    \label{fig:domain_size_ERA_reldiff_maps}
\end{figure}

\clearpage
 
\section*{Appendix to chapter \ref{chap:monthly}}
% Article

%f
\begin{figure}[htbp]
    \centering
    \includegraphics[width=\textwidth]{images/chap4/article/18_stations_TS.png}
    \caption{Time series of river discharge for the \irr and \noirr simulations and GRDC observations.}
    \label{fig:TS_discharge_18stations}
\end{figure}

%f
\begin{figure}[htbp]
    \centering
    \includegraphics[width=\textwidth]{images/chap4/article/18_stations_SC.png}
    \caption{Mean seasonal cycle of river discharge for the \irr and \noirr simulations and GRDC observations. A mask is applied to the simulations to filter out months without corresponding observation data.}
    \label{fig:SC_discharge_18stations}
\end{figure}


% Climate change 

%figure : from chap 4 future vs present. Seasonnal cycle of variables for 3 sims
\begin{figure}[htbp]
    \centering
    \begin{tabular}{cc}
        %precip
        \begin{subfigure}[b]{0.5\textwidth}
            \caption{}
            \includegraphics[width=\textwidth]{images/chap4/future/SC_precip_presfutirr.png}
        \end{subfigure} &
        %evap
        \begin{subfigure}[b]{0.5\textwidth}
            \caption{}
            \includegraphics[width=\textwidth]{images/chap4/future/SC_evap_presfutirr.png}
        \end{subfigure} \\

        %t2m
        \begin{subfigure}[b]{0.5\textwidth}
            \caption{}
            \includegraphics[width=\textwidth]{images/chap4/future/SC_t2m_presfutirr.png}
        \end{subfigure} &
        %fluxsens
        \begin{subfigure}[b]{0.5\textwidth}
            \caption{}
            \includegraphics[width=\textwidth]{images/chap4/future/SC_fluxsens_presfutirr.png}
        \end{subfigure} \\

        %q2m
        \begin{subfigure}[b]{0.5\textwidth}
            \caption{}
            \includegraphics[width=\textwidth]{images/chap4/future/SC_q2m_presfutirr.png}
        \end{subfigure} &
        %rh2m
        \begin{subfigure}[b]{0.5\textwidth}
            \caption{}
            \includegraphics[width=\textwidth]{images/chap4/future/SC_rh2m_presfutirr.png}
        \end{subfigure} \\

        %pblh
        \begin{subfigure}[b]{0.5\textwidth}
            \caption{}
            \includegraphics[width=\textwidth]{images/chap4/future/SC_s_pblh_presfutirr.png}
        \end{subfigure} &
        %lcl
        \begin{subfigure}[b]{0.5\textwidth}
            \caption{}
            \includegraphics[width=\textwidth]{images/chap4/future/SC_s_lcl_presfutirr.png}
        \end{subfigure} \\
    \end{tabular}
    \caption{}
    \label{fig:SC_future_irr}
\end{figure}

%figure : diff maps JJA (noirr, pres - future)
% \begin{figure}[htbp]
%     \centering
%     \begin{tabular}{cc}
%         %precip
%         \begin{subfigure}[b]{0.5\textwidth}
%             \caption{}
%             \includegraphics[width=\textwidth]{images/chap4/future/diffmap_JJA_precip_presfut.png}
%         \end{subfigure} &
%         %evap
%         \begin{subfigure}[b]{0.5\textwidth}
%             \caption{}
%             \includegraphics[width=\textwidth]{images/chap4/future/diffmap_JJA_evap_presfut.png}
%         \end{subfigure} \\

%         %t2m
%         \begin{subfigure}[b]{0.5\textwidth}
%             \caption{}
%             \includegraphics[width=\textwidth]{images/chap4/future/diffmap_JJA_t2m_presfut.png}
%         \end{subfigure} &
%         %fluxsens
%         \begin{subfigure}[b]{0.5\textwidth}
%             \caption{}
%             \includegraphics[width=\textwidth]{images/chap4/future/diffmap_JJA_fluxsens_presfut.png}
%         \end{subfigure} \\

%         %q2m
%         \begin{subfigure}[b]{0.5\textwidth}
%             \caption{}
%             \includegraphics[width=\textwidth]{images/chap4/future/diffmap_JJA_q2m_presfut.png}
%         \end{subfigure} &
%         %rh2m
%         \begin{subfigure}[b]{0.5\textwidth}
%             \caption{}
%             \includegraphics[width=\textwidth]{images/chap4/future/diffmap_JJA_rh2m_presfut.png}
%         \end{subfigure} \\

%         %pblh
%         \begin{subfigure}[b]{0.5\textwidth}
%             \caption{}
%             \includegraphics[width=\textwidth]{images/chap4/future/diffmap_JJA_s_pblh_presfut.png}
%         \end{subfigure} &
%         %lcl
%         \begin{subfigure}[b]{0.5\textwidth}
%             \caption{}
%             \includegraphics[width=\textwidth]{images/chap4/future/diffmap_JJA_s_lcl_presfut.png}
%         \end{subfigure} \\
%     \end{tabular}
%     \caption{JJA difference (\futnoirr - \presnoirr)}
%     \label{fig:diffmaps_JJA_present_future}
% \end{figure}

%figure : relative diff maps annual (noirr, pres - future)
\begin{figure}[htbp]
    \centering
    \begin{tabular}{cc}
        %precip
        \begin{subfigure}[b]{0.5\textwidth}
            \caption{}
            \includegraphics[width=\textwidth]{images/chap4/future/reldiffmap_precip_presfut.png}
        \end{subfigure} &
        %evap
        \begin{subfigure}[b]{0.5\textwidth}
            \caption{}
            \includegraphics[width=\textwidth]{images/chap4/future/reldiffmap_evap_presfut.png}
        \end{subfigure} \\

        %t2m
        \begin{subfigure}[b]{0.5\textwidth}
            \caption{}
            \includegraphics[width=\textwidth]{images/chap4/future/reldiffmap_psol_presfut.png}
        \end{subfigure} &
        %fluxsens
        \begin{subfigure}[b]{0.5\textwidth}
            \caption{}
            \includegraphics[width=\textwidth]{images/chap4/future/reldiffmap_fluxsens_presfut.png}
        \end{subfigure} \\

        %q2m
        \begin{subfigure}[b]{0.5\textwidth}
            \caption{}
            \includegraphics[width=\textwidth]{images/chap4/future/reldiffmap_q2m_presfut.png}
        \end{subfigure} &
        %rh2m
        \begin{subfigure}[b]{0.5\textwidth}
            \caption{}
            \includegraphics[width=\textwidth]{images/chap4/future/reldiffmap_rh2m_presfut.png}
        \end{subfigure} \\

        %pblh
        \begin{subfigure}[b]{0.5\textwidth}
            \caption{}
            \includegraphics[width=\textwidth]{images/chap4/future/reldiffmap_s_pblh_presfut.png}
        \end{subfigure} &
        %lcl
        \begin{subfigure}[b]{0.5\textwidth}
            \caption{}
            \includegraphics[width=\textwidth]{images/chap4/future/reldiffmap_s_lcl_presfut.png}
        \end{subfigure} \\
    \end{tabular}
    \caption{Relative difference, \futnoirr - \presnoirr}
    \label{fig:reldiffmaps_present_future}
\end{figure}

%figure : diff maps JJA (future, irr - no_irr)
% \begin{figure}[htbp]
%     \centering
%     \begin{tabular}{cc}
%         %precip
%         \begin{subfigure}[b]{0.5\textwidth}
%             \caption{}
%             \includegraphics[width=\textwidth]{images/chap4/future/diffmap_JJA_precip_futirr.png}
%         \end{subfigure} &
%         %evap
%         \begin{subfigure}[b]{0.5\textwidth}
%             \caption{}
%             \includegraphics[width=\textwidth]{images/chap4/future/diffmap_JJA_evap_futirr.png}
%         \end{subfigure} \\

%         %t2m
%         \begin{subfigure}[b]{0.5\textwidth}
%             \caption{}
%             \includegraphics[width=\textwidth]{images/chap4/future/diffmap_JJA_t2m_futirr.png}
%         \end{subfigure} &
%         %fluxsens
%         \begin{subfigure}[b]{0.5\textwidth}
%             \caption{}
%             \includegraphics[width=\textwidth]{images/chap4/future/diffmap_JJA_fluxsens_futirr.png}
%         \end{subfigure} \\

%         %q2m
%         \begin{subfigure}[b]{0.5\textwidth}
%             \caption{}
%             \includegraphics[width=\textwidth]{images/chap4/future/diffmap_JJA_q2m_futirr.png}
%         \end{subfigure} &
%         %rh2m
%         \begin{subfigure}[b]{0.5\textwidth}
%             \caption{}
%             \includegraphics[width=\textwidth]{images/chap4/future/diffmap_JJA_rh2m_futirr.png}
%         \end{subfigure} \\

%         %pblh
%         \begin{subfigure}[b]{0.5\textwidth}
%             \caption{}
%             \includegraphics[width=\textwidth]{images/chap4/future/diffmap_JJA_s_pblh_futirr.png}
%         \end{subfigure} &
%         %lcl
%         \begin{subfigure}[b]{0.5\textwidth}
%             \caption{}
%             \includegraphics[width=\textwidth]{images/chap4/future/diffmap_JJA_s_lcl_futirr.png}
%         \end{subfigure} \\
%     \end{tabular}
%     \caption{JJA difference (\futnoirr - \futirr)}
%     \label{fig:diffmaps_JJA_future_irr}
% \end{figure}


%figure : rel diff maps (future, irr - no_irr)
\begin{figure}[htbp]
    \centering
    \begin{tabular}{cc}
        %precip
        \begin{subfigure}[b]{0.5\textwidth}
            \caption{}
            \includegraphics[width=\textwidth]{images/chap4/future/reldiffmap_precip_futirr.png}
        \end{subfigure} &
        %evap
        \begin{subfigure}[b]{0.5\textwidth}
            \caption{}
            \includegraphics[width=\textwidth]{images/chap4/future/reldiffmap_evap_futirr.png}
        \end{subfigure} \\

        %t2m
        \begin{subfigure}[b]{0.5\textwidth}
            \caption{}
            \includegraphics[width=\textwidth]{images/chap4/future/reldiffmap_psol_futirr.png}
        \end{subfigure} &
        %fluxsens
        \begin{subfigure}[b]{0.5\textwidth}
            \caption{}
            \includegraphics[width=\textwidth]{images/chap4/future/reldiffmap_fluxsens_futirr.png}
        \end{subfigure} \\

        %q2m
        \begin{subfigure}[b]{0.5\textwidth}
            \caption{}
            \includegraphics[width=\textwidth]{images/chap4/future/reldiffmap_q2m_futirr.png}
        \end{subfigure} &
        %rh2m
        \begin{subfigure}[b]{0.5\textwidth}
            \caption{}
            \includegraphics[width=\textwidth]{images/chap4/future/reldiffmap_rh2m_futirr.png}
        \end{subfigure} \\

        %pblh
        \begin{subfigure}[b]{0.5\textwidth}
            \caption{}
            \includegraphics[width=\textwidth]{images/chap4/future/reldiffmap_s_pblh_futirr.png}
        \end{subfigure} &
        %lcl
        \begin{subfigure}[b]{0.5\textwidth}
            \caption{}
            \includegraphics[width=\textwidth]{images/chap4/future/reldiffmap_s_lcl_futirr.png}
        \end{subfigure} \\
    \end{tabular}
    \caption{Annual relative difference (\futnoirr - \futirr)}
    \label{fig:reldiffmaps_future_irr}
\end{figure}

%figure : Aridity Index diff and rel diff maps
\begin{figure}[htbp]
    \centering
    \begin{tabular}{cc}
        %present vs future (noirr)
        \begin{subfigure}[b]{0.5\textwidth}
            \caption{Absolute change in aridity index under climate change}
            \includegraphics[width=\textwidth]{images/chap4/future/diffmap_aridity_index_presfut.png}
        \end{subfigure} &
        \begin{subfigure}[b]{0.5\textwidth}
            \caption{Relative change in aridity index under climate change}
            \includegraphics[width=\textwidth]{images/chap4/future/reldiffmap_aridity_index_presfut.png}
        \end{subfigure} \\

        %irr vs noirr (future)
        \begin{subfigure}[b]{0.5\textwidth}
            \caption{Absolute change in future aridity index in the presence of irrigation}
            \includegraphics[width=\textwidth]{images/chap4/future/diffmap_aridity_index_futirr.png}
        \end{subfigure} &
        \begin{subfigure}[b]{0.5\textwidth}
            \caption{Relative change in future aridity index in the presence of irrigation}
            \includegraphics[width=\textwidth]{images/chap4/future/reldiffmap_aridity_index_futirr.png}
        \end{subfigure} \\
    \end{tabular}
    \caption{Aridity index changes under climate change and in the presence of irrigation.}
    \label{fig:diffmaps_aridity}
\end{figure}

\clearpage

\section*{Appendix to chapter \ref{chap:liaise}}

\begin{figure}[hbtp]
    \centering
    \begin{tabular}{cc}
        \begin{subfigure}[t]{0.5\textwidth}
            \caption{}
            \includegraphics[width=\textwidth]{images/chap6/SOP_TS_DC/time_series_cendrosa_SWdnSFC.png}
        \end{subfigure} &
        \begin{subfigure}[t]{0.5\textwidth}
            \caption{}
            \includegraphics[width=\textwidth]{images/chap6/SOP_TS_DC/diurnal_cycle_cendrosa_SWdnSFC.png}
        \end{subfigure} \\
        
        \begin{subfigure}[t]{0.5\textwidth}
            \caption{}
            \includegraphics[width=\textwidth]{images/chap6/SOP_TS_DC/time_series_cendrosa_LWdnSFC.png}
        \end{subfigure} &
        \begin{subfigure}[t]{0.5\textwidth}
            \caption{}
            \includegraphics[width=\textwidth]{images/chap6/SOP_TS_DC/diurnal_cycle_cendrosa_LWdnSFC.png}
        \end{subfigure} \\

        \begin{subfigure}[t]{0.5\textwidth}
            \caption{}
            \includegraphics[width=\textwidth]{images/chap6/SOP_TS_DC/time_series_elsplans_SWdnSFC.png}
        \end{subfigure} &
        \begin{subfigure}[t]{0.5\textwidth}
            \caption{}
            \includegraphics[width=\textwidth]{images/chap6/SOP_TS_DC/diurnal_cycle_elsplans_SWdnSFC.png}
        \end{subfigure} \\
        
        \begin{subfigure}[t]{0.5\textwidth}
            \caption{}
            \includegraphics[width=\textwidth]{images/chap6/SOP_TS_DC/time_series_elsplans_LWdnSFC.png}
        \end{subfigure} &
        \begin{subfigure}[t]{0.5\textwidth}
            \caption{}
            \includegraphics[width=\textwidth]{images/chap6/SOP_TS_DC/diurnal_cycle_elsplans_LWdnSFC.png}
        \end{subfigure} \\
    \end{tabular}
    \caption{Time series and mean diurnal cycle of radiative fluxes on both sites, 14-30 July 2021.}
    \label{fig:bothsites_rad}
\end{figure}

%Fig : surface variables ElsPlans
\begin{figure}[hbtp]
    \centering
    \begin{tabular}{cc}
        %t2m, q2m
        \begin{subfigure}[t]{0.5\textwidth}
            \caption{}
            \includegraphics[width=\textwidth]{images/chap6/IOP_TS/TS_2021-07-15_elsplans_t2m.png}
        \end{subfigure} &
        \begin{subfigure}[t]{0.5\textwidth}
            \caption{}
            \includegraphics[width=\textwidth]{images/chap6/IOP_TS/TS_2021-07-20_elsplans_t2m.png}
        \end{subfigure} \\
        \begin{subfigure}[t]{0.5\textwidth}
            \caption{}
            \includegraphics[width=\textwidth]{images/chap6/IOP_TS/TS_2021-07-15_elsplans_q2m.png}
        \end{subfigure} &
        \begin{subfigure}[t]{0.5\textwidth}
            \caption{}
            \includegraphics[width=\textwidth]{images/chap6/IOP_TS/TS_2021-07-20_elsplans_q2m.png}
        \end{subfigure} \\
        %wind speed
        \begin{subfigure}[t]{0.5\textwidth}
            \caption{}
            \includegraphics[width=\textwidth]{images/chap6/IOP_TS/TS_2021-07-15_elsplans_wind_speed_10m.png}
        \end{subfigure} &
        \begin{subfigure}[t]{0.5\textwidth}
            \caption{}
            \includegraphics[width=\textwidth]{images/chap6/IOP_TS/TS_2021-07-20_elsplans_wind_speed_10m.png}
        \end{subfigure} \\
        \begin{subfigure}[t]{0.5\textwidth}
            \caption{}
            \includegraphics[width=\textwidth]{images/chap6/IOP_TS/TS_2021-07-15_elsplans_wind_direction_10m.png}
        \end{subfigure} &
        \begin{subfigure}[t]{0.5\textwidth}
            \caption{}
            \includegraphics[width=\textwidth]{images/chap6/IOP_TS/TS_2021-07-20_elsplans_wind_direction_10m.png}
        \end{subfigure} \\
    \end{tabular}
    \caption{}
    \label{fig:iop_days_TS_surfvars_elsplans}
\end{figure}

%Fig : energy fluxes ElsPlans
\begin{figure}[hbtp]
    \centering
    \begin{tabular}{cc}
        %turb fluxes
        \begin{subfigure}[t]{0.5\textwidth}
            \caption{}
            \includegraphics[width=\textwidth]{images/chap6/IOP_TS/TS_2021-07-15_elsplans_flat.png}
        \end{subfigure} &
        \begin{subfigure}[t]{0.5\textwidth}
            \caption{}
            \includegraphics[width=\textwidth]{images/chap6/IOP_TS/TS_2021-07-20_elsplans_flat.png}
        \end{subfigure} \\
        \begin{subfigure}[t]{0.5\textwidth}
            \caption{}
            \includegraphics[width=\textwidth]{images/chap6/IOP_TS/TS_2021-07-15_elsplans_sens.png}
        \end{subfigure} &
        \begin{subfigure}[t]{0.5\textwidth}
            \caption{}
            \includegraphics[width=\textwidth]{images/chap6/IOP_TS/TS_2021-07-20_elsplans_sens.png}
        \end{subfigure} \\
        %rad fluxes
        \begin{subfigure}[t]{0.5\textwidth}
            \caption{}
            \includegraphics[width=\textwidth]{images/chap6/IOP_TS/TS_2021-07-15_elsplans_SWdnSFC.png}
        \end{subfigure} &
        \begin{subfigure}[t]{0.5\textwidth}
            \caption{}
            \includegraphics[width=\textwidth]{images/chap6/IOP_TS/TS_2021-07-20_elsplans_SWdnSFC.png}
        \end{subfigure} \\
        \begin{subfigure}[t]{0.5\textwidth}
            \caption{}
            \includegraphics[width=\textwidth]{images/chap6/IOP_TS/TS_2021-07-15_elsplans_LWdnSFC.png}
        \end{subfigure} &
        \begin{subfigure}[t]{0.5\textwidth}
            \caption{}
            \includegraphics[width=\textwidth]{images/chap6/IOP_TS/TS_2021-07-20_elsplans_LWdnSFC.png}
        \end{subfigure} 
    \end{tabular}
    \caption{}
    \label{fig:iop_days_TS_energy_elsplans}
\end{figure}

%Fig : bins flux lat
\begin{figure}[hbtp]
    \centering
    \makebox[\textwidth][c]{%
    \begin{tabular}{cc}
        \begin{subfigure}[t]{0.48\textwidth}
            \caption{La Cendrosa, 15 July at 12UTC}
            \includegraphics[width=\textwidth]{images/chap6/IOP_bins/bins_flat_2021-07-15T12:00:00_cendrosa.png}
        \end{subfigure}
        \begin{subfigure}[t]{0.48\textwidth}
            \caption{La Cendrosa, 20 July at 12UTC}
            \includegraphics[width=\textwidth]{images/chap6/IOP_bins/bins_flat_2021-07-20T12:00:00_cendrosa.png}
        \end{subfigure} \\
        \begin{subfigure}[t]{0.48\textwidth}
            \caption{Els Plans, 15 July at 12UTC}
            \includegraphics[width=\textwidth]{images/chap6/IOP_bins/bins_flat_2021-07-15T12:00:00_elsplans.png}
        \end{subfigure}
        \begin{subfigure}[t]{0.48\textwidth}
            \caption{Els Plans, 20 July at 12UTC}
            \includegraphics[width=\textwidth]{images/chap6/IOP_bins/bins_flat_2021-07-20T12:00:00_elsplans.png}
        \end{subfigure}
    \end{tabular}
    }
    \caption{Distribution of surface latent heat flux at 12UTC in \mesomean at La Cendrosa (a-b) and Els Plans (c-d) on 15 July and 20 July.}
    \label{fig:flat_bins}
\end{figure}

%Fig : profiles with sens bins min/max winds
\begin{figure}[hbtp]
    \centering
    \makebox[\textwidth][c]{%
    \begin{tabular}{@{}cccc@{}}
        %cendrosa
        %1507
        \begin{subfigure}[t]{0.380\textwidth}
            \caption{}
            \includegraphics[width=\textwidth]{images/chap6/profiles/profile_cendrosa_wind_speed_1507_sensbins.png}
        \end{subfigure} &
        \begin{subfigure}[t]{0.283\textwidth}
            \caption{}
            \includegraphics[width=\textwidth]{images/chap6/profiles/profile_cendrosa_wind_direction_1507_sensbins.png}
        \end{subfigure} &
        %2007
        \begin{subfigure}[t]{0.283\textwidth}
            \caption{}
            \includegraphics[width=\textwidth]{images/chap6/profiles/profile_cendrosa_wind_speed_2007_sensbins.png}
        \end{subfigure} &
        \begin{subfigure}[t]{0.283\textwidth}
            \caption{}
            \includegraphics[width=\textwidth]{images/chap6/profiles/profile_cendrosa_wind_direction_2007_sensbins.png}
        \end{subfigure} \\
        %elsplans
        %1507
        \begin{subfigure}[t]{0.382\textwidth}
            \caption{}
            \includegraphics[width=\textwidth]{images/chap6/profiles/profile_elsplans_wind_speed_1507_sensbins.png}
        \end{subfigure} &
        \begin{subfigure}[t]{0.283\textwidth}
            \caption{}
            \includegraphics[width=\textwidth]{images/chap6/profiles/profile_elsplans_wind_direction_1507_sensbins.png}
        \end{subfigure} &
        %2007
        \begin{subfigure}[t]{0.283\textwidth}
            \caption{}
            \includegraphics[width=\textwidth]{images/chap6/profiles/profile_elsplans_wind_speed_2007_sensbins.png}
        \end{subfigure} &
        \begin{subfigure}[t]{0.283\textwidth}
            \caption{}
            \includegraphics[width=\textwidth]{images/chap6/profiles/profile_elsplans_wind_direction_2007_sensbins.png}
        \end{subfigure} \\
    \end{tabular}
    }
    \caption{Vertical wind profiles at 12UTC on 15 and 20 July, at La Cendrosa (a-d) and Els Plans (e-h). Dashed lines show the mean profiles for the minimum and maximum sensible heat fux bins of Fig. \ref{fig:sens_bins}.}
    \label{fig:profiles_winds_sensbins}
\end{figure}

%Fig : MesoNH IOP vertical winds
\begin{figure}[hbtp]
    \centering
    \begin{tabular}{cc}
        %10m
        \begin{subfigure}[t]{0.5\textwidth}
            \caption{15 July at 10 m}
            \includegraphics[width=\textwidth]{images/chap6/IOP_maps/mesoNH_vertwind_10m_2021-07-15T12:00:00.png}
        \end{subfigure} &
        \begin{subfigure}[t]{0.5\textwidth}
            \caption{20 July at 10 m}
            \includegraphics[width=\textwidth]{images/chap6/IOP_maps/mesoNH_vertwind_10m_2021-07-20T12:00:00.png}
        \end{subfigure} \\
        %950hPa
        \begin{subfigure}[t]{0.5\textwidth}
            \caption{15 July at 950 hPa}
            \includegraphics[width=\textwidth]{images/chap6/IOP_maps/mesoNH_vertwind_950_2021-07-15T12:00:00.png}
        \end{subfigure} &
        \begin{subfigure}[t]{0.5\textwidth}
            \caption{20 July at 950 hPa}
            \includegraphics[width=\textwidth]{images/chap6/IOP_maps/mesoNH_vertwind_950_2021-07-20T12:00:00.png}
        \end{subfigure} \\
        %900hPa
        \begin{subfigure}[t]{0.5\textwidth}
            \caption{15 July at 900 hPa}
            \includegraphics[width=\textwidth]{images/chap6/IOP_maps/mesoNH_vertwind_900_2021-07-15T12:00:00.png}
        \end{subfigure} &
        \begin{subfigure}[t]{0.5\textwidth}
            \caption{20 July at 900 hPa}
            \includegraphics[width=\textwidth]{images/chap6/IOP_maps/mesoNH_vertwind_900_2021-07-20T12:00:00.png}
        \end{subfigure} \\
        %850hPa
        \begin{subfigure}[t]{0.5\textwidth}
            \caption{15 July at 850 hPa}
            \includegraphics[width=\textwidth]{images/chap6/IOP_maps/mesoNH_vertwind_850_2021-07-15T12:00:00.png}
        \end{subfigure} &
        \begin{subfigure}[t]{0.5\textwidth}
            \caption{20 July at 850 hPa}
            \includegraphics[width=\textwidth]{images/chap6/IOP_maps/mesoNH_vertwind_850_2021-07-20T12:00:00.png}
        \end{subfigure} \\
    \end{tabular}
    \caption{Vertical wind speed simulated by Meso-NH over the LIAISE observations sites at 12UTC on 15 and 20 July, at 10m (a-b), 950hPa (c-d), 900hPa (e-f), and 850hPa (g-h).}
    \label{fig:iop_days_vertwinds}
\end{figure}


%Fig : LMDZ IOP winds (10m + 850hPa)
\begin{figure}[hbtp]
    \centering
    \begin{tabular}{cc}
        %10m
        \begin{subfigure}[t]{0.5\textwidth}
            \caption{15 July at 10 m}
            \includegraphics[width=\textwidth]{images/chap6/IOP_maps/lmdz_wind10m_2021-07-15_12UTC.png}
        \end{subfigure} &
        \begin{subfigure}[t]{0.5\textwidth}
            \caption{20 July at 10 m}
            \includegraphics[width=\textwidth]{images/chap6/IOP_maps/lmdz_wind10m_2021-07-20_12UTC.png}
        \end{subfigure} \\
        %850hPa
        \begin{subfigure}[t]{0.5\textwidth}
            \caption{15 July at 850 hPa}
            \includegraphics[width=\textwidth]{images/chap6/IOP_maps/lmdz_wind850_2021-07-15_12UTC.png}
        \end{subfigure} &
        \begin{subfigure}[t]{0.5\textwidth}
            \caption{20 July at 850 hPa}
            \includegraphics[width=\textwidth]{images/chap6/IOP_maps/lmdz_wind850_2021-07-20_12UTC.png}
        \end{subfigure} \\
    \end{tabular}
    \caption{Wind speed simulated by the ICOLMDZOR LAM over the LIAISE observations sites at 12UTC on 15 and 20 July, at 10m (a-b) and 850hPa (c-d).}
    \label{fig:iop_days_LMDZ_winds}
\end{figure}
\clearpage

%end of appendix

% List of Figures and Tables (with clickable links)
\newpage
\listoffigures
\addcontentsline{toc}{chapter}{List of figures} % Add to TOC

\newpage
\listoftables
\addcontentsline{toc}{chapter}{List of tables} % Add to TOC

% Bibliography (with entry in TOC)
\newpage
\printbibliography[heading=bibintoc, title={Bibliography}]

\end{document}