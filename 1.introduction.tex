\section{Irrigation in the world and in the Iberian Peninsula}
\label{sec:irrig_context}
Irrigation is an ancestral agricultural practice which consists in providing additional water to cultivated soil to favour crop growth.
This practice is widespread globally in various forms and enables agriculture to maintain yields that would not be achievable in many regions otherwise \citep{siebert_quantifying_2010}.
Multiple techniques have been developed for irrigation, as a response to different contexts regarding available water quantities, technological developments, and the type of crops grown (Fig. \ref{fig:irrig_methods}).
Gravity irrigation (also called surface irrigation), the most ancient method, is conducted by diverting water from natural sources or artificial storage reservoirs to the fields. In particular, flood irrigation methods cover entire fields with water, saturating the upper layers of the soil. It is rather simple to implement as it requires little infrastructure and technology, but remains largely inefficient as a large share of the water is evaporated without directly benefiting the crops, or leaves the field as surface runoff or by infiltrating.
For an equivalent volume of water, sprinkler irrigation enables higher crop transpiration by spraying the water in the air above the fields, avoiding large infiltration and runoff (Fig. \ref{fig:irrig_efficiency}). However, evaporation remains high because part of it is intercepted by the leaves or evaporated in the atmosphere instead of directly reaching the soil and roots of crops, like natural rainfall. This method requires more infrastructure to cover large surfaces and external energy to power the sprinklers.
Finally, drip irrigation uses pipes to provide smaller amounts of water directly at the base of the crops. If managed properly, this method has a much higher water use efficiency since it avoids excessive runoff, infiltration and evaporation of water. Its downside is that it requires a lot of equipment that has to be maintained to remain efficient, but that does not prevent this method from being increasingly used \citep{mpanga_decade_2021, pool_flood_2021}. Beyond the method used, improved efficiency does not necessarily lead to lower withdrawals, as a consequence of a frequent rebound effect, and might even increase them (Jevons paradox) as analysed in \citet{grafton_paradox_2018}.

\begin{figure}[hbtp]
    \centering
    \includegraphics[width=0.5\textwidth]{images/intro/irrigation_methods_lunel.png}
    \caption{Schematic representation of the three main irrigation techniques: (a) flood irrigation, (b) sprinkler irrigation, and (c) drip irrigation. Taken from \citet{lunel_interactions_2024}.}
    \label{fig:irrig_methods}
\end{figure}

\begin{figure}[hbtp]
    \centering
    \includegraphics[width=0.8\textwidth]{images/intro/irrigation_efficiency_grafton.png}
    \caption{Estimated share of irrigation water contributions to crop transpiration, evaporation, surface runoff and infiltration, for the three main irrigation methods. Taken from \citet{grafton_paradox_2018}.}
    \label{fig:irrig_efficiency}
\end{figure}

\subsection{The increasing use and impacts of irrigation}

Irrigated fields are estimated to cover about 20\% of global cropland (3.5 million km$^2$), which account for 40\% of the food produced in the world according to the United Nations Food and Agriculture Organization \citep{UNESCO2019,mcdermid_irrigation_2023}.
The historical reconstruction by \citet{siebert_global_2015} estimates that the total irrigated area increased fivefold during the 20th century due to population growth and industrialization, rising from 63 Mha in 1900 to 306 Mha in 2005. Certain regions stand out, such as South Asia, the Western United States of America (USA), Eastern China, and Western Europe (Figure \ref{irrig_evolution_map}).
This trend has continued in the 21st century, with an 11\% increase in irrigated areas from 2000 (297 Mha) to 2015 (330 Mha) according to \citet{mehta_half_2024}, who insist on the fact that at least half of this expansion occurred in water-stressed regions, raising sustainability concerns.
In the future, irrigation demand is projected to keep rising in most scenarios, especially during the summer in the northern hemisphere as a consequence of climate change \citep{wada_multimodel_2013,busschaert_net_2022}. 

\begin{figure}[ht]
    \centering
    \includegraphics[width=\textwidth]{images/intro/irrig_evolution_Siebert.png}
    \caption{Percentage of area equipped for irrigation in 1900, 1960, and 2005 according to the Historical Irrigation Dataset (HID). Taken from \citet{siebert_global_2015}.}
    \label{irrig_evolution_map}
\end{figure}

Irrigation already has a major impact on water resources management, representing about 70\% of freshwater withdrawals globally, and 84-90\% of annual freshwater consumption \citep[``withdrawn water that is evaporated, transpired, incorporated into products or crops, or consumed by humans or livestock", ][]{ mcdermid_irrigation_2023, qin_flexibility_2019}.
Although disentangling its impacts from climate variability, climate change and other anthropogenic processes is challenging, it has been identified as an important driver of groundwater depletion \citep{siebert_groundwater_2010,smith_estimating_2017, thomas_identifying_2019}. 
Similarly, the large withdrawals and infrastructures associated with irrigation have been shown to reduce streamflow \citep{biemans_impact_2011, pokhrel_recent_2016, vicente-serrano_climate_2019}. 
Through interactions with the atmosphere, irrigation can also influence climate, on both regional \citep{nocco_observation_2019} and global scales \citep{puma_effects_2010, cook_irrigation_2015,arboleda-obando_feedback_2023, arboleda-obando_joint_2025}. This aspect is a major focus of this thesis, and a more detailed review of known impacts and processes is provided in Section \ref{sec:irrig_landatmosphere}.
From a broader perspective, considering that irrigation is often instrumental in enabling the expansion of agriculture, which has a wide range of other impacts (land use changes, disruption of biodiversity, concentration of chemical components and pollutants, direct and indirect CO$_2$ emissions), the influence of irrigation on natural systems go beyond freshwater resources \citep{campbell2017agriculture}.

All the impacts mentioned above remain very complex to isolate, especially since large uncertainties remain in the quantification and modelling of irrigation itself, as analysed in \citet{mcdermid_irrigation_2023}:
\begin{quote}
    ``Major uncertainties and gaps remain in irrigation research. High-resolution spatio-temporal data sets of the area actually irrigated annually, crop species and calendars, irrigation methods, amounts and timing are critical to model irrigation. However, no large area data sets exist for all these parameters. Furthermore, uncertainties in existing irrigated area estimates, a critical input for irrigation models, introduce large variation in irrigation predictions that might be partly irreducible."
\end{quote}
In spite of this complexity in the processes involved, \citet{puy_irrigated_2021} found that the irrigation withdrawals estimated by the Food and Agriculture Organization (FAO) or simulated by global models are largely proportional to irrigated areas. This conclusion suggests that model refinements to account for all specificities of irrigation might be overly complex.

\subsection{Irrigation in a fast-changing semi-arid climate: the case of the Iberian Peninsula}
The Iberian Peninsula is a continental area of approximately 582,480 km$^2$, bordered by the Atlantic Ocean to the north and west, and by the Mediterranean Sea to the east and south. The Pyrenees mountain range forms its natural boundary with the rest of continental Europe. Other topographical features include the Cantabrian range along the northern coast, an elevated plateau which regroups the Iberian and Central mountain systems, and the Baetic system in the south-east which contains the Sierra Nevada.
The Peninsula comprises five major river basins: Ebro, Douro, Tagus, Guadiana, and Guadalquivir, identified in Fig. \ref{fig:IP_map_fonseca}.

%peninsula topography
\begin{figure}[h!]
    \centering
    \begin{subfigure}{0.9\textwidth}
        \includegraphics[width=\linewidth]{images/intro/ip_topography_fonseca_mountains.png}
    \end{subfigure}
    \begin{subfigure}{0.9\textwidth}
        \includegraphics[width=\linewidth]{images/intro/land_cover_fonseca.png}
    \end{subfigure}
    \caption{The Iberian Peninsula: (a) topography and major rivers, (b) water-dependent land cover in major basins according to the 2020 \href{https://land.copernicus.eu/en/products/corine-land-cover}{CORINE} database. Adapted from \citet{fonseca_agricultural_2022}.}
    \label{fig:IP_map_fonseca}
\end{figure}

Irrigation has long been present in this region, as demonstrated by archaeological studies of the Bronze Age \citep[2140-1500 BCE, ][]{mora-gonzalez_isotopic_2016} or in the Roman and Islamic eras \citep[250-800, ][]{butzer_irrigation_1985}. 
Nowadays, it is an essential agricultural practice, as illustrated by the share of land covers identified as water-dependent in \citet{fonseca_agricultural_2022}, shown in Fig. \ref{fig:IP_map_fonseca}b. In 2020, the share of these water-dependent land covers is 13.7\% for the Ebro watershed, 8.3 \% in the Douro watershed, 8.0\% in the Tagus watershed, 16.9\% in the Guadiana watershed, and 39.1\% in the Guadalquivir watershed.
According to the \href{https://data.apps.fao.org/aquastat/?lang=en}{AQUASTAT} database \citep[presented in][and frequently updated since]{frenken_aquastat_2012}, annual irrigation withdrawals in Spain and Portugal were estimated at 22.8 km$^3$ on average over the period 2010-2022.
The Ebro and Guadalquivir basins are particularly known for their intensive irrigation practices, and irrigated areas are still expanding in the 21st century, with respective increases of 40\% and 53\% between 1990 and 2020 \citep{fonseca_agricultural_2022}. A large share of these cultivated land covers falls into semi-arid areas like the Ebro valley ("steppe" climates in the Köppen-Geiger classification), or temperate with dry summers like the Guadalquivir valley (Fig. \ref{fig:IP_climate}a).

%peninsula climate and precip
\begin{figure}[h!]
    \centering
    \begin{subfigure}{0.8\textwidth}
        \includegraphics[width=\linewidth]{images/intro/koppen_IP_correa.png}
    \end{subfigure}
    \vspace*{0.5cm}
    
    \begin{subfigure}{0.8\textwidth}
        \includegraphics[width=\linewidth]{images/intro/precip_IP_1970-2000.png}
    \end{subfigure}
    \caption{(a) Köppen-Geiger classification of the Iberian Peninsula (1991-2020), taken from \citet{correa_analysis_2024} based on \citet{beck_high-resolution_2023}. (b) Annual mean precipitation (1970-2000) according to the Iberian climate atlas \citep{cunha_atlas_2011}.}
    \label{fig:IP_climate}
\end{figure}

These climates explain the high reliance on artificial methods to sustain crop  water supply throughout the year. As a result, diverse irrigation infrastructures have been developed across regions, adapted to local conditions and needs.
River dams with a wide range of reservoir sizes can store water in winter and spring when surface runoff and river discharge are high (see Fig. \ref{fig:selected_stations} and associated discussion on river dams), making it available later in the year for irrigation and other services such as hydropower generation.
According to the \citet{ICOLD2020}, there are at least 1064 large dams in Spain, making it one of the top 10 dam-building countries \citep{sadki_implementation_2023}.
Mountainous areas receive most of the Peninsula's precipitation (Fig. \ref{fig:IP_climate}b), including in the form of snow, and canals are used to transport water to cultivated areas in the drier valleys. The Canal d'Urgell \citep{farran_urgell_2024} in the Segre sub-basin of the Ebro watershed, further described in Chapter \ref{chap:liaise}, is a very good example of this type of infrastructure.
Finally, groundwater pumping is also a major water source for irrigation. It was estimated to provide water for one third of irrigated areas in Spain in \citet{de_stefano_groundwater_2015}, and its general role in freshwater supplies is still increasing \citep{llamas_groundwater_2015}. Figure \ref{fig:irrig_inputs}b shows a map, used as input for the simulations presented in this thesis, that quantifies the relative importance of surface and groundwater withdrawals for irrigation over the Iberian Peninsula \citep[based on ][]{siebert_groundwater_2010}.
Some of these resources, particularly in the arid regions of the south-east, come from so-called \textit{fossil} aquifers, and the term \textit{groundwater mining} is used for withdrawals in aquifers where recharge takes more than 50 years \citep{custodio_groundwater_2016}.
The intensification of these practices is responsible for rapid groundwater depletion which poses risks for the quality and future availability of water resources, raising economic and ethical concerns \citep{custodio_groundwater_2017}.

Other concerns for the sustainability of irrigation arise from ongoing and future climate change in the region. The Intergovernmental Panel on Climate Change 
\citep{RN1} identified arid and semi-arid regions, like the Mediterranean basin \citep{giorgi_climate_2006}, as very sensitive to adverse impacts of a global temperature increase, some of which are already underway and visible in observations.
This includes more frequent and intense extreme events \citep{dominguez-castro_high_2019}, but also long-term increases in temperature \citep{pena-angulo_seasonal_2021,gonzalez-hidalgo_variability_2022}. Using observation data in Spain over the last 100 years, \citet{gonzalez-hidalgo_seasonal_2024} identified a drying trend, that mostly occurs in the beginning of the 20th century, as no significant change was found over the period 1976-2015. Yet, they identified recent shifts in spatial and seasonal patterns which could disrupt the availability of water resources over the Peninsula, mainly a decrease in spring and winter precipitation in western Spain, compensated by increases in autumn (Fig. \ref{fig:IP_precip_trends}). 

\begin{figure}[hbtp]
    \centering
    \begin{subfigure}{\textwidth}
        \caption{Annual and seasonal precipitation trends (1916-2015).}
        \includegraphics[width=\linewidth]{images/intro/precip_trend_gonzalez1.png}
    \end{subfigure}
    \vspace*{0.5cm}
    
    \begin{subfigure}{0.8\textwidth}
        \caption{Changes in dominant precipitation season over Spain.}
        \includegraphics[width=\linewidth]{images/intro/precip_trend_gonzalez2.png}
    \end{subfigure}
    \caption{Figures taken from \citet{gonzalez-hidalgo_seasonal_2024} showing changes in precipitation identified over Spain. (a) Annual and seasonal precipitation trends from 1916 to 2015. Four-class classification: positive significant (dark blue), positive not significant (light blue), negative not significant (light red) and negative significant (dark red). Mann-Kendall test, significance level $\alpha$=0.10. (b) Sankey diagram of changes in dominant precipitation regimes between four 25-year sub-periods. The percentage for each season indicates the areal fraction of the country where this season is the dominant precipitation season.}
    \label{fig:IP_precip_trends}
\end{figure}

For the future, global change scenarios generally predict a warming and decrease in annual precipitation over the region \citep{pereira_temperature_2021, arjdal_future_2023}, with associated consequences on soil moisture, evapotranspiration, and vegetation \citep{RN1, nunes_effects_2023}.
In particular, the concepts of desertification and aridification, are being increasingly studied in the region, to analyse ongoing trends \citep{paniagua_aridity_2019, begueria_aridity_2025}, and evaluate adaptation policies \citep{van_leeuwen_evolution_2019, MITECO2022}. For instance, a shift in agricultural practices, favouring rain-fed agriculture is considered essential to reduce the detrimental impacts of irrigation on regional water security \citep{eekhout_how_2024}. In the near future, the ongoing shift from flood to drip irrigation in the region of Valencia was predicted to have a larger influence on the decrease of groundwater recharge than climate change \citep{pool_flood_2021}, highlighting the need for comprehensive assessments of the impacts of irrigation. Using a distributed hydrological model, \citet{von_gunten_estimating_2015} studied a sub-catchment of the Ebro basin which transitioned from rain-fed to sprinkler-irrigated agriculture in the 21st century, by adducting water from outside the catchment. Simulations under present and future conditions with and without irrigation showed increases in groundwater storage over irrigated areas, but also identified a higher sensitivity of groundwater storage and river flow to decreases in precipitation under climate change in the presence of irrigation, and insisted on the need to account for atmospheric feedback to enable precise projections.

\hfill

This thesis aims to contribute to the understanding of the relations between irrigation, climate, and the water cycle over the Iberian Peninsula. 
The following sections first present the main drivers of the climate system and the processes at the interface between land surface and atmosphere, through which irrigation can interact with the regional climate. The modelling tools used to study the climate and its evolution are then described, along with the importance and representation of land-atmosphere interactions in these models. Finally, the established impacts of irrigation on these interactions and remaining knowledge gaps are discussed, providing the necessary context for the questions addressed in this thesis.

% \clearpage

\section{Near-surface climate and land-atmosphere interactions}
\label{sec:l-a_interactions}
%distinction avec la météo (en temps que science, pas outils)
%variables d'intérêt: température precip vents humidité, ressources en eau, glace
%focus sur la surface (choix de la thèse) 
% bilans d'eau et energie
%explication de ce qu'est l'ABL (tout ou juste le début ?)
% impacts humidité du sol sur l'atmosphère

Climatology is the study of climate, defined as the statistics of weather conditions \citep{cnrs_climat_2017}, \textit{i.e.} the state of the atmosphere at a given time and place.
The atmospheric phenomena studied cover a wide range of spatio-temporal scales \citep{malardel2005fondamentaux}. 
The \textit{planetary scale} regroups persistent circulations over several weeks or months that cover a large share of the globe (of characteristic spatial dimension $\sim$10.000 km) such as the Hadley \citep{Hadley1735} and Ferrel \citep{Ferrel1856} cells.
\textit{Synoptic scale} processes designate large structures ($\sim$1000 km) that last for several days, such as high and low pressure systems.
Phenomena that range from $\sim$5-100 km with a characteristic timescale of a few hours are called \textit{mesoscale} processes (\textit{e.g} sea breeze circulations). 
Below that scale, short-lived phenomena are regrouped in the \textit{microscale} ($\sim$1-1000 m).

Climatology is fundamentally distinct from meteorology, although they both aim to describe the state of the atmosphere, due to the temporal scales considered. 
Meteorology focuses on short-term deterministic phenomena that govern the weather at a given time, and will influence its evolution in the following hours or days. Its operational application is to forecast weather conditions in the near future (typically up to two weeks). 
Climatology, in contrast, focuses on the statistics of weather conditions over decadal to centennial timescales (typically 30 years or more). It aims to characterise the long-term mean state, variability, and the frequency of weather events (including extreme ones). 

The main atmospheric variables of interest for human societies are near-surface temperature, humidity, wind (both its speed and direction), cloud cover and precipitation. 
These variables have long been measured and recorded around the globe, particularly since the International Meteorological Organization was founded in 1873 (later evolving into the World Meteorological Organization in 1950).
Their evolution on climatic timescales is largely driven by the energy balance at the global scale \citep{hartmann2015global}, and the climate system is often described as transporting solar energy, mostly received near the equator, to the rest of the planet \citep{edwards_history_2011}.
This global energy balance is the result of interactions between the main components of the climate system: the atmosphere, the hydrosphere (oceans and freshwater), the cryosphere (ice caps, glaciers, permafrost and sea ice), the lithosphere (land surface) and the biosphere (living beings).
It relies on essential intermediate processes such as atmospheric and oceanic circulations (determined by pressure and density gradients and the Earth's rotation), biogeochemical cycles (water, carbon, nitrogen), and surface energy fluxes \citep{hartmann2015global}.

This thesis focuses on interactions between land surfaces and the atmosphere to study the impacts of irrigation on the climate and water cycle of the Iberian Peninsula.
Land-atmosphere interactions can have significant impacts on meteorological (air temperature and humidity, precipitation, wind) and hydrological (runoff, stream flow, soil moisture) variables, and on the vegetation \citep{bonan2015ecological}.
The two systems influence each other's water and energy budgets, and multiple feedback loops can be identified depending on the spatial and temporal scales considered, many of which include soil moisture \citep{seneviratne_investigating_2010}, which is directly impacted by irrigation.
The following sections describe the water and energy budgets at the Earth's surface and identify the key variables and processes in land-atmosphere interactions.

\subsection{Water and energy budgets at the surface}
\label{sec:water_energy_budgets}
To understand the components of the land-atmosphere interactions, it is necessary to recall the main fluxes of water and energy at the surface. 
Figure \ref{fig:budgets} depicts the various components of these two budgets.

\begin{figure}[ht]
    \centering
    \includegraphics[width=\textwidth]{images/intro/budgets.png}
    \caption{Surface water and energy budgets. Based on similar figures from \citet{seneviratne_investigating_2010,lunel_interactions_2024}
    }
    \label{fig:budgets}
\end{figure}

\noindent The surface water budget accounts for:
\begin{itemize}
    \item \textbf{Precipitation} ($P$), transfers from the atmosphere to the surface in liquid or solid form.
    \item \textbf{Evapotranspiration} ($E$), transfer of liquid or solid surface water to the atmosphere in gaseous form. It is the sum of direct evaporation and of transpiration, a process in which vegetation releases water is has collected in the soil. Evaporation occurs on bare soil, but also on water intercepted by vegetation (interception) and on snow surfaces (sublimation).
    \item \textbf{Drainage} of water in the soil to lower layers, denoted  as $D$. 
    \item \textbf{Surface runoff} ($R_{surf}$), water that does not infiltrate in the soil and flows out of the area considered.
\end{itemize}

The equation governing the evolution of water quantity in the upper soil layer (denoted here as $W_{surf}$) is:
\begin{equation}
    \frac{dW_{surf}}{dt} = P - E - R_{surf} - D
\end{equation}

\hfill

\noindent The surface energy budget includes:
\begin{itemize}
    \item \textbf{Shortwave radiation} (SW), with an incoming term ($SW_{dn}$) corresponding to incident solar radiation and an outgoing term ($SW_{up}$) corresponding to the portion reflected by the surface.

    The difference between these two terms and the albedo of the considered surface are defined as:

    $SW_{net} = SW_{dn} - SW_{up}$

    $\alpha_{SW} = SW_{up}/SW_{dn}$.
    \item \textbf{Longwave radiation} (LW), with an incoming term ($LW_{dn}$) corresponding to the infrared radiation emitted by clouds and atmospheric gases that reaches the surface, and an outgoing term ($LW_{up}$) corresponding mostly to the infrared radiation emitted by the surface based on its temperature, as well as a partial reflection of the incoming LW radiation.

    The difference between these two terms is defined as $LW_{net} = LW_{dn} - LW_{up}$.
    
    \item \textbf{Sensible heat flux} ($H$), which is a thermal energy transfer between the surface and the air.
    \item \textbf{Latent heat flux}, which corresponds to the energy used to evaporate water at the surface and the energy provided by the surface to the atmosphere under latent form. This energy flux is directly related to evapotranspiration (water flux) through the enthalpy of vaporization ($\lambda$), and is therefore denoted as $\lambda E$.
    \item \textbf{Heat flux into the soil} ($G$), which is a thermal conduction transfer between the considered surface layer and the lower soil layers.
\end{itemize}

The equation governing the evolution of energy in the surface soil layer (denoted here as $E_{surf}$) is:
\begin{equation}
    \frac{dE_{surf}}{dt} = R_{n} - G - \lambda E - H
\end{equation}
where $R_{n} = SW_{net} + LW_{net}$ is the net radiation at the surface.

\subsection{The atmospheric boundary layer}

The atmospheric boundary layer (ABL) is defined as the lower part of the atmosphere  directly influenced by the presence of the surface. 
The height of the boundary layer varies during the diurnal cycle depending on local conditions, and can measure a few tens of metres at night and up to a few kilometres during the day in arid regions.
This thesis only discusses the convective boundary layer over land surfaces, although several other categories of ABL are typically defined, such as the nocturnal, cloud-topped, or marine boundary layers \citep{garratt1994atmospheric}.

\subsubsection*{ABL stability and structure}
Vertical motions of air in the ABL are controlled by buoyancy forces that arise from the presence of vertical density gradients, and by mechanical turbulence (winds and turbulence).
The boundary layer is considered stable if the vertical gradient of density from the surface is negative (heavier air near the surface), and unstable if it is positive (lighter air near the surface), which is the case in the convective boundary layer.
The intensity of this instability is modulated by the presence of horizontal wind shear (wind speed gradient), which can limit the ascending motions induced by buoyant forces.
Vertical gradients of air density are primarily driven by temperature and water vapour content (which is lighter than dry air), making them essential variables for the description of the ABL. However, atmospheric pressure decreases with altitude and induces a decrease in temperature, which does not reflect changes in buoyancy. Therefore, rather than the absolute temperature ($T$, in K), gradients of potential temperature are used, to account for the variations in atmospheric pressure. This variable ($\theta$, in K), dependent on pressure $P$ (hPa) is the temperature that a parcel of air would attain if adiabatically brought to a standard reference pressure $P_0$ (usually 1000 hPa):
\begin{equation}
    \theta = T \left(\frac{P}{P_0}\right)^{R/C_p}
\end{equation}
where $R$ is the specific gas constant of air and $C_p$ is the specific heat capacity at a constant pressure. For air, $R/C_p=0.286$.

\hfill

The lowest part of the boundary layer is called the \textit{surface layer}. 
In these first few tens of metres, the influence of the Earth's rotation on motions of air is negligible compared to that of the surface, which generates strong gradients of wind, temperature and humidity. The empirical similarity theory developed by Monin and Obukhov describes these variables in this layer \citep{monin1954osnovnye}. 
The wind speed generally follows a logarithmic profile, which is very dependent on the roughness of the surface and background wind.
On a sunny day, the surface layer is warmed by the soil via conductive and convective energy transfers quantified by the sensible heat flux, and moistened by evapotranspiration. Close the surface, the air is therefore warmer, moister, and therefore less dense than the air above, leading to the formation of thermal updraughts that transport moisture and energy (as heat and in latent form). 
These ascending motions, along with turbulent diffusion processes, contribute to the mixing of air masses above the surface layer. Most of the ABL is therefore constituted by the \textit{mixed layer}, directly above the surface layer, which exhibits vertically homogeneous profiles of humidity, potential temperature, and, to a lesser extent, of wind (Fig. \ref{fig:abl_structure}).
This layer is capped by the \textit{entrainment layer}, which is typically characterised by a rapid increase in potential temperature, a decrease in humidity, and faster winds. 
This marks the transition to the free troposphere, where the influence of surface conditions is weak in comparison to large-scale conditions.

\begin{figure}[ht]
    \centering
    \includegraphics[width=0.7\textwidth]{images/intro/abl_lunel.png}
    \caption{Typical vertical profiles of potential temperature ($\theta$), specific humidity ($q$) and wind speed ($V$) in the convective atmospheric boundary layer. Taken from \citet{lunel_interactions_2024}, based on \citet{stull2012introduction}.}
    \label{fig:abl_structure}
\end{figure}

\subsubsection*{ABL development over heterogeneous terrain}

The development of the ABL during the day is driven by vertical motions of air, and therefore largely dependent on the surface sensible heat flux, which quantifies the warming of near-surface air. This warming leads to the creation of thermals, that merge into structured convective updraughts \citep[$\sim$100 m wide,][]{garratt1994atmospheric}. In these plumes, the air rises rapidly, which is compensated by subsidence of the surrounding air masses (downdraughts). 
The presence of these structures can be strongly influenced by the distribution of sensible heat flux at the surface. They are also very sensitive to the characteristics of wind in the ABL, which can enhance turbulent mixing, disrupt vertical gradients of potential temperature and humidity, or increase ABL extent via the local convergence of air masses that result in upward motions. 
Consequently, the presence of surface heterogeneities, including those created by human activities like urban areas, cultivated land, or artificial water bodies can affect the structure of the ABL in a wide range of ways that might never be fully understood or modelled \citep{bou-zeid_persistent_2020}. 
The review by \citet{garratt_review_1994} already mentioned the importance of their organisation (Fig. \ref{fig:abl_heterog}), and identified 10 km as a threshold length scale to observe different ABL structures over heterogeneous terrain. 
More recently, \citet{bou-zeid_persistent_2020} distinguished four classes of heterogeneities: semi-infinite interfaces with a single transition between two large surfaces ($>$ 100 km); large individual patches ($\sim$10 km, such as cities or lakes); repeating patterns of small patches ($\sim$1 km); and unstructured heterogeneities. 
The large contrasts in the first and second cases can initiate mesoscale circulations, like sea and lake breezes \citep{crosman_sea_2010,kenny_numerical_2017}, and similar phenomena at the transition between cities and rural landscape \citep{hidalgo_urban-breeze_2008,omidvar_plume_2020} or along large rivers \citep{zhang_large-eddy_2023}. In the third category, if heterogeneous patches are smaller than the ABL characteristic height, it is often possible to assume that differences are limited to the surface layer but disappear in the mixed layer, as shown for type A in Fig. \ref{fig:abl_heterog}. The fourth type of heterogeneities, although most common, is still poorly understood and accounted for in models.
In all these cases, the strength and direction of the ambient low-level wind is essential, affecting advection terms, and the presence of internal boundary layers (rather homogeneous sub-layers within the mixed layer) or secondary circulations \citep{anderson_turbulent_2020, zhang_large-eddy_2023}. 

\begin{figure}[ht]
    \centering
    \includegraphics[width=0.9\textwidth]{images/intro/abl_heterog_garratt.png}
    \caption{Schematised ABL behaviour over two types of land surfaces. In type A, the surface structure is disorganised at length scales of 10 km or less, in type B, there is organisation at length scales greater than 10 km. Taken from \citet{garratt_review_1994}.}
    \label{fig:abl_heterog}
\end{figure}

\subsection{Role of soil moisture in land-atmosphere interactions}

The term \textit{interactions} between the surface and the atmosphere encompasses multiple couplings \citep[this word is used here as the one-way controls of one variable on another, as in][]{seneviratne_investigating_2010} and feedbacks (two-way interactions resulting of multiple reciprocal couplings). Soil moisture, which is artificially increased in irrigated fields, plays a central role through its direct and indirect interactions with evapotranspiration, precipitation, and the surface energy budget.

\subsubsection*{Couplings between soil moisture and evapotranspiration}

Soil moisture is a key factor in the description of evapotranspiration regimes initially established by \citet{Budyko_1956, Budyko_1974}. Three main regimes are identified for the evolution of the evaporative fraction (defined here as $EF = \frac{\lambda E}{R_n}$, although $EF = \frac{\lambda E}{\lambda E+H}$ is sometimes used). Figure \ref{fig:evap_regimes} represents these regimes.

\begin{itemize}
    \item If soil moisture is below a threshold $\theta_{WILT}$, called the wilting point, plants cannot extract water to transpire, which often corresponds to a situation where bare soil no longer evaporates. Evapotranspiration, and thus the evaporative fraction, is therefore near-zero (evaporation of intercepted water can technically still occur). This first regime is called the \textbf{dry regime}.
    \item If soil moisture is above a critical threshold $\theta_{CRIT}$, soil moisture has no impact on $EF$, which is maximal. Evaporation is limited by the available incident energy, it is the \textbf{wet regime}.
    \item Between $\theta_{WILT}$ and $\theta_{CRIT}$, there is a \textbf{transition regime} where evapotranspiration is primarily conditioned by soil moisture. 
\end{itemize}

A decrease in soil moisture will therefore generally lead to a decrease in evapotranspiration, as figured by the A arrow in Fig. \ref{fig:sm_coupling}.

\begin{figure}[ht]
    \centering
    \includegraphics[width=0.7\textwidth]{images/intro/evap_regimes.png}
    \caption{Representation of different evapotranspiration regimes. Taken from \citet{seneviratne_investigating_2010}.}
    \label{fig:evap_regimes}
\end{figure}

There are also several controls of evapotranspiration on soil moisture. 
First, an increase in evapotranspiration directly leads to a decrease in soil moisture. 
In a transitional regime, this reduces evapotranspiration and constitutes a first negative (stabilizing) feedback loop attenuating strong variations of both soil moisture and evapotranspiration (combination of A and C arrows in Fig. \ref{fig:sm_coupling}). 
Evapotranspiration also contributes to increasing air humidity and reducing the vertical humidity gradient between the air and the surface, which tends to limit evaporation \citep{allen_crop_2000}, forming another negative feedback loop. 
A more complex feedback involving temperature also exists, and is described below.

\subsubsection*{Couplings between soil moisture and temperature}

Through its influence on evapotranspiration, soil moisture directly impacts the partitioning of energy between the two turbulent fluxes at the surface ($\lambda E$ and $H$). 
This can have an influence on the temperature in the surface layer and on the development of the ABL, which both largely depend on the sensible heat flux. 
For a given net radiation $R_n$ and a nearly constant soil heat flux (e.g., over 24 hours if there is equilibrium between daytime and nighttime), the remaining energy is distributed between the two turbulent fluxes. 
The evaporative fraction $EF$ and the Bowen ratio (defined as $B = \frac{H}{\lambda E}$) quantify this partitioning. 
If the latent heat flux is very high compared to the sensible heat flux (high $EF$, low $B$), the air temperature in the surface layer remains low because the energy is primarily used for evapotranspiration, and the soil transfers little heat to the air. Conversely, if the Bowen ratio is high, a larger portion of the incident energy is transmitted as heat, leading to a higher air temperature near the surface and more pronounced development of the boundary layer \citep{betts_fife_1995}.
In reality, the processes are more complex since the surface temperature also reacts to changes in latent heat flux, therefore affecting the $LW_{up}$ flux which means that $R_n$ is not constant. Yet, a reciprocal behaviour of the latent and sensible heat fluxes has been identified in multiple observations and modelling experiments \citep{betts_fife_1995, seneviratne_investigating_2010}.
Considering the coupling of evapotranspiration with soil moisture, this means that a higher soil moisture generally leads to colder surface and air temperature, while a decrease in soil moisture and evapotranspiration leads to a warming of the lower atmosphere (B arrow in Fig. \ref{fig:sm_coupling}).

However, the thermal properties of the soil are also affected by soil moisture, as wet soil has greater thermal inertia than dry soil. In arid and semi-arid regions, this can significantly impact air temperature by affecting nighttime cooling \citep{ait-mesbah_role_2015}. The absence of solar radiation at night usually leads to radiative cooling of the soil and air in the surface layer. However, for a wetter soil, this cooling will be less pronounced due to higher thermal inertia. 
During daytime, the impact of this inertia is often negligible compared to incoming solar energy, and other processes dominate the evolution of air temperature. Still, on a daily average, an increase in soil moisture can lead to an increase in air temperature since night-time cooling is reduced \citep{cheruy_role_2017}.

It is also established that an increase in air temperature leads to higher evaporative demand \citep{jarvis_stomatal_1986}. 
If evaporation decreases, the warming described above will increase evapotranspiration 
(unless soil moisture is already below the wilting point), thus decreasing soil moisture (C arrows in Fig. \ref{fig:sm_coupling}). 
This can form a positive feedback loop where dry soil leads to high air temperatures and enhanced evapotranspiration, resulting in even drier soil and possibly initiating extreme drought events \citep{quesada_asymmetric_2012}. 
If no more water is available (dry regime) or if there are no more gradients between the soil surface and the air, the influence of air temperature on soil moisture becomes neutral, and the feedback loop is interrupted (hatched arrow in Fig. \ref{fig:sm_coupling}).

\begin{figure}[ht]
    \centering
    \includegraphics[width=0.5\textwidth]{images/intro/sm-et-t2m_coupling_seneviratne.png}
    \caption{Processes contributing to soil moisture–temperature coupling and feedback loop. Positive arrows (red) indicate processes that lead to a drying/warming in response to a negative soil moisture anomaly and blue arrows denote potential negative feedbacks (the hatched blue arrow indicates a negative feedback that is not always active). Note that possible links to the radiation (clouds and water vapour) and thermal inertia of the soil are not included. The displayed relationships similarly apply for positive soil moisture anomalies (leading to negative temperature anomalies). Taken from \citet{seneviratne_investigating_2010}.}
    \label{fig:sm_coupling}
\end{figure}

\subsubsection*{Couplings between soil moisture and precipitation}

The coupling of soil moisture with precipitation is rather straightforward, as rain or snowfall directly increases soil moisture where it occurs.
In contrast, the coupling of precipitation with soil moisture encompasses some of the most complex processes of land-atmosphere interactions. 
Precipitation occurs when the air becomes saturated with water vapour (reaching 100\% relative humidity), so that water condenses and precipitates. Two processes, possibly acting together, can lead water vapour saturation: cooling (which notably occurs in ascending motions) and moistening (either due to local processes or moisture advection). 
This mainly occurs as the consequence of convective processes (intense vertical motions associated with heavy localized precipitation), or large-scale upward motions associated with less intense stratiform precipitation. 
In the coupling of precipitation with soil moisture conditions, several opposing effects are at play, with a large importance of spatial heterogeneities of soil moisture at the surface, which may derive from the diversity of vegetation, soil types, orographic features, and anthropogenic processes like irrigation.

Increases in soil moisture and evapotranspiration have been associated with direct increases in precipitation in both modelling and observational studies \citep{koster_observational_2003, guo_glace_2006, wei_dissecting_2012, findell_probability_2011}, constituting a positive feedback loop \citep[moisture recycling, as presented in ][]{eltahir_precipitation_1996}.
However, high soil moisture can lead to a stabilisation of the ABL by reducing the sensible heat flux.
This stabilisation inhibits vertical development and convective processes involved in cloud formation and precipitation \citep{findell_atmospheric_2003-1, ek_influence_2004}. 
This constitutes a negative feedback loop where convective rainfall is more likely to occur over drier soil patches, which was noticed in observations \citep{taylor_afternoon_2012, klein_dry_2020}. 
These findings suggest that indirect processes related to boundary layer structure and heterogeneities can locally exceed the direct process of atmospheric moisture recycling, although demonstrating causality in such observational studies remains challenging \citep{salvucci_investigating_2002, guillod_land-surface_2014}. Building on the purely spatial analysis of \citet{taylor_afternoon_2012}, a spatio-temporal analysis of correlations between soil moisture and precipitation highlighted the importance of temporal variability \citep{guillod_reconciling_2015}. It showed that while precipitation is more frequently triggered over drier areas, it occurs on days that are wetter relative to the season and region concerned.
Finally, spatial heterogeneities in soil moisture have also been identified as factors influencing precipitation through mesoscale circulations that can either favour or inhibit convection triggering \citep{findell_atmospheric_2003, taylor_frequency_2011, rochetin_morphology_2017}.

\hfill

The specific impacts of irrigation on the climate, via the land-atmosphere processes described here, will be further reviewed in Section \ref{sec:irrig_landatmosphere}.
Before that, the next section gives an overview of the modelling tools used to study the climate, and how they account for land-atmosphere interactions.

\section{Numerical climate modelling}

For centuries, climatology remained a field of science based on observations, followed by the formulation of conceptual models, physical interpretation of wind patterns \citep{Halley1686, Hadley1735, Dove1837, Ferrel1856} and mathematical representations of the global energy balance and radiative transfer processes \citep{edwards_history_2011}.
The motions of the atmosphere (and of oceans) are largely driven by fluids mechanics, which can be described by the Navier-Stokes equations, derived in the 19th century. However these equations do not have any known analytical solution, which forbids their direct use to predict motions of air, water and other components of the atmosphere. 
Modern climate modelling originated in the 1950s with the development of computer simulations which enable numerically estimating solutions of these equations.

\subsection{From AGCMs to ESMs}

\begin{figure}[hbtp]
    \centering
    \includegraphics[width=0.7\textwidth]{images/intro/GCM_structure_kotamarthi.png}
    \caption{Representation of the structure of an AGCM. Parametrizations of the physics are applied to the vertical columns and the dynamics manages exchanges between columns. Taken from \citep{kotamarthi_downscaling_2021}. 
    }
    \label{fig:GCM}
\end{figure}

In atmospheric general circulation models (AGCMs), a simplified version of the Navier-Stokes equations, referred to as \textit{primitive equations} \citep[initially formulated in][]{bjerknes1910}, is used to represent the complex motions of the atmosphere.\footnote{Ocean general circulation models (OGCMs) based on similar principles also exist, but this thesis will not discuss them in details as it focuses on the interactions of the atmosphere with the land surface.}
These equations notably rely on the hydrostatic hypothesis, stating that the pressure at any height is equal to the weight of the air directly above it, which simplifies the computations by ignoring rapid vertical motions.
In an AGCM, the globe is discretized into grid cells ranging from a few tens to a few hundreds of kilometres \citep[scales at which this assumption is considered valid,][]{stull2012introduction}, and each atmospheric column is also discretized vertically. Using an appropriate time step to guarantee numerical stability \citep[such as Courant-Friedrichs-Lewy (CFL) conditions,][]{courant_partial_1967}, this enables approximating solutions of the primitive equations to represent the atmospheric circulation. 
This part of the AGCM is often referred to as \textit{the dynamical core} or simply \textit{the dynamics} of the model.

However, several major processes of the climate system are not described by fluid mechanics and require additional \textit{parametrizations}, which compute their average effect in each vertical column, independently of neighbouring grid cells (Fig \ref{fig:GCM}). 
This part of the AGCM is often referred to as \textit{the physics} of the model.
Most importantly, it represents all radiative emission and transfer processes, which largely dictate the energy budget of the atmosphere.
Thermodynamic processes involved in the phase changes of water are also essential to represent energy transfers between phases, cloud formation and precipitation. Parametrizations are also used to account for the processes that occur at a smaller scale than the resolution of the AGCM, or that are not accounted for in the primitive equations and cannot be described by the dynamical core, such as turbulent diffusion, shallow and deep convection, sub-grid orographic processes and gravity waves. 
Finally, parametrizations are used to represent interactions of the atmosphere with the various surfaces it can be interfaced with (sea surface, ice caps or sea ice, and continental surfaces), in particular to compute the surface water and energy budgets described in Section \ref{sec:water_energy_budgets}.

Historically, AGCMs used a very simplified representation of the other components of the climate system.
Considering the importance of oceans in the global energy balance, they were soon associated with ocean models in coupled atmosphere-ocean GCMs \citep[][]{manabe_climate_1969}.
Over the past five decades, climate models have evolved by further coupling atmosphere-ocean GCMs to specific land surfaces models (LSMs), ice-sheets models, atmospheric chemistry models, and explicitly representing biogeochemical cycles, especially the carbon cycle (Fig. \ref{fig:GCM_processes}).
Such climate models are often referred to as Earth system models (ESMs). They are not only used to simulate atmospheric variables, but can have a much wider range of applications in geoscience, palaeoclimatology, oceanography, glaciology, subsurface hydrology, biology, biogeochemistry... 
Regarding continental surfaces, the complexity of land surface models has gradually increased, for example, to better describe the continental water cycle (including river discharge and groundwater storage), and account for anthropic activities like irrigation or urbanisation.
To limit computing power, it is common to use the AGCM and LSM together but without coupling them to an ocean model. Instead, reference values for sea surface temperature (SST) and sea ice concentrations (SIC), obtained from observations or previous fully-coupled simulations, can be used. Similarly, ocean, ice-sheet or land surface models can be run as standalone models using atmospheric forcings instead of a dynamical coupling with an AGCM. 

\begin{figure}[hbtp]
    \centering
    \includegraphics[width=0.9\textwidth]{images/intro/climate_modelling_evolution_kotamarthi.png}
    \caption{Gradual implementation of new processes in climate models. Taken from \citep{kotamarthi_downscaling_2021}. 
    }
    \label{fig:GCM_processes}
\end{figure}

\subsection{A hierarchy of atmospheric models and scales}

A major limitation of AGCMs is computing power, which calls for a trade-off between the spatial and temporal resolution and the period covered by the simulation, to limit the actual time it will require to run on a given (super)computer. 
In most global climate simulations, their horizontal resolution is usually $\sim$100 km to limit the computing time required and enable multi-decadal studies, accounting for long-term feedbacks between the atmosphere and the other Earth system components. 
However, a wide range of other atmospheric models exist, that achieve much higher spatial resolutions while maintain acceptable computing times, either by simulating shorter periods, or by simulating only a small domain instead of the entire planet.
This section gives a brief overview of the existing atmospheric model classes, their applications, and their interest for climate modelling.

Regional climate models (RCMs) are mostly based on the same structure and assumptions as AGCMs but are used on restricted domains (typically a few hundred of kilometres). The ICOLMDZ limited area model (used in this thesis and presented in Section \ref{sec:ICOLMDZLAM}) is an example of RCM.
Running simulations over a limited area requires lateral boundary conditions (LBC) to take into account large-scale atmospheric features without simulating the global circulation.
The influence of the LBC is a well known problem for regional climate modelling or dynamical downscaling. 
\citet{denis_sensitivity_2003} analysed the dependency of the simulated climate to the resolution of the LBC and found that a 12 times lower spatial resolution in the forcing compared to the domain could still yield acceptable performance. This study also explored the influence of the temporal sampling and little difference between a 3-hour and a 6-hour sampling, but a degradation of performance with a 12-hour sampling frequency.
\citet{wu_estimating_2005} performed regional climate model simulations with LBC obtained from 16 different sources and found them to be more impactful than the initial conditions, and that climate variables such as precipitation could be affected by the choice of forcing source over the whole simulation domain. 
The sensitivity of variables such as precipitation or sea level pressure to the LBC was found to be improved with larger simulation domains \citep{koltzow_importance_2011}, although \citet{leduc_sensitivity_2011} showed that this effect was very dependent on the local climate and even on the season.
Using forcing data from the exact same model at the same resolution, \citet{colin_sensitivity_2010} found that model performance was not degraded on precipitation, and particulalry its extreme values, even using a rather small domain. 

Althouhg using a limited domain introduces such complex sensitivities, it also enables running simulations over a region on climatic timescales at higher spatial resolutions, without exceeding reasonable computation time.
This means that model developers and users must be aware of the scale-dependent hypotheses and parameters of the model to ensure that they remain suitable.
In particular, the hydrostatic hypothesis used in dynamical cores of AGCM and RCMs is considered valid only above 10-kilometre resolution, and surely invalid below 4-kilometre resolution \citep{stull2012introduction, guichard_short_2017}.
More generally, the scale separation between processes that are effectively resolved by the dynamics and processes that must be parametrized cannot be universal across scales. For instance, when model resolution reaches the average size of a single cloud, statistical parametrizations of the cloud population then become partly unadapted, yet the dynamical core is not fully capable of properly representing their distribution. Multiple \textit{grey zones} are therefore identified for processes that were historically parametrized in atmospheric models, as scales within which new representations need to be developed \citep{wei_research_2024,frassoni_building_2018,chow_crossing_2019}.

At higher spatial resolutions, multiple terms like storm-resolving, convection-permitting, or cloud-resolving models (CRMs) regroup models that explicitly represent convective clouds. The Meso-NH model \citep{lac_overview_2018}, developed by Météo France and referred to in Chapter \ref{chap:liaise}, falls in this category. These models are typically run at resolutions ranging from a few tens of metres to a few kilometres and are based on non-hydrostatic physics. 
As explained in a review of CRMs by \citet{guichard_short_2017}:
\begin{quote}
    ``CRMs are fine-scale limited-area numerical models whose major characteristic is to provide explicit simulations of the mesoscale dynamics associated with convective clouds. They integrate parametrizations in order to represent major sub-grid processes (turbulence, microphysics, radiative processes). However, unlike GCMs, their grid size allows numerous couplings arising between convective motions and physical processes to be resolved. [...] their utilization has proved very fruitful to the understanding of several convective cloud-related issues that cannot be satisfactorily addressed with observations alone; they are also now widely used as 'numerical laboratories' which guide and help the development of cloud and convection parametrizations for larger-scale models."
\end{quote}
More recently, global CRMs have also been implemented, and although they cannot be used for long climate runs at the moment, they are expected to play an important role in the study and understanding of both weather and climate in the future \citep{satoh_global_2019, lee_weaker_2024}.
%option:il commence à y avoir des modèles globaux à l echelle km https://www.dkrz.de/en/projects-and-partners/projects/focus/climate-simulations-with-km-resolution, projet nextGEMS au Centre Euroéen
Nowadays, most numerical weather prediction (NWP) systems combine CRMs and observation data assimilation methods to produce weather forecasts. These models are typically run at kilometre-scale resolutions, and the 100-metre scale is being investigated and considered for future deployments \citep{lean_hectometric_2024}. NWP is an entire field of atmospheric modelling that is related to meteorology and operational forecasting rather than climate science, but the physical processes involved and key variables of interest are fundamentally the same. Therefore, models used for NWP can also contribute to research activities to improve general knowledge in atmospheric science, as is often done with the Weather Research and Forecasting model \citep[WRF,][]{skamarock2021description}.

At even finer resolution, down to the metre scale, are large eddy simulations (LES). They consist in a different class of models that achieve much higher performance in the representation of turbulent motions by using different assumptions to numerically approximate the Navier-Stokes equations. These models still filter some of the finer-scale motions and have their own parametrizations for these processes, but can be of great use to understand processes and design parametrizations for CRMs or AGCMs. 
In particular, AGCM can be used in single-column (also called 1D) configurations, that do not need the dynamical core but take lateral advection terms as inputs, to compare the performance of the physics to LES simulations of a similar atmospheric column \citep{couvreux_process-based_2021}.
Finally, the ultimate refinement of computational fluid dynamics is the use direct numerical simulations (DNS), which do not require any parametrization of turbulence, but are very demanding in computing power, and therefore limited to metre-scale domains and very short simulation periods.

To summarise, although not all model types nor conceptual abstraction levels have been mentioned here, atmospheric modelling relies on a hierarchy of models which can support each other for both research and operational purposes \citep{maher_model_2019}. In particular, coarser models like AGCMs (or fully-coupled ESMs) can be used to provide boundary conditions for limited area models at higher resolutions, which are in turn used to improve process-understanding and parametrizations in coarser models.
These approaches have been particularly successful at representing the state of climate, and at predicting climate change.

%option:il manque le concept de réanalyse...
%option:peut être ressortir la notion de modèle meso-scale mais elle n'est pas si bien définie que ça (pour des modèles plus HR que RCM mais pas NH/CRM pour autant?)
%option:tableau agnès ou figure pour illustrer les echelles concernées et processus représentés ou non

\subsection{Predicting climate change}
As early as 1824, Joseph Fourier, followed by Claude Pouillet, theorised that some components of the atmosphere can influence the temperature of the air more than others, which was demonstrated in 1838 by Eunice Newton Foote for water vapour and carbon dioxide (CO$_2$). This was linked to infrared absorption and emissions by the experimental work of \citep{tyndall_xxiii_1861}, and \citep{arrhenius_xxxi_1896} conducted the first estimate of the global temperature increase caused by a hypothetical doubling of CO$_2$ in the atmosphere.
Since the industrial revolution in the 19th century, anthropic activities have increased the amounts of greenhouse gases in the atmosphere, mainly as a consequence of fossil fuel combustion and by destruction of natural sinks to give way to agricultural land \citep{RN1}.
By the end of the 1950's, it was established that concentration of CO$_2$ in the atmosphere was increasing and monitoring efforts intensified \citep{pales_concentration_1965}.
Indeed, it was known that although water vapour is overwhelmingly dominant in the atmospheric composition, other greenhouse gases such as CO$_2$, could have a significant impact on global energy balance, and therefore temperature, by absorbing infrared radiation in a different part of the spectrum \citep{plass_carbon_1956}.
Concurrent research in palaeoclimatology using ice cores showed that past evolutions of greenhouse gases in the atmosphere had never been this fast over the last 30.000 years, and were all associated with large-scale changes in global climate \citep{lorius_30000-yr_1979}, which would later be extended to the last 800.000 years \citep{jouzel_orbital_2007}.
At this time, concerns arose about the possible rise of global temperatures and the impacts it could have on natural ecosystems and human activities \citep{broecker_climatic_1975}.

As this knowledge progressed coincidentally with capabilities in numerical climate modelling, models started to be used not only to reproduce past and present climate, but also to simulate future climate scenarios. 
In particular, Syukuro Manabe simulated the impact of greenhouse gases on the thermal equilibrium of the atmosphere in single-column models \citep{manabe_thermal_1964}, before conducting groundbreaking work with coupled ocean-atmosphere GCMs to simulate the response of climate to an increase of atmospheric CO$_2$ \citep{manabe_sensitivity_1980}. 
Such modelling experiments quickly confirmed the risk of rapid increases in global temperature, and hinted at the global and regional consequences on ice cap and glacier melting, sea level rise, and disruption of the water cycle, among others \citep{hansen_global_1988}. 
In 1988, the International Panel on Climate Change (IPCC) was established by the World Meteorological Organisation (WMO) and the United Nations Environment Program (UNEP) to unify efforts in climate change science, and on the socio-economic impacts of global warming. Since its first report \citep{IPCC_1990_AR1_WG1} the IPCC has been carrying on in its mission to "provide governments at all levels with scientific information that they can use to develop climate policies".

At the end of the 20th and the beginning of the 21st century, many developments in the field of remote sensing enabled the monitoring of climate variables from satellites, which complemented ground-based observations. This largely contributed to the ability to isolate the signal of anthropogenic climate change from natural climate variability in recent observations, retroactively validating climate change predictions established in the previous decades \citep{hansen_earths_2005,yang_role_2013}. 
In 2021, Syukuro Manabe shared one half of the Nobel Prize in Physics with Klaus Hasselmann "for the physical modelling of Earth's climate, quantifying variability and reliably predicting global warming". 
Hasselmann's work focused on the statistical methods that allowed to disentangle the variations in weather from long-term changes of climate induced by external forcings, such as anthropogenic greenhouse gases emissions, and identify a \textit{fingerprint} of climate change \citep{hasselmann_optimal_1993,hasselmann_stochastic_2022}.

\begin{figure}[hbtp]
    \centering
    \includegraphics[width=\textwidth]{images/intro/IPCC_AR6_WGI_SPM_Figure_8.png}
    \caption{Selected indicators of global climate change under five scenarios used in the Summary for policymakers of the 6th Assessment Report \citep{IPCC_2023}.}
    \label{fig:ipcc_ssp}
\end{figure}

Early climate change studies focused on idealised scenarios, such as the doubling of atmospheric CO$_2$, but the IPCC has since introduced new standards for future climate simulations. For its 6th Assessment Report \citep{RN1}, Shared Socio-economic Pathways (SSPs) were defined, encompassing different trajectories for population, economic development, education, and technological change to represent a range of plausible futures for human society. They are used with Representative Concentration Pathways (RCPs), which quantify the additional radiative forcing on the global energy budget induced by greenhouse gases emissions in each scenario.
This approach allows climate modellers to assess a wide spectrum of radiative forcing scenarios, providing policymakers with a range of possible evolutions of the climate system (Fig. \ref{fig:ipcc_ssp}).
In the Coupled Model Intercomparison Project \citep[CMIP,][]{eyring_overview_2016}, similar simulations are performed by multiple climate models for several SSP-RCP scenarios. 
This allows the IPCC to elaborate conclusions that are not reliant on a single model, and to account for the uncertainties induced by modelling choices in each climate model.
Moreover, a given model also performs multiple simulations for each scenario with slightly different initial conditions, constituting a simulation ensemble. Considering the chaotic nature of the climate system, changes in initial conditions will lead to different results. Studying the outputs of a simulation ensemble as a statistic distribution of plausible trajectories of the climate system is a way to account for its internal variability and to quantify the uncertainties in its predicted evolution. 

% \clearpage

\subsection{Land-atmosphere interactions in climate models}

In climate models, the representation of land-atmosphere interactions involves both the atmospheric model and the land surface model, and the modelling choices for their coupling.
From the perspective of an atmospheric model, latent and sensible heat fluxes at the surface constitute necessary boundary conditions to solve turbulent diffusion equations in the atmospheric column. These conditions notably depend on spatial heterogeneities of soil moisture, and impact essential atmospheric variables such as air temperature, wind, and humidity. From the perspective of the LSM, the precipitation computed by the atmospheric model constitutes a water input for the soil column, while surface layer characteristics (humidity, temperature, wind) condition evapotranspiration demand.

\subsubsection*{Land-atmosphere coupling hotspots}

Various experiments have been designed to quantify the importance of surface coupling processes for atmospheric models. In particular, the GLACE experiments \citep{koster_glace_2006} compared atmospheric simulations with different prescribed soil moisture conditions to isolate the influence of soil moisture on summer precipitation. This led to the identification of hotspots: regions where this coupling is particularly pronounced. These are mainly semi-arid regions (Sahel, Great Plains in the USA) where the transitional evaporation regime described in Figure \ref{fig:evap_regimes} is more frequent than dry and wet regimes \citep{koster_regions_2004}.
This was confirmed by other modelling studies that also identified various mechanisms through which land surface conditions can impact the atmosphere in these coupling hotspots, and metrics to quantify them \citep{dirmeyer_terrestrial_2011,santanello_landatmosphere_2018, zou_precipitation_2023, hay-chapman_novel_2023}.
The GLACE-CMIP5 experiments \citep{seneviratne_impact_2013} extended these conclusions by highlighting the importance of this coupling in the response to global warming in these hotspots \citep{berg_interannual_2015}.
\citet{dirmeyer_intensified_2014} also found that the controls exerted by the land surface on boundary layer development would strengthen under climate change, amplifying the sensibility to land use change and climate extremes in these hotspots.
This investigation was pursued in the CMIP6 framework with the the Land Surface, Snow and Soil moisture Model Intercomparison Project \citep[LS3MIP,][]{van_den_hurk_ls3mip_2016}, particularly the Land Feedback MIP (LFMIP) experiments. Similarly to GLACE-CMIP5, simulations were run with prescribed surface conditions and compared to the coupled simulations of CMIP6 to assess the response of the atmosphere. The higher sensitivity of precipitation to changes in soil moisture in transition zones between wet and dry climates was confirmed, as well as the more intense response of 2-metre temperature and precipitation to moisture changes under a strong climate change scenario \citep{catalano_land-surface_2021}.
Other analyses confirmed that the mean climate simulated by current ESMs is still highly dependent on the representation of land processes \citep[][]{may_role_2023, zarakas_land_2024}, and of the atmospheric feedbacks they induce \citep{lague_separating_2019}.

\subsubsection*{Representation of sub-grid heterogeneities}

A particular challenge in modelling land-atmosphere interactions within an ESM arises from the size of the grid cells used (generally $\sim$100 km for global climate simulations). 
At this scale, numerous sub-grid heterogeneities exist at the surface, requiring the aggregation or averaging of highly diverse land-atmosphere interactions depending on vegetation cover, elevation, or anthropogenic factors (cities, irrigation).
For example, regarding the triggering of convective rainfall in heterogeneous areas, \citet{moon_soil_2019} showed that CMIP5 models correctly reproduced the positive temporal feedbacks (rain on wetter days) described in \citet{guillod_reconciling_2015} but not the negative spatial feedbacks identified in \citet{taylor_afternoon_2012}. The importance of land-atmosphere coupling has also been highlighted by \citet{lee_weaker_2024}, showing that CRMs were much more capable of reproducing these negative spatial feedbacks than coarse climate models.

LSMs usually rely on input databases that describe surface heterogeneities such as maps of soil texture, vegetation and other land covers.
Several approaches have been developed to use this information to provide atmospheric models with lower boundary conditions (budgets for momentum, energy, water, and other chemical species), which are summarised in Fig. \ref{fig:mosaic_approach}.

\begin{figure}[ht]
    \centering
    \includegraphics[width=0.8\textwidth]{images/mosaic_approach.jpg}
    \caption{ Conceptual view of the different paradigms to represent the land-atmosphere coupling over heterogeneous surface in models, from the simplest on the left to the more complex to the right. Note that the spatial organization of the heterogeneity and horizontal scale of the different patches are not taken into account. Taken from \citet{lohou_model_2025}}
    \label{fig:mosaic_approach}
\end{figure}

The simplest option is to compute only one value for these variables, aiming to represent the grid-cell average flux. The easiest way to do this is to take into account only the \textit{dominant land cover} and assume that it covers the entire grid cell \citep{chen_coupling_2001}, which does not account for sub-grid heterogeneities, and is all the more limited in highly heterogeneous covers as demonstrated in WRF by \citet{li_development_2013}. Approaches that distinguish several soil tiles depending on land cover are now used in most models. For vegetation, this is often done using vegetation classes, or plant functional types (PFTs), that regroup plants in categories (\textit{e.g.} deciduous trees, conifers, types of crops) and allocate them characteristic parameters that can be used to describe their height, leaf area index (LAI), or albedo \citep{sellers_revised_1996}. 
The \textit{parameter aggregation} (also called \textit{composite} or \textit{mixture}) method uses the parameters and the areal fraction of the tiles to determine effective parameters (that are not necessarily the result of an arithmetic average). Here, all the tiles are interfaced to the same atmospheric grid cell and coupling variables are only computed once for the grid cell \citep{koster_comparative_1992}. This is the method used in the ORCHIDEE LSM used in this thesis (described in Chapter \ref{chap:methods}). 
Conversely, in the \textit{mosaic approach}, surface budgets are computed independently for each tile, and averaged afterwards to provide a single value to the atmospheric model \citep{koster_modeling_1992,xi_assessment_2024}. This notably enables more relevant comparisons of models to point-based observations over a specific land cover in heterogeneous terrain, since the variables computed in the corresponding tile can be used \citep{ament_improved_2006}.
Finally, the mosaic approach has been extended to the atmospheric column by dividing it into sub-columns \citep{salmun_observational_2007, vrese_explicit_2016} to explicitly propagate the differences induced by land covers to the atmosphere.

The modelling choices for this representation of heterogeneities determines the variables computed by the LSM, and can significantly affect the behaviour of the atmosphere. For instance, studying soil moisture heterogeneities in California using the WRF model with several LSMs, \citet{alexander_simulating_2022} found that the choice of LSMs dominated variations in the response of the ABL, rather than the parametrization of turbulence.
Efforts are still ongoing to evaluate the existing approaches and suggest new ways to account for sub-grid heterogeneities \citep{huang_representing_2022, waterman_two-column_2024}, which has been identified as a major challenge for future LSMs \citep{fisher_perspectives_2020}.
The association of dedicated observations with modelling at several scales helps to document the impact of these heterogeneities and explore better ways to account for them in models. The  Models and Observations for Surface-Atmosphere Interactions (MOSAI, \cite{lohou_model_2025}) project relies on measurement campaigns specifically designed to study land-atmosphere interactions over heterogeneous land covers in different seasons \footnote{In August 2023, I had the opportunity to take part to one of the three special observation periods of the MOSAI project in Lannemezan, although it was not in the scope of my PhD thesis to work with data from this project}.
It aims at comparing multiple ESMs and NWP models with these observations and deriving new parametrizations to account for surface heterogeneities. 

\subsubsection*{Limitations in representing land-atmosphere interactions}

It is now well known that accurately representing surface-atmosphere couplings is essential to accurately represent climate, particularly temperature and precipitation extremes \citep{jaeger_impact_2011,van_den_hurk_acceleration_2011, wehrli_identifying_2019, al-yaari_heatwave_2023}. Furthermore, the study of CMIP5 simulations has revealed a warm bias in most models, particularly in the Mediterranean basin \citep{christensen_temperature_2012, mueller_systematic_2014}. By comparing CMIP5 model biases with satellite observations over the USA, \citet{al-yaari_satellite-based_2019} also showed that these temperature biases are correlated with soil moisture biases. More generally, surface interaction processes have been identified as a crucial factor in the appearance of these biases, especially the partitioning of energy between latent and sensible heat fluxes and the underestimation of evapotranspiration \citep{cheruy_combined_2013, cheruy_role_2014,williams_land-atmosphere_2016,ma_causes_2018}. 

New developments were carried out in most climate models between CMIP5 and CMIP6 \citep[e.g.][]{lawrence_community_2019, cheruy_improved_2020}, which contributed to reduce several biases associated to land-atmosphere interactions. 
Improvements and better inter-model agreement have been noted for surface radiative fluxes \citep{wild_global_2020} and for energy and water fluxes in general except for the Amazon, the Tibetan Plateau, and with limitations for turbulent fluxes in dry regions \citep{li_evaluation_2021}. 
The CMIP6 ensemble is found to perform better than any single model for evapotranspiration, which remains generally overestimated over continents, as well as evaporative fractions \citep{wang_evaluation_2021, yuan_understanding_2022}.
Regional analyses found improvements in the representation of precipitation and soil moisture over the USA \citep{srivastava_evaluation_2020, yuan_historical_2021}, but excessive land-atmosphere coupling strength in East and Southern Africa \citep{mwanthi_representation_2024}. 
A comparison of the CMIP6 ensemble to the ERA5 reanalysis and observation data showed that a warm bias is still present in Southern Europe although mostly decorrelated from soil moisture biases \citep{osso_assessment_2023}. 
Substantial model uncertainty on land-surface conditions still remains in CMIP6 models \citep{yuan_historical_2021}, and the importance of land-use and land-cover representation to properly represent surface energy partitioning and precipitation has been renewed for this generation of climate models \citep{devanand_land_2020,singh_land-use_2024}. In particular, only three CMIP6 models represented irrigation and \citet{al-yaari_role_2022} showed that over intensely irrigated areas, they capture the trends of increasing latent heat flux seen in satellite-based products whereas the others show no trend, or even a decrease in latent heat flux. These findings justify efforts to account for irrigation in ESMs and understand its impacts on land-atmosphere interactions.

\section{Observed and modelled impacts of irrigation on land-atmosphere interactions}
\label{sec:irrig_landatmosphere}

Human activities can influence the various components of land-atmosphere interactions. This is particularly the case with irrigation, which directly modifies soil moisture through artificial water inputs and can create strong spatial heterogeneities. 

\subsection{Observations and high-resolution modelling}

Observational studies have established that irrigation leads to a moister and cooler atmosphere near the surface \citep{bonfils_empirical_2007, mcdermid_irrigation_2023}, which is explained by an increase in the latent heat flux and a decrease in the sensible heat flux in irrigated areas \citep{rappin_landatmosphere_2022, boone_land_2025}.
Based on observation records in California, \citet{bonfils_empirical_2007} estimated a decrease of 1.8 to 3°C in the average summer temperature since the development of irrigation practices. This cooling effect can limit temperature values during extreme meteorological events \citep{thiery_present-day_2017, thiery_warming_2020}. However, the cooling due to increased latent heat flux can be partially compensated on a daily scale by a weakening of nighttime cooling due to thermal inertia \citep{chen_irrigation_2018}.
In the American Midwest, \cite{nocco_observation_2019} showed that the coupling effects could impact the regional climate and even mask the rise in temperature induced by global warming. Different regional responses of precipitation have also been observed, such as a decrease in irrigated areas \citep{alter_rainfall_2015} or an increase in downwind regions \citep{deangelis_evidence_2010}.
The other impacts on the atmosphere (structure of the boundary layer, mesoscale circulations, precipitation) are more complex to observe and characterise because they do not always occur solely over the irrigated area.

To get more insights than what point-based or satellite observations provide and analyse the atmospheric processes involved, mesoscale modelling studies have often been used to complement observational campaigns. 
The Great Plains irrigation experiment \citep[GRAINEX,][]{rappin_great_2021} measurements and simulations with the WRF model revealed a lower ABL over irrigated areas, which is sustained by irrigation throughout the growing season independently of background atmospheric conditions \citep{lachenmeier_irrigated_2024}, and a reduction of existing mesoscale slope-induced circulations in the presence of irrigation \citep{rappin_landatmosphere_2022, phillips_influence_2022}. Computing land-atmosphere coupling metrics on GRAINEX sites, irrigated fields were found to be more favourable to convection than non-irrigated ones in the day \citep{whitesel_assessing_2024}. In modelling experiments over the GRAINEX study area, the choice of the irrigation fraction map and its resolution were identified as important drivers of land-atmosphere interactions \citep{lawston-parker_investigating_2023}. They attributed it to the importance of boundaries between irrigated and non-irrigated zones, and emphasized the need to account for soil moisture heterogeneities induced by irrigation.
Irrigted and rain-fed sites in the Ebro Valley (northern Spain) were monitored during the Land Surface Interactions with the Atmosphere over the Iberian Semi-Arid Environment campaign \citep[LIAISE,][, which will be further presented in Chapter \ref{chap:liaise}]{boone_land_2019, boone_land_2025}. Meso-NH simulations at 2-kilometre and 400-metre resolution were conducted over the LIAISE domain. They reproduced the observed differences between irrigated areas and neighbouring areas in turbulent fluxes and air temperature and were greatly improved by representing irrigation \citep{lunel_irrigation_2024}. They also highlighted a weakening of the regional sea-breeze regime due to irrigation, which reduces the pressure gradient force \citep{lunel_marinada_2024}.  

Multiple mesoscale modelling studies were also conducted without being associated to specific observation campaigns, particularly with the WRF atmospheric model and the Noah LSM, showing that the representation of irrigation improves model performance, especially limiting warm biases \citep{qian_modeling_2013, yang_impact_2017,qian_neglecting_2020, liu_simulating_2021}. Over the American Great Plains, they identified a lowering of the boundary layer and lifting condensation level \citep{qian_modeling_2013}, and a reduction of the summertime precipitation deficit by increasing the frequency of mesoscale convective systems \citep{qian_neglecting_2020}. In the Colorado River basin, \citet{yang_impact_2017} showed an increase in precipitation linked to irrigation due to moisture recycling over the Sierra Nevada mountains. Similar processes of regional precipitation recycling were identified by \citet{zhang_us_2025} in the US Corn Belt region with a larger impact on drier years.
Conversely, a negative feedback loop on precipitation in Saudi Arabia was described in \citep{lo_intense_2021}, with moisture convergence leading to increased precipitation in a remote area west of the irrigated region. Very heterogeneous effects on simulated precipitation were also found in China \citep{liu_simulating_2021}, where the importance of indirect effects on climate through vegetation greening was also identified \citep{liu_irrigation-induced_2023}. In the Po valley in Italy, CRM simulations over the summer 2015 showed that irrigation prevented the soil from drying during a heatwave, moistened the air, and reduced near-surface atmospheric warming by 2.5-3K in July \citep{valmassoi_regional_2020}. This impact propagated to the boundary layer, which was lowered from 1500 m to 1000 m but did not lead to any precipitation feedback.

These results were obtained with very diverse representations of irrigation since not all models target the same objectives. For example, the experiments with WRF-Noah or Meso-NH-SURFEX \citep{lunel_irrigation_2024} used idealized representations that force soil moisture to remain at field capacity in irrigated areas to limit stress for the vegetation. This type of modelling is suitable for short simulations over a limited domain but  water-conservative approaches are necessary in ESMs to run long-term global simulations (coupled with the oceans) while keeping a consistent water cycle representation \citep{valmassoi_review_2022}. 

\subsection{Impacts of irrigation in ESMs}

The impacts of irrigation on specific regions have been studied for several decades, but interest in its impacts on global climate, and therefore its inclusion in ESMs, is more recent \citep{boucher_direct_2004}. 
By comparing coupled simulations with and without irrigation, \citet{sacks_effects_2009} found no influence on the global mean temperature but small regional cooling (less than 1 K) at mid-latitudes. 
In more recent studies, LSMs that represent irrigation were shown to represent the shift in energy partitioning between the turbulent fluxes, achieving large increases in latent heat flux \citep{pokhrel_incorporating_2012, al-yaari_role_2022, arboleda-obando_validation_2024}. 
Irrigation modelling was found to induce a cooling on yearly average values, with seasonal and regional impacts that remain highly varied \citep{puma_effects_2010, cook_irrigation_2015}. 
In general, regional effects within global simulations are mostly visible and analysed in irrigation hotspots such as India, Eastern China and the USA, although \citet{chen_global_2019} pointed that the Community Earth System Model \citep[CESM, ][]{danabasoglu_community_2020} tends to exacerbate the effects in these regions of strong land-atmosphere coupling compared to observations. 
Global simulations also allow the study of remote connection and long term patterns. 
Using the global Community Atmosphere Model (CAM) coupled with the Community Land Model (CLM) to study the local and remote impacts of irrigation in California's Central Valley,  \citet{lo_irrigation_2013} identified an intensification of the water cycle in the south-western USA. They described an anthropogenic regional recycling loop with increased precipitation in the Colorado basin, that sustains water adduction from the Colorado river to California.
On even larger scales, \citet{de_vrese_asian_2016} highlighted a link between irrigation in Asia and increased precipitation in East Africa, while \citet{wei_where_2013} identified moisture-importing and -exporting regions at the global scale. 
Running 30-years simulations with and without irrigation, \citet{guimberteau_global_2012} studied the temporal disruptions of seasonal phenomena, and identified a delay in the onset of the Indian summer monsoon due to regional irrigation effects.

Using CMIP6 simulations under climate change and estimates of irrigation withdrawals to compute the evolution of groundwater resources, \citet{costantini_projected_2023} identified regions where the impact of climate change could limit future groundwater withdrawals, and estimated that at least 27\% of the world population would be exposed to water scarcity issues by 2100. 
\citet{arboleda-obando_joint_2025} conducted a global study of the impacts of irrigation under climate change with the Institut Pierre-Simon Laplace climate model (IPSL-CM, some components of which were used in this thesis, see Chapter \ref{chap:methods}), and identified more intense water depletion in irrigated zones, but also atmospheric feedbacks that could increase precipitation in their non-irrigated surroundings. This study projects that irrigation will be limited by future hydroclimate conditions or contribute to an increase in tensions over water use over one third of irrigated areas, including the Mediterranean basin.
In a similar setup, \citet{cook_divergent_2020} concluded that even under a strong climate change scenario, irrigation would keep moderating warming and drying but warned that for regions like Mexico and the Eastern Mediterranean, ``current irrigation rates will be unable to compensate for expected moisture losses by the end of the 21st century, meaning that any irrigation benefit will be temporary".

According to \citet{de_vrese_uncertainties_2018}, uncertainties on the impacts of irrigation on climate can still stem from modelling choices at the interface between LSMs and AGCMs. 
They found that aggregating surface fluxes in the LSM could limit or even reverse the moistening impact of irrigation on the atmosphere.
To limit uncertainties on the impact of irrigation on climate and the water cycle, \citet{de_vrese_uncertainties_2018, cook_divergent_2020} also both highlighted the importance of accounting for available resources and methods for irrigation withdrawals, which is not represented in all models.
Indeed, using water-conservative representations of irrigation also allows the study of the continental water cycle as a whole, accounting for impacts of irrigation withdrawals, on the condition that they are not overly simplified by removing seawater (as done in the CESM for instance). 
Elaborate models can now differentiate between irrigation methods, withdrawals from surface or groundwater resources and represent storage and adduction, offering an integrated vision of the anthropised water cycle \citep{decharme_simple_2025}.
For instance, \citet{leng_significant_2017} showed that the responses in groundwater levels and river flows to irrigation were very contrasted and highly dependent on the irrigation methods and sources used. \citet{yao_implementation_2022} confirmed these findings, and also showed that representing distinct irrigation methods in the CLM improved the agreement with surface fluxes measurements.

\hfill

It is important to note that most ESMs participating in CMIP6 did not include irrigation modelling. In heavily irrigated regions, models representing irrigation show different historical trends in several essential climate variables (soil moisture, latent heat flux) that better match satellite observations \citep{al-yaari_role_2022}. 
Regarding the IPSL-CM used in this thesis, \citet{mizuochi_multi-variable_2020} found that in its CMIP6 simulations, the absence of irrigation was a major source of error, on soil moisture and evapotranspiration, which was amplified and propagated to errors in precipitation in coupled configuration, owing to atmospheric feedbacks.
The  Irrigation model intercomparison project (IRRMIP) is designed to study in detail the various impacts of irrigation representation in ESMs to better understand the uncertainty of responses to this modelling \citep{yao_irrigation-expansion-induced_2023}. So far, the first results of this multi-model analysis have confirmed that irrigation expansion contributes to a rapid increase in withdrawals and to a cooling of near-surface temperatures \citep{yao_impacts_2025}. It was also associated to less frequent heat extremes in irrigated regions, although the concurrent moistening increases the exposure of local populations to moist-heat stress \citep{yao_impacts_2025-1}.

\section{Scientific questions and thesis outline}

This chapter has shown that land-atmosphere coupling processes are a major element of the climate system, and a challenge for climate models, especially at coarse resolutions. 
Their high influence on near-surface atmospheric variables and precipitation, which  directly affect human and non-human health and well-being, already make them a relevant object of study. This interest is strengthened by the uncertainties still associated to their representation and by the impact they can have on the results of current or future climate simulations. 
In this context, irrigation, a widespread agricultural practice that many believe  bound to keep expanding in the future, can alter natural balances of the water and energy budgets.
Withdrawals from surface and groundwater resources disrupt the continental water cycle, and soil moisture is also directly modified by irrigation, possibly affecting land-atmosphere feedback processes.
This justifies the increasing interest in including a representation of irrigation in ESMs (via LSMs), to improve model performance and provide relevant predictions of the impacts of climate change in intensely irrigated regions.

Although there are exceptions, two main types of modelling studies can be distinguished in the literature, which both have their limitations.
On the one hand, climate modelling studies often focus on global average impacts of irrigation on climate and do not delve into details on the intermediary processes affected by irrigation such as the structure of the ABL, partly due to the lack of high frequency outputs \citep{findell_accurate_2024}. Although some regional impacts can be studied, the coarse resolution of ESMs often limits their analysis to the most intensely irrigated regions such as southern Asia or the USA, and does not enable a discussion of sub-grid heterogeneities induced by irrigation. 
Also, since coupled simulations with an AGCM are costly, hydrological studies of the impact of irrigation on the water cycle are sometimes conducted with a standalone LSM which does not account for atmospheric feedbacks, including the ones on irrigation demand.
On the other hand, higher-resolution simulations with CRMs can accurately capture the spatial distribution of irrigated and non-irrigated areas. This enables a more precise representation of the impacts of irrigation on near-surface conditions and the ABL, as well as more direct comparisons with point-based observations. However, their descriptions of irrigation are often idealised and tailor-made to represent local surface conditions, and do not enable an analysis of the full water cycle. Furthermore, they are often run for a few days or weeks only, which means they are not suited to study impacts on climatic timescales.

This PhD work therefore aims to analyse the regional impacts of irrigation on climate, on both continental and atmospheric components of the water cycle, and on the atmospheric boundary layer at the diurnal scale.
It leverages a new limited area model (LAM) configuration developed at IPSL to perform simulations with the atmospheric and land surface components of the global IPSL climate model, ICOLMDZ and ORCHIDEE.
This regional climate model, called ICOLMDZOR LAM, can provide insight on the parametrizations of the global IPSL-CM while running with a higher resolution and lower computational costs than in global applications.

Until now, a large share of irrigation modelling studies have focused on the USA (and to some extent eastern China and southern Asia), and hinted that further research in different contexts was required to strengthen their conclusions, especially on atmospheric feedbacks which can be very dependent on mesoscale patterns.
This study focuses on the Iberian Peninsula, which does not present irrigated surfaces as large as these regions but remains a hotspot for irrigation in Europe, with major concerns regarding its sustainability in a changing climate.
Although it was not identified by \cite{koster_regions_2004} as a region of strong surface-atmosphere coupling (probably owing to its small size relative to the resolution of the model used and to other identified hotspots), its semi-arid climate is expected to make land surface-atmosphere interactions very sensitive to the disruptions induced by irrigation.

\hfill

\noindent Using regional climate simulations over this region, this thesis addresses the following questions:

\begin{itemize}
    \item Does simulated irrigation contribute to a better representation of the continental and atmospheric water cycle over the Iberian Peninsula?
    \item How does irrigation affect the regional climate of the Iberian Peninsula, in current conditions and under climate change?
    \item To what extent can a regional climate model account for the diurnal effects of irrigation on land-atmosphere interactions and the ABL, over a heterogeneous landscape?
    \item What are the limitations and perspectives for improvement of the ICOLMDZOR LAM in simulating the climate of the Iberian Peninsula and the impacts of irrigation? 
\end{itemize}

Chapter \ref{chap:methods} describes the components of the ICOLMDZOR LAM and the simulation setups used in this study. 
Chapter \ref{chap:routing} presents the evaluation and parameter tuning of the river routing and irrigation scheme conducted over the Iberian Peninsula.
Chapter \ref{chap:forcing} gives some insight into the influence of the lateral boundary conditions of the LAM regarding land-atmosphere coupling variables, which was explored during Mariame Maiga's master's internship. 
Chapter \ref{chap:monthly} analyses the impacts of irrigation over the Iberian Peninsula in recent climate conditions and in the second half of the century under a strong climate change scenario. 
Based on the same simulations, Chapter \ref{chap:liaise} focuses on the study area and period of the LIAISE campaign to compare the LAM to observations and a mesoscale simulation and analyse the impacts of irrigation on the ABL.
Finally, Chapter \ref{chap:conclusion} summarises the main outcomes and perspectives of this thesis.

%fin

