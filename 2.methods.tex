This work uses the atmosphere and land surface components of the IPSL-CM, which has been a regular participant in CMIP exercises, including CMIP6 \citep{boucher_presentation_2020}. 

\section{ICOLMDZ General Circulation Model}
\subsection{General structure and parameterizations}
The atmospheric component of the model is the association of the dynamical core DYNAMICO \citep{dubos_dynamico-10_2015}, which uses an icosahedral grid, and the LMDZ6A physics used for CMIP6: version NPv6.2 with 79 vertical levels \citep{hourdin_lmdz6a_2020}. The physics of the model are run every 15 mn and include the following parameterizations:
\begin{itemize}
    \item a surface layer description based on \citet{louis_parametric_1979} and \citet{king_sensitivity_2001}; 
    \item an Eddy-Diffusivity Mass Flux (EDMF) scheme of boundary layer vertical transfer composed of a turbulent diffusion scheme based on \citet{yamada_simulations_1983} with recent improvements described in \citet{vignon_modeling_2018}, and a thermal plume model for shallow convection \citep{hourdin_unified_2019}; 
    \item a stochastic triggering scheme for deep convection \citep{rochetin_deep_2014, rochetin_deep_2014-1}; 
    \item a large scale condensation scheme based on a statistical distribution of subgrid total water content, from which cloud fraction and water contents are derived \citep{madeleine_improved_2020}; 
    \item radiative transfer model RRTM \citep{mlawer_radiative_1997}.
\end{itemize}

%todo:detail forcings
All simulations were run with prescribed sea surface temperature and sea ice content from the AMIP dataset.


\subsection{Limited Area Model configuration}
Simulations are run in a Limited Area Model (LAM) configuration, first used and well described in \citet{raillard_leveraging_2024}.
Lateral boundary conditions for the LAM are read at each time step of the dynamics, and are taken from ERA5 reanalysis hourly values at 0.25° resolution \citep{hersbach_era5_2020}.
The domain comprises 3 zones: a raw forcing zone which contains values directly given by the forcing, a transition zone  where the model is nudged toward the forcing with decreasing strength, and a free zone at the center of the domain where there is no direct influence of the lateral forcing.
In this study, the LAM domain is a hexagon centered on (40.4°N, -3.7°E) and of radius 1500 km (Fig. \ref{fig:domain_full_hex}). The radius is composed of 60 grid cells, so the diameter of a cell is 25 km. 
Outputs of the model are natively on a hexagonal grid but are interpolated to a more traditional longitude-latitude grid of similar resolution, in order to simplify post-treatment and comparisons to evaluation products.

\section{ORCHIDEE Land Surface Model}
\subsection{General structure, running modes, and input data}
ORCHIDEE (Organizing Carbon and Hydrology In Dynamic EcosystEms) is a land surface model. Various branches exist within the model, as the base structure presented in \citet{krinner_dynamic_2005} has been adapted and extended to meet different research objectives.
This work uses branch 2.2, which serves as the land component of the IPSL climate model and has notably been used for CMIP6 simulations \citep{boucher_presentation_2020}. Coupling with an atmosphere and ocean model imposes significant computational constraints, leading to the simplification or omission of certain processes for global climate simulations.
The description of this model version provided here is partly based on Section 2 of Aurélien Campoy’s PhD thesis \citep{campoy_influence_2013}.

ORCHIDEE v2.2 includes several modules:
\begin{itemize}
    \item SECHIBA (\textit{Schématisation des Échanges Hydriques à l’Interface entre la Biosphère et l’Atmosphère}, \cite{ducoudre_sechiba_1993}). Initially integrated into the surface scheme of the LMDZ GCM, this module represents surface energy and water balances, including interactions with the atmosphere. It models water infiltration into the soil and horizontal transfers, allowing for river representation.
    \item STOMATE \citep{krinner_dynamic_2005}. This module simulates biochemical surface processes such as photosynthesis and phenology evolution, enabling the representation of seasonal variations in transpiration.
    \item LPJ \citep{sitch_evaluation_2003}. This module dynamically evolves vegetation to represent land-use changes. In this work, to reduce computational costs, this module is inactive, and vegetation evolution is instead prescribed using annual land-cover maps from \citet{belward1999igbp}.
\end{itemize}

ORCHIDEE can interface with the atmosphere in either forced mode (also called offline) or coupled mode. 
In the first case, meteorological forcing (usually obtained from a reanalysis) provides values for downward radiation (shortwave and longwave), precipitation (rain and snow), air temperature and specific humidity at 2m, wind speed at 10m (eastward and northward components), and surface pressure. 
In the second case, ORCHIDEE is coupled with an atmospheric model that calculates these variables in real time while receiving certain variables computed by ORCHIDEE (surface roughness, albedo and turbulent fluxes). 
Within the IPSL climate model, ORCHIDEE can be coupled with the atmospheric physics model LMDZ in a standard configuration called LMDZOR, using a semi-implicit coupling scheme \citep{polcher_proposal_1998}. %option : detail that this schemes enables computing thermal and turbulent diffusion in soil and atmospheric column (respectively).
ORCHIDEE discretizes the simulation domain into grid cells, with a resolution that adapts to the atmospheric forcing or the grid of the coupled atmospheric model.

In addition to meteorological variables, the model requires other input data. Each grid cell must be assigned a soil texture from three main classes (sandy loam, silt loam, clay loam), which determines various hydrological and thermal soil parameters. 
For CMIP6 simulations, the map from \citet{zobler87802world} was used to assign the dominant  texture in each grid cell, with a resolution of 1°. For the coupled simulations described in this work, the dominant USDA texture was obtain from the map \citet{reynolds_estimating_2000}, which has a resolution of 5 arcmin.

To characterize vegetation, ORCHIDEE defines 15 Plant Functional Types (PFTs), each associated with a set of characteristic parameters such as average height, leaf area index, and albedo. For CMIP6 simulations, the fraction of each PFT in a grid cell is derived from a map based on the LUHv2 database \citep{lurton_implementation_2020}. Each grid cell is divided into three columns, grouping multiple PFTs: one for bare soil, one for tall vegetation (trees), and one for low vegetation (crops and grasses). A separate water budget is computed for each soil tile but only one energy budget per grid cell.

\subsection{Water and energy budgets}
\subsubsection{from article LAM}
The ICOLMDZ GCM is coupled to the Land Surface Model (LSM) ORCHIDEE v2.2 \citep{cheruy_improved_2020}. The spatial grid of the LSM is the same as for the GCM and the timestep is dictated by the atmospheric physics (15 mn).
% This LSM represents a 2-m soil column discretized over 11 vertical levels of increasing depth. 
It computes the water and energy budgets at the surface and simulates water infiltration and thermal diffusion in the soil column. 
% Each grid cell is assigned the dominant USDA soil texture according to the 5 arcmin resolution map from \citet{reynolds_estimating_2000}.
Vegetation is described using 15 Plant Functional Types (PFTs) on a 0.1° resolution input map based on the Land Use Harmonization 2 (LUHv2) dataset \citep{hurtt_harmonization_2020, lurton_implementation_2020}. PFTs are distributed in each grid cell over 3 soil tiles for bare soil, forests, and low vegetation (which notably includes C3 and C4 crops).
A separate water budget is computed for each soil tile but only one energy budget per grid cell, using a composite approach with aggregated parameters (roughness length, albedo) to compute surface temperature and turbulent fluxes. 
%option:other parameters aggregated ? Beta for Evap (resistance to evaporation ? other name ?)
%option:implicit coupling for turbulent and thermal diffusion into the ground -> jugé plutôt pas nécessaire

\subsubsection{from CSI + translation}
This section describes how the different terms of water and energy budgets identified in the introduction are modeled in ORCHIDEE.  

To determine the soil water content in a grid cell, ORCHIDEE first distinguishes between the fraction of precipitation intercepted by vegetation and the fraction that actually reaches the ground. In the case of liquid precipitation (snow is treated separately), it further differentiates between the portion that infiltrates into the soil and surface runoff.  

Soil hydrology is based on the one-dimensional Richards equation, with a 2-m soil column discretized over 11 vertical levels of increasing depth \citep{de_rosnay_impact_2002, dorgeval_sensitivity_2008}. A free drainage condition is applied at the bottom of the column. ORCHIDEE models the infiltration front velocity, which progressively saturates the layers. The values of hydraulic conductivity and diffusivity are calculated based on soil texture according to \citet{mualem_new_1976, van_genuchten_closed-form_1980}.

The missing term to complete the surface water balance is evapotranspiration $E$, which also appears in the energy balance via the latent heat flux $\lambda E$. The total evaporation value depends on potential evaporation, which represents the evaporation that would occur over a water surface and thus serves as an upper limit. It is calculated using the formulation of \citet{Budyko_1956}:  

\begin{equation}
    E_{pot} = \frac{\rho_{air} (q_{sat}(T_S) - q_{air})}{r_a}
\end{equation}

where $\rho_{air}$ is the air density, $q_{air}$ is the specific humidity of the air, $q_{sat}(T_S)$ is the specific humidity of saturated air at the surface temperature $T_S$, and $r_a$ is the aerodynamic resistance to evaporation.  
Evapotranspiration $E$ is obtained from the ratio $\beta = \frac{E}{E_{pot}}$, which depends on four terms associated with different components of evapotranspiration:  

\begin{itemize}
    \item Snow sublimation, which only occurs in certain regions and seasons.  
    \item Evaporation of water intercepted by vegetation, which depends on vegetation structure.
    \item Vegetation transpiration, which involves the stomatal resistance of vegetation.  
    \item Bare soil evaporation, which is the minimum between $E_{pot}^*$, the potential evaporation reduced according to \citet{milly_potential_1992}, and $Q_{up}$, the maximum volume that can be extracted from the soil column.  
\end{itemize}

The factor $\beta$ is calculated as a conductance that depends on the evaporation resistances of each of these terms and the fractions of the grid cell occupied by vegetation and bare soil.  

% \hfill

% Bare soil evaporation is expressed as a relationship between supply and demand:
% \begin{equation}
%     E_g = min(E^*_{pot}, Q_{up})
% \end{equation}
% where $Q_{up}$ is the maximum volume that can be extracted from the soil column, and $E^*_{pot}$ is the reduced potential evaporation from \citet{Milly_1992}. Note that in most cases, $E^*_{pot} < E_{pot}$.
%NB Provide more details on the calculation of Qup? The available water volume is estimated by integrating Richards' equation over the soil column to maintain a water volume in each layer above......

% \hfill

% An alternative version exists in ORCHIDEE, which allows limiting the demand by modeling a bare soil resistance to evaporation $r_{soil}$ based on \citet{Sellers_1992}:
% \begin{equation}
%     E_g = min(\frac{E^*_{pot}}{1+\frac{r_{soil}}{r_a}}, Q_{up})
% \end{equation}
% \begin{equation}
%     r_{soil} = exp(8.206 - 4.255 \frac{W_L}{W_L^s})
% \end{equation}
% where $W_L$ is the moisture content in the top four layers of the soil column, and $W_L^s$ is the saturation moisture content in these same layers.

In the energy balance, ORCHIDEE does not modify the incoming radiation terms but determines the outgoing terms by calculating the albedo (which is split into two values, one for infrared and one for visible light) and the surface temperature, which controls the outgoing longwave radiation. 

For turbulent fluxes, ORCHIDEE calculates roughness lengths, which affect surface drag coefficients ($C_{drag}$). Two roughness lengths are distinguished:  
\begin{itemize}
    \item $z_{0m}$, which represents the height above the ground where wind speed is zero.
    \item $z_{0h}$, below which temperature is assumed to be equal to the soil temperature. 
\end{itemize} 

The default parameterization in ORCHIDEE is based on \citet{su_evaluation_2001} and dynamically evaluates both roughness lengths using empirical formulations. However, some of the obtained values are not always consistent (particularly for $z_{0h}$), and an older parameterization is sometimes used. This simpler approach relies on two prescribed parameters for each PFT:  
\begin{itemize}
    \item The ratio between $z_{0m}$ and the canopy height, with a typical value of $\frac{1}{15}$.
    \item The ratio between $z_{0m}$ and $z_{0h}$, with a typical value of $\frac{1}{10}$.  
\end{itemize}

\subsection{Routing scheme}
\subsubsection{from article}
River discharge is simulated using a routing scheme based on the one described in \citet{ngo-duc_validation_2007}.
This scheme solves water transfers directly on the MERIT Digital Elevation Model (DEM) \citep{yamazaki_merit_2019} which provides topographic data at a 2-km resolution. Each DEM grid cell contains three linear reservoirs that represent surface runoff, groundwater and rivers. Surface runoff and drainage computed for each ORCHIDEE soil column are interpolated to the routing grid to feed the surface runoff and groundwater reservoirs (respectively). All three reservoirs then flow into the river reservoir of the downstream grid cell, with a residence time that depends on the slope and on a time constant specific to each reservoir. 
In this scheme, the river reservoir is the only representation of horizontal water transfers in ORCHIDEE.

\subsubsection{from CSI + translation}
The routing scheme is a sub-module of SECHIBA, which circulates water between various cascading reservoirs to represent horizontal transfers and transport it to the oceans \citep{ducharne_development_2003, ngo-duc_validation_2007}. 
Routing is necessary to maintain water conservation on a global scale in coupled simulations with the ocean, and it also allows for simulating river discharge and representing groundwater volumes. 
This scheme generally runs at a larger time step than the rest of the model (once per day by default); however, the irrigation modeling (detailed in the next section) requires routing to run at each time step of the ORCHIDEE model.

The scheme subdivides the simulation domain into Hydrological Transfer Units (HTU) and uses a Digital Elevation Model (DEM, constructed from topographic data) to define flow directions and characterize the HTUs (dimensions, slope, upstream/downstream relationships).
Within each HTU, three linear reservoirs are distinguished: the fast reservoir (surface runoff), the slow reservoir (groundwater), and the stream reservoir (rivers). These reservoirs differ based on their respective time constants, which determine the residence time as a function of slope (denoted as TCST\_FAST, TCST\_SLOW, and TCST\_STREAM in Figure \ref{fig:routing_principles}).
The slow reservoir is fed by drainage, the fast reservoir by runoff, and all three reservoirs flow into the river reservoir in the downstream grid cell. The river reservoir is therefore the only one connected to neighboring cells, as no other horizontal transfer is modeled in ORCHIDEE. This assumption may have limitations at high resolutions as it ignores direct groundwater transfers between grid cells.

The outgoing volume $Q$ from a reservoir at a routing time step is expressed in terms of the reservoir volume in the grid cell, $V$, the time constant $TCST$ ($day \cdot km^{-1}$), a topographic index $topoindex$ ($km$) provided by the DEM, and the routing time step $dt_{routing}$ ($s$):

\begin{equation}
    Q = \frac{V}{topoindex \times TCST} \times \frac{dt_{routing}}{86400}
\end{equation}

\begin{figure}[ht]
    \centering
    \includegraphics[width=1\linewidth]{images/routing_structure.png}
    \caption{Operating principles of the \textit{interp\_topo} routing}
    \label{fig:routing_principles}
\end{figure}

Based on the same flow physics, different versions of routing have been developed, corresponding to different relationships between the DEM grid, the ORCHIDEE grid, and the definition of the HTUs used:

\begin{itemize}
\item \textbf{\textit{Subgrid\_halfdeg} routing}, which was the default option in branch 2.2 until recently. It is designed to work with a specific DEM at a resolution of 0.5°. Additionally, it requires that an ORCHIDEE grid cell contains an integer number of DEM grid cells, which limits its use to resolutions higher than 0.5°.

\item \textbf{\textit{Interp\_topo} routing}, recently implemented into ORCHIDEE to become the default version in IPSL-CM7. It is designed to adapt to any DEM and to be completely independent of ORCHIDEE’s resolution and even the shape of the grid cells. This is particularly necessary for compatibility with the new hexagonal-grid atmospheric dynamics model (DYNAMICO) of IPSL-CM7.

In this routing, HTUs are exactly the DEM grid cells, requiring interpolation of runoff and drainage from the ORCHIDEE grid to the DEM grid, where they serve as inputs for the fast and slow reservoirs, respectively. Horizontal transfers between river reservoirs of the HTUs are thus performed directly on the DEM grid. Once these transfers are completed, water volumes in the reservoirs are re-interpolated from the DEM back to the ORCHIDEE grid, allowing them to be used for irrigation within ORCHIDEE grid cells.

\item \textbf{\textit{Subgrid\_HTU} routing}, developed before the introduction of the \textit{interp\_topo} routing to operate at a higher resolution than \textit{subgrid\_halfdeg} routing. In this version, HTUs are constructed from a high-resolution DEM by aggregating grid cells. However, this routing still maintains the constraint of having an integer number of HTUs within each ORCHIDEE grid cell, making it incompatible with DYNAMICO. I have not used it in simulations, but this manuscript occasionally refers to it.
\end{itemize}

\subsection{Irrigation scheme}
\subsubsection{from article}
Irrigation is modeled using the scheme extensively described in \citet{arboleda-obando_validation_2024}, based on a water-conservative supply-and-demand approach.
It computes a moisture deficit by comparing SM in the upper layers of the ORCHIDEE soil column (corresponding to the root zone) to a target SM described as a fraction of the SM at field capacity.
In the default version of the global model, this target is set to 90 \% of SM at field capacity, but it reflects a wide variety of irrigation practices, including flooding in rice paddies. In the Iberian Peninsula, the irrigation methods are less water-intensive, and after a calibration of the routing and irrigation schemes (not shown), the target value was adjusted to 60 \% of SM at field capacity. 
To avoid computing irrigation requirements on grid cells without plants, the SM deficit is set to zero if the Leaf Area Index (LAI) is below a given threshold $LAI_{min}$.
At each time step, the SM deficit is combined to the irrigated fraction of each ORCHIDEE grid cell (based on the HID database of \citet{siebert_quantifying_2010} at 5 arc-min resolution) to define the irrigation requirement. 
This demand is then compared to the available water in the various reservoirs of the routing scheme. Water is withdrawn preferentially from either surface water (runoff and rivers) or groundwater depending on the nature of irrigation equipments, according to a map of areas equipped for irrigation \citep{siebert_groundwater_2010}.
The amount of water withdrawn from the reservoirs is then added at the top of the ORCHIDEE soil column at the next time step and infiltrates.
%option:show here map of irrig_frac and AEI ? -> I think they are fine with results on irrigation for better visualization of the links

\subsubsection{from CSI}
A scheme for representing irrigation in ORCHIDEE has recently been developed, presented, and validated in \citet{arboleda-obando_validation_2024}. An irrigation representation already existed in ORCHIDEE \citep{de_rosnay_integrated_2003, guimberteau_global_2012}, but this work relies solely on this new modeling approach detailed here and summarized in Figure \ref{fig:schema_pedro}.

\begin{figure}[t]
    \centering
    \includegraphics[width=1\textwidth]{images/schema_pedro.png}
    \caption{Principles of the new irrigation scheme. Extracted from \citet{arboleda-obando_validation_2024}.}
    \label{fig:schema_pedro}
\end{figure}

\hfill

First, a root zone is defined using parameter $Root_{lim} \in [0;1]$, which determines the fraction of the root system that must be included in the zone. This results in the inclusion of a certain number of vertical layers depending on the cumulative root density. It is important to note that irrigation only applies to the column of low vegetation PFTs (grasses and crops), and the root zone is defined exclusively within this column.

A target soil moisture value is also defined to sustain plant growth. It is expressed as a fraction of the soil moisture at field capacity water content (parameter $\beta \in [0;1]$).

Given this target value and the root zone, a soil moisture deficit $D$ (mm) is calculated as the sum of deficits across all layers of the root zone:
\begin{equation}
    D = \sum_{i \in Rootzone} min(0,\beta \times W_i^{fc} - W_i)
\end{equation}
where $W_i$ and $W_i^{fc}$ are respectively the soil moisture and field capacity soil moisture of layer $i$ (in mm), .

In the absence of plants (notably in winter), it is not relevant to compute a soil moisture deficit and an irrigation demand. A parameter $LAI_{lim}$ is therefore defined as a threshold LAI value below which the soil moisture deficit is zero.

For a SECHIBA module time step $dt$ (by default 15 minutes in ICOLMDZOR simulations), the water demand to be added to the column to represent irrigation is thus $D/dt$ (mm/h). However, if the hourly irrigation rate is significantly higher than the infiltration rate into the soil, the model may produce excessive runoff that is not representative of reality. To prevent this, even if the moisture deficit is very high, the hourly irrigation rate is capped by a maximum value $I_{max}$.

Finally, the water demand for the column is weighted by the fraction of the grid cell that is irrigated: $f_{irr}$. This fraction is obtained from a map of irrigated fractions, which can be updated over the course of the simulation.
This irrigated fraction must not exceed the fraction of the soil column containing grasses and crops, as irrigation is only applied to this column.

The irrigation demand $I_{req}$ is therefore:
\begin{equation}
    I_{req} = f_{irr} min(D/dt, I_{max})
\end{equation}

Once this demand is determined, the irrigation scheme searches for available water in the three reservoirs of the routing scheme mapped on the ORCHIDEE grid.
The volume of water available for irrigation is expressed as:
\begin{equation}
    A_w = f_{sw} (a_1 S_1 + a_2 S_2)+ f_{gw}a_3 S_3
\end{equation}
where $S_i$ represents the water volume (in mm) in each of the reservoirs (rivers, surface water, groundwater), and for each reservoir, a parameter $a_i \in [0;1]$ limits the total volume that can be withdrawn. This ensures that a certain amount of water remains in each reservoir, representing physical constraints and environmental regulations on irrigation withdrawals.
The fractions $f_{sw}$ and $f_{gw}$, ranging from 0 to 1, represent the accessibility of different reservoirs (surface water and groundwater, respectively) within the grid cell. The global map from \citet{siebert_groundwater_2010} of irrigation-equipped areas is used to define these fractions. A key feature of this map is that a given area cannot be equipped for both surface water and groundwater use simultaneously, which translates to $f_{sw} + f_{gw} =1$.

Once the demand $I_{req}$ and supply $A_w$ are calculated, the applied irrigation is determined as:
\begin{equation}
    I = min(A_w/dt, I_{req})
\end{equation}

Water is primarily withdrawn from the river reservoir, and only if this is insufficient to meet the demand are the other reservoirs tapped, in the limit of $f_{sw}$ and $f_{gw}$.
This withdrawn water is added in the code simultaneously with water infiltration into the soil, simulating a gravity-fed or drip irrigation method.

\hfill

To summarize the functioning of this irrigation scheme, the following inputs are used:
\begin{itemize}
    \item Map of irrigated fractions, to obtain $f_{irr}$.
    \item Map of irrigation-equipped areas, to obtain $f_{sw}$ and $f_{gw}$.
\end{itemize}

Additionally, to use this scheme, the following parameters must be set:
\begin{itemize}
    \item $Root_{lim}$, which determines the portion of the root system considered in the soil moisture deficit calculation.
    \item $\beta$, which defines the target soil moisture.
    \item $LAI_{lim}$, the minimum LAI threshold for irrigation activation.
    \item $I_{max}$, which limits the hourly irrigation rate.
    \item $a_i$, which restricts the volume that can be withdrawn from each reservoir.
    % \item $a_{i,add}$, which limits the maximum portion that can be supplied from the river reservoir of a neighboring grid cell.
\end{itemize}

The sensitivity analyses described in \citet{arboleda-obando_validation_2024} have shown that the parameter $\beta$ has the greatest impact on the volume of water withdrawn for irrigation. These analyses have allowed the determination of default values for all parameters to best represent the impact of irrigation on global climate. However, these values can be adjusted for simulations focused on a specific region.
