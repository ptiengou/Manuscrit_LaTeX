\section{Chapter introduction}

Considering its impacts on the energy budget at the surface, the effects of irrigation follow a marked diurnal cycle, with distinct effects in nighttime and daytime and varying amplitude depending on insolation. 
%option:could reuse ref from Intro
Average monthly or seasonal studies (like the work presented in Chapter \ref{chap:monthly}) cannot allow a detailed analysis of these impacts, and simulation outputs at a finer temporal resolution must be used.

Although it's based on the same simulation setup as Chapter \ref{chap:monthly}, this chapter focuses on the month on July 2021, and on part of the Ebro Valley. This study area and period were selected to include the measurement sites and special observation period (SOP) of the LIAISE field campaign. This enables direct comparison to local observations of temperature, humidity, wind and turbulent fluxes on irrigated and rainfed sites, at the surface and in the boundary layer. Furthermore, several mesoscale modelling experiments were conducted over this area and period, in particular using the mesoscale MesoNH model at 2-km resolution. These simulations served as an intermediate between punctual observations and the regional climate model, bringing new perspective on the following questions:

\begin{itemize}
    \item How does the ICOLMDZOR-LAM perform over the LIAISE study area, relative to observations and mesoscale simulations?
    \item To what extent does simulated irrigation improve the performance of the LAM? What limits its ability to reproduce observations on irrigated and rainfed sites at the diurnal scale?
    \item Are the effects of simulated irrigation limited to surface variables or do they propagate to the vertical structure of the ABL? 
    \item Are the effects of simulated irrigation only visible over the irrigated areas or do they also affect neighbouring rainfed areas?
    \item Over such a heterogeneous terrain, how can representativity of the LAM be defined and to what extent is it achieved?
\end{itemize}

This chapter first presents the LIAISE field campaign with details on the measurements used, mesoscale simulations that were conducted over the study area, and the main results of this project.
Then, the ICOLMDZOR LAM simulations used for comparison over LIAISE sites are described, before presenting results on land-atmosphere coupling variables and the vertical structure of the atmosphere.
%todo: improve/rephrase after final structure

\section{The LIAISE field campaign}
This description of the Land Surface Interactions with the Atmosphere over the Iberian Semi-Arid Environment (LIAISE) project is partly based on the various articles which presented the campaign : \citet{boone_land_2019,boone_updates_2021,boone_land_2025}, as well as on the dedicated section in Tanguy Lunel's PhD thesis \citep{lunel_interactions_2024}.

%objectives
LIAISE is an international research campaign aimed at improving the understanding of land-atmosphere-hydrology interactions in a semiarid region characterized by strong surface heterogeneity owing to contrasts between the natural landscape and irrigation-dependent agriculture, and the limitations of models to represent all aspects of the terrestrial water cycle.

%period
Although the campaign included a longer monitoring of surface variables, this work focuses on the special observation period (SOP) which occurred from July  15–29 2021. This SOP has been selected since it is the period where contrasts between irrigated and natural surfaces are generally at or near their maximum. At this time, synoptic scale winds are typically light and from the west, and a  thermal (heat) low pressure area generally forms over the region with daily maximum temperatures in the mid-to-upper  30s °C.
Within this SOP, Intensive Observation Periods (IOPs) were selected, aiming for relatively clear days characterized by a well-mixed  atmospheric boundary layer with weak synoptic scale forcing, so that surface-induced local to regional scale circulations were most likely to be present and detectable. 

%area
The study domain for LIAISE is the Ebro basin in northeastern Spain, which is bound to the north by the Pyrenees and to the south by the Iberian System.
It presents a semiarid hot, dry  Mediterranean climate, with a very sharp delineation between a vast, nearly continuous intensively-irrigated region and the generally drier rainfed zone to the east of the study domain.
%sites
Two supersites were defined at La Cendrosa, within the irrigated zone, and Els Plans, within the rainfed zone. Each was equipped with a 50-meter mast to measure temperature, humidity, wind speed, radiative and turbulent fluxes (using eddy-covariance) at various heights.
This work used 2-meter measurements for most variables, and 10-m wind speed.
On IOP days, hourly radiosondes were launched, roughly from  sunrise through early evening, probing the ABL up to about 3 km.

\begin{figure}[hbtp]
    \centering
    \includegraphics[width=\textwidth]{images/chap5/liaise_sites_picture.png}
    \caption{Instrumented sites of La Cendrosa (a) and Els Plans (b) in July 2021 \citep[taken from ][]{lunel_interactions_2024}.}
    \label{fig:liaise_sites_photos}
\end{figure}


%choice of IOP days, synoptic conditions
%  This thermal situation has  a strong impact on surface winds. Formation of some shallow cumulus is possible, but moist convection, if present, is  generally confined to the surrounding mountain ranges. Owing to the proximity to the Mediterranean Sea to the east, sea  breeze (SB) formation is quite frequent, but its inland progression is slowed by the presence of the Catalan pre-coastal and  coastal ranges that separate the Ebro basin from the sea. The  intensity of the westerlies and strength of the heat low are also  contributing factors to the SB front propagation and intensity. Generally speaking, the SB front usually arrived sometime  between 16:00 and 19:00 local time over the study zone with  the dry zone sites impacted earliest. The SB passage was seen  in the observations as a low level wind shift (to winds with a  significant easterly component) over the entire region, while in  the east, horizontally-scanning lidar observations also revealed  a significant increase in low-level moisture with its arrival. The  sea breeze generally coincided with a collapsing ABL in the late  afternoon. Days with a predicted early arrival of the SB were  not designated as IOPs during the daily briefings. Finally, one  rain event did occur on July 26, which was associated with the  passage of a synoptic scale trough: local precipitation totals of  around 30 mm were recorded over the irrigated zone; however,  the convective cells propagated to the northeast and the dry  zone received relatively little rainfall, thereby reinforcing the  wet-dry zone contrast during the subsequent days of the SOP.

%measurements / instruments
%irrigation practices and timing. Urgell canal adduction
%overview of synoptic conditions, rainfall
% simulation experiment with conceptual and mesoscale models
%main outcomes so far : Mangan, Lunel (2 sites), Lunel Marinada, intermodel comparison

%option:voir si irait bien quelque part: 
%Most of the water used for agriculture, approximately 75%, is stored in reservoirs while the rest is maintained by the snow pack in the mountains.

\section{Simulation experiments}

Several simulations are compared to analyse the diurnal cycle of land-atmosphere coupling variables in the Ebro Valley.
First, the two simulations studied in Section \ref{sec:article1} (\noirr and \irr) were analysed and compared to LIAISE observations using hourly outputs over the month of July 2021.

Then, sensitivity experiments were conducted to increase irrigation over grid cell corresponding to the irrigated site of La Cendrosa, leading to the simulation henceforth referred to as \irrboost. This simulation was only run over the month of July, starting from the same state as \noirr and \irr on July 1st 2021, and involved significant changes in both on the computation of irrigation demand, and on the water availability. 
Regarding demand, the irrigated fraction was slightly increased to make sure that all of the soiltile dedicated to low vegetation and crops was considered as irrigated land. This soiltile represents 83.5\% of the grid cell, the rest being occupied by trees (12.4\%) and bare soil (4.1\%). This fraction was also reduced to 0\% on the Els Plans site to make sure there would be no irrigation demand computed in this grid cell.
More importantly, the \betairrig parameter, which had been set to 0.6 after the offline calibration in Chapter \ref{chap:routing} to avoid excessive depletion of the groundwater and river reservoirs, was increased to 1. This means that the irrigation scheme is aiming to maintain soil moisture in the root zone (first 64cm of soil) to field capacity, making sure that plants would not be limited in their growth by available soil moisture. 
Regarding available water, since this was a short simulation run (one month), very large amounts of water were added in the three routing reservoirs at the simulation initial state, making virtually inifinite amounts of water available for irrigation in the LIAISE region. Obviously, this method is not realistic, nor applicable to longer climate runs, and was only used to test the limits of the irrigation scheme and explore the sensitivities of land-atmosphere interactions in a context where irrigation would be driven by the water demand rather than the supply. 
When comparing to reality, this can also be seen as a way to compensate for the absence of water adduction in the ICOLMDZOR LAM simulations, a practice that is determinant in the actual supply of irrigated water in the area. %todo:see if this explained in previous sec, refer to it
%todo:see also if irrigation simulation method used by tanguy has been detailed (because that's what it does too)

The three ICOLMDZOR LAM simulations (\noirr, \irr, \irrboost) are compared to observations and to the MesoNH simulation from July 14th to July 30th. This covers all the LIAISE SOP and allows for the stabilization of irrigation volumes in the \irrboost simulation, since no long-term spinup was conducted with this setup.

\hfill

%todo:choice of grid cell -> figure
For each site, an ICOLMDZOR grid cell was selected for the comparison to observations. For Els Plans, the grid cell containing the exact site location seemd appropriate since it was a very lightly irrigated cell. however, the exact position of the La Cendrosa site was not in an intensely irrigated cell, which is why the neighbouring grid cell was selected.

%fig:sites and corresponding grid cells
\begin{figure}[hbtp]
    \centering
    \begin{tabular}{cc}
        \begin{subfigure}[t]{0.44\textwidth}
            \caption{Visible image of the LIAISE study area as seen by \textit{Sentinel-2} on July 22nd.}
            \includegraphics[width=\textwidth]{images/chap5/liaise_overview_lunel.png}
        \end{subfigure} &

        \begin{subfigure}[t]{0.5\textwidth}
            \caption{Monthly average irrigation simulated by ORCHIDEE (July 2021, \irr simulation).}
            \includegraphics[width=\textwidth]{images/chap5/liaise_sites_irrig_ORC.png}
        \end{subfigure} 
    \end{tabular} 

    \begin{subfigure}[t]{0.75\textwidth}
            \caption{Average latent heat flux simulated by MesoNH (14-30 July 2021, with irrigation). Red hexagons show the ICOLMDZOR grid cells for each site.}
            \includegraphics[width=\textwidth]{images/chap5/liaise_sites_mean_mesoNH.png}
        \end{subfigure} 
    
    \caption{Correspondance between actual location of La Cendrosa and Els Plans sites (red dots), selected ICOLMDZOR grid cells (centered on green dots), and MesoNH grid. (a) was taken from \citet{lunel_irrigation_2024}.}
    \label{fig:liaise_sites_grid_cells}
\end{figure}

%todo:mesoNH (if not presented in previous section ?)
Comparison to the MesoNH simulations was performed using two values for each site. 
The first one, referred to as \mesoexact, takes the variables from the exact grid cell corresponding to the site location based on the GPS coordinates of the site, as done in \citet{lunel_irrigation_2024}. 
The second, referred to as \mesomean, is an aggregation of all the MesoNH grid cells contained into the selected ICOLMDZOR grid cell for each site. The mean value is shown, as well as an envelope encompassing the 25th and 75th percentiles, when it is appropriate.
On Fig. \ref{fig:liaise_sites_grid_cells}c, the two hexagonal ICOLMDZOR grid cells are drawn upon the average latent heat flux simulated by MesoNH. It can already be noticed that within the grid cell for La Cendrosa, there are some MesoNH grid cells with very low values of ET, and that in the grid cell for Els Plans, there are some MesoNH grid cells which fall into the irrigated zone and exhibit high ET. This will have an influence on the \mesomean values and on the spread of MesoNH surface variables within one ICOLMDZOR grid cell. 
%option:mesoMean is not meant to be more representative of the obs than mesoExact, but to bring information on heterogeneity within one ICOLMDZOR grid cell

\section{Near-surface variables over the SOP}
\label{sec:sop}

Here, surface variables simulated by ICOLMDZOR and MesoNH are compared to the observations on both sites from July 14th to July 30th, the span of the MesoNH simulation, which corresponds to the LIAISE SOP.

\hfill
%%LA CENDROSA%%

At La Cendrosa (Fig. \ref{fig:cendrosa_surfacevars}), observed turbulent fluxes(in black) exhibit a clear diurnal cycle driven by solar irradiation, with a maximum value between 12 and 13UTC.
In the \noirr simulation (in red), ICOLMDZOR simulates a negligible latent heat flux (Fig. \ref{fig:cendrosa_surfacevars}b) and largely overestimates the sensible heat flux (+300\% at 12UTC on Fig. \ref{fig:cendrosa_surfacevars}d). 
These major biases are partly improved in the \irr simulation (in blue), showing that additional soil moisture brought by irrigation can really improve turbulent fluxes. However, latent heat flux simulated in \irr follows a good trajectory from 5UTC to 9UTC but then drops and remains largely underestimated until the evening. This is due to a failure of the irrigation parametrization which cannot sustain the irrigation demand throughout the day since the routing reservoirs (particularly the river reservoirs) cannot provide sufficient irrigation withdrawals. 
In the \irrboost sensitivity experiment (in green), virtually infinite amounts of water are available in the reservoirs and ET is no longer constrained by soil moisture. Both turbulent fluxes follow a diurnal cycle that is consistent with observations, although a small underestimation of latent heat flux and an overestimation of sensible heat flux remain compared to the observations. 
However, the turbulent fluxes in \irrboost are very close to those of \mesomean (in yellow, with the envelope representing the 25th and 75th percentiles of the distribution), suggesting that ICOLMDZOR correctly represents the average fluxes over the grid cell, which does not only include intensely irrigated areas.
It can be noted here that \mesoexact (in purple), corresponding to the exact MesoNH grid cell of the observation site, overestimates latent heat flux on average over the period. This is mostly due to the beginning of the SOP, where the alfalfa crops at La Cendrosa were still freshly cut and had not recovered their original size, LAI and transpiration levels.

The diurnal cycle of 2-meter temperature is also driven by solar irradiation but its shape is slightly shifted with a peak in the afternoon around 4UTC (Fig. \ref{fig:cendrosa_surfacevars}f). Figure \ref{fig:cendrosa_surfacevars}e shows that temperatures were consistently rising throughout the SOP, from July 14th to 22nd, and that this evolution is well captured by ICOLMDZOR and MesoNH.
On average over the SOP, all simulations exhibit a warm bias, of at least 1.5° at night, where \mesoexact is closer to ICOLMDZOR than to observations, and of varying amplitude during the day. The \irr simulation has the strongest warm bias, which is consistent with the simulated turbulent fluxes, and this bias is slightly reduced in \irr. In \irrboost, it is largely improved, with a peak temperature 2°C lower than \irr, and simulated values that follow \mesomean throughout most of the day. In the evening, \irrboost even falls closer to observed values than to \mesomean, which suggests that if turbulent fluxes are correctly simulated, the structural nighttime warm bias is smaller in ICOLMDZOR than in MesoNH.
The diurnal cycle for 2-meter specific humidity is less structured than turbulent fluxes or temperatures.
On Fig. \ref{fig:cendrosa_surfacevars}h, MesoNH appears to present a dry bias a night, with an underestimation of specific humidity in \mesoexact as strong as in \mesomean. With ICOLMDZOR, a similar bias is present in \noirr, but it is absent in \irr and \irrboost, likely thanks to local ET enabled by irrigation.
In daytime, \mesoexact matches observed values from 8UTC to 18UTC, and \mesomean follows a similar structure through the diurnal cycle. 
ICOLMDZOR, on the contrary, presents an incorrect diurnal cycle of specific humidity, with an extensive drying in daytime. The strength of this drying is largest in \noirr but \irrboost still falls below the values of \mesomean, at the limit of the 25th percentile. 
The differences between ICOLMDZOR and observed values increase largely from July 20th, where humidity drops in the simulations but not in reality or in MesoNH simulation (Fig. \ref{fig:cendrosa_surfacevars}g).
This shows that in certain daytime conditions, increasing ET by accounting for irrigation is not sufficient to simulate a realistic 2-meter specific humidity. In particular, incorrect advection terms and surface wind can neutralise the effect of this local improvement of land-atmosphere interactions.

This hypothesis is corroborated by the 10-meter wind regime at La Cendrosa (Fig. \ref{fig:bothsites_wind}a-d), which is quite well structured in the beginning of the SOP with low wind speeds, but shows important variations in direction and an increased wind speed from July 20th. ICOLMDZOR seems to be capturing the variations quite well until July 19th but clearly misses part of the regime changes afterwards. 
On average over the SOP, \mesoexact follows observations closely, and the diurnal cycle simulated by ICOLMDZOR has the same structure as \mesomean, particularly for wind direction.
Irrigation only has an impact on the 10-meter wind speed: in \noirr it is overestimated, and that bias is partly reduced in \irr and \irrboost. This was expected since in the presence of irrigation, the vegetation is more developed and LAI is larger, leading to a larger surface drag and therefore lower surface wind speed. 
Wind direction is not impacted by irrigation (red, blue and green lines overlap in Fig. \ref{fig:bothsites_wind}d), which was also expected since only a few grid cells are intensely irrigated in the region, which limits the possibilities of influencing the dynamics of the LAM.

%fig : Cendrosa turbulent fluxes + t2m, q2m
\begin{figure}[hbtp]
    \centering
    \begin{tabular}{cc}
        \begin{subfigure}[t]{0.5\textwidth}
            \caption{}
            \includegraphics[width=\textwidth]{images/chap5/SOP_TS_DC/time_series_cendrosa_flat.png}
        \end{subfigure} &
        \begin{subfigure}[t]{0.5\textwidth}
            \caption{}
            \includegraphics[width=\textwidth]{images/chap5/SOP_TS_DC/diurnal_cycle_cendrosa_flat.png}
        \end{subfigure} \\
        
        % \vspace{1em} % Add vertical space between the rows
        \begin{subfigure}[t]{0.5\textwidth}
            \caption{}
            \includegraphics[width=\textwidth]{images/chap5/SOP_TS_DC/time_series_cendrosa_sens.png}
        \end{subfigure} &
        \begin{subfigure}[t]{0.5\textwidth}
            \caption{}
            \includegraphics[width=\textwidth]{images/chap5/SOP_TS_DC/diurnal_cycle_cendrosa_sens.png}
        \end{subfigure} \\

        \begin{subfigure}[t]{0.5\textwidth}
            \caption{}
            \includegraphics[width=\textwidth]{images/chap5/SOP_TS_DC/time_series_cendrosa_t2m.png}
        \end{subfigure} &
        \begin{subfigure}[t]{0.5\textwidth}
            \caption{}
            \includegraphics[width=\textwidth]{images/chap5/SOP_TS_DC/diurnal_cycle_cendrosa_t2m.png}
        \end{subfigure} \\
        
        % \vspace{1em} % Add vertical space between the rows
        \begin{subfigure}[t]{0.5\textwidth}
            \caption{}
            \includegraphics[width=\textwidth]{images/chap5/SOP_TS_DC/time_series_cendrosa_q2m.png}
        \end{subfigure} &
        \begin{subfigure}[t]{0.5\textwidth}
            \caption{}
            \includegraphics[width=\textwidth]{images/chap5/SOP_TS_DC/diurnal_cycle_cendrosa_q2m.png}
        \end{subfigure} \\
    \end{tabular}
    \caption{Time series and mean diurnal cycle of surface turbulent fluxes at La Cendrosa (irrigated site), July 14-30 2021. The envelope for \mesomean represents the 25th and 75th percentiles of the distribution.}
    \label{fig:cendrosa_surfacevars}
\end{figure}


%Fig : Wind on both sites
\begin{figure}[hbtp]
    \centering
    \begin{tabular}{cc}
        \begin{subfigure}[t]{0.5\textwidth}
            \caption{}
            \includegraphics[width=\textwidth]{images/chap5/SOP_TS_DC/time_series_cendrosa_wind_speed_10m.png}
        \end{subfigure} &
        \begin{subfigure}[t]{0.5\textwidth}
            \caption{}
            \includegraphics[width=\textwidth]{images/chap5/SOP_TS_DC/diurnal_cycle_cendrosa_wind_speed_10m.png}
        \end{subfigure} \\
        
        \begin{subfigure}[t]{0.5\textwidth}
            \caption{}
            \includegraphics[width=\textwidth]{images/chap5/SOP_TS_DC/time_series_cendrosa_wind_direction_10m.png}
        \end{subfigure} &
        \begin{subfigure}[t]{0.5\textwidth}
            \caption{}
            \includegraphics[width=\textwidth]{images/chap5/SOP_TS_DC/diurnal_cycle_cendrosa_wind_direction_10m.png}
        \end{subfigure} \\

        \begin{subfigure}[t]{0.5\textwidth}
            \caption{}
            \includegraphics[width=\textwidth]{images/chap5/SOP_TS_DC/time_series_elsplans_wind_speed_10m.png}
        \end{subfigure} &
        \begin{subfigure}[t]{0.5\textwidth}
            \caption{}
            \includegraphics[width=\textwidth]{images/chap5/SOP_TS_DC/diurnal_cycle_elsplans_wind_speed_10m.png}
        \end{subfigure} \\
        
        \begin{subfigure}[t]{0.5\textwidth}
            \caption{}
            \includegraphics[width=\textwidth]{images/chap5/SOP_TS_DC/time_series_elsplans_wind_direction_10m.png}
        \end{subfigure} &
        \begin{subfigure}[t]{0.5\textwidth}
            \caption{}
            \includegraphics[width=\textwidth]{images/chap5/SOP_TS_DC/diurnal_cycle_elsplans_wind_direction_10m.png}
        \end{subfigure} \\
    \end{tabular}
    \caption{Time series and mean diurnal cycle of wind speed and direction on both sites, July 14-30 2021. The envelope for \mesomean represents the 25th and 75th percentiles of the distribution.}
    \label{fig:bothsites_wind}
\end{figure}

\hfill

%%ELS PLANS%%
Looking into near-surface conditions at the Els Plans site (Fig. \ref{fig:elsplans_surfacevars}) serves as a useful comparison to assess the general performance of both models in a context that is largely independent of soil moisture.
The energy partitioning between turbulent fluxes is very different, since the average observed latent heat flux is below 20 W \persqm while the sensible heat flux is three times larger than at La Cendrosa (Fig. \ref{fig:elsplans_surfacevars}b, d). 
The three ICOLMDZ simulations are very similar and the lines for \noirr and \irr completely overlap, while latent heat flux is slightly increased (5-10 W \persqm) in \irrboost since ORCHIDEE still computes a small irrigation for this grid cell. 
%option:explain the details ? Demand is 0 but then interpolated to DEM, satisfied on some grid cells near the border of the hexagon and reinterpolated to ORC heaxagonal grid.
ICOLMDZOR underestimates the already weak latent heat flux and overestimates the sensible heat flux with a peak value higher than 400 W \persqm whereas the observed one is around 300 W \persqm. 
Regarding the MesoNH simulation, the \mesoexact diurnal cycle of latent heat flux looks a bit different from the observed one, with a higher and sharper peak around 12UTC instead of the smooth shape seen in observations. The diurnal cycle of sensible heat flux matches the observed shape but presents an overestimation from 11UTC until the evening.
The latent heat flux of \mesomean is clearly driven by the few irrigated grid cells that fall into the hexagon (Fig. \ref{fig:liaise_sites_grid_cells}c), leading the mean value to exceed the 75th percentile. These outliers have much less influence on the sensible heat flux which shows a smaller spread and is even closer to observations than \mesoexact.
On the contrary to La Cendrosa, it was not really expected to see the \irr or \irrboost simulation matching \mesomean because the input map of irrigated fraction was voluntarily modified to limit irrigation on this grid cell. This was done with the initial objectives of creating a constrated situation compared to La Cendrosa and achieving a better match with on-site observations, but clearly made the comparison of grid-cell average fluxes less relevant.
%option:add that no big reservoir here so no irrigation at all if not cheating ? (and too much if cheating ?)

The average diurnal cycle of 2-meter temperature (Fig. \ref{fig:elsplans_surfacevars}f) confirms the presence of a warm bias in MesoNH at night (more than 2°C), and shows a slight underestimation of the peak value in the afternoon. 
%option: further links to fsens, LWup, Tsurf ?
%NB : sur les deux sites, LWdn est sous estimé de ~15W/m² en journée par les deux modèles (sauf \noirr sur La Cendrosa car air plus chaud), probablement ce qui permet d'être bon sur fsens alors que flat est surestimé (manque d'énergie incidente donc pas de compensation parfaite entre les deux)
It also highlights the very good performance of ICOLMDZOR on this non-irrigated site, although the peak is slightly too early compared to the observed one, as also seen at La Cendrosa.
The observed evolution of 2-meter specific humidity over the day is very different from La Cendrosa, since it is highest at night and steadliy decreases throughout the day with a minimal peak at 16UTC on average over the SOP (Fig. \ref{fig:elsplans_surfacevars}h).
As seen in La Cendrosa, MesoNH exhibits a dry bias at night but simulates correct humidity in the day. 
The diurnal evolution simulated by ICOLMDZOR is quite similar to the one at La Cendrosa, except that at Els Plans it is more in agreement with observations and MesoNH. On average over the SOP, ICOLMDZOR strongly underestimates 2-meter specific humidity, but as visible in Fig. \ref{fig:elsplans_surfacevars}g, this is mostly due to the second part of the SOP where excessive drying occurs in all three ICOLMDZOR simulations. 
This supports the idea that the limitations in capturing near-surface humidity variations at La Cendrosa are mostly non-local and largely induced by large-scale advection. 
Finally, it can be noticed that irrigation has a visible impact on specific humidity, although not very large in comparison to the dry bias, also pointing to non-local effects on this variable since irrigation is very small on the Els Plans grid cell in ICOLMDOZOR.%todo:link to irrig TS

Observed wind speed at Els Plans is higher than at La Cendrosa (Fig. \ref{fig:bothsites_wind}f), which relflects the influence on roughness on a site with lower vegtation (Fig. \ref{fig:liaise_sites_photos}). On average, variations in direction are of smaller amplitude than at La Cendrosa (Fig. \ref{fig:bothsites_wind}h) but some quick regime changes can still be identified on most days (Fig. \ref{fig:bothsites_wind}g).
In ICOLMDZOR, the simulated wind speed and direction are very similar at Els Plans and La Cendrosa, which is not surprising since the grid cells are next to each other, but there are no more dfferences between the three simulations. This confirms the hypothesis that the wind speed differences at La Cendrosa were mainly the consequence of increased roughness in \irrboost and \irr. On average, ICOLMDZOR underestimates wind speed but reproduces the diurnal changes in wind direction. As seen at La Cendrosa, it captures the variations in wind direction much better in the beginning of the SOP than after the regime change on July 20th. On this second part of the SOP, a lot of regime changes are not represented by ICOLMDZOR and \mesomean, while only some of them are simulated in \mesoexact.

%Fig : Els Plans turbulent fluxes + t2m, q2m
\begin{figure}[hbtp]
    \centering
    \begin{tabular}{cc}
        \begin{subfigure}[t]{0.5\textwidth}
            \caption{}
            \includegraphics[width=\textwidth]{images/chap5/SOP_TS_DC/time_series_elsplans_flat.png}
        \end{subfigure} &
        \begin{subfigure}[t]{0.5\textwidth}
            \caption{}
            \includegraphics[width=\textwidth]{images/chap5/SOP_TS_DC/diurnal_cycle_elsplans_flat.png}
        \end{subfigure} \\
        
        % \vspace{1em} % Add vertical space between the rows
        \begin{subfigure}[t]{0.5\textwidth}
            \caption{}
            \includegraphics[width=\textwidth]{images/chap5/SOP_TS_DC/time_series_elsplans_sens.png}
        \end{subfigure} &
        \begin{subfigure}[t]{0.5\textwidth}
            \caption{}
            \includegraphics[width=\textwidth]{images/chap5/SOP_TS_DC/diurnal_cycle_elsplans_sens.png}
        \end{subfigure} \\

        \begin{subfigure}[t]{0.5\textwidth}
            \caption{}
            \includegraphics[width=\textwidth]{images/chap5/SOP_TS_DC/time_series_elsplans_t2m.png}
        \end{subfigure} &
        \begin{subfigure}[t]{0.5\textwidth}
            \caption{}
            \includegraphics[width=\textwidth]{images/chap5/SOP_TS_DC/diurnal_cycle_elsplans_t2m.png}
        \end{subfigure} \\
        
        % \vspace{1em} % Add vertical space between the rows
        \begin{subfigure}[t]{0.5\textwidth}
            \caption{}
            \includegraphics[width=\textwidth]{images/chap5/SOP_TS_DC/time_series_elsplans_q2m.png}
        \end{subfigure} &
        \begin{subfigure}[t]{0.5\textwidth}
            \caption{}
            \includegraphics[width=\textwidth]{images/chap5/SOP_TS_DC/diurnal_cycle_elsplans_q2m.png}
        \end{subfigure} \\
    \end{tabular}
    \caption{Time series and mean diurnal cycle of surface turbulent fluxes at Els Plans (rainfed site), July 14-30 2021. The envelope for \mesomean represents the 25th and 75th percentiles of the distribution.}
    \label{fig:elsplans_surfacevars}
\end{figure}

\hfill

%%CCL%%
The key takeaways from this comparison of near-surface variables over the SOP is that the irrigation parametrization can significantly improve the results at La Cendrosa, especially if irrigation demand is fully satisfied, as done in the \irrboost sensitivity experiment.
In particular, even if the observations cannot be matched perfectly, near-surface variables simulated by ICOLMDZOR are very close to the \mesomean values. Considering that the MesoNH simulation is a relevant reference (ensured by the good performance of \mesoexact), it means that a good representation of grid-cell average values is achieved in the \irrboost simulation. 
In the second part of the SOP, a dry bias in ICOLMDZOR was identified, which could be improved be not entirely corrected by irrigation. Since this limitation is also visible at the rainfed site of Els Plans, and occurs simultaneously with a change in near-surface wind regime, it was hypothesized to be mainly the consequence of non-local effects, such as insufficient moisture advection over the LIAISE study area. The next section provides a study of the vertical structure of the ABL over two days, July 15th and July 20th, to investigate the differences between the first and second part of the SOP, and characterise the impact of irrigation and surface heterogeneities in these two different contexts.
\clearpage %todo:keep ?

\section{Vertical structure of the atmosphere}
\label{sec:iop}

Out of the seven IOP days over which radiosoundings were conducted on both sites, two were selected (July 15th and 20th) to investigate the atmospheric behaviour of the ICOLMDZOR LAM and the impact of irrigation on the vertical structure of the ABL.

\subsection{Surface conditions on selected IOP days}
%todo:reorganize, describe obs for all vars and then comment on model performance ?
%mesoexact is very very good on most vars, making it an interesting reference

July 15th was selected as a representative IOP day for the beginning of the SOP, showing similarities with observed profiles on the 16th and 17th. As the SOP progressed, 2-meter temperature increased and the wind regime became much more changeable, with more influence from the sea breeze circulation.
The three following IOP days (20th, 21st, 22nd) therefore presented different conditions, and this is why July 20th was also selected to analyse the impacts of irrigation on the ABL under different conditions.
The time series at La Cendrosa over both days show that July 20th was a warmer day than July 15th (Fig. \ref{fig:iop_days_TS_energy}a,b) and that the lower atmospher was moister (Fig. \ref{fig:iop_days_TS_energy}c,d).
%LMDZ biases in t2m, q2m
At the surface, ICOLMDZOR presents a warm bias on July 15th but almost not temperature bias in the \irrboost simulation on July 20th. As previously noticed in general over the SOP, 2-m specific humidity is overestimated by ICOLMDZOR at night, but decreases during the day. In \irrboost, it remains slightly above observed and MesoNH values during the day on July 15th, but on July 20th it is underestimated.
%link to flat
This may partly be explained by the simulated latent heat flux.%todo:meh
In the first part of the SOP, the alfalfa crops at La Cendrosa were still freshly cut and had not recovered their original size, LAI and transpiration levels. %todo:rephrase to refer to previous explanation
Since the models do not account for this information, the latent heat flux is overestimated by MesoNH (both \mesoexact and \mesomean are above the observations) and ICOLMDZOR (in the \irrboost simulation), whereas the sensible heat flux is underestimated (Fig. \ref{fig:iop_days_TS_energy}e, g). 
On July 20th, observed latent heat flux is higher and sensible heat flux lower, even presenting negative values in the afternoon.
It is difficult to disentangle the increase due to the vegetation growth from the increased evaporative demand due to the warming, but in \mesoexact, the increase is rather small.%blah
For both turbulent fluxes, as noticed in Section \ref{sec:sop}, the \irrboost simulation matches really well the \mesomean aggregated value throughout the day. It is also the case for 2-meter temperature but not really for specific humidity and 10m winds, which might result from limitations of the dynamics rather than the LMDZ physics.

% However, MesoNH does not overestimate 2-meter specific humidity, and  


%Fig : energy fluxes Cendrosa
\begin{figure}[hbtp]
    \centering
    \begin{tabular}{cc}
        %rad fluxes
        \begin{subfigure}[t]{0.5\textwidth}
            \caption{}
            \includegraphics[width=\textwidth]{images/chap5/IOP_TS/TS_2021-07-15_cendrosa_SWdnSFC.png}
        \end{subfigure} &
        \begin{subfigure}[t]{0.5\textwidth}
            \caption{}
            \includegraphics[width=\textwidth]{images/chap5/IOP_TS/TS_2021-07-20_cendrosa_SWdnSFC.png}
        \end{subfigure} \\
        \begin{subfigure}[t]{0.5\textwidth}
            \caption{}
            \includegraphics[width=\textwidth]{images/chap5/IOP_TS/TS_2021-07-15_cendrosa_LWdnSFC.png}
        \end{subfigure} &
        \begin{subfigure}[t]{0.5\textwidth}
            \caption{}
            \includegraphics[width=\textwidth]{images/chap5/IOP_TS/TS_2021-07-20_cendrosa_LWdnSFC.png}
        \end{subfigure} \\

        %turb fluxes
        \begin{subfigure}[t]{0.5\textwidth}
            \caption{}
            \includegraphics[width=\textwidth]{images/chap5/IOP_TS/TS_2021-07-15_cendrosa_flat.png}
        \end{subfigure} &
        \begin{subfigure}[t]{0.5\textwidth}
            \caption{}
            \includegraphics[width=\textwidth]{images/chap5/IOP_TS/TS_2021-07-20_cendrosa_flat.png}
        \end{subfigure} \\
        \begin{subfigure}[t]{0.5\textwidth}
            \caption{}
            \includegraphics[width=\textwidth]{images/chap5/IOP_TS/TS_2021-07-15_cendrosa_sens.png}
        \end{subfigure} &
        \begin{subfigure}[t]{0.5\textwidth}
            \caption{}
            \includegraphics[width=\textwidth]{images/chap5/IOP_TS/TS_2021-07-20_cendrosa_sens.png}
        \end{subfigure} \\
    \end{tabular}
    \caption{}
    \label{fig:iop_days_TS_energy}
\end{figure}

%Fig : surface variables Cendrosa
\begin{figure}[hbtp]
    \centering
    \begin{tabular}{cc}
        %t2m, q2m
        \begin{subfigure}[t]{0.5\textwidth}
            \caption{}
            \includegraphics[width=\textwidth]{images/chap5/IOP_TS/TS_2021-07-15_cendrosa_t2m.png}
        \end{subfigure} &
        \begin{subfigure}[t]{0.5\textwidth}
            \caption{}
            \includegraphics[width=\textwidth]{images/chap5/IOP_TS/TS_2021-07-20_cendrosa_t2m.png}
        \end{subfigure} \\
        \begin{subfigure}[t]{0.5\textwidth}
            \caption{}
            \includegraphics[width=\textwidth]{images/chap5/IOP_TS/TS_2021-07-15_cendrosa_q2m.png}
        \end{subfigure} &
        \begin{subfigure}[t]{0.5\textwidth}
            \caption{}
            \includegraphics[width=\textwidth]{images/chap5/IOP_TS/TS_2021-07-20_cendrosa_q2m.png}
        \end{subfigure} \\

        %turb fluxes
        \begin{subfigure}[t]{0.5\textwidth}
            \caption{}
            \includegraphics[width=\textwidth]{images/chap5/IOP_TS/TS_2021-07-15_cendrosa_wind_speed_10m.png}
        \end{subfigure} &
        \begin{subfigure}[t]{0.5\textwidth}
            \caption{}
            \includegraphics[width=\textwidth]{images/chap5/IOP_TS/TS_2021-07-20_cendrosa_wind_speed_10m.png}
        \end{subfigure} \\
        \begin{subfigure}[t]{0.5\textwidth}
            \caption{}
            \includegraphics[width=\textwidth]{images/chap5/IOP_TS/TS_2021-07-15_cendrosa_wind_direction_10m.png}
        \end{subfigure} &
        \begin{subfigure}[t]{0.5\textwidth}
            \caption{}
            \includegraphics[width=\textwidth]{images/chap5/IOP_TS/TS_2021-07-20_cendrosa_wind_direction_10m.png}
        \end{subfigure} \\
    \end{tabular}
    \caption{}
    \label{fig:iop_days_TS_surfvars}
\end{figure}

%todo:figure of winds at 12UTC (10m and 850hPa) ideally mesoNH and LMDZ
\clearpage

\subsection{Vertical profiles at 12UTC}
%July 15th
%noirr : ABL too high in ICOLMDZ (also visible at Els Plans), fsens too high ?
%ICOLMDZ wind speed lower than obs but mostly agrees with mesoNH, wind direction is Ok
%irrboost : ABL height clearly lowered, getting much closer to mesoMean, although not fully, still missing few 100m
%irrigation only slightly reduced warm bias, moistens whole ABL (closer to mesoExact than mesoMean or obs)

%no impact at ElsPlans, not a surprise

%Fig : profiles 1507 12UTC
\begin{figure}[hbtp]
    \centering
    \makebox[\textwidth][c]{%
    \begin{tabular}{@{}cccc@{}}
        %cendrosa
        \begin{subfigure}[t]{0.382\textwidth}
            \caption{}
            \includegraphics[width=\textwidth]{images/chap5/profiles/profile_cendrosa_theta_1507_.png}
        \end{subfigure} &
        \begin{subfigure}[t]{0.289\textwidth}
            \caption{}
            \includegraphics[width=\textwidth]{images/chap5/profiles/profile_cendrosa_ovap_1507_.png}
        \end{subfigure} &
        \begin{subfigure}[t]{0.283\textwidth}
            \caption{}
            \includegraphics[width=\textwidth]{images/chap5/profiles/profile_cendrosa_wind_speed_1507_.png}
        \end{subfigure} &
        \begin{subfigure}[t]{0.283\textwidth}
            \caption{}
            \includegraphics[width=\textwidth]{images/chap5/profiles/profile_cendrosa_wind_direction_1507_.png}
        \end{subfigure} \\
        %elsplans
        \begin{subfigure}[t]{0.382\textwidth}
            \caption{}
            \includegraphics[width=\textwidth]{images/chap5/profiles/profile_elsplans_theta_1507_.png}
        \end{subfigure} &
        \begin{subfigure}[t]{0.289\textwidth}
            \caption{}
            \includegraphics[width=\textwidth]{images/chap5/profiles/profile_elsplans_ovap_1507_.png}
        \end{subfigure} &
        \begin{subfigure}[t]{0.283\textwidth}
            \caption{}
            \includegraphics[width=\textwidth]{images/chap5/profiles/profile_elsplans_wind_speed_1507_.png}
        \end{subfigure} &
        \begin{subfigure}[t]{0.283\textwidth}
            \caption{}
            \includegraphics[width=\textwidth]{images/chap5/profiles/profile_elsplans_wind_direction_1507_.png}
        \end{subfigure} \\
    \end{tabular}
    }
    \caption{Vertical profiles at 12UTC on July 15th, at La Cendrosa (a-d) and Els Plans (e-h).}
    \label{fig:profiles_cendrosa_1507}
\end{figure}


%July 20th
%noirr : ABL too low (even more striking at els plans), although fsens strongly overestimated, consequence of dryness ?
%irrboost : reduces warm bias (half) and dry bias (matches obs but mesoMean even moister
%irrboost : lower ABL, making it worse...
%winds: link with ABL structure, missing lower jet at Els Plans

%Fig : profiles 2007 12UTC
\begin{figure}[hbtp]
    \centering
    \makebox[\textwidth][c]{%
    \begin{tabular}{@{}cccc@{}}
        %cendrosa
        \begin{subfigure}[t]{0.382\textwidth}
            \caption{}
            \includegraphics[width=\textwidth]{images/chap5/profiles/profile_cendrosa_theta_2007_.png}
        \end{subfigure} &
        \begin{subfigure}[t]{0.289\textwidth}
            \caption{}
            \includegraphics[width=\textwidth]{images/chap5/profiles/profile_cendrosa_ovap_2007_.png}
        \end{subfigure} &
        \begin{subfigure}[t]{0.283\textwidth}
            \caption{}
            \includegraphics[width=\textwidth]{images/chap5/profiles/profile_cendrosa_wind_speed_2007_.png}
        \end{subfigure} &
        \begin{subfigure}[t]{0.283\textwidth}
            \caption{}
            \includegraphics[width=\textwidth]{images/chap5/profiles/profile_cendrosa_wind_direction_2007_.png}
        \end{subfigure} \\
        %elsplans
        \begin{subfigure}[t]{0.382\textwidth}
            \caption{}
            \includegraphics[width=\textwidth]{images/chap5/profiles/profile_elsplans_theta_2007_.png}
        \end{subfigure} &
        \begin{subfigure}[t]{0.289\textwidth}
            \caption{}
            \includegraphics[width=\textwidth]{images/chap5/profiles/profile_elsplans_ovap_2007_.png}
        \end{subfigure} &
        %winds
        \begin{subfigure}[t]{0.283\textwidth}
            \caption{}
            \includegraphics[width=\textwidth]{images/chap5/profiles/profile_elsplans_wind_speed_2007_.png}
        \end{subfigure} &
        \begin{subfigure}[t]{0.283\textwidth}
            \caption{}
            \includegraphics[width=\textwidth]{images/chap5/profiles/profile_elsplans_wind_direction_2007_.png}
        \end{subfigure} \\
    \end{tabular}
    }
    \caption{Vertical profiles at 12UTC on July 20th, at La Cendrosa (a-d) and Els Plans (e-h).}
    \label{fig:profiles_cendrosa_2007}
\end{figure}

\clearpage

\section{Importance of surface fluxes heterogeneities}
\clearpage

\section{Chapter conclusions}
%irrigation without boost does not have very significant impact
%with boostirr, manage to get much closer to observations 

%no impact on Els Plans site, not a surprise


%discussion
% flat in irrboost is mostly driven by bare soil evap : realistic ? error compensation? need further developments in plant behaviour ?
%irrboost very simplified test, setup not good for dynamical effects, very strong irrig for some grid cells and no irrig at Els Plans
%setup does not allow a study of heterogeneity at Els Plans but it was not the main aim here, it rather serves as a reference for noirr case


\clearpage

\section{Chapter appendix}

\begin{figure}[hbtp]
    \centering
    \begin{tabular}{cc}
        \begin{subfigure}[t]{0.5\textwidth}
            \caption{}
            \includegraphics[width=\textwidth]{images/chap5/SOP_TS_DC/time_series_cendrosa_SWdnSFC.png}
        \end{subfigure} &
        \begin{subfigure}[t]{0.5\textwidth}
            \caption{}
            \includegraphics[width=\textwidth]{images/chap5/SOP_TS_DC/diurnal_cycle_cendrosa_SWdnSFC.png}
        \end{subfigure} \\
        
        \begin{subfigure}[t]{0.5\textwidth}
            \caption{}
            \includegraphics[width=\textwidth]{images/chap5/SOP_TS_DC/time_series_cendrosa_LWdnSFC.png}
        \end{subfigure} &
        \begin{subfigure}[t]{0.5\textwidth}
            \caption{}
            \includegraphics[width=\textwidth]{images/chap5/SOP_TS_DC/diurnal_cycle_cendrosa_LWdnSFC.png}
        \end{subfigure} \\

        \begin{subfigure}[t]{0.5\textwidth}
            \caption{}
            \includegraphics[width=\textwidth]{images/chap5/SOP_TS_DC/time_series_elsplans_SWdnSFC.png}
        \end{subfigure} &
        \begin{subfigure}[t]{0.5\textwidth}
            \caption{}
            \includegraphics[width=\textwidth]{images/chap5/SOP_TS_DC/diurnal_cycle_elsplans_SWdnSFC.png}
        \end{subfigure} \\
        
        \begin{subfigure}[t]{0.5\textwidth}
            \caption{}
            \includegraphics[width=\textwidth]{images/chap5/SOP_TS_DC/time_series_elsplans_LWdnSFC.png}
        \end{subfigure} &
        \begin{subfigure}[t]{0.5\textwidth}
            \caption{}
            \includegraphics[width=\textwidth]{images/chap5/SOP_TS_DC/diurnal_cycle_elsplans_LWdnSFC.png}
        \end{subfigure} \\
    \end{tabular}
    \caption{Time series and mean diurnal cycle of radiative fluxes on both sites, July 14-30 2021.}
    \label{fig:bothsites_rad}
\end{figure}


%Fig : energy fluxes ElsPlans
\begin{figure}[hbtp]
    \centering
    \begin{tabular}{cc}
        %rad fluxes
        \begin{subfigure}[t]{0.5\textwidth}
            \caption{}
            \includegraphics[width=\textwidth]{images/chap5/IOP_TS/TS_2021-07-15_elsplans_SWdnSFC.png}
        \end{subfigure} &
        \begin{subfigure}[t]{0.5\textwidth}
            \caption{}
            \includegraphics[width=\textwidth]{images/chap5/IOP_TS/TS_2021-07-20_elsplans_SWdnSFC.png}
        \end{subfigure} \\
        \begin{subfigure}[t]{0.5\textwidth}
            \caption{}
            \includegraphics[width=\textwidth]{images/chap5/IOP_TS/TS_2021-07-15_elsplans_LWdnSFC.png}
        \end{subfigure} &
        \begin{subfigure}[t]{0.5\textwidth}
            \caption{}
            \includegraphics[width=\textwidth]{images/chap5/IOP_TS/TS_2021-07-20_elsplans_LWdnSFC.png}
        \end{subfigure} \\

        %turb fluxes
        \begin{subfigure}[t]{0.5\textwidth}
            \caption{}
            \includegraphics[width=\textwidth]{images/chap5/IOP_TS/TS_2021-07-15_elsplans_flat.png}
        \end{subfigure} &
        \begin{subfigure}[t]{0.5\textwidth}
            \caption{}
            \includegraphics[width=\textwidth]{images/chap5/IOP_TS/TS_2021-07-20_elsplans_flat.png}
        \end{subfigure} \\
        \begin{subfigure}[t]{0.5\textwidth}
            \caption{}
            \includegraphics[width=\textwidth]{images/chap5/IOP_TS/TS_2021-07-15_elsplans_sens.png}
        \end{subfigure} &
        \begin{subfigure}[t]{0.5\textwidth}
            \caption{}
            \includegraphics[width=\textwidth]{images/chap5/IOP_TS/TS_2021-07-20_elsplans_sens.png}
        \end{subfigure} \\
    \end{tabular}
    \caption{}
    \label{fig:iop_days_TS_energy_elsplans}
\end{figure}

%Fig : surface variables ElsPlans
\begin{figure}[hbtp]
    \centering
    \begin{tabular}{cc}
        %t2m, q2m
        \begin{subfigure}[t]{0.5\textwidth}
            \caption{}
            \includegraphics[width=\textwidth]{images/chap5/IOP_TS/TS_2021-07-15_elsplans_t2m.png}
        \end{subfigure} &
        \begin{subfigure}[t]{0.5\textwidth}
            \caption{}
            \includegraphics[width=\textwidth]{images/chap5/IOP_TS/TS_2021-07-20_elsplans_t2m.png}
        \end{subfigure} \\
        \begin{subfigure}[t]{0.5\textwidth}
            \caption{}
            \includegraphics[width=\textwidth]{images/chap5/IOP_TS/TS_2021-07-15_elsplans_q2m.png}
        \end{subfigure} &
        \begin{subfigure}[t]{0.5\textwidth}
            \caption{}
            \includegraphics[width=\textwidth]{images/chap5/IOP_TS/TS_2021-07-20_elsplans_q2m.png}
        \end{subfigure} \\

        %turb fluxes
        \begin{subfigure}[t]{0.5\textwidth}
            \caption{}
            \includegraphics[width=\textwidth]{images/chap5/IOP_TS/TS_2021-07-15_elsplans_wind_speed_10m.png}
        \end{subfigure} &
        \begin{subfigure}[t]{0.5\textwidth}
            \caption{}
            \includegraphics[width=\textwidth]{images/chap5/IOP_TS/TS_2021-07-20_elsplans_wind_speed_10m.png}
        \end{subfigure} \\
        \begin{subfigure}[t]{0.5\textwidth}
            \caption{}
            \includegraphics[width=\textwidth]{images/chap5/IOP_TS/TS_2021-07-15_elsplans_wind_direction_10m.png}
        \end{subfigure} &
        \begin{subfigure}[t]{0.5\textwidth}
            \caption{}
            \includegraphics[width=\textwidth]{images/chap5/IOP_TS/TS_2021-07-20_elsplans_wind_direction_10m.png}
        \end{subfigure} \\
    \end{tabular}
    \caption{}
    \label{fig:iop_days_TS_surfvars_elsplans}
\end{figure}
%todo:relocate appendix